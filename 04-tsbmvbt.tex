%%=====================================================================
%% Time-Split and Multiversion B-trees
%%=====================================================================
\chapter{Time-Split and Multiversion \Btree{}s}
\label{chapter:tsbmvbt}

The previous chapters have discussed the theory behind multiversion
databases (that is, partially persistent transaction-time databases), and 
reviewed some approached for indexing such data.
So far, none of the structures introduced have been optimal,
according to our definition (\defref{def:optimal-mv}).
In this chapter, we review three of the more recent multiversion index
structures, one of which are optimal.
In addition, we shortly discuss a multiversion database system that uses one
of these structures and is built on top of the commercial Microsoft SQL
Server.

We begin this chapter in \secref{sec:tsbmvbt:common} by listing some of
the common design ideas shared by all the efficient structures reviewed in
this chapter. 
Then, in \secref{sec:tsbmvbt:tsb}, we review the first of these structures,
the time-split \Btree\
of Lomet and
Salzberg (\TSBtree~\cite{lomet:1989:tsb,lomet:1990:tsb-performance}). 
After that, \secref{sec:tsbmvbt:immortaldb} describes
Immortal~DB~\cite{lomet:2005:immortaldb,lomet:2006:transactiontime,lomet:2008:version-compression,lomet:2009:improving},
which is based on the \TSBtree. 
Immortal~DB is a multiversion database management system that Microsoft is
researching and developing on top of the Microsoft SQL Server.
In \secref{sec:tsbmvbt:mvbt}, we review the multiversion \Btree\ of Becker 
et~al.\ (MVBT,~\cite{becker:1993:optimal,becker:1996:mvbt}), the first optimal
multiversion index structure.
Finally, \secref{sec:tsbmvbt:mvas} describes the multiversion access method
of Varman and Verma (MVAS,~\cite{varman:1997:multiversion}), which was
developed at about the same time as the MVBT, and shares many characteristics
of the MVBT\@.
Varman and Verma use a slightly more relaxed definition of optimality for
multiversion indexes, and the MVAS is not optimal according to our definition.



%% Common Design Bases
%%---------------------------------------------------------------------
\section{Common Design Bases}
\label{sec:tsbmvbt:common}

All of the index structures presented in this chapter have a similar
structure: the database pages cover regions in key-version space, the
index structures form a directed acyclic graph of database pages (although
these structures may still be called \emph{trees}), and for each
version, there exists a search tree~$S_v$ (see definition below) that is
used to locate the entries belonging to that version.
Let us call these multiversion structures region-based multiversion index
structures:

\thmskip
\begin{definition}
A \emph{region-based multiversion index} is a multiversion database
index structure in which all the pages cover regions in key-version space.
The structure of the pages forms a directed acyclic graph.
Let \kvr{p} denote the key-version region of a page $p$. 
Each parent page~$p$ at level~$l$ contains links to child pages $Q$ at
level $l-1$ so that $q \in Q \Leftrightarrow \kvr{p} \cap \kvr{q}
\neq \emptymark$. 
The key-version regions of pages on the same level of the graph do not
overlap.
\end{definition}
\thmskip
\begin{definition}
\label{def:search-tree}
For each version~$v$ in a region-based multiversion index, there is
a \emph{search tree}~$S_v$ that is a subgraph of the entire multiversion
index graph.
The subgraph~$S_v$ is a tree and all the entries of the data items that are
alive at version~$v$ are located in the pages of~$S_v$. 
An example of a search tree $S_{10}$ is shown in
\figref{fig:sv-mv-index-comparison:mv} on
page~\pageref{fig:sv-mv-index-comparison:mv}.
For each search tree~$S_v$ and all levels~$l$ of~$S_v$, the pages
that belong to $S_v$ at the same level~$l$ partition the entire key-space into
disjoint regions.
Each search tree thus covers the entire key space at each level of the search
tree.
\end{definition}
\thmskip
\begin{definition}
\label{def:consistent-balanced}
A region-based multiversion index structure is said to be \emph{structurally
consistent}, if all the index-specific invariants of the structure
hold; and \emph{balanced}, if 
(1)~it is structurally consistent;
(2)~all the pages of the search tree~$S_v$ contain at least a minimum number
of entries that are alive at version~$v$, for each version~$v$; and
(3)~for any search tree~$S_v$, all the root-to-leaf paths of~$S_v$ are of the
same length.
\end{definition}
\thmskip

Note that our definition of a balanced index structure requires that the
lengths of all the search paths within any one version are of the same
length. 
In practice, this is guaranteed by designing the structure-modification
operations so that the index never becomes unbalanced.

\begin{figure}[htb]
\begin{center}
\begin{tikzpicture}[x=3.5em,y=3.7em,every node/.style={font=\small}]

%\fill[blue!50] (-1,-1) rectangle (5,3);

\node[page] (p1) at (0,0) {$p_1$};
\node[page] (p2) at (1,0) {$p_2$};
\node[page] (p3) at (2,0) {$p_3$};
\node[page] (p4) at (3,0) {$p_4$};
\node[page] (p5) at (4,0) {$p_5$};
\node[page] (p6) at (0.5,1) {$p_6$};
\node[page] (p7) at (1.8,1) {$p_7$};
\node[page] (p8) at (3.2,1) {$p_8$};
\node[page] (p9) at (2.5,2) {$p_9$};

\node[db] (roots) at (0, 2) {$\rootstarp$};

\draw (p6.south) -- (p1.north);
\draw (p6.south) -- (p2.north);

\draw (p7.south) -- (p2.north);
\draw (p7.south) -- (p3.north);

\draw (p8.south) -- (p4.north);
\draw (p8.south) -- (p5.north);

\draw (p9.south) -- (p7.north);
\draw (p9.south) -- (p8.north);

\draw[->,densely dotted] (roots) to[bend left=10] (p9.west);
\draw[->,densely dotted] (roots) to[bend left=10] (p6.north);

\node[anchor=west] at (4.5,0.2) {\footnotesize Leaf pages};
\draw[dashed] (-1,0.5) -- (7,0.5);
\node[anchor=west] at (4.5,0.8) {\footnotesize Index pages};

% \draw[densely dotted] ($(p6) + (0,0.5)$) to[controls=+(180:1) and +(0:0.32)] 
%   ($(p1) + (-0.2,0.5)$) to[controls=+(180:0.32) and +(90:0.2)]
%   ($(p1) + (-0.5,-0.2)$) to[controls=+(270:0.2) and +(180:1)]
%   ($(p1) + (0.5,-0.5)$) to[controls=+(0:1) and +(270:0.1)]
%   ($(p2) + (0.5,-0.2)$) to[controls=+(90:0.1) and +(0:0.32)]
%   ($(p2) + (0.2,0.5)$) to[controls=+(180:0.32) and +(0:1)]
%   ($(p6) + (0,0.5)$);
% 
% \node[pin={[pin edge=solid,pin distance=1em]110:{\small $S_v$}}] 
%   at ($(p1) + (-0.05,0.9)$) {};

\end{tikzpicture}

\figcaption{Structure of a balanced multiversion index}{}
\label{fig:mv-index-example}
\end{center}
\end{figure}

\figref{fig:sv-mv-index-comparison:mv} in the previous chapter shows the
general structure of these multiversion index structures, and
\figref{fig:mv-index-example} shows the structure of a balanced
region-based multiversion index that has an auxiliary \rootstar\ structure
for locating the roots of different search trees.
Note that search trees $S_{v_1}$ and $S_{v_2}$ rooted at pages $p_6$ and
$p_9$ have a different height. 
Let us also define what we mean by live entries and live pages in the
multiversion indexes:

\thmskip
\begin{definition}
\label{def:alive-entries-pages}
For all versions~$v$, an entry that represents a data item is \emph{alive} at
version~$v$ if the data item is alive at version~$v$; and a database page is
\emph{alive} at version~$v$ if it is part of the search
tree~$S_v$.
\end{definition}
\thmskip


Salzberg et~al.\ have recently published a general framework for indexing
fully persistent transaction-time data~\cite{salzberg:2004:framework} (recall 
\defref{def:fully-persistent}). 
Although designed to include fully persistent indexes, the framework
encompasses many of the features present in the efficient indexes reviewed in
this chapter.
Such features are, for example, maintaining a minimum number of live entries
(\defref{def:alive-at-version}) in each page and consolidating (merging)
those pages where the number of live entries falls below the acceptable
limit.
Because the framework is designed for full persistence, some of its features
are however unnecessarily complicated for partial persistence.
More specifically, the framework describes a version-tree structure that is
used to determine which data item entries are along the same version branch.
This adds a significant overhead to finding the correct version of a data
item from a database page. 
Furthermore, some of the pages in the index structure described in the
framework can contain \emph{ghost pages}, which are pages that only contain
\emph{null markers} that are used to specify endpoints of the life spans
of data items.
As is shown with the multiversion \Btree\ (described in
\secref{sec:tsbmvbt:mvbt}), these pages can be avoided if the
structure is only partially persistent.


%% Time-split B-tree
%%---------------------------------------------------------------------
\section{Time-split \Btree}
\label{sec:tsbmvbt:tsb}
\label{def:tsb}

The \emph{time-split \Btree} (\TSBtree) of Lomet and
Salzberg~\cite{lomet:1989:tsb,lomet:1990:tsb-performance} is a multiversion
index structure that is based on Easton's write-once \Btree\ (see
\secpageref{def:wobt}). 
The \TSBtree\ structure forms a directed acyclic graph of database pages.
There is a single root page that is shared by all versions, and the 
height of the search tree~$S_v$ is therefore the same for all
versions~$v$.
The root page of the \TSBtree\ covers the entire key-version space.
The graph structure is formed by splitting pages that have become full.
Each page can be either \emph{key-split} or \emph{time-split}.
Splitting a page~$p$ creates a new page~$p'$ and the entries are distributed
between $p$ and $p'$ based on the new dimensions of the pages. 
If the key-version range of a child page (or the life span of a
leaf-page entry) extends past the split boundary, the child page identifier
(or leaf-page entry) is duplicated to both of the pages $p$ and $p'$.

There are two types of pages in the \TSBtree: \emph{current pages} and
\emph{historical pages}.
The life spans of all current pages cover the current version; that is, the
current pages are alive.
The time-split operation creates a new historical page $p'$.
When time-splitting page~$p$, entries with deletion times smaller than the
split boundary are moved to the historical page~$p'$, entries with life spans
covering the split boundary are copied to the historical page, and the rest
of the entries are left in the current page~$p$. 
The page~$p$ thus remains in use for current-version queries, and the
historical page~$p'$ can be moved to a tertiary storage for archival.
This is possible, because the \TSBtree\ maintains the following invariant:

\thmskip
\begin{invariant}
\label{inv:tsb:one-parent}
In the \TSBtree, current pages have at most a single parent. 
Historical pages can have multiple parents.
\end{invariant}
\thmskip

If such an invariant were not enforced, there might be an unbounded number
of parent pages that need to be updated when a page is split.
However, this invariant imposes a restriction on the time-split operation:
none of the live entries on page~$p$ must be copied to the historical
page~$p'$. 
A page~$p$ therefore cannot be time-split into a historical page~$p'$ based
on a version~$v$ that is covered by the life span of a live entry; that is,  
if $p$ contains an entry with a life span $[v_1, \infty)$ such that $v_1 <
v$. 
By adhering to these rules, the following invariant can be maintained, and
the historical pages can be directly written to a write-once media
during a time-split.

\thmskip
\begin{invariant}
\label{inv:tsb:historical-immutable}
In the \TSBtree, historical pages are never modified.
\end{invariant}
\thmskip

When fine-tuning the \TSBtree, it is possible to make a choice on how often 
key splits and time splits are used.
If the goal is to minimize the index size, key splits should be used more
often.
If the performance of the current-version queries is important, version
splits should be more frequently applied.
This is a natural way for adjusting the behaviour of the index for varying
database loads.
The original \TSBtree~\cite{lomet:1989:tsb} was designed mainly for
scenarios where data is never deleted, and pages therefore do not need to be
merged.
This limitation still exists, and it means that the key ranges of pages can
only shrink, and never expand. 

The \TSBtree\ that was adapted for the Immortal~DB (discussed in the next
section) manages data deletion by inserting new entries that mark item
deletion. 
While this approach works, it means that key splits and time splits
alone cannot guarantee that pages have a minimum number of live entries for
all the versions that are stored in the pages.
For optimality, the access costs of the operations must be
logarithmic in the number of entries that are alive at the queried version.
This is guaranteed if the index structure is balanced (see
\defref{def:consistent-balanced}), so that each page~$p$ with a life span
$[v_1, v_2)$ contains at least \minlive\phantomsection\label{def:minlive}
entries that are alive at version~$v$, for each $v \in [v_1, v_2)$.
%and all the root-to-leaf paths of the search tree $S_v$ are of the same
%length. 
Here, \minlive\ is an integer configuration variable that must be larger than
one so that the fan-out at each level of the search tree is greater than one,
thus guaranteeing a logarithmic height for the search tree and maintaining
the performance of range queries.
The variable \minlive\ must furthermore be linearly dependent on the page
capacity~\capacity, so that the cost differs only by a constant when the base
of the logarithm is~\capacity.

Lomet and Salzberg~\cite{lomet:1989:tsb,lomet:1990:tsb-performance} do not
give asymptotic bounds for the costs of the user actions, but rather
derive exact formulas for calculating the size of the index structure and
for the amount of redundancy in the index. 
A general space complexity class for multiversion indexes is \Oh{n/\capacity}
database pages, where $n$ is the number of updates performed in the history of
the database; the \TSBtree\ belongs to this complexity
class~\cite{salzberg:1999:comparison}.
Regarding time complexity, we formulate some general bounds for the costs of
the actions in the \TSBtree\ here:

\thmskip
\begin{theorem}
All single-key actions in the \TSBtree\ targeting any version~$v$ have a cost
of \OhT{\log_\capacity m} pages, where $m$ is the number of updates performed
on the \TSBtree\ during its history. 
The worst-case cost of a key-range query action that queries the range $[k_1,
k_2)$ of version~$v$ is \OhT{\log_\capacity m + \entries{k} / \capacity},
where $\entries{k}$ is the maximum possible amount of discrete keys in the
queried range (for databases that store integer keys, $\entries{k} = k_2 -
k_1$).
\end{theorem}
\begin{proof}
Because the \TSBtree\ has a single root page, the height of the index
structure is determined by the total number of entries indexed by the entire
\TSBtree, which in turn is dependent on the number of updates performed on
the index.
The cost of the single-key operations derives directly from the traversal
from the root page to the correct leaf page.
For the range query, the number of page accesses is not bounded by the size
of the retrieved item set, because it is possible that there are leaf pages
that contain only deleted entries that are not relevant to the query but still
need to be processed. 
In the worst case, the index contains $\entries{k}$ live entries in the
queried range at version~$v_1$; that is, the index contains a live entry for
each possible key in the queried range. 
Assuming that the history contains no deletions, then $m =
\entries{v_1}$ (when version~$v_1$ is the latest version), and the range
query for version~$v_1$ is optimal (as per \defref{def:optimal-mv}), because
$\entries{k} = r$, the size of the retrieved item set.
If all the entries are deleted at version~$v_2$, the entries are marked
deleted but the leaf pages are not merged.
The range-query operation targeting version~$v_2$ must process as many pages
as the query that targets version~$v_1$, even though the pages do not
contain any entries that are relevant to the query.
\end{proof}
\thmskip

Lomet and Salzberg have evaluated the performance of several different
splitting policies~\cite{lomet:1990:tsb-performance}. 
According to their experiments, the policy called Isolated-Key-Split
(\abbr{IKS})\phantomsection\label{def:iks} is a good choice if rewritable
media (write-many, read-many, or \abbr{WMRM}
media\phantomsection\label{def:wmrm}) is available and
inexpensive\footnote{Their exact wording was that the cost of
\abbr{WORM} storage should be less than a factor of ten cheaper than 
\abbr{WMRM} storage cost~\cite{lomet:1990:tsb-performance}.}.
With this policy, pages are key-split whenever more than two thirds of the
entries in the page are alive; and time-split otherwise.
When time-splitting a page, the split is always performed based on the
latest committed version.
This policy optimizes the size of the index structure, and was designed at
a time when \abbr{WMRM} media was not as inexpensive as it is currently.
With the Immortal DB more recent split policies have been introduced. 
These are reviewed in the next section.   

From the beginning, it has been assumed that lazy timestamping (see
\secpageref{def:lazy-ts}) is used with the \TSBtree.
The initial \TSBtree\ article~\cite{lomet:1989:tsb} does not explain how and
when the records should be timestamped with the correct commit-time versions,
but rather suggests that Stonebraker's approach for the PostgreSQL
database system~\cite{stonebraker:1987:postgres} could be used.
In this approach, each relation may be assigned to one of the following three
levels: no archive, light archive, and heavy archive. 
From these, \emph{no archive} practically means that the relation is a
single-version relation, and no access to past states is possible.
For the other settings, the entries are first stamped with the transaction
identifier, and the mapping from commit-time versions to transaction
identifiers is inserted in a special relation when the transaction commits. 
With the \emph{light archive} setting, each time a historical entry is
requested, the transaction identifier is mapped to the commit-time version by
reading the proper value from the special relation.
With the \emph{heavy archive} setting, the mapping is performed on the first
access, and the historical entry is updated with the commit-time version so
that subsequent accesses do not need to load the mapping from the special
relation.
The Immortal DB (discussed in the next section) more clearly defines that a
timestamping scheme similar to the heavy archive option is used to lazily
timestamp the entries.

In conclusion, the TBS-tree is a practical index structure that can be easily
fine-tuned for different data sets, but it does not guarantee optimal
access costs for updates or queries of any version.



%% Immortal DB
%%---------------------------------------------------------------------
\section{Immortal DB}
\label{sec:tsbmvbt:immortaldb}
\label{def:immortaldb}

Lomet et~al.\ have chosen the time-split \Btree~(\TSBtree) as the
basis when implementing multiversion support to the \emph{Immortal~DB}
multiversion database
system~\cite{lomet:2005:immortaldb,lomet:2006:transactiontime,lomet:2008:version-compression,lomet:2009:improving}.
The multiversion database is built on top of existing Microsoft SQL Server
code. 
An important aspect of the implementation was that the existing program code
must be compatible with the new multiversion code.
The multiversion functionality has been introduced gradually, first by
chaining versions together in the original SQL Server \Btree\ index, and then
by replacing the \Btree\ based index with the
\TSBtree~\cite{lomet:2006:transactiontime,lomet:2008:version-compression}.
Immortal DB therefore serves as an example that multiversion functionality
can be gradually added to an existing database system.

The Immortal DB also introduced changes to the \TSBtree\ index structure.
From our point of view, the important ones are the more specific explanation
of lazy timestamping and the new splitting policies designed to cluster the
data items that are alive at the current version more efficiently.
The articles furthermore specify how the entries are stored in
the data
pages~\cite{lomet:2006:transactiontime,lomet:2008:version-compression} and
how the pages can be compressed~\cite{lomet:2008:version-compression}.

The entries in Immortal DB are initially timestamped with the
temporary transaction identifier.
When the transaction commits, a mapping between the transaction identifier and
the commit-time version of the transaction is inserted into a \emph{persistent
timestamp table} (\abbr{PTT})\phantomsection\label{def:ptt}\@.
This table is stored in a \Btree\ index so that the contents are made
persistent and can be recovered in the event of a system crash.
Because all entries need to be timestamped during the first access, there
will be many queries to the \abbr{PTT}\@.
The Immortal DB therefore also maintains a main-memory-based \emph{volatile
timestamp table} (\abbr{VTT})\phantomsection\label{def:vtt} that serves as a
cache for the \abbr{PTT}\@.
In addition to storing these mappings, the \abbr{VTT} also contains a
reference counter for determining how many entries there are that have
not yet been timestamped.
When new data items are inserted to the database, the reference count is
incremented.
When entries in the database pages are timestamped after the transaction that
inserted them has committed, the reference counter is decremented.
When the reference counter reaches zero, the mapping is removed from both
the \abbr{VTT} and the \abbr{PTT}\@.
The \abbr{PTT} is thus used only to provide recoverability: if the system
fails, the \abbr{VTT} can be rebuilt from the \abbr{PTT} contents.
The reference counter, however, is not present in the \abbr{PTT}, and 
is thus lost if the system fails.
This can cause some entries to remain indefinitely in the \abbr{PTT} and
\abbr{VTT}, because it is not known if there still exists entries that 
need to be timestamped with the commit-time version.

The Immortal DB articles also specify two new split policies for the
\TSBtree, namely the \abbr{WOB}-tree split policy and the deferred split
policy. 
The \emph{\abbr{WOB}-tree split
policy}~\cite{lomet:2008:version-compression} updates the \abbr{IKS} split
policy (see \secpageref{def:iks}) so that whenever a page contains more than
two thirds of live entries, it is first time-split and afterwards key-split.
Otherwise the page is time-split, as with the \abbr{IKS} split policy.
The split threshold (two thirds of the entries in this example) is now a
configuration variable, but Lomet et~al.\ use two thirds when there is no
compression in the pages~\cite{lomet:2008:version-compression}.
The \emph{deferred split policy}~\cite{lomet:2009:improving} defers the
key-split that is performed after the time-split in the \abbr{WOB}-tree split
policy.
That is, if a page that needs to be split contains enough live entries
according to the threshold, then the page is time-split and a key-split is
deferred by marking the page.
When the marked page next needs to be split, it is key-split without first
time-splitting it.
Unmarked pages that have fewer live entries than the threshold are
simply time-split, as in the \abbr{WOB}-tree and \abbr{IKS} split policies.

In addition to the deferred key-split, the deferred split policy introduces
batch updates to the \abbr{PTT} table. 
Normally, whenever a transaction commits, the mapping from transaction
identifiers to commit-time versions is inserted to the \abbr{PTT}, thus
adding extra \abbr{I/O} operations to each commit operation.
With the deferred split policy, the mappings are only inserted into the
\abbr{VTT} during a commit operation.
The \abbr{PTT} is then updated later on with a larger batch of mappings. 
This reduces the number of \abbr{I/O} operations as the \abbr{PTT} entries
are clustered next to each other and also because some of the entries may
already have been removed from the \abbr{VTT} and are therefore never
inserted into the \abbr{PTT} at all.
This happens if all the entries a transaction~$T$ inserted have been accessed
and timestamped before the next batch of updates is applied into the
\abbr{PTT}\@. 
In this situation, the transaction entry has already been removed from the
\abbr{VTT} and is never inserted into the \abbr{PTT}.

With Immortal DB, the focus on optimizing the TBS-tree has moved from
optimizing space usage into optimizing both space usage and query
performance \cite{lomet:2009:improving,lomet:2008:version-compression}.  
The original TBS-tree (see previous section) was designed for data that is
never deleted~\cite{lomet:1989:tsb}.
When data deletion is allowed, maintaining the number of live entries for
each page is more challenging.
Even though the new Immortal DB splitting policies are better suited for
transaction histories that also contain deletions, the fact remains that pages
are not merged, and thus key ranges cannot expand.
If a current page of the \TSBtree\ covers the key range $[k_1, k_2)$, all
the items in this range are deleted, and no new items are inserted that fall
into that range, then the current page will contain no live entries.
It will still be part of the current-version database, and key-range queries
will process it (also recall the discussion in
\secref{sec:mv-index:optimality}). 
The \TSBtree\ and the Immortal DB are therefore suboptimal multiversion index
structures, and range-query performance may degrade when deletions are
present.



%% Multiversion B-tree
%%---------------------------------------------------------------------
\section{Multiversion \Btree}
\label{sec:tsbmvbt:mvbt}
\label{def:mvbt}

The first optimal multiversion index structure
presented was the multiversion \Btree\ (MVBT) of Becker
et~al.~\cite{becker:1993:optimal,becker:1996:mvbt}.
The MVBT follows a single-update model, in which each update to
the index creates a new version of the index.
The update cannot be rolled back, so the MVBT structure is not transactional.

We can model a single multi-action updating transaction operating on
the MVBT, if we accept that the transaction cannot roll back. 
The updates performed by the updating transaction~$T$ obtain versions that
form a contiguous range $[v_1, v_2]$.
This means that the intermediate states $v: v_1 < v < v_2$ also
persist, even though they are never queried as they are internal to the
transaction. 
For example, suppose that a transaction~$T$ first inserts a data item
with key~$k$, and then updates it again and again.
If the last action of the transaction with commit-time version~$v$ on the
data item with key~$k$ is an update with value $w$, then an optimal
multiversion index only records the data item $(k, v, w)$.
In contrast, the MVBT stores information of uncommitted versions of the data
item that never will be committed.

There are some benefits to this model, however.
Concurrency control is straightforward, because only one updating
transaction is allowed to operate on the index at a time.
As we will show in \secref{sec:tmvbt:multi-action-tx}, multiple read-only
transactions can be allowed to operate on the index concurrently with the
single updating transaction.
Because there is only a single updating transaction, and every update
operation creates a new version of the index, the most recent committed
version can be maintained in a single variable.
The variable~\comver\ denotes the current version of the MVBT index.
At the beginning of each action, \comver\ is incremented, and each operation
on the index is then tagged with the incremented value.
For correct operation, the read-only transactions may only query the
latest version after the active updating transaction has committed.
Therefore the implementation should maintain a separate variable that records
the latest committed version that precedes the version of the active updating
transaction. 
This variable is then read by the read-only transactions, and updated when
the active updating transaction commits. 
For simplicity of the explanations, we will assume that \comver\ denotes the
latest committed version as seen by the updating transaction, and that it is 
incremented at the beginning of the updating transaction.

As in the \TSBtree, pages of the MVBT cover axis-aligned rectangles of
key-version space.
The key-version rectangle of a page~$p$ is unambiguously defined by its
components: the key range~$\kr{p}$\phantomsection\label{def:kr}, and
the version range, or life span~$\vr{p}$\phantomsection\label{def:vr}.
The pages form a directed acyclic graph with leaf pages at level one, and
index pages on consecutively higher levels\footnote{Becker et~al.\ say that
leaf pages are at level zero, but as this is just a matter of convention we
have modified the definition here to better suit the convention used in this
dissertation.}.
In the MVBT, life spans are explicitly stored in each entry in the index
structure.
Leaf-page entries are thus tuples of the form $(k, \vrmark{v}, w)$, where 
$k$ is the key, $\vrmark{v} = [v_1, v_2)$ is the life span, and $w$ is the
associated data stored in the data item.
When a leaf-page entry is first inserted into the index, its life span is set
to $\vrmark{v} \leftarrow [\comver, \infty)$, indicating that it has been
inserted at the current version and that it has not yet been deleted.
If an entry with a life span $\vrmark{v} = [v_1, \infty)$ is deleted, the
life span is replaced with $\vrmark{v} \leftarrow [v_1, \comver)$, thus
marking that the entry was deleted at the current version.
The life span of a leaf-page entry is therefore not static, but it changes
when the item is deleted.
After an item has been deleted (or updated by replacing it with another
item with a different associated value), the life span of the historical
entry becomes static.

For each level~$l$ in the MVBT index, the pages of level~$l$ partition the
key-version space into disjoint rectangles. 
The pages of level~$l$ cover the entire key-version space, except for
those ranges of versions~$v$ for which the height of the search tree $S_v$
was lower than~$l$. 
An example of this is shown in \figref{fig:mvbt-space-partition}. 
In the example, the search tree~$S_0$ of version~$v_0$ contains only the
single leaf-level root page $p_1$, and therefore the MVBT index does not
contain any pages at the second level before version~$v_1$.
At version $v_1$, the root page is split into pages~$p_2$ and~$p_3$, and
therefore a new root page~$p_9$ at the next level is created.

\begin{figure}[htb]
\begin{center}
\subfigure[Level 1]{\begin{tikzpicture}[x=2em,y=2em]
  \small
  \coordinate (o) at (0,0);
  \coordinate (ok) at ($(o) + (0,3.7)$);
  \coordinate (ov) at ($(o) + (6.7,0)$);
  \coordinate[label=below:{$v_0$}] (v0) at (o); 
  \coordinate[label=below:{$v_1$}] (v1) at ($(o) + (1,0)$); 
  \coordinate[label=below:{$v_2$}] (v2) at ($(o) + (2,0)$);
  \coordinate[label=below:{$v_3$}] (v3) at ($(o) + (3,0)$);
  \coordinate[label=below:{$v_4$}] (v4) at ($(o) + (4,0)$);
  \coordinate[label=below:{$v_5$}] (v5) at ($(o) + (5,0)$);
  \coordinate[label=below:{$v_6$}] (v6) at ($(o) + (6,0)$);
 
  \draw[opendash] decorate[opendeco] { (0,3) -- (1,3) };
  \node at (0.5, 1.5) {$p_1$};
  %\draw (o) rectangle node {$p_1$} (1,3);

  \draw (1,0) rectangle node {$p_2$} (2,2);

  %\draw (1,2) rectangle node {$p_3$} (4,3);
  \draw (1,2) -- (1,3);
  \draw[opendash] decorate[opendeco] { (1,3) -- (4,3) };
  \node at (2.5, 2.5) {$p_3$};

  \draw (2,1) rectangle node {$p_4$} (4,2);
  \draw (2,0) rectangle node {$p_5$} (5,1);
  
  %\draw (4,1) rectangle node {$p_6$} (5,3);
  \draw (4,2) -- (4,3);
  \draw[opendash] decorate[opendeco] { (4,3) -- (5,3) };
  \node at (4.5, 2) {$p_6$};
  \draw (5,1) -- (5,3);
  
  \draw (5,0) -- (6,0);
  \draw (5,2) -- (6,2);
  \draw[opendash] decorate[opendeco] { (5,3) -- (6,3) };

  \node at (5.5,2.5) {$p_7$};
  \node at (5.5,1) {$p_8$};

  \draw[opendash] decorate[opendeco] { (6,0) -- (6,2) };
  \draw[opendash] decorate[opendeco] { (6,2) -- (6,3) };

  \draw[->] (o) -- node[below=1em,lbl] {Versions} (ov);
  \draw[->] (o) -- node[above,sloped,lbl] {Keys} (ok);
\end{tikzpicture}

\label{fig:mvbt-space-partition:level-1}}
\subfigure[Level 2]{\begin{tikzpicture}[x=2em,y=2em]
  \small
  \coordinate (o) at (0,0);
  \coordinate (ok) at ($(o) + (0,3.7)$);
  \coordinate (ov) at ($(o) + (6.7,0)$);
  \coordinate[label=below:{$v_0$}] (v0) at (o); 
  \coordinate[label=below:{$v_1$}] (v1) at ($(o) + (1,0)$); 
  \coordinate[label=below:{$v_2$}] (v2) at ($(o) + (2,0)$);
  \coordinate[label=below:{$v_3$}] (v3) at ($(o) + (3,0)$);
  \coordinate[label=below:{$v_4$}] (v4) at ($(o) + (4,0)$);
  \coordinate[label=below:{$v_5$}] (v5) at ($(o) + (5,0)$);
  \coordinate[label=below:{$v_6$}] (v6) at ($(o) + (6,0)$);

  %\draw (1,0) rectangle node {$p_9$} (5,3);
  \draw (1,0) -- (1,3);
  \draw (4,0) -- (4,3);
  \draw[opendash] decorate [opendeco] { (1,3) -- (4,3) };
 
  \draw[opendash] decorate [opendeco] { (4,3) -- (6,3) };

  \draw[opendash] decorate [opendeco] { (6,0) -- (6,3) };

  \node at (5,1.5) {$p_{10}$};
  \node at (2.5,1.5) {$p_{9}$};

  % Router in page p9 to page p5 at the lower level
  \draw[dashed] (2.1,0.1) rectangle node {$\ptr p_5$} (3.9,0.9);
  % Router in page p10 to page p5 at the lower level
  \draw[dashed] (4.1,0.1) rectangle node {$\ptr p_5$} (4.9,0.9);

  \draw[->] (o) -- node[below=1em,lbl] {Versions} (ov);
  \draw[->] (o) -- node[above,sloped,lbl] {Keys} (ok);
\end{tikzpicture}

\label{fig:mvbt-space-partition:level-2}}
\figcaption{Partitioning of the key-version space in the MVBT}%
{In the image, $\comver = v_6$, which is indicated by the open ends of the
life spans of pages $p_7$, $p_8$, and $p_{10}$.
The level \num{2} image shows routers to page $p_5$ in pages $p_9$ and
$p_{10}$.}
\label{fig:mvbt-space-partition}
\end{center}
\end{figure}

The index pages contain pointers, or \emph{routers}, to child
pages~\cite{becker:1993:optimal,becker:1996:mvbt}.
A pointer from a parent page~$p$ to a child page~$p'$ is represented by a
tuple of the form $(\vrmark{k}, \vrmark{v}, p')$, where $p'$ is the
page identifier of the child page~$p'$.
The key range~\vrmark{k} and life span~\vrmark{v} form the router
$(\vrmark{k}, \vrmark{v})$ that guides the search to the child page. 
The router is set to the intersection of the key ranges and life spans of
the parent page and the child page.
This is illustrated in \figref{fig:mvbt-space-partition:level-2}, where the
routers to the child page~$p_5$ are shown in both parent pages~$p_9$
and~$p_{10}$.
For illustration, the page identifiers stored in index pages are prefixed with
a marker~\ptr. 
A parent page~$p$ contains routers to each child page~$p'$ such that
$\kr{p} \cap \kr{p'} \neq \emptymark$ and $\vr{p} \cap \vr{p'} \neq
\emptymark$, and the format of the MVBT index therefore follows the general
convention shown in \figref{fig:sv-mv-index-comparison:mv} on
page~\pageref{fig:sv-mv-index-comparison:mv}.

The structure-modification operations on the MVBT are based on the
\emph{version-split} operation, in which a live page~$p$ is \emph{killed};
that is, $p$~is split at the current version~\comver, and a new live
copy~$p'$ is created. 
The life span $[v,\infty)$ of the old page~$p$ is cropped to
$[v,\comver)$, and the life span of the new page~$p'$ is set to
$[\comver,\infty)$. 
All the live entries in~$p$ are copied to~$p'$. 
Page~$p$ is now considered a \textit{dead page} and is only used for
historical queries, and the new page~$p'$ is used for current-version
queries.
This operation is the same for index pages and leaf pages.
All structure-modification operations target the new page~$p'$, and 
the killed page~$p$ becomes static.
Dead pages are never modified again, although they might be deleted
later to save space (see the discussion of purging old versions in
the \abbr{MVBT} article by Becker
et~al.~\cite{becker:1996:mvbt,becker:1993:optimal}). 
This is stated more formally below.

\thmskip
\begin{invariant}
\label{inv:mvbt-static-dead-page}
Dead pages in the MVBT can only be modified by a purging process that is run
to remove old versions.
If an old version is removed, it can no longer be queried.
From the viewpoint of a user transaction querying for any version, all
encountered dead pages are static.
\end{invariant}
\thmskip

There are two basic structure-modification operations in the MVBT: page split
and page merge.
A page split is triggered when an insertion is attempted into a page
that is full and cannot accommodate the new entry, and a page merge is
triggered when the number of live entries in a page falls below \minlive.
To avoid thrashing between splitting and merging the same page, all pages in
the MVBT must contain between $\minsplit = \minlive +
s$\phantomsection\label{def:minsplit} and $\maxsplit = \capacity -
s$\phantomsection\label{def:maxsplit} entries immediately after any
structure-modification operation, where $s$ is a split-tolerance variable. 
The variable~$s$ determines how many updates must at least be performed
on the page before a new structure-modification operation is required.

When a page~$p$ is \emph{split}, it is first version-split into a new live
page~$p'$.
As described above, page~$p'$ contains the live entries copied from page~$p$.
If the number of live entries in page $p'$ is now between \minsplit\ and
\maxsplit, the operation terminates, because the page can accommodate at
least~$s$ more updates before it needs to be either split or merged.
If the number of live entries is above \maxsplit, page~$p'$ is further
key-split into two pages. 
The key-split operation is identical to a standard \Btree\ key split.
This is possible also for index pages, because at this point the
new page~$p'$ contains only live entries, and thus there are
no entries with overlapping key ranges.
If the number of live entries on page~$p'$ is below \minsplit, the page
must be merged with a sibling page. 
In this case, a sibling page~$p''$ is located, version-split, and merged with
the page~$p'$.
If the merged page contains more than \maxsplit\ entries, it is further
key-split into two pages, similar to the key-split situation above.

The \emph{page merge} operation, which is triggered when the number of live
entries on a page~$p$ falls below \minlive, is identical to the merge
operation that takes place after a version-split.
That is, page~$p$ is first version-split, then a sibling page~$p'$ is located,
version-split, and the resulting live pages are merged.
The merged page might again have more than \maxsplit\ entries, in which case
it is further key-split into two pages.
In this situation, the merge operation resembles the key-redistribution
operation of the \Btree.

% \thmskip
% \begin{theorem}
% Each structure-modification operation on the MVBT transforms a structurally
% consistent and balanced MVBT index into a structurally consistent and balanced
% index, and all structure-modification operations on the MVBT are
% deadlock-free.
% \end{theorem}
% \thmskip
% 
% For a more detailed description of these operations and the proof to the
% theorem, we refer to the extended algorithms in \secref{sec:tmvbt:smo}.
 
% Invariant: historical entries remain in place
Because no historical data is removed from the killed page~$p$ in all the
structure-modification operations, and all the live entries are copied to the
new page~$p'$, we can deduce an even stronger invariant for the MVBT\@:

\thmskip
\begin{invariant}
\label{inv:mvbt-static-entries}
All entries in the MVBT pages remain in place. 
They are never moved to another page.
Only the deletion time of an entry may be changed, always from $\infty$ to
the current version~\comver.
\end{invariant}
\thmskip

% Creation of new root pages to the root*
When a non-leaf page is version-split, the copying of entries
creates a new parent, in addition to the old one(s), for the
child pages pointed to by the copied entries.
As noted before, the \abbr{MVBT} is not a tree but a directed acyclic
graph, which may have several roots.
When a root page is split, a new root page is created, but the old root page
is still used as a starting point when searching for historical entries.
To accomplish this, a separate structure called
\rootstar\ (see \defref{def:rootstar}) is used to store all the different
roots of the MVBT\@. 
The \rootstar\ structure can be implemented, for example, as a
\Btree\ that contains the page identifiers of the different root pages,
indexed by their creation versions. 
It is assumed that the number of different roots is small, and that the
\rootstar\ structure can fully reside in main memory.

The optimality of the MVBT is based on maintaining the following invariant:

\thmskip
\begin{invariant}
\label{inv:mvbt-live-count}
Assuming that each version of the MVBT index consists of only a single
update, then for all versions~$v$ and all pages~$p$, either
(1)~page~$p$ contains at least \minlive\ entries that are alive at
version~$v$, where \minlive\ is a configuration variable that is linearly
dependent on the page capacity~\capacity; or 
(2)~page~$p$ is a root page of a search tree~$S_v$ and $p$ contains at least
two entries that are alive at version~$v$; or 
(3)~page~$p$ is the single page of a search tree~$S_v$ that has a height of
one; or
(4)~page~$p$ is not part of the search tree~$S_v$ and therefore
contains no entries that are alive at version~$v$.
\end{invariant}
\thmskip

The first case of this invariant states that each page either contains enough
entries of the queried version so that all queries (including the
key-range query) are efficient, or the page is not part of the search tree of
the queried version and is therefore not processed at all.
The two other cases are special cases listed for completeness: a root page may
contain as few as two live entries, if there are no more pages at the next
lower level.
Similarly, if all the $n$ live entries of a version~$v$ fit on a single page,
the search tree~$S_v$ contains only a single page that has exactly $n$ live
entries.
Because this invariant is maintained, each search tree~$S_v$ is
asymptotically equivalent to a single-version \Btree\ index, and all queries
in the MVBT index are optimal (see \defref{def:optimal-mv}), if the root
page of the search tree of the queried version is known.

Like the \TSBtree, the space complexity of the MVBT index is also
\Oh{n/\capacity} database pages, where $n$ is the number of updates in the
database history~\cite{salzberg:1999:comparison,becker:1996:mvbt}. 
The time complexity of the MVBT actions has been shown by Becker
et~al.~\cite{becker:1993:optimal,becker:1996:mvbt}, and we reproduce the 
results here:

\thmskip
\begin{theorem}
\label{thm:mvbt-cost}
Assume that every transaction consists of only one update, the transactions
all run in a serial order and they all commit.
The cost (see \defref{def:action-cost}) of the current-version single-key
operations (key query, key insertion, and key deletion) in the MVBT index,
when the roots of the queried search trees are known, is
\OhT{\log_\capacity \entries{\comver}} pages,
the cost of the single-key query operation for version~$v$ is
\OhT{\log_\capacity \entries{v}} pages,
and the cost of the key-range query operation for version~$v$ is
\OhT{\log_\capacity \entries{v} + r/\capacity} pages, where $\entries{v}$
denotes the number of data items that are alive at version~$v$, $r$ is the
number of live entries in the queried range and \capacity\ is the page
capacity.
\end{theorem} 
\begin{proof}
By \invref{inv:mvbt-live-count}, each page of the search tree~$S_v$ of
version~$v$ in the MVBT contains at least \minlive\ entries of version~$v$,
except possibly the root page of~$S_v$.
The MVBT is always balanced, as shown by Becker
et~al.~\cite{becker:1993:optimal,becker:1996:mvbt}.
The height of the search tree~$S_v$ is thus \OhT{\log_\capacity \entries{v}},
because there are $\entries{v}$ entries that are alive at version~$v$, and
\minlive\ is a linear function of the page capacity~\capacity\@.
This explains the logarithmic part $\log_\capacity \entries{v}$ of the costs,
as the search tree must be traversed from the root to the correct leaf node.
The cost of the range query operation is \OhT{\log_\capacity \entries{v} +
r/\capacity} by \thmref{thm:btree-range-cost}, because each
page of the search tree~$S_v$ contains at least \minlive\ entries that are
alive at version~$v$.
\end{proof}
\thmskip

In the discussion above, we have omitted the page accesses required for
locating the root page of version~$v$, when performing a query that targets a
version~$v$.
As mentioned earlier, the MVBT uses a separate \rootstar\ structure to store
the root page identifiers of different versions.
In practice, the \rootstar\ is either a table or a single-version \Btree,
indexed by the version during which the root of the search tree has changed.
For example, suppose that the \rootstar\ contains the following entries:
$\{ (v_0, p_1), (v_3, p_5), (v_{20}, p_{14}) \}$.
Reading from this, versions $[v_0, v_3)$ have page~$p_1$ as
their root page, versions $[v_3, v_{20})$ use page~$p_5$, and the rest
of the versions $[v_{20}, \infty)$ use page~$p_{14}$.

Let us now discuss the cost of locating the correct root page for a user
action.   
If the \rootstar\ is stored in a \Btree\ index, locating the current page
requires access to \OhT{\log_\capacity n} pages, where $n$ is the number of
page identifiers stored in the \rootstar.
If the \rootstar\ is stored as a table, the correct root page identifier can
be located by a binary search that accesses \OhT{\log_2 n} table slots,
requiring access to \OhT{(\log_2 n)/\capacity} pages.
The number of page identifiers in the \rootstar\ is determined by how often
the root page of the current version search tree has changed.
As noted by Becker et~al.~\cite{becker:1996:mvbt}, if the MVBT index has
been created by a sequence of insertions only, the leftmost path of the
current version search tree $S_{\comver}$ contains all the root pages of
the MVBT index, and the number of different roots is therefore $n =
\log_\capacity m = \log_\capacity \entries{\comver}$, where $m$ is the number
of data items stored in the entire index structure.
We therefore expect that the number of different roots will be small in
practice.
It is possible, however, that the number of roots is much larger.

The worst-case scenario would be to alternatively insert an entry in one
version into an originally empty database, and delete it in the next. 
If the index implementation explicitly stores null root page identifiers $(v,
\deletemark)$ to the \rootstar\ to denote that the search tree of
version~$v$ is empty, then the number of roots is directly dependent on the
number of data items stored in the index ($n = \OhT{m}$), because there must
then exist a page identifier for each version in the \rootstar. 
If the implementation reuses the root page identifier of the previous version
leaf-level root page, then the page identifier in the \rootstar\ does not
change for each update.  
The root page must however change after \OhT{\capacity} updates, because the
single root page can only accommodate that many entries.
The number of root pages is therefore at least $n = \OhT{m/\capacity}$, which
is still linearly dependent on the number of versions, assuming that the page
capacity is a constant.
In this pathological worst case, locating the correct version
from the \rootstar\ can have an asymptotic cost that is higher than the cost
of the operation itself.
However, the root page does not need to be separately located for each query
operation.

For updating transactions, and when querying for the most recent version, the
root page of version~\comver\ is required to begin the search tree
traversal.
For efficiency, the MVBT index implementation should cache the page
identifier of the current-version root page. 
Because there is only one updating transaction, and read-only transactions
may not target the version that is being updated, the page identifier can be
stored in a single variable without using locks to protect its value.
Therefore, there is no additional cost for performing any update action, or
any query action that targets the latest version.
Once the updating transaction completes, creating version~$v$, the root page
of version~$v$ remains static, and thus all entries representing
historical versions in the \rootstar\ can be cached.
Each read-only transaction that targets a version~$v < \comver$
needs to locate the page identifier of the root page of the search
tree $S_v$ only once.
The correct page identifier can also be provided by the context.
We therefore feel justified in stating that locating the root
page of the queried version from the \rootstar\ has a negligible effect on
overall query performance.

% MVBT problem scenario
We stated earlier on that the MVBT follows a single-update model, so that
only one update may receive the same version.
Let us now consider what happens if we try to index more than one update
with the same version in the MVBT index.
Consider, for example, simply inserting entries with consecutive keys
$1,2,3,\ldots,n$ to the MVBT\@.
The scenario is shown in \figref{fig:mvbt-invalid-split}, with an
illustrative page capacity of three entries per page.
The first three entries are inserted to page $p_1$, therefore filling the
page.
The insertion of the fourth item leads to an overflow which triggers
a page split.
The page split begins with a version-split operation, which in turn causes
the life span of the old page and its entries to degenerate into an empty
interval $[1,1)$.
This page does not hold any relevant information as it is no longer a part of
any version of the database. 
The pages created earlier by the same transaction are the cause of this
problem.
As it turns out, this problem can be remedied by redesigning the algorithms
as we will show in \secref{sec:tmvbt:smo}.
In this problem scenario, the page could be key-split directly, without first
version-splitting it.

\begin{figure}[htb]
\begin{center}
  \subfigure[Insert key~1]{
\:\:
\begin{tikzpicture}
\node[treepage] (p1) {%
  \treepagecontents{4.5}{$[\!-\infty,\!\infty\!),\![1,\!\infty\!)$}%
  {$(1, [1,\!\infty\!), w_1\!)$}};
\treepageheader{p1}{$p_1$}
\end{tikzpicture}
\:\:
}
\subfigure[Insert keys~2--3]{
\:\:
\begin{tikzpicture}
\node[treepage] (p1) {%
  \treepagecontents{4.5}{$[\!-\infty,\!\infty\!),\![1,\!\infty\!)$}%
  {$(1, [1,\!\infty\!), w_1\!)$\\
   $(2, [1,\!\infty\!), w_2\!)$\\
   $(3, [1,\!\infty\!), w_3\!)$}};
\treepageheader{p1}{$p_1$}
\end{tikzpicture}
\:\:
}\\
\subfigure[Insert key~4]{
\begin{tikzpicture}
\node[treepage] (p1) {%
  \treepagecontents{4.5}{$[\!-\infty,\!\infty\!),$\sout{$[1,\!1)$}}%
  {$(1,\,$\sout{$[1,\!1)$}$, w_1\!)$\\
   $(2,\,$\sout{$[1,\!1)$}$, w_2\!)$\\
   $(3,\,$\sout{$[1,\!1)$}$, w_3\!)$}};
\treepageheader{p1}{$p_1$}

\node[treepage,right=1em of p1] (p2) {%
  \treepagecontents{4.5}{$[\!-\infty,\!3),\![1,\!\infty\!)$}%
  {$(1, [1,\!\infty\!), w_1\!)$\\
   $(2, [1,\!\infty\!), w_2\!)$}};
\treepageheader{p2}{$p_2$}

\node[treepage,right=1em of p2] (p3) {%
  \treepagecontents{4.5}{$[3,\!\infty\!),\![1,\!\infty\!)$}%
  {$(3, [1,\!\infty\!), w_3\!)$\\
   $(4, [1,\!\infty\!), w_4\!)$}};
\treepageheader{p3}{$p_3$}
\end{tikzpicture}
}

  \figcaption{\abbr{MVBT} problem scenario}{Insertion of key~4 causes
  an invalid split. 
  The format of the page header is key range, life span; 
  and the format of the entries is (key, life span, data).}
  \label{fig:mvbt-invalid-split}
\end{center}
\end{figure}

% Version-range queries in the MVBT
Although we have concentrated on queries with a fixed version 
(that is, \qtype{$x$/$-$/point} queries, see \secref{sec:mv-data:queries}),
the MVBT can also be modified so that version-range queries
(\qtype{$x$/$-$/range} queries) can be performed efficiently on the index
structure.
The modifications are explained by van~den~Bercken and
Seeger~\cite{bercken:1996:multiversion}.
In practice, page identifiers of temporal predecessor pages are added
to each page. 
A page~$p'$ is a said to be the \emph{temporal predecessor} of page~$p$ if
the live entries of~$p'$ were copied to~$p$ during a
version-split operation; that is, if the key ranges of $p$
and $p'$ overlap and the deletion time of~$p'$ is the creation time of $p$.
 
When the page identifiers of temporal predecessor are maintained,   
queries that target ranges of versions can be processed as follows.
First, the pages that cover the last version of the queried version range 
are located. 
The preceding versions on the range can then be located by using the
page identifiers of temporal predecessors.
It is sufficient to allocate space for at most two page identifiers in
each page, because each page of the MVBT index can have at most two temporal
predecessors. 
This can be verified by enumerating all the possible
structure-modification operations.
For more details, refer to the article by
van~den~Bercken and Seeger~\cite{bercken:1996:multiversion}.

% Conclusion
In conclusion, the MVBT structure is an optimal
multiversion index structure, if the root page of the searched version is
assumed to be known (and in practice it can be located with negligible cost),
but it has three shortcomings: the updates created by multi-action
transactions cannot be assigned the same version number, only a single
updating transaction can operate on the index at a time, and the transaction
cannot be rolled back.
In the next chapter, we present a redesign of the MVBT algorithms that
allows multiple updates to be performed within the same transaction, so that
each update is assigned the same version, and the transaction can be rolled
back.



%% Multiversion Access Structure
%%---------------------------------------------------------------------
\section{Multiversion Access Structure}
\label{sec:tsbmvbt:mvas}

Varman and Verma have also described a multiversion index structure that is
optimal according to their definition.
Their structure is called the \emph{multiversion access structure}
(MVAS~\cite{varman:1997:multiversion}\label{def:mvas}), and its structure
is similar to the structure of the MVBT\@.
The optimality definition of Varman and Verma is less restrictive than our
definition (\defref{def:optimal-mv}), and therefore we cannot consider the MVAS
to be optimal.
To distinguish the optimality concepts, let us define the optimality of the
MVAS separately.
According to Varman and Verma, an optimal multiversion
structure has a cost of \Oh{\log_\capacity m} MVAS pages to locate a queried
key~$k$ in a given version~$v$, where $m$ is the total number of updates in the
database history (or, equivalently, the total number of unique data item
entries in the index).
We will call this \emph{$m$-optimality}; that is, we say that the MVAS is
\emph{$m$-optimal}.
This is in contrast to our required cost of \Oh{\log_\capacity \entries{v}}
index structure pages, where \entries{v} denotes the number of data items alive
at version~$v$.

With this altered definition, the MVAS is $m$-optimal for the query types we
require of multiversion indexes (that is, queries of the
type \qtype{$x$/$-$/point}; see Sections~\ref{sec:mv-data:queries}
and~\ref{sec:mv-data:mv-data}), and also for single-key and snapshot
version-range queries (i.e., \qtype{$x$/$-$/range} queries, where $x$ is either
\qtype{point} or~$*$).
Version-range queries for key-ranges (e.g., queries of the type ``retrieve
all data items with keys in the range $[k_1, k_2)$ that were alive between
the versions $v_1$ and $v_2$'') are however not $m$-optimal.

The structure of the MVAS is close to that of the MVBT, with the
following notable exceptions: 
(1)~the key-range of a page can be altered for later versions, 
(2)~there is no \rootstar\ structure, 
(3)~leaf pages are linked together in page creation order, and
(4)~the index incorporates an access list structure to facilitate
version-range queries. 
These differences will be discussed in more detail below.
In other respects, the MVAS shares the shortcomings of the MVBT: the
version of the index structure must change after each update, and the
structure cannot by updated concurrently by multiple transactions.

% Key-ranges can be altered and pages reused
As described in the previous section, when a structure-modification operation
is triggered in the MVBT on page~$p$, all the involved pages are 
version-split before the entries are distributed into new pages.
If necessary, a sibling page~$p'$ is located, version-split, and the entries
are redistributed between the two new pages.
The sibling page~$p'$ might still have usable space left, however.
Instead of always creating a copy of the sibling page~$p'$, the MVAS reuses
the physical page~$p'$, and simply inserts a new router to the parent.
The MVAS is thus a multigraph with possibly more than one edge between its
nodes.
The searches for previous versions use the historical router in the index
page, and the searches for newer versions use the new router.
This means that the key ranges of pages can change, and the regions of
key-version space that the MVAS pages cover are not rectangles.
While this method does use the space more efficiently, the asymptotic space
cost remains unchanged~\cite{varman:1997:multiversion}, and the index
structure is a bit more complicated.
Nevertheless, a space-optimization scheme such as this may give some
practical benefits, and it could be applied to the MVBT also, if the
structural invariants were updated.

% No rootstar, traversal costs logarithmic in the number of updates
The MVAS does not have a separate \rootstar\ structure to track the roots of
different versions, but rather uses a single root page like the \TSBtree.
The search tree height is thus the same for each version, and the
root-to-leaf path length cannot become shorter even if entries are deleted
from the current version.
The MVAS does track the single root for the current version separately, so
that updates and queries that target the current version have the
asymptotically lower cost of \OhT{\log_\capacity \entries{v}} pages.

% Leaf pages linked together
To facilitate snapshot queries of different versions (i.e., queries that
fetch all the entries that are alive at a given version), the MVAS has sibling
links between leaf pages.
The pages are linked to each other in the order of their creation time, so
that a leaf page~$p$ has a link to the leaf page~$p'$ if $p'$~is the
next leaf page that is created after $p$~was created.
Varman and Verma do not explicitly define how these links are maintained, but
we assume that the page identifier of the latest leaf page that has been
created is maintained, and used to locate the page~$p$ that needs to be
updated when a new page~$p'$ is created.
This adds some complexity to concurrency control and deadlock avoidance when
a latching protocol is used to protect the integrity of database pages.

The leaf-page links are used to efficiently locate all entries that are alive
between any two specified versions~$v_1$ and $v_2$ such that $v_1 < v_2$.
The snapshot version-range query (\qtype{$*$/$-$/range}) is composed of two
subqueries: first, the entire set of entries alive at version $v_1$ is
queried.
This query is an $m$-optimal range query, similar to the key-range query
in the MVBT index. 
After that, the leaf-page sibling links are followed, starting from the
most recently created leaf page that was encountered during the first
subquery.
Because the leaf pages are linked in increasing creation order, all entries
that have been created between versions $v_1$ and $v_2$ must be located in
the leaf pages that were created before $v_2$; or in the pages that were
alive at version $v_1$, and thus were processed during the first subquery.
This method is thus an efficient way of locating all the entries that were
alive between the two given versions.

% Access lists
Finally, the MVAS contains a separate index structure called \emph{access
list} that is used to efficiently locate all the data items with key~$k$
that were alive between two versions~$v_1$ and~$v_2$.
The access list is a separate index structure that resembles the versioned
\Btree\ of \secref{sec:mv-index:btree}; that is, the entries are ordered
first by the keys, and then by their versions, in reverse version
order.
The different versions of each key are thus clustered close to each other. 
Because there is now a total ordering between the entries in the access list,
the leaf pages of the structure can be linked together, as is often done in
\Btree{}s.
The problem of finding all the data items with key~$k$ that were alive
between two versions now reduces to finding the correct version of the entry
with key~$k$ from the access list and then traversing the sibling links to
locate all the versions between the queried range.
If the correct starting point can be found, this operation is $m$-optimal.
For this purpose, the entries of the MVAS and the entries of the access list
are linked together with two-way links.

When querying for the history of a key~$k$ between versions~$v_1$ and~$v_2$,
the MVAS index is used to locate the entry that is alive at version~$v_2$.
The search then follows the link in this entry to the access list, and
traverses the access list by the sibling links to locate all the versions of
the entry between~$v_1$ and~$v_2$.
However, an entry that was alive during or after version~$v_1$ might have been
deleted before version~$v_2$, so that the MVAS search will not find any entry
and thus does not find a pointer to the access list. 
This entry is still alive at a version $v : v_1 \leq v < v_2$, so it should
be included in the result set of the query operation.
In this case, the \Btree\ index of the access list is used to locate the
first entry in the queried history.
The two-way links between the MVAS and the access list are needed because
leaf pages in both structures may be split and the entries may thus be moved
around. 
The links then need to be followed to update the references in the other
structure.
There is also an extra level of indirection in the MVAS, because an entry may
be copied to multiple pages due to a version-split operation that creates
copies of entries with life spans that cross the split boundary.
Only the first entry in the MVAS contains a link to the access list, and the
successive copies of the entry contain links to the first entry.
When the address of an access list entry changes, it is then sufficient only
to change the address in the first MVAS entry.

There is, however, yet another complication due to the access list.
When entries are added to the access list, some nearby pointer from the MVAS
is followed to locate the correct page of the access list to insert the
corresponding entry there.
This is done so that it is not required to traverse the \Btree\ index to
locate the correct page.
Recall that the MVAS separately tracks the root page of the current version,
so that update actions have the optimal cost of \OhT{\log_\capacity
\entries{v}} pages.
The access list size is, however, dependent on the total number of updates
performed in the history, $m$, and traversing the \Btree\ index of the access
list thus requires \OhT{\log_\capacity m} pages. 
This additional cost would increase the cost of the update operations above
the asymptotically optimal cost.
The pages of the access list therefore have two-way links between parent
pages and child pages, so that leaf pages can be split without having first
searched the path from the root to the leaf page.
This means that whenever an index page of the access list is split, each child
page whose pointer is moved to the new page needs to be updated.
In addition to being costly because there are \OhT{\capacity} pages to update,
the approach is also problematic for concurrency control, because the pages
all need to be latched at the same time.
Varman and Verma show that the amortized complexity of this operation is
constant, because it can only occur after enough other operations have been
performed~\cite{varman:1997:multiversion}, and the approach thus retains the
amortized asymptotic optimality of the update actions.
Like the \TSBtree\ and MVBT, the space complexity of the MVAS (including the
access list structure) is \Oh{n/\capacity} database
pages, where $n$ is the number of updates in the database
history~\cite{varman:1997:multiversion}.

As a conclusion, the MVAS is a structure that is closely related to the
MVBT index. 
Similarly to the MVBT, transaction rollback and recovery cannot be optimally
performed on the MVAS\@.
There are several differences in the structures, but all the features of the
MVAS could be implemented on the MVBT as well.
The access list structure, while preserving the asymptotic behaviour of
the algorithms, incurs a constant overhead to the update operations and
an occasional high cost when an index page is split and half its
child pages need to be updated.
In addition to taking much time, the index-page-split operation has to latch
many pages at the same time, thus reducing the concurrency of the structure.
