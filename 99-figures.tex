\chapter{First images}
\label{chapter:dummy-layout-text}

% Explanation of the details of transactions on multiversion data.
% The transaction model.
% No implementation (index structure) details.

The theory of general transaction processing in traditional 
databases is well-defined and mature, and the basic principles are
well presented in many
textbooks~\cite{bernstein:1987:cc-n-r,gray:1993:transactionprocessing,papadimitriou:1986:cc-theory}.
In this chapter, we concentrate on the theory of multiversion
transaction processing, and highlight the differences to the classical
transaction theory.
The traditional read-write model assumes that transactions are sequences
of reads and writes on data items, without distinguishing item deletions and
insertions from
updates~\cite{bernstein:1987:cc-n-r,papadimitriou:1986:cc-theory}.
The theory of transaction processing in this dissertation is based on
the recoverable transaction model presented by Sippu and
Soisalon-Soininen~\cite{sippu:2001:theory}, which in turn is based on the
model proposed by C.~Mohan~\cite{mohan:1990:aries-kvl,mohan:1992:aries-im}.
In this model, data-item insertions and deletions are made explicit,
and structure-modification operations are included in the model.
We assume the partially persistent transaction-time model, as
described in the introduction.

\begin{figure}[htb]
\centering
\subfigure[Before]{%
\begin{tikzpicture}[x=\txfigx,y=\txfigy,every node/.style={font=\footnotesize}]
  \coordinate (o) at (0,0);
  \coordinate (oo) at (0,0.2);
  \coordinate (ok) at ($(oo) + (0,5.6)$);

  \coordinate[label=right:{$(1,\alpha)$}] (a) at ($(o) + (0,1)$);
  \coordinate[label=right:{$(2,\beta)$}] (b) at ($(a) + (0,1)$);
  \coordinate[label=right:{$(3,\gamma)$}] (c) at ($(b) + (0,1)$);
  \coordinate[label=right:{$(4,\delta)$}] (d) at ($(c) + (0,1)$);
  
  \foreach \point in {a,b,c,d}
    \fill[black] (\point) circle (2pt);

  \draw[->] (oo) -- node[lbl,above] {Keys} (ok);
\end{tikzpicture}
%\:
%\raisebox{2.5\txfigy}{$\Rightarrow$}
%\:
}
\subfigure[After]{%
\begin{tikzpicture}[x=\txfigx,y=\txfigy,every node/.style={font=\footnotesize}]
  \coordinate (o) at (0,0);
  \coordinate (oo) at (0,0.2);
  \coordinate (ok) at ($(oo) + (0,5.6)$);

  \coordinate[label=right:{$(1,\alpha')$}] (a) at ($(o) + (0,1)$);
  \coordinate[label=right:{$(3,\gamma)$}] (c) at ($(a) + (0,2)$);
  \coordinate[label=right:{$(4,\delta)$}] (d) at ($(c) + (0,1)$);
  \coordinate[label=right:{$(5,\epsilon)$}] (e) at ($(d) + (0,1)$);

  \foreach \point in {a,c,d,e}
    \fill[black] (\point) circle (2pt);

  \draw[->] (oo) -- node[lbl,above] {Keys} (ok);
\end{tikzpicture}%
}

\figcaption{Transaction in a snapshot database}{The transaction has
executed the schedule $b w[a,\alpha,\alpha'] d[b] w[e,\epsilon] c$.}
\label{fig:tx-snapshot}
\end{figure}

\begin{figure}[htb]
\centering
\subfigure[Before]{%
\begin{tikzpicture}[x=\txfigx,y=\txfigy,every node/.style={font=\footnotesize}]
  \coordinate (o) at (0,0);
  \coordinate (oo) at (0,0.2);
  \coordinate (ox) at ($(oo) + (0,5.6)$);
  \coordinate (ot) at ($(oo) + (5,0)$);

  \coordinate (a) at ($(o) + (0,1)$);
  \coordinate (b) at ($(o) + (1,2)$);
  \coordinate (c) at ($(b) + (-1,1)$);
  \coordinate (d) at ($(c) + (2,1)$);
  
  \draw[|-|] (a) -- node[above] {$(1,\alpha)$} ($(a) + (2,0)$);
  \draw[|-|] (b) -- node[above] {$(2,\beta)$} ($(b) + (3,0)$);
  \draw[|-|] (c) -- node[above] {$(3,\gamma)$} ($(c) + (1,0)$);
  \draw[|-|] (d) -- node[above] {$(4,\delta)$} ($(d) + (2,0)$);

  \draw[->] (oo) -- node[lbl,above] {Keys} (ox);
  \draw[->] (oo) -- node[lbl,below] {Time} (ot);
\end{tikzpicture}
%\:
%\raisebox{\txfigy}{$\Rightarrow$}
%\:
}
\subfigure[After]{%
\begin{tikzpicture}[x=\txfigx,y=\txfigy,every node/.style={font=\footnotesize}]
  \coordinate (o) at (0,0);
  \coordinate (oo) at (0,0.2);
  \coordinate (ox) at ($(oo) + (0,5.6)$);
  \coordinate (ot) at ($(oo) + (5,0)$);

  \coordinate (a) at ($(o) + (1,1)$);
  \coordinate (c) at ($(o) + (0,3)$);
  \coordinate (d) at ($(c) + (2,1)$);
  \coordinate (e) at ($(d) + (-1,1)$);

  \draw[|-|] (a) -- node[above] {$(1,\alpha')$} ($(a) + (3,0)$);
  \draw[|-|] (c) -- node[above] {$(3,\gamma)$} ($(c) + (1,0)$);
  \draw[|-|] (d) -- node[above] {$(4,\delta)$} ($(d) + (2,0)$);
  \draw[|-|] (e) -- node[above] {$(5,\epsilon)$} ($(e) + (2,0)$);

  \draw[->] (oo) -- node[lbl,above] {Keys} (ox);
  \draw[->] (oo) -- node[lbl,below] {Time} (ot);

\end{tikzpicture}%
}
\figcaption{Transaction in a valid-time database}{The transaction has
executed the schedule $b w[a,\alpha,\alpha'] d[b] w[e,\epsilon] c$.}
\label{fig:tx-valid}
\end{figure}

\begin{figure}[htb]
\begin{center}
\subfigure[A single-version \Btree\ index]{
\begin{tikzpicture}[x=4em,y=3.7em,label distance=.3em]
  \tiny

  \node[rnode,draw] (p1) at (0,2)
    {\centerrows{4.2em}{$p_1$\\$[-\infty,\infty)$}};
  \node[rnode,right=0 of p1] (p1text) 
    {\raggedrows{10em}{= Page identifier\\= Key range}};
  
  \node[rnode,draw] (p2) at (-1.5,1) 
    {\centerrows{4.2em}{$p_2$\\$[-\infty,0)$}};
  \node[rnode,draw] (p3) at (0,1) 
    {\centerrows{4.2em}{$p_3$\\$[0,100)$}};
  \node[rnode,draw] (p4) at (1.5,1) 
    {\centerrows{4.2em}{$p_4$\\$[100,\infty)$}};

  \node[rnode,draw] (p5) at (-1,0) 
    {\centerrows{4.2em}{$p_5$\\$[0,25)$}};
  \node[rnode,draw] (p6) at (0,0) 
    {\centerrows{4.2em}{$p_6$\\$[25,75)$}};
  \node[rnode,draw] (p7) at (1,0) 
    {\centerrows{4.2em}{$p_7$\\$[75,100)$}};

  % Connectors
  \foreach \x in {p2,p3,p4}
    \draw[-] (p1.south) -- (\x.north);
  \foreach \x in {p5,p6,p7}
    \draw[-] (p3.south) -- (\x.north);

  % Not shown pages
  \newdimen\btloy
  \newdimen\btlox

  \pgfextracty{\btloy}{\pgfpointanchor{p5}{north}}
  \pgfextractx{\btlox}{\pgfpointanchor{p2}{south}}
  \draw[-] (p2.south) -- ($(\btlox,\btloy) + (-.2,0)$);
  \draw[-] (p2.south) -- ($(\btlox,\btloy) + (-.5,0)$);
  \draw[-] (p2.south) -- ($(\btlox,\btloy) + (-.8,0)$);
  \node at ($(\btlox,\btloy) + (-.5,-3ex)$) {\ldots};

  \pgfextracty{\btloy}{\pgfpointanchor{p5}{north}}
  \pgfextractx{\btlox}{\pgfpointanchor{p4}{south}}
  \draw[-] (p4.south) -- ($(\btlox,\btloy) + (.2,0)$);
  \draw[-] (p4.south) -- ($(\btlox,\btloy) + (.5,0)$);
  \draw[-] (p4.south) -- ($(\btlox,\btloy) + (.8,0)$);
  \node at ($(\btlox,\btloy) + (.5,-3ex)$) {\ldots};

  % Key ranges and key-space cover
  \coordinate[right=1 of p4] (l2p1);
  \coordinate[label=below:{$0$}] (l2p2) at ($(l2p1) + (0.25,0)$);
  \coordinate[label=below:{$100$}] (l2p3) at ($(l2p2) + (1,0)$);
  \coordinate (l2p4) at ($(l2p3) + (0.25,0)$);

  \draw[-|] (l2p1) -- node[above] {$p_2$} (l2p2);
  \draw[|-|] (l2p2) -- node[above] {$p_3$} (l2p3);
  \draw[|-] (l2p3) -- node[above] {$p_4$} (l2p4);

  \coordinate[above=1 of l2p1] (l3p1);
  \coordinate (l3p2) at ($(l3p1) + (1.5,0)$);
  \draw[-] (l3p1) -- node[above] {$p_1$} node[below] {Keys} (l3p2);

  \coordinate[below=1 of l2p1] (l1p1);
  \coordinate[label=below:{$0$}] (l1p2) at ($(l1p1) + (0.25,0)$);
  \coordinate[label=below:{$25$}] (l1p3) at ($(l1p2) + (0.25,0)$);
  \coordinate[label=below:{$75$}] (l1p4) at ($(l1p3) + (0.5,0)$);
  \coordinate[label=below:{$100$}] (l1p5) at ($(l1p4) + (0.25,0)$);
  \coordinate (l1p6) at ($(l1p5) + (0.25,0)$);

  \draw[-|] (l1p1) -- (l1p2);
  \draw[|-|] (l1p2) -- node[above] {$p_5$} (l1p3);
  \draw[|-|] (l1p3) -- node[above] {$p_6$} (l1p4);
  \draw[|-|] (l1p4) -- node[above] {$p_7$} (l1p5);
  \draw[|-] (l1p5) -- (l1p6);
  
\end{tikzpicture}
}
\subfigure[A common design in multiversion indexes]{
\label{fig:sv-mv-index-comparison:mv}
\begin{tikzpicture}[x=4em,y=3.7em,label distance=0pt]
  \tiny

  \node[rnode,draw] (p1) at (0,2)
    {\centerrows{4.2em}{$p_1$\\$[-\infty,\infty)$\\$[-\infty,\infty)$}};
  \node[rnode,right=0 of p1] (p1text) 
    {\raggedrows{10em}{= Page identifier\\= Key range\\= Life span}};
  
  \node[rnode,draw] (p2) at (-1.5,1) 
    {\centerrows{4.2em}{$p_2$\\$[-\infty,\infty)$\\$[-\infty,10)$}};
  \node[rnode,draw] (p3) at (0,1) 
    {\centerrows{4.2em}{$p_3$\\$[-\infty,100)$\\$[10,\infty)$}};
  \node[rnode,draw] (p4) at (1.5,1) 
    {\centerrows{4.2em}{$p_4$\\$[100,\infty)$\\$[10,\infty)$}};

  \node[rnode,draw] (p5) at (-1,0) 
    {\centerrows{4.2em}{$p_5$\\$[-\infty,25)$\\$[-\infty,\infty)$}};
  \node[rnode,draw] (p6) at (0,0) 
    {\centerrows{4.2em}{$p_6$\\$[25,75)$\\$[10,\infty)$}};
  \node[rnode,draw] (p7) at (1,0) 
    {\centerrows{4.2em}{$p_7$\\$[75,100)$\\$[10,\infty)$}};

  % Connectors
  \foreach \x in {p2,p3,p4}
    \draw[-] (p1.south) -- (\x.north);
  \foreach \x in {p5,p6,p7}
    \draw[-] (p3.south) -- (\x.north);
  
  \draw (p2.south) -- (p5.north);

  % Not shown pages
  \newdimen\tmpx
  \newdimen\tmpy

  \pgfextracty{\tmpy}{\pgfpointanchor{p5}{north}}
  \pgfextractx{\tmpx}{\pgfpointanchor{p2}{south}}
  \draw[-] (p2.south) -- ($(\tmpx,\tmpy) + (-.2,0)$);
  \draw[-] (p2.south) -- ($(\tmpx,\tmpy) + (-.5,0)$);
  \draw[-] (p2.south) -- ($(\tmpx,\tmpy) + (-.8,0)$);
  \node at ($(\tmpx,\tmpy) + (-.5,-3ex)$) {\ldots};

  \pgfextracty{\tmpy}{\pgfpointanchor{p5}{north}}
  \pgfextractx{\tmpx}{\pgfpointanchor{p4}{south}}
  \draw[-] (p4.south) -- ($(\tmpx,\tmpy) + (.2,0)$);
  \draw[-] (p4.south) -- ($(\tmpx,\tmpy) + (.5,0)$);
  \draw[-] (p4.south) -- ($(\tmpx,\tmpy) + (.8,0)$);
  \node at ($(\tmpx,\tmpy) + (.5,-3ex)$) {\ldots};

  \pgfextracty{\tmpy}{\pgfpointanchor{p7}{south}}
  \pgfextractx{\tmpx}{\pgfpointanchor{p4}{east}}

  \path[draw,dashed,rounded corners] ($(p1.north west) + (-2ex,2ex)$) -- 
    ($(p1.north east) + (2ex,2ex)$) --
    ($(p3.north east) + (2ex,2ex)$) --
    ($(p4.north east) + (2ex,2ex)$) --
    ($(\tmpx,\tmpy) + (2ex,-2ex)$) -- 
      node[below,near start] {Search tree of version \num{10}} 
    ($(p5.south west) + (-2ex,-2ex)$) --
    ($(p5.north west) + (-2ex,2ex)$) --
    ($(p6.north west) + (-2ex,2ex)$) -- cycle;

  % Key ranges and key-space cover
  % Level 2
  \coordinate[right=1 of p4,label=left:{$10$}] (l2p1);
  \coordinate (l2p2) at ($(l2p1) + (1.25,0)$);
  \coordinate (l2p3) at ($(l2p2) + (0.25,0)$);
  \coordinate[label=above:{$100$}] (l2p2u) at ($(l2p2) + (0,.25)$);

  \draw (l2p1) rectangle node {$p_3$} 
    (l2p2u) rectangle node {$p_4$} 
    (l2p3) rectangle node {$p_2$}
    ($(l2p1) + (0,-.25)$); 

  % Level 3 (root page)
  \coordinate[above=1 of l2p1] (l3p1);
  \coordinate (l3p2) at ($(l3p1) + (1.5,0)$);
  \draw ($(l3p1) + (0,-0.3)$) rectangle
     node {$p_1$} 
     node[below=1.5em] {\tiny Keys}
     ($(l3p2) + (0,0.3)$);
  \node[rotate=90,left=0pt of l3p1,anchor=south] {\tiny Versions};

  % Level 1 (leaf pages)
  \coordinate[below=1 of l2p1,label=left:{$10$}] (l1p1);
  \coordinate (l1p2) at ($(l1p1) + (0.34,0)$);
  \coordinate[label=above:{$25$}] (l1p2u) at ($(l1p2) + (0,.25)$); 
  \coordinate (l1p3) at ($(l1p2) + (0.50,0)$);
  \coordinate[label=above:{$75$}] (l1p3u) at ($(l1p3) + (0,.25)$); 
  \coordinate (l1p4) at ($(l1p3) + (0.41,0)$);
  \coordinate[label=above:{$100$}] (l1p4u) at ($(l1p4) + (0,.25)$); 
  \coordinate (l1p5) at ($(l1p4) + (0.25,0)$);

  \draw ($(l1p1) + (0,.25)$) rectangle ($(l1p5) + (0,-.25)$);
  \draw ($(l1p1) + (0,-.25)$) rectangle node {$p_5$} 
    (l1p2u) rectangle node {$p_6$} 
    (l1p3) rectangle node {$p_7$} 
    (l1p4u);

%   \draw[-|] (l1p1) -- (l1p2);
%   \draw[|-|] (l1p2) -- node[above] {$p_5$} (l1p3);
%   \draw[|-|] (l1p3) -- node[above] {$p_6$} (l1p4);
%   \draw[|-|] (l1p4) -- node[above] {$p_7$} (l1p5);
%   \draw[|-] (l1p5) -- (l1p6);
%   
\end{tikzpicture}
}
\figcaption{Comparison of single-version and multiversion indexes}{}
\label{fig:sv-mv-index-comparison}
\end{center}
\end{figure}

\begin{figure}[htb]
\begin{center}
  \begin{tikzpicture}[x=3.5em,y=3.7em,% 
vlabel/.style={font=\scriptsize,left=0.2em,near start}]

%\fill[blue!50] (-1,-1) rectangle (5,3);

\node[page,dead] (p1) at (0,0) {};
\node[page] (p2) at (1,0) {};
\node[page,dead] (p3) at (2,0) {};
\node[page] (p4) at (3,0) {};
\node[page] (p5) at (4,0) {};
\node[page,dead] (p6) at (5,0) {};

\node[page,dead] (p7) at (0.8,1) {};
\node[page,dead] (p8) at (2.2,1) {};

\node[page] (p9) at (3.5,1) {};

\node[page,dead] (p10) at (1.5,2) {};

\node[db] (roots) at (4.5, 2.5) {\footnotesize$\rootstarp$};

\draw (p7.south) -- (p1.north);
\draw (p7.south) -- (p2.north);
\draw (p8.south) -- (p3.north);
\draw (p8.south) -- (p4.north);
\draw (p10.south) -- (p7.north);
\draw (p10.south) -- (p8.north);

\draw (p9.south) -- (p2.north);
\draw (p9.south) -- (p4.north);
\draw (p9.south) -- (p5.north);

 
\draw[->,densely dotted] (roots) to[bend left=10] (p6.north);
\draw[->] (roots) to[bend right=10] 
           node[vlabel] {\parbox{4em}{\center Selected version}}
           (p9.north); 
\draw[->,densely dotted] (roots) to[bend right=10] (p10.east);

\end{tikzpicture}

  \figcaption{Search trees of different heights}
  {Search trees of different versions in a multiversion index can
  have different heights.}
  \label{fig:mv-height-decrease}
\end{center}
\end{figure}

\begin{figure}[htb]
\begin{center}
\begin{tikzpicture}[x=3.5em,y=3.7em,every node/.style={font=\small}]

%\fill[blue!50] (-1,-1) rectangle (5,3);

\node[page] (p1) at (0,0) {$p_1$};
\node[page] (p2) at (1,0) {$p_2$};
\node[page] (p3) at (2,0) {$p_3$};
\node[page] (p4) at (3,0) {$p_4$};
\node[page] (p5) at (4,0) {$p_5$};
\node[page] (p6) at (0.5,1) {$p_6$};
\node[page] (p7) at (1.8,1) {$p_7$};
\node[page] (p8) at (3.2,1) {$p_8$};
\node[page] (p9) at (2.5,2) {$p_9$};

\node[db] (roots) at (0, 2) {$\rootstarp$};

\draw (p6.south) -- (p1.north);
\draw (p6.south) -- (p2.north);

\draw (p7.south) -- (p2.north);
\draw (p7.south) -- (p3.north);

\draw (p8.south) -- (p4.north);
\draw (p8.south) -- (p5.north);

\draw (p9.south) -- (p7.north);
\draw (p9.south) -- (p8.north);

\draw[->,densely dotted] (roots) to[bend left=10] (p9.west);
\draw[->,densely dotted] (roots) to[bend left=10] (p6.north);

\node[anchor=west] at (4.5,0.2) {\footnotesize Leaf pages};
\draw[dashed] (-1,0.5) -- (7,0.5);
\node[anchor=west] at (4.5,0.8) {\footnotesize Index pages};

% \draw[densely dotted] ($(p6) + (0,0.5)$) to[controls=+(180:1) and +(0:0.32)] 
%   ($(p1) + (-0.2,0.5)$) to[controls=+(180:0.32) and +(90:0.2)]
%   ($(p1) + (-0.5,-0.2)$) to[controls=+(270:0.2) and +(180:1)]
%   ($(p1) + (0.5,-0.5)$) to[controls=+(0:1) and +(270:0.1)]
%   ($(p2) + (0.5,-0.2)$) to[controls=+(90:0.1) and +(0:0.32)]
%   ($(p2) + (0.2,0.5)$) to[controls=+(180:0.32) and +(0:1)]
%   ($(p6) + (0,0.5)$);
% 
% \node[pin={[pin edge=solid,pin distance=1em]110:{\small $S_v$}}] 
%   at ($(p1) + (-0.05,0.9)$) {};

\end{tikzpicture}

\figcaption{Structure of a balanced multiversion index}{}
\label{fig:mv-index-example}
\end{center}
\end{figure}

\chapter{Second images}

\begin{figure}[htb]
\begin{center}
\subfigure[Level 1]{\begin{tikzpicture}[x=2em,y=2em]
  \small
  \coordinate (o) at (0,0);
  \coordinate (ok) at ($(o) + (0,3.7)$);
  \coordinate (ov) at ($(o) + (6.7,0)$);
  \coordinate[label=below:{$v_0$}] (v0) at (o); 
  \coordinate[label=below:{$v_1$}] (v1) at ($(o) + (1,0)$); 
  \coordinate[label=below:{$v_2$}] (v2) at ($(o) + (2,0)$);
  \coordinate[label=below:{$v_3$}] (v3) at ($(o) + (3,0)$);
  \coordinate[label=below:{$v_4$}] (v4) at ($(o) + (4,0)$);
  \coordinate[label=below:{$v_5$}] (v5) at ($(o) + (5,0)$);
  \coordinate[label=below:{$v_6$}] (v6) at ($(o) + (6,0)$);
 
  \draw[opendash] decorate[opendeco] { (0,3) -- (1,3) };
  \node at (0.5, 1.5) {$p_1$};
  %\draw (o) rectangle node {$p_1$} (1,3);

  \draw (1,0) rectangle node {$p_2$} (2,2);

  %\draw (1,2) rectangle node {$p_3$} (4,3);
  \draw (1,2) -- (1,3);
  \draw[opendash] decorate[opendeco] { (1,3) -- (4,3) };
  \node at (2.5, 2.5) {$p_3$};

  \draw (2,1) rectangle node {$p_4$} (4,2);
  \draw (2,0) rectangle node {$p_5$} (5,1);
  
  %\draw (4,1) rectangle node {$p_6$} (5,3);
  \draw (4,2) -- (4,3);
  \draw[opendash] decorate[opendeco] { (4,3) -- (5,3) };
  \node at (4.5, 2) {$p_6$};
  \draw (5,1) -- (5,3);
  
  \draw (5,0) -- (6,0);
  \draw (5,2) -- (6,2);
  \draw[opendash] decorate[opendeco] { (5,3) -- (6,3) };

  \node at (5.5,2.5) {$p_7$};
  \node at (5.5,1) {$p_8$};

  \draw[opendash] decorate[opendeco] { (6,0) -- (6,2) };
  \draw[opendash] decorate[opendeco] { (6,2) -- (6,3) };

  \draw[->] (o) -- node[below=1em,lbl] {Versions} (ov);
  \draw[->] (o) -- node[above,sloped,lbl] {Keys} (ok);
\end{tikzpicture}

\label{fig:mvbt-space-partition:level-1}}
\subfigure[Level 2]{\begin{tikzpicture}[x=2em,y=2em]
  \small
  \coordinate (o) at (0,0);
  \coordinate (ok) at ($(o) + (0,3.7)$);
  \coordinate (ov) at ($(o) + (6.7,0)$);
  \coordinate[label=below:{$v_0$}] (v0) at (o); 
  \coordinate[label=below:{$v_1$}] (v1) at ($(o) + (1,0)$); 
  \coordinate[label=below:{$v_2$}] (v2) at ($(o) + (2,0)$);
  \coordinate[label=below:{$v_3$}] (v3) at ($(o) + (3,0)$);
  \coordinate[label=below:{$v_4$}] (v4) at ($(o) + (4,0)$);
  \coordinate[label=below:{$v_5$}] (v5) at ($(o) + (5,0)$);
  \coordinate[label=below:{$v_6$}] (v6) at ($(o) + (6,0)$);

  %\draw (1,0) rectangle node {$p_9$} (5,3);
  \draw (1,0) -- (1,3);
  \draw (4,0) -- (4,3);
  \draw[opendash] decorate [opendeco] { (1,3) -- (4,3) };
 
  \draw[opendash] decorate [opendeco] { (4,3) -- (6,3) };

  \draw[opendash] decorate [opendeco] { (6,0) -- (6,3) };

  \node at (5,1.5) {$p_{10}$};
  \node at (2.5,1.5) {$p_{9}$};

  % Router in page p9 to page p5 at the lower level
  \draw[dashed] (2.1,0.1) rectangle node {$\ptr p_5$} (3.9,0.9);
  % Router in page p10 to page p5 at the lower level
  \draw[dashed] (4.1,0.1) rectangle node {$\ptr p_5$} (4.9,0.9);

  \draw[->] (o) -- node[below=1em,lbl] {Versions} (ov);
  \draw[->] (o) -- node[above,sloped,lbl] {Keys} (ok);
\end{tikzpicture}

\label{fig:mvbt-space-partition:level-2}}
\figcaption{Partitioning of the key-version space in the MVBT}%
{In the image, $\comver = v_6$, which is indicated by the open ends of the
version ranges of pages $p_7$, $p_8$, and $p_{10}$.
The level \num{2} image shows routers to page $p_5$ in pages $p_9$ and
$p_{10}$.}
\label{fig:mvbt-space-partition}
\end{center}
\end{figure}

\begin{figure}[htb]
\begin{center}
  \subfigure[Insert key~1]{
\:\:
\begin{tikzpicture}
\node[treepage] (p1) {%
  \treepagecontents{4.5}{$[\!-\infty,\!\infty\!),\![1,\!\infty\!)$}%
  {$(1, [1,\!\infty\!), w_1\!)$}};
\treepageheader{p1}{$p_1$}
\end{tikzpicture}
\:\:
}
\subfigure[Insert keys~2--3]{
\:\:
\begin{tikzpicture}
\node[treepage] (p1) {%
  \treepagecontents{4.5}{$[\!-\infty,\!\infty\!),\![1,\!\infty\!)$}%
  {$(1, [1,\!\infty\!), w_1\!)$\\
   $(2, [1,\!\infty\!), w_2\!)$\\
   $(3, [1,\!\infty\!), w_3\!)$}};
\treepageheader{p1}{$p_1$}
\end{tikzpicture}
\:\:
}\\
\subfigure[Insert key~4]{
\begin{tikzpicture}
\node[treepage] (p1) {%
  \treepagecontents{4.5}{$[\!-\infty,\!\infty\!),$\sout{$[1,\!1)$}}%
  {$(1,\,$\sout{$[1,\!1)$}$, w_1\!)$\\
   $(2,\,$\sout{$[1,\!1)$}$, w_2\!)$\\
   $(3,\,$\sout{$[1,\!1)$}$, w_3\!)$}};
\treepageheader{p1}{$p_1$}

\node[treepage,right=1em of p1] (p2) {%
  \treepagecontents{4.5}{$[\!-\infty,\!3),\![1,\!\infty\!)$}%
  {$(1, [1,\!\infty\!), w_1\!)$\\
   $(2, [1,\!\infty\!), w_2\!)$}};
\treepageheader{p2}{$p_2$}

\node[treepage,right=1em of p2] (p3) {%
  \treepagecontents{4.5}{$[3,\!\infty\!),\![1,\!\infty\!)$}%
  {$(3, [1,\!\infty\!), w_3\!)$\\
   $(4, [1,\!\infty\!), w_4\!)$}};
\treepageheader{p3}{$p_3$}
\end{tikzpicture}
}

  \figcaption{\abbr{MVBT} problem scenario}{Insertion of key~4 causes
  an invalid split.}
  \label{fig:mvbt-invalid-split}
\end{center}
\end{figure}

\begin{figure}[htb]
\begin{center}
  \subfigure[Insert~key~1]{
\:\:
\begin{tikzpicture}
\node[treepage] (p1) {%
  \treepagecontents{4.5}{$[\!-\infty,\!\infty\!),\![1,\!\infty\!)$}%
  {$(1, [1,\!\infty\!), w_1\!)$}};
\treepageheader{p1}{$p_1$}
\end{tikzpicture}
\:\:}
\subfigure[Insert~keys~2--3]{
\:\:
\begin{tikzpicture}
\node[treepage] (p1) {%
  \treepagecontents{4.5}{$[\!-\infty,\!\infty\!),\![1,\!\infty\!)$}%
  {$(1, [1,\!\infty\!), w_1\!)$\\
   $(2, [1,\!\infty\!), w_2\!)$\\
   $(3, [1,\!\infty\!), w_3\!)$}};
\treepageheader{p1}{$p_1$}
\end{tikzpicture}
\:\:
}\\
\subfigure[Insert key~4]{
\begin{tikzpicture}
\node[treepage] (p1) {%
  \treepagecontents{4.5}{$[\!-\infty,\!3),\![1,\!\infty\!)$}%
  {$(1, [1,\!\infty\!), w_1\!)$\\
   $(2, [1,\!\infty\!), w_2\!)$}};
\treepageheader{p1}{$p_1$}

\node[treepage,right=1em of p1] (p2) {%
  \treepagecontents{4.5}{$[3,\!\infty\!),\![1,\!\infty\!)$}%
  {$(3, [1,\!\infty\!), w_3\!)$\\
   $(4, [1,\!\infty\!), w_4\!)$}};
\treepageheader{p2}{$p_2$}

\end{tikzpicture}
}

  \figcaption{Key split without version split in the TMVBT}%
  {A key-split is triggered by the insertion of key~\num{4}.}
  \label{fig:mvbt-invalid-split-solved}
\end{center}
\end{figure}

\begin{figure}[!htb]
\begin{center}
\subfigure[Active index entries]{\begin{tikzpicture}[x=2em,y=2em,every node/.style={font=\small}]
  \small
  \coordinate (o) at (0,0);
  \coordinate (ok) at ($(o) + (0,5)$);
  \coordinate (ov) at ($(o) + (3,0)$);
  \coordinate (v0) at (o); 
  \coordinate[label=below:{\actverp}] (v1) at ($(o) + (1,0)$); 

  \draw (1,0) -- (1,4.5);
%  \draw (1,1) -- (2,1);
%  \draw (1,3) -- (2,3);

  \draw[opendash] decorate [opendeco] { (1,4.5) -- (2,4.5) };

  \draw[opendash] decorate [opendeco] { (2,0) -- (2,1) };
  \draw[opendash] decorate [opendeco] { (2,1) -- (2,3) };
  \draw[opendash] decorate [opendeco] { (2,3) -- (2,4.5) };

  \draw[dashed] { (1.1,0.1) -- (2,0.1) };
  \draw[dashed] { (1.1,0.1) -- (1.1,0.95) };
  \draw[dashed] { (1.1,0.95) -- (2,0.95) };

  \draw[dashed] { (1.1,1.05) -- (2,1.05) };
  \draw[dashed] { (1.1,1.05) -- (1.1,2.95) };
  \draw[dashed] { (1.1,2.95) -- (2,2.95) };

  \draw[dashed] { (1.1,3.05) -- (2,3.05) };
  \draw[dashed] { (1.1,3.05) -- (1.1,4.5) };

  \node at (1.55,0.5) {$\ptr p_1$};
  \node at (1.55,2) {$\ptr p_2$};
  \node at (1.55,3.75) {$\ptr p_3$};

  \draw[->] (o) -- node[below=1em,lbl] {Versions} (ov);
  \draw[->] (o) -- node[above,sloped,lbl] {Keys} (ok);
\end{tikzpicture}

\label{fig:tmvbt-active-entries:index}}
\subfigure[Active leaf entries]{\begin{tikzpicture}[x=2em,y=2em,every node/.style={font=\small}]
  \small
  \coordinate (o) at (0,0);
  \coordinate (ok) at ($(o) + (0,5)$);
  \coordinate (ov) at ($(o) + (3,0)$);
  \coordinate (v0) at (o); 
  \coordinate[label=below:{\actverp}] (v1) at ($(o) + (1,0)$); 

  \draw[|->] (1,1) -- node[lbl,above] {$e_1$} (2,1);
  \draw[|->] (1,2) -- node[lbl,above] {$e_2$} (2,2);
  \draw[|->] (1,3.5) -- node[lbl,above] {$e_3$} (2,3.5);

  \draw[dashed,->] (1,0) -- (1,4.5);

  \draw[->] (o) -- node[below=1em,lbl] {Versions} (ov);
  \draw[->] (o) -- node[above,sloped,lbl] {Keys} (ok);
\end{tikzpicture}

\label{fig:tmvbt-active-entries:leaf}}
\figcaption{Active entries in the TMVBT index}%
{The index page contains three index entries with routers to pages $p_1$,
$p_2$, and $p_3$. 
The leaf page contains three entries, namely $e_1$, $e_2$, and $e_3$.}
\label{fig:tmvbt-active-entries}
\end{center}
\end{figure}

\chapter{Third images}

\begin{figure}[!hbt]
\begin{center}
  \begin{tikzpicture}[x=0.135em,y=-0.058em]
\node[treepage] (p6) at (29,5) {%
\treeindexpagecontents{6.5}{$[\!-\infty,\!\infty\!),[1,\!\infty\!)$}%
  {%%
   $([\!-\infty,\!4),[1,\!\infty\!),p_{3})$\\
   $([4,\!\infty\!),[1,\!\infty\!),p_{5})$}};
\treepageheader{p6}{$p_{6}$};
\node[treepage] (p3) at (5,146) {%
\treepagecontents{6.5}{$[\!-\infty,\!4),[1,\!\infty\!)$}%
  {%%
   $(1,[1,\!\infty\!))$\\
   $(2,[1,\!\infty\!))$\\
   $(3,[1,\!\infty\!))$}};
\treepageheader{p3}{$p_{3}$};
\node[treepage] (p5) at (53,146) {%
\treepagecontents{6.5}{$[4,\!\infty\!),[1,\!\infty\!)$}%
  {%%
   $(4,[1,\!\infty\!))$\\
   $(5,[1,\!\infty\!))$\\
   $(6,[1,\!\infty\!))$\\
   $(7,[2,\!\infty\!))$\\
   $(8,[2,\!\infty\!))$}};
\treepageheader{p5}{$p_{5}$};
\draw[parentlink] (p6.south) -- ($(p3header.north) + (0,-2)$);
\draw[parentlink] (p6.south) -- ($(p5header.north) + (0,-2)$);

\end{tikzpicture}

  \figcaption{Example of a TMVBT index after insertions}
  {
  The page header shows the page identifier followed by (key range, version
  range); 
  the format of index-page entries is (key range, life span, page
  identifier); and 
  the format of leaf-page entries is (key, life span, data), but
  the associated data has been left out for clarity. 
  This TMVBT has been created by transaction~$T_1$
  inserting keys~\range{1}{6} and transaction~$T_2$ inserting
  keys~\num{7} and~\num{8}.
  }
  \label{fig:example-oper-1}
\end{center}
\end{figure}

\begin{figure}[!htb]
\begin{center}
  \begin{tikzpicture}[x=0.125em,y=-0.058em]
\node[treepage] (p6) at (77,5) {%
\treeindexpagecontents{6.5}{$[\!-\infty,\!\infty\!),[1,\!\infty\!)$}%
  {%%
   $([\!-\infty,\!4),[1,\!\infty\!),p_{3})$\\
   $([4,\!\infty\!),[1,\!2),p_{5})$\\
   $([4,\!7),[2,\!\infty\!),p_{7})$\\
   $([7,\!\infty\!),[2,\!\infty\!),p_{8})$}};
\treepageheader{p6}{$p_{6}$};
\node[treepage] (p3) at (5,146) {%
\treepagecontents{6.5}{$[\!-\infty,\!4),[1,\!\infty\!)$}%
  {%%
   $(1,[1,\!\infty\!))$\\
   $(2,[1,\!\infty\!))$\\
   $(3,[1,\!\infty\!))$}};
\treepageheader{p3}{$p_{3}$};
\node[treepage,dead] (p5) at (53,146) {%
\treepagecontents{6.5}{$[4,\!\infty\!),[1,\!2)$}%
  {%%
   $(4,[1,\!2))$\\
   $(5,[1,\!2))$\\
   $(6,[1,\!2))$}};
\treepageheader{p5}{$p_{5}$};
\node[treepage] (p7) at (101,146) {%
\treepagecontents{6.5}{$[4,\!7),[2,\!\infty\!)$}%
  {%%
   $(4,[2,\!\infty\!))$\\
   $(5,[2,\!\infty\!))$\\
   $(6,[2,\!\infty\!))$}};
\treepageheader{p7}{$p_{7}$};
\node[treepage] (p8) at (149,146) {%
\treepagecontents{6.5}{$[7,\!\infty\!),[2,\!\infty\!)$}%
  {%%
   $(7,[2,\!\infty\!))$\\
   $(8,[2,\!\infty\!))$\\
   $(9,[2,\!\infty\!))$}};
\treepageheader{p8}{$p_{8}$};
\draw[parentlink] (p6.south) -- ($(p3header.north) + (0,-2)$);
\draw[parentlink] (p6.south) -- ($(p5header.north) + (0,-2)$);
\draw[parentlink] (p6.south) -- ($(p7header.north) + (0,-2)$);
\draw[parentlink] (p6.south) -- ($(p8header.north) + (0,-2)$);

\end{tikzpicture}

  \figcaption{\abbr{TMVBT} after inserting entry with key~\num{9}}
  {
  The format of the figure is the same as in \figref{fig:example-oper-1}.
  White rectangles denote live pages, and gray rectangles denote dead pages.
  Transaction~$T_2$ has caused a version-split on~$p_5$ by inserting
  key~\num{9}.}
  \label{fig:example-oper-2}
\end{center}
\end{figure}

\begin{figure}[!htb]
\begin{center}
  \begin{tikzpicture}[x=0.13em,y=-0.058em]
\node[treepage,dead] (p6) at (29,5) {%
\treeindexpagecontents{6.5}{$[\!-\infty,\!\infty\!),[1,\!2)$}%
  {%%
   $([\!-\infty,\!4),[1,\!2),p_{3})$\\
   $([4,\!\infty\!),[1,\!2),p_{5})$}};
\treepageheader{p6}{$p_{6}$};
\node[treepage,dead] (p3) at (5,146) {%
\treepagecontents{6.5}{$[\!-\infty,\!4),[1,\!2)$}%
  {%%
   $(1,[1,\!2))$\\
   $(2,[1,\!2))$\\
   $(3,[1,\!2))$}};
\treepageheader{p3}{$p_{3}$};
\node[treepage,dead] (p5) at (53,146) {%
\treepagecontents{6.5}{$[4,\!\infty\!),[1,\!2)$}%
  {%%
   $(4,[1,\!2))$\\
   $(5,[1,\!2))$\\
   $(6,[1,\!2))$}};
\treepageheader{p5}{$p_{5}$};
\node[treepage] (p8) at (101,146) {%
\treepagecontents{6.5}{$[\!-\infty,\!\infty\!),[2,\!\infty\!)$}%
  {%%
   $(1,[2,\!\infty\!))$\\
   $(2,[2,\!\infty\!))$\\
   $(3,[2,\!\infty\!))$}};
\treepageheader{p8}{$p_{8}$};
\draw[parentlink] (p6.south) -- ($(p3header.north) + (0,-2)$);
\draw[parentlink] (p6.south) -- ($(p5header.north) + (0,-2)$);

\end{tikzpicture}

  \figcaption{\abbr{TMVBT} after deleting most of the entries}
  {Transaction~$T_2$ deleted keys~\range{4}{9}, thus shrinking
  the current-version search tree to a single page.}
  \label{fig:example-oper-3}
\end{center}
\end{figure}


\begin{figure*}[htb]
\begin{center}
  \begin{tikzpicture}[x=0.13em,y=-0.058em]
\node[treepage] (p6) at (53,5) {%
\treeindexpagecontents{6.5}{$[\!-\infty,\!\infty\!),[1,\!\infty\!)$}%
  {%%
   $([\!-\infty,\!4),[1,\!\infty\!),p_{3})$\\
   $([4,\!7),[1,\!\infty\!),p_{5})$\\
   $([7,\!\infty\!),[1,\!\infty\!),p_{7})$}};
\treepageheader{p6}{$p_{6}$};
\node[treepage] (p3) at (5,146) {%
\treepagecontents{6.5}{$[\!-\infty,\!4),[1,\!\infty\!)$}%
  {%%
   $(1,[1,\!\infty\!))$\\
   $(2,[1,\!\infty\!))$\\
   $(3,[1,\!\infty\!))$}};
\treepageheader{p3}{$p_{3}$};
\node[treepage] (p5) at (53,146) {%
\treepagecontents{6.5}{$[4,\!7),[1,\!\infty\!)$}%
  {%%
   $(4,[1,\!\infty\!))$\\
   $(5,[1,\!\infty\!))$\\
   $(6,[1,\!\infty\!))$}};
\treepageheader{p5}{$p_{5}$};
\node[treepage] (p7) at (101,146) {%
\treepagecontents{6.5}{$[7,\!\infty\!),[1,\!\infty\!)$}%
  {%%
   $(7,[1,\!\infty\!))$\\
   $(8,[1,\!\infty\!))$\\
   $(9,[1,\!\infty\!))$}};
\treepageheader{p7}{$p_{7}$};
\draw[parentlink] (p6.south) -- ($(p3header.north) + (0,-2)$);
\draw[parentlink] (p6.south) -- ($(p5header.north) + (0,-2)$);
\draw[parentlink] (p6.south) -- ($(p7header.north) + (0,-2)$);

\end{tikzpicture}

  \figcaption{Example of a TMVBT index after insertions}{In this
  figure, transaction $T_1$ has inserted keys \range{1}{9}.}
  \label{fig:tmvbt-example:1}
\end{center}
\end{figure*}

\begin{figure*}[htb]
\begin{center}
  \begin{tikzpicture}[x=0.125em,y=-0.058em]
\node[treepage] (p6) at (77,5) {%
\treeindexpagecontents{6.5}{$[\!-\infty,\!\infty\!),[1,\!\infty\!)$}%
  {%%
   $([\!-\infty,\!4),[1,\!\infty\!),p_{3})$\\
   $([4,\!7),[1,\!2),p_{5})$\\
   $([4,\!\infty\!),[2,\!\infty\!),p_{8})$\\
   $([7,\!\infty\!),[1,\!2),p_{7})$}};
\treepageheader{p6}{$p_{6}$};
\node[treepage] (p3) at (5,146) {%
\treepagecontents{6.5}{$[\!-\infty,\!4),[1,\!\infty\!)$}%
  {%%
   $(1,[1,\!\infty\!))$\\
   $(2,[1,\!\infty\!))$\\
   $(3,[1,\!\infty\!))$}};
\treepageheader{p3}{$p_{3}$};
\node[treepage,dead] (p5) at (53,146) {%
\treepagecontents{6.5}{$[4,\!7),[1,\!2)$}%
  {%%
   $(4,[1,\!2))$\\
   $(5,[1,\!2))$\\
   $(6,[1,\!2))$}};
\treepageheader{p5}{$p_{5}$};
\node[treepage] (p8) at (101,146) {%
\treepagecontents{6.5}{$[4,\!\infty\!),[2,\!\infty\!)$}%
  {%%
   $(4,[2,\!\infty\!))$\\
   $(5,[2,\!\infty\!))$\\
   $(6,[2,\!\infty\!))$}};
\treepageheader{p8}{$p_{8}$};
\node[treepage,dead] (p7) at (149,146) {%
\treepagecontents{6.5}{$[7,\!\infty\!),[1,\!2)$}%
  {%%
   $(7,[1,\!2))$\\
   $(8,[1,\!2))$\\
   $(9,[1,\!2))$}};
\treepageheader{p7}{$p_{7}$};
\draw[parentlink] (p6.south) -- ($(p3header.north) + (0,-2)$);
\draw[parentlink] (p6.south) -- ($(p5header.north) + (0,-2)$);
\draw[parentlink] (p6.south) -- ($(p8header.north) + (0,-2)$);
\draw[parentlink] (p6.south) -- ($(p7header.north) + (0,-2)$);

\end{tikzpicture}

  \figcaption{Example of a TMVBT index after some deletions}{In this
  figure, transaction $T_2$ has deleted keys \range{7}{9}.}
  \label{fig:tmvbt-example:2}
\end{center}
\end{figure*}

\begin{figure*}[htb]
\begin{center}
  \begin{tikzpicture}[x=0.108em,y=-0.058em]
\node[treepage,dead] (p6) at (53,5) {%
\treeindexpagecontents{6.5}{$[\!-\infty,\!\infty\!),[1,\!2)$}%
  {%%
   $([\!-\infty,\!4),[1,\!2),p_{3})$\\
   $([4,\!7),[1,\!2),p_{5})$\\
   $([7,\!\infty\!),[1,\!2),p_{7})$}};
\treepageheader{p6}{$p_{6}$};
\node[treepage] (p3) at (5,146) {%
\treepagecontents{6.5}{$[\!-\infty,\!4),[1,\!\infty\!)$}%
  {%%
   $(1,[1,\!\infty\!))$\\
   $(2,[1,\!\infty\!))$\\
   $(3,[1,\!\infty\!))$}};
\treepageheader{p3}{$p_{3}$};
\node[treepage,dead] (p5) at (53,146) {%
\treepagecontents{6.5}{$[4,\!7),[1,\!2)$}%
  {%%
   $(4,[1,\!2))$\\
   $(5,[1,\!2))$\\
   $(6,[1,\!2))$}};
\treepageheader{p5}{$p_{5}$};
\node[treepage,dead] (p7) at (101,146) {%
\treepagecontents{6.5}{$[7,\!\infty\!),[1,\!2)$}%
  {%%
   $(7,[1,\!2))$\\
   $(8,[1,\!2))$\\
   $(9,[1,\!2))$}};
\treepageheader{p7}{$p_{7}$};
\node[treepage] (p10) at (125,5) {%
\treeindexpagecontents{6.5}{$[\!-\infty,\!\infty\!),[2,\!\infty\!)$}%
  {%%
   $([\!-\infty,\!4),[2,\!\infty\!),p_{3})$\\
   $([4,\!10),[2,\!\infty\!),p_{8})$\\
   $([10,\!13),[2,\!\infty\!),p_{9})$\\
   $([13,\!\infty\!),[2,\!\infty\!),p_{11})$}};
\treepageheader{p10}{$p_{10}$};
\node[treepage] (p8) at (149,146) {%
\treepagecontents{6.5}{$[4,\!10),[2,\!\infty\!)$}%
  {%%
   $(4,[2,\!\infty\!))$\\
   $(5,[2,\!\infty\!))$\\
   $(6,[2,\!\infty\!))$}};
\treepageheader{p8}{$p_{8}$};
\node[treepage] (p9) at (197,146) {%
\treepagecontents{6.5}{$[10,\!13),[2,\!\infty\!)$}%
  {%%
   $(10,[2,\!\infty\!))$\\
   $(11,[2,\!\infty\!))$\\
   $(12,[2,\!\infty\!))$}};
\treepageheader{p9}{$p_{9}$};
\node[treepage] (p11) at (245,146) {%
\treepagecontents{6.5}{$[13,\!\infty\!),[2,\!\infty\!)$}%
  {%%
   $(13,[2,\!\infty\!))$\\
   $(14,[2,\!\infty\!))$\\
   $(15,[2,\!\infty\!))$}};
\treepageheader{p11}{$p_{11}$};
\draw[parentlink] (p6.south) -- ($(p3header.north) + (0,-2)$);
\draw[parentlink] (p6.south) -- ($(p5header.north) + (0,-2)$);
\draw[parentlink] (p6.south) -- ($(p7header.north) + (0,-2)$);
\draw[parentlink] (p10.south) -- ($(p3header.north) + (0,-2)$);
\draw[parentlink] (p10.south) -- ($(p8header.north) + (0,-2)$);
\draw[parentlink] (p10.south) -- ($(p9header.north) + (0,-2)$);
\draw[parentlink] (p10.south) -- ($(p11header.north) + (0,-2)$);

\end{tikzpicture}

  \figcaption{Example of a TMVBT index after more insertions}{In this
  figure, transaction $T_2$ has inserted keys \range{10}{15}.}
  \label{fig:tmvbt-example:3}
\end{center}
\end{figure*}

\begin{figure}[!htb]
\begin{center}
  \begin{tikzpicture}[x=\tmvbtsmox,y=\tmvbtsmoy]
  \small
\coordinate (lo) at (0,0);
\coordinate (hi) at (0,3);
\coordinate (loact) at (2,0);
\coordinate (hiact) at (2,3);
\coordinate (loinf) at (3,0);
\coordinate (hiinf) at (3,3);

  \pattern[bgpattern] (lo) rectangle (hiact);

  \node[anchor=base] at ($(loact) + (0.5, 1.3)$) {$p$};

  \draw (loact) -- (hiact);
  \draw (loact) -- (loinf);
  \draw (hiact) -- (hiinf);
  \draw[opendash] decorate [opendeco] { (loinf) -- (hiinf) };

\end{tikzpicture}
\raisebox{1.25\tmvbtsmoy}{$\rightarrow$}
\begin{tikzpicture}[x=\tmvbtsmox,y=\tmvbtsmoy]
  \small
\coordinate (lo) at (0,0);
\coordinate (hi) at (0,3);
\coordinate (loact) at (2,0);
\coordinate (hiact) at (2,3);
\coordinate (loinf) at (3,0);
\coordinate (hiinf) at (3,3);

  \pattern[bgpattern] (lo) rectangle (hiact);

  \node[anchor=base] at ($(loact) + (0.5, 2.3)$) {$p$};
  \node[anchor=base] at ($(loact) + (0.5, 0.8)$) {$p'$};

  \draw (loact) -- (hiact);
  \draw (loact) -- (loinf);
  \draw (hiact) -- (hiinf);
  \draw ($(loact)+(0,2)$) -- ($(loinf)+(0,2)$);
  \draw[opendash] decorate [opendeco] { (loinf) -- ($(loinf)+(0,2)$) };
  \draw[opendash] decorate [opendeco] { ($(loinf)+(0,2)$) -- (hiinf) };

\end{tikzpicture}

  \figcaption{Key-splitting an active page~$p$}%
  {The horizontal axis represents version ranges, and the vertical axis key
  ranges.}
 % [$p$ active, $\liveentries{p} = \capacity$]
  \label{fig:split-active} 
\end{center}
\end{figure}

\chapter{Fourth images}

\begin{figure}[!htb]
\begin{center}
  \subfigure[$\minsplit \leq \liveentries{p} \leq
  \maxsplit$]{\label{fig:split:is1}
    \begin{tikzpicture}[x=\tmvbtsmox,y=\tmvbtsmoy]
  \small
\coordinate (lo) at (0,0);
\coordinate (hi) at (0,3);
\coordinate (loact) at (2,0);
\coordinate (hiact) at (2,3);
\coordinate (loinf) at (3,0);
\coordinate (hiinf) at (3,3);


  \node[anchor=base] at ($(lo) + (1.5, 1.3)$) {$p$};

  \draw (lo) -- (hi);
  \draw (lo) -- (loinf);
  \draw (hi) -- (hiinf);
  \draw[opendash] decorate [opendeco] { (loinf) -- (hiinf) };

\end{tikzpicture}
\raisebox{1.25\tmvbtsmoy}{$\rightarrow$}
\begin{tikzpicture}[x=\tmvbtsmox,y=\tmvbtsmoy]
  \small
\coordinate (lo) at (0,0);
\coordinate (hi) at (0,3);
\coordinate (loact) at (2,0);
\coordinate (hiact) at (2,3);
\coordinate (loinf) at (3,0);
\coordinate (hiinf) at (3,3);


  \node[anchor=base] at ($(lo) + (1, 1.3)$) {$p$};
  \node[anchor=base] at ($(loact) + (0.5, 1.3)$) {$p'$};

  \draw (lo) -- (hi);
  \draw (lo) -- (loinf);
  \draw (hi) -- (hiinf);
  \draw (hiact) -- (loact);
  \draw[opendash] decorate [opendeco] { (loinf) -- (hiinf) };

\end{tikzpicture}
}
  \subfigure[$\liveentries{p} > \maxsplit$]{\label{fig:split:is2} 
    \begin{tikzpicture}[x=\tmvbtsmox,y=\tmvbtsmoy]
  \small
\coordinate (lo) at (0,0);
\coordinate (hi) at (0,3);
\coordinate (loact) at (2,0);
\coordinate (hiact) at (2,3);
\coordinate (loinf) at (3,0);
\coordinate (hiinf) at (3,3);


  \node[anchor=base] at ($(lo) + (1.5, 1.3)$) {$p$};

  \draw (lo) -- (hi);
  \draw (lo) -- (loinf);
  \draw (hi) -- (hiinf);
  \draw[opendash] decorate [opendeco] { (loinf) -- (hiinf) };

\end{tikzpicture}
\raisebox{1.25\tmvbtsmoy}{$\rightarrow$}
\begin{tikzpicture}[x=\tmvbtsmox,y=\tmvbtsmoy]
  \small
\coordinate (lo) at (0,0);
\coordinate (hi) at (0,3);
\coordinate (loact) at (2,0);
\coordinate (hiact) at (2,3);
\coordinate (loinf) at (3,0);
\coordinate (hiinf) at (3,3);


  \node[anchor=base] at ($(lo) + (1, 1.3)$) {$p$};
  \node[anchor=base] at ($(loact) + (0.5, 2.3)$) {$p'$};
  \node[anchor=base] at ($(loact) + (0.5, 0.8)$) {$p''$};

  \draw (lo) -- (hi);
  \draw (lo) -- (loinf);
  \draw (hi) -- (hiinf);
  \draw (hiact) -- (loact);
  \draw ($(loact)+(0,2)$) -- ($(loinf)+(0,2)$);
  \draw[opendash] decorate [opendeco] { (loinf) -- ($(loinf)+(0,2)$) };
  \draw[opendash] decorate [opendeco] { ($(loinf)+(0,2)$) -- (hiinf) };

\end{tikzpicture}
}
  \\
  \subfigure[$s$ inactive, $\liveentries{p} < \minsplit$ and 
    $\liveentries{p} + \liveentries{s} \leq
  \maxsplit$]{\label{fig:split:imi1}
    \begin{tikzpicture}[x=\tmvbtsmox,y=\tmvbtsmoy]
  \small
\coordinate (lo) at (0,0);
\coordinate (hi) at (0,3);
\coordinate (loact) at (2,0);
\coordinate (hiact) at (2,3);
\coordinate (loinf) at (3,0);
\coordinate (hiinf) at (3,3);


  \node[anchor=base] at ($(lo) + (1.5, 1.8)$) {$p$};
  \node[anchor=base] at ($(lo) + (1.5, 0.3)$) {$s$};

  \draw (lo) -- (hi);
  \draw (lo) -- (loinf);
  \draw (hi) -- (hiinf);
  \draw ($(lo) + (0,1)$) -- ($(loinf) + (0,1)$);
  \draw[opendash] decorate [opendeco] { (loinf) -- ($(loinf) + (0,1)$) };
  \draw[opendash] decorate [opendeco] { ($(loinf) + (0,1)$) -- (hiinf) };

\end{tikzpicture}
\raisebox{1.25\tmvbtsmoy}{$\rightarrow$}
\begin{tikzpicture}[x=\tmvbtsmox,y=\tmvbtsmoy]
  \small
\coordinate (lo) at (0,0);
\coordinate (hi) at (0,3);
\coordinate (loact) at (2,0);
\coordinate (hiact) at (2,3);
\coordinate (loinf) at (3,0);
\coordinate (hiinf) at (3,3);


  \node[anchor=base] at ($(lo) + (1, 1.8)$) {$p$};
  \node[anchor=base] at ($(lo) + (1, 0.3)$) {$s$};

  \node[anchor=base] at ($(loact) + (0.5, 1.3)$) {$p'$};

  \draw (lo) -- (hi);
  \draw (lo) -- (loinf);
  \draw (hi) -- (hiinf);
  \draw ($(o) + (0,1)$) -- ($(loact) + (0,1)$);
  \draw (hiact) -- (loact);
  \draw[opendash] decorate [opendeco] { (loinf) -- (hiinf) };

\end{tikzpicture}
}
  \subfigure[$s$ inactive, $\liveentries{p} < \minsplit$ and  
    $\liveentries{p} + \liveentries{s} >
  \maxsplit$]{\label{fig:split:imi2}
    \begin{tikzpicture}[x=\tmvbtsmox,y=\tmvbtsmoy]
  \small
\coordinate (lo) at (0,0);
\coordinate (hi) at (0,3);
\coordinate (loact) at (2,0);
\coordinate (hiact) at (2,3);
\coordinate (loinf) at (3,0);
\coordinate (hiinf) at (3,3);


  \node[anchor=base] at ($(lo) + (1.5, 2.3)$) {$p$};
  \node[anchor=base] at ($(lo) + (1.5, 0.8)$) {$s$};

  \draw (lo) -- (hi);
  \draw (lo) -- (loinf);
  \draw (hi) -- (hiinf);
  \draw ($(lo) + (0,2)$) -- ($(loinf) + (0,2)$);
  \draw[opendash] decorate [opendeco] { (loinf) -- ($(loinf) + (0,2)$) };
  \draw[opendash] decorate [opendeco] { ($(loinf) + (0,2)$) -- (hiinf) };

\end{tikzpicture}
\raisebox{1.25\tmvbtsmoy}{$\rightarrow$}
\begin{tikzpicture}[x=\tmvbtsmox,y=\tmvbtsmoy]
  \small
\coordinate (lo) at (0,0);
\coordinate (hi) at (0,3);
\coordinate (loact) at (2,0);
\coordinate (hiact) at (2,3);
\coordinate (loinf) at (3,0);
\coordinate (hiinf) at (3,3);


  \node[anchor=base] at ($(lo) + (1, 2.3)$) {$p$};
  \node[anchor=base] at ($(lo) + (1, 0.8)$) {$s$};

  \node[anchor=base] at ($(loact) + (0.5, 1.8)$) {$p'$};
  \node[anchor=base] at ($(loact) + (0.5, 0.3)$) {$p''$};

  \draw (lo) -- (hi);
  \draw (lo) -- (loinf);
  \draw (hi) -- (hiinf);
  \draw ($(o) + (0,2)$) -- ($(loact) + (0,2)$);
  \draw (hiact) -- (loact);
  \draw ($(loact) + (0,1)$) -- ($(loinf) + (0,1)$);

  \draw[opendash] decorate [opendeco] { (loinf) -- ($(loinf) + (0,2)$) };
  \draw[opendash] decorate [opendeco] { ($(loinf) + (0,2)$) -- (hiinf) };

\end{tikzpicture}
}
  \\
  \subfigure[$s$ active, $\liveentries{p} < \minsplit$ and  
    $\liveentries{p} + \liveentries{s} \leq
  \maxsplit$]{\label{fig:split:ima1}
    \begin{tikzpicture}[x=\tmvbtsmox,y=\tmvbtsmoy]
  \small
\coordinate (lo) at (0,0);
\coordinate (hi) at (0,3);
\coordinate (loact) at (2,0);
\coordinate (hiact) at (2,3);
\coordinate (loinf) at (3,0);
\coordinate (hiinf) at (3,3);


  \node[anchor=base] at ($(lo) + (1.5, 1.8)$) {$p$};
  \node[anchor=base] at ($(loact) + (0.5, 0.3)$) {$s$};

  \pattern[bgpattern] (lo) rectangle ($(loact) + (0,1)$);

  \draw ($(lo) + (0,1)$) -- (hi);
  \draw ($(lo) + (0,1)$) -- ($(loinf) + (0,1)$);
  \draw (hi) -- (hiinf);
  \draw[opendash] decorate [opendeco] { (loinf) -- ($(loinf) + (0,1)$) };
  \draw[opendash] decorate [opendeco] { ($(loinf) + (0,1)$) -- (hiinf) };
  \draw (loact) -- (loinf);
  \draw (loact) -- ($(loact) + (0,1)$);

\end{tikzpicture}
\raisebox{1.25\tmvbtsmoy}{$\rightarrow$}
\begin{tikzpicture}[x=\tmvbtsmox,y=\tmvbtsmoy]
  \small
\coordinate (lo) at (0,0);
\coordinate (hi) at (0,3);
\coordinate (loact) at (2,0);
\coordinate (hiact) at (2,3);
\coordinate (loinf) at (3,0);
\coordinate (hiinf) at (3,3);


  \node[anchor=base] at ($(lo) + (1, 1.8)$) {$p$};
  \node[anchor=base] at ($(loact) + (0.5, 1.3)$) {$p'$};

  \pattern[bgpattern] (lo) rectangle ($(loact) + (0,1)$);

  \draw ($(lo) + (0,1)$) -- (hi);
  \draw ($(lo) + (0,1)$) -- ($(loact) + (0,1)$);
  \draw (hi) -- (hiinf);
  \draw[opendash] decorate [opendeco] { (loinf) -- (hiinf) };
  \draw (loact) -- (loinf);
  \draw (loact) -- (hiact);

\end{tikzpicture}
}
  \subfigure[$s$ active, $\liveentries{p} < \minsplit$ and  
    $\liveentries{p} + \liveentries{s} >
  \maxsplit$]{\label{fig:split:ima2}
    \begin{tikzpicture}[x=\tmvbtsmox,y=\tmvbtsmoy]
  \small
\coordinate (lo) at (0,0);
\coordinate (hi) at (0,3);
\coordinate (loact) at (2,0);
\coordinate (hiact) at (2,3);
\coordinate (loinf) at (3,0);
\coordinate (hiinf) at (3,3);


  \node[anchor=base] at ($(lo) + (1.5, 2.3)$) {$p$};
  \node[anchor=base] at ($(loact) + (0.5, 0.8)$) {$s$};

  \pattern[bgpattern] (lo) rectangle ($(loact) + (0,2)$);

  \draw ($(lo) + (0,2)$) -- (hi);
  \draw ($(lo) + (0,2)$) -- ($(loinf) + (0,2)$);
  \draw (hi) -- (hiinf);
  \draw[opendash] decorate [opendeco] { (loinf) -- ($(loinf) + (0,2)$) };
  \draw[opendash] decorate [opendeco] { ($(loinf) + (0,2)$) -- (hiinf) };
  \draw (loact) -- (loinf);
  \draw (loact) -- ($(loact) + (0,2)$);

\end{tikzpicture}
\raisebox{1.25\tmvbtsmoy}{$\rightarrow$}
\begin{tikzpicture}[x=\tmvbtsmox,y=\tmvbtsmoy]
  \small
\coordinate (lo) at (0,0);
\coordinate (hi) at (0,3);
\coordinate (loact) at (2,0);
\coordinate (hiact) at (2,3);
\coordinate (loinf) at (3,0);
\coordinate (hiinf) at (3,3);


  \node[anchor=base] at ($(lo) + (1, 2.3)$) {$p$};

  \node[anchor=base] at ($(loact) + (0.5, 1.8)$) {$p'$};
  \node[anchor=base] at ($(loact) + (0.5, 0.3)$) {$p''$};

  \pattern[bgpattern] (lo) rectangle ($(loact) + (0,2)$);

  \draw ($(lo) + (0,2)$) -- (hi);
  \draw ($(lo) + (0,2)$) -- ($(loact) + (0,2)$);
  \draw (hi) -- (hiinf);
  \draw (loact) -- (loinf);
  \draw (loact) -- (hiact);

  \draw ($(loact) + (0,1)$) -- ($(loinf) + (0,1)$);
  \draw[opendash] decorate [opendeco] { (loinf) -- ($(loinf) + (0,2)$) };
  \draw[opendash] decorate [opendeco] { ($(loinf) + (0,2)$) -- (hiinf) };

\end{tikzpicture}
}
  \figcaption{Version-splitting an inactive page~$p$}
  {The horizontal axis represents version ranges, and the vertical
  axis key ranges. 
  Case 
  (a)~represents a version split, 
  (b)~a version split followed by a key split,
  (c)~a version split followed by a merge with an inactive sibling, 
  (d)~a version split followed by a redistribution of live entries
  with an inactive sibling, 
  (e)~a version split followed by a merge with an active
  sibling, and 
  (f)~a version split followed by a redistribution of live entries
  with an active sibling.}
  \label{fig:split}
\end{center}
\end{figure}

\begin{figure}[!htb]
\begin{center}
  \subfigure[$p$ inactive, $s$ active, $\liveentries{p} + \liveentries{s}
  \leq \maxsplit$]{\label{fig:merge:ima1}
    \begin{tikzpicture}[x=\tmvbtsmox,y=\tmvbtsmoy]
  \small
\coordinate (lo) at (0,0);
\coordinate (hi) at (0,3);
\coordinate (loact) at (2,0);
\coordinate (hiact) at (2,3);
\coordinate (loinf) at (3,0);
\coordinate (hiinf) at (3,3);

  \pattern[bgpattern] (lo) rectangle ($(loact) + (0,2)$);

  \node[anchor=base] at ($(lo) + (1.5, 2.3)$) {$p$};
  \node[anchor=base] at ($(loact) + (0.5, 0.8)$) {$s$};

  \draw (loact) -- (loinf);
  \draw (loact) -- ($(loact) + (0,2)$);
  \draw (hi) -- (hiinf);
  \draw ($(lo)+(0,2)$) -- ($(loinf)+(0,2)$);
  \draw ($(lo)+(0,2)$) -- (hi);
  \draw[opendash] decorate [opendeco] { (loinf) -- ($(loinf)+(0,2)$) };
  \draw[opendash] decorate [opendeco] { ($(loinf)+(0,2)$) -- (hiinf) };
\end{tikzpicture}
\raisebox{1.25\tmvbtsmoy}{$\rightarrow$}
\begin{tikzpicture}[x=\tmvbtsmox,y=\tmvbtsmoy]
\small
\coordinate (lo) at (0,0);
\coordinate (hi) at (0,3);
\coordinate (loact) at (2,0);
\coordinate (hiact) at (2,3);
\coordinate (loinf) at (3,0);
\coordinate (hiinf) at (3,3);

  \pattern[bgpattern] (lo) rectangle ($(loact) + (0,2)$);

  \node[anchor=base] at ($(loact) + (0.5, 1.3)$) {$p'$};
  \node[anchor=base] at ($(lo) + (1, 2.3)$) {$p$};

  \draw (hi) rectangle ($(loact) + (0,2)$); 

  \draw (loact) -- ($(loact) + (0,2)$);
  \draw (loact) -- (loinf);
  \draw (hiact) -- (hiinf);
  \draw[opendash] decorate [opendeco] { (loinf) -- (hiinf) };
\end{tikzpicture}
}
  \subfigure[$p$ inactive, $s$ active, $\liveentries{p} + \liveentries{s} >
  \maxsplit$]{\label{fig:merge:ima2}
    \begin{tikzpicture}[x=\tmvbtsmox,y=\tmvbtsmoy]
  \small
\coordinate (lo) at (0,0);
\coordinate (hi) at (0,3);
\coordinate (loact) at (2,0);
\coordinate (hiact) at (2,3);
\coordinate (loinf) at (3,0);
\coordinate (hiinf) at (3,3);

  \pattern[bgpattern] (lo) rectangle ($(loact) + (0,2)$);

  \node[anchor=base] at ($(lo) + (1.5, 2.3)$) {$p$};
  \node[anchor=base] at ($(loact) + (0.5, 0.8)$) {$s$};

  \draw (loact) -- (loinf);
  \draw (loact) -- ($(loact) + (0,2)$);
  \draw (hi) -- (hiinf);
  \draw ($(lo)+(0,2)$) -- ($(loinf)+(0,2)$);
  \draw ($(lo)+(0,2)$) -- (hi);
  \draw[opendash] decorate [opendeco] { (loinf) -- ($(loinf)+(0,2)$) };
  \draw[opendash] decorate [opendeco] { ($(loinf)+(0,2)$) -- (hiinf) };
\end{tikzpicture}
\raisebox{1.25\tmvbtsmoy}{$\rightarrow$}
\begin{tikzpicture}[x=\tmvbtsmox,y=\tmvbtsmoy]
\small
\coordinate (lo) at (0,0);
\coordinate (hi) at (0,3);
\coordinate (loact) at (2,0);
\coordinate (hiact) at (2,3);
\coordinate (loinf) at (3,0);
\coordinate (hiinf) at (3,3);

  \pattern[bgpattern] (lo) rectangle ($(loact) + (0,2)$);

  \node[anchor=base] at ($(lo) + (1, 2.3)$) {$p$};
  \node[anchor=base] at ($(loact) + (0.5, 1.8)$) {$p'$};
  \node[anchor=base] at ($(loact) + (0.5, 0.3)$) {$p''$};

  \draw (hi) rectangle ($(loact) + (0,2)$); 

  \draw (loact) -- ($(loact) + (0,2)$);
  \draw (loact) -- (loinf);
  \draw (hiact) -- (hiinf);
  \draw ($(loact) + (0,1)$) -- ($(loinf) + (0,1)$);
  \draw[opendash] decorate [opendeco] { (loinf) -- ($(loinf)+(0,1)$) };
  \draw[opendash] decorate [opendeco] { ($(loinf)+(0,1)$) -- (hiinf) };
\end{tikzpicture}
}
  \subfigure[$p$ inactive, $s$ inactive, $\liveentries{p} + \liveentries{s} \leq
  \maxsplit$]{\label{fig:merge:imi1} 
    \begin{tikzpicture}[x=\tmvbtsmox,y=\tmvbtsmoy]
  \small
\coordinate (lo) at (0,0);
\coordinate (hi) at (0,3);
\coordinate (loact) at (2,0);
\coordinate (hiact) at (2,3);
\coordinate (loinf) at (3,0);
\coordinate (hiinf) at (3,3);


  \node[anchor=base] at ($(lo) + (1.5, 2.3)$) {$p$};
  \node[anchor=base] at ($(lo) + (1.5, 0.8)$) {$s$};

  \draw (lo) -- (loinf);
  \draw (lo) -- (hi);
  \draw (hi) -- (hiinf);
  \draw ($(lo)+(0,2)$) -- ($(loinf)+(0,2)$);
  \draw[opendash] decorate [opendeco] { (loinf) -- ($(loinf)+(0,2)$) };
  \draw[opendash] decorate [opendeco] { ($(loinf)+(0,2)$) -- (hiinf) };
\end{tikzpicture}
\raisebox{1.25\tmvbtsmoy}{$\rightarrow$}
\begin{tikzpicture}[x=\tmvbtsmox,y=\tmvbtsmoy]
\small
\coordinate (lo) at (0,0);
\coordinate (hi) at (0,3);
\coordinate (loact) at (2,0);
\coordinate (hiact) at (2,3);
\coordinate (loinf) at (3,0);
\coordinate (hiinf) at (3,3);

  \node[anchor=base] at ($(lo) + (1, 2.3)$) {$p$};
  \node[anchor=base] at ($(lo) + (1, 0.8)$) {$s$};
  \node[anchor=base] at ($(loact) + (0.5, 1.3)$) {$p'$};

  \draw (lo) -- (loinf);
  \draw (lo) -- (hi);
  \draw (hi) -- (hiinf);

  \draw ($(lo) + (0,2)$) -- ($(loact) + (0,2)$);

  \draw (loact) -- (hiact);
  \draw (loact) -- (loinf);
  \draw (hiact) -- (hiinf);
  \draw[opendash] decorate [opendeco] { (loinf) -- (hiinf) };
\end{tikzpicture}
}
  \subfigure[$p$ inactive, $s$ inactive, $\liveentries{p} + \liveentries{s} >
  \maxsplit$]{\label{fig:merge:imi2}
    \begin{tikzpicture}[x=\tmvbtsmox,y=\tmvbtsmoy]
  \small
\coordinate (lo) at (0,0);
\coordinate (hi) at (0,3);
\coordinate (loact) at (2,0);
\coordinate (hiact) at (2,3);
\coordinate (loinf) at (3,0);
\coordinate (hiinf) at (3,3);


  \node[anchor=base] at ($(lo) + (1.5, 2.3)$) {$p$};
  \node[anchor=base] at ($(lo) + (1.5, 0.8)$) {$s$};

  \draw (lo) -- (loinf);
  \draw (lo) -- (hi);
  \draw (hi) -- (hiinf);
  \draw ($(lo)+(0,2)$) -- ($(loinf)+(0,2)$);
  \draw[opendash] decorate [opendeco] { (loinf) -- ($(loinf)+(0,2)$) };
  \draw[opendash] decorate [opendeco] { ($(loinf)+(0,2)$) -- (hiinf) };
\end{tikzpicture}
\raisebox{1.25\tmvbtsmoy}{$\rightarrow$}
\begin{tikzpicture}[x=\tmvbtsmox,y=\tmvbtsmoy]
\small
\coordinate (lo) at (0,0);
\coordinate (hi) at (0,3);
\coordinate (loact) at (2,0);
\coordinate (hiact) at (2,3);
\coordinate (loinf) at (3,0);
\coordinate (hiinf) at (3,3);

  \node[anchor=base] at ($(lo) + (1, 2.3)$) {$p$};
  \node[anchor=base] at ($(lo) + (1, 0.8)$) {$s$};
  \node[anchor=base] at ($(loact) + (0.5, 1.8)$) {$p'$};
  \node[anchor=base] at ($(loact) + (0.5, 0.3)$) {$p''$};

  \draw (lo) -- (loinf);
  \draw (lo) -- (hi);
  \draw (hi) -- (hiinf);

  \draw ($(lo) + (0,2)$) -- ($(loact) + (0,2)$);

  \draw (loact) -- (hiact);
  \draw ($(loact) + (0,1)$) -- ($(loinf) + (0,1)$);
  \draw[opendash] decorate [opendeco] { (loinf) -- ($(loinf)+(0,1)$) };
  \draw[opendash] decorate [opendeco] { ($(loinf)+(0,1)$) -- (hiinf) };
\end{tikzpicture}
}
  \figcaption{Merging an inactive page~$p$}
  {The horizontal axis represents version ranges, and the vertical
  axis key ranges.
  Case
  (a)~a represents a version split followed by a merge,
  (b)~a version split followed by a redistribution of live entries,
  (c)~a version split followed by a merge with an inactive
  sibling, and
  (d)~a version split followed by a redistribution of live entries
  with an inactive sibling.}
  \label{fig:merge-inactive}
\end{center}
\end{figure}


The theory of general transaction processing in traditional 
databases is well-defined and mature, and the basic principles are
well presented in many
textbooks~\cite{bernstein:1987:cc-n-r,gray:1993:transactionprocessing,papadimitriou:1986:cc-theory}.
In this chapter, we concentrate on the theory of multiversion
transaction processing, and highlight the differences to the classical
transaction theory.
The traditional read-write model assumes that transactions are sequences
of reads and writes on data items, without distinguishing item deletions and
insertions from
updates~\cite{bernstein:1987:cc-n-r,papadimitriou:1986:cc-theory}.
The theory of transaction processing in this dissertation is based on
the recoverable transaction model presented by Sippu and
Soisalon-Soininen~\cite{sippu:2001:theory}, which in turn is based on the
model proposed by C.~Mohan~\cite{mohan:1990:aries-kvl,mohan:1992:aries-im}.
In this model, data-item insertions and deletions are made explicit,
and structure-modification operations are included in the model.
We assume the partially persistent transaction-time model, as
described in the introduction.

\begin{figure}[htb]
\begin{center}
  \begin{tikzpicture}[x=1.5em,y=1.5em,%
every node/.style={font=\scriptsize},%
actr/.style={draw,shape=actor,inner sep=1ex,minimum width=3em},%
action/.style={draw,ellipse,minimum width=5.2em,minimum
height=2em,font=\tiny,fill=white!100},% 
actiontext/.style={font=\tiny},
userarr/.style={densely dashed},
decoration={markings,mark=at position .75 with {\arrow[line
width=1pt]{>>}}}, 
actarr/.style={postaction={decorate}},
border/.style={draw,shape=rounded rectangle,inner sep=1ex,densely
dotted,thick}]

\node[border,minimum width=15em,minimum height=14em] (borders) {};
\node[above=0pt of borders,anchor=north] {CMVBT};

\node[db,minimum height=5em,anchor=north west] (tmvbt) at ($(borders.north
west) + (1.5ex,-3em)$) {TMVBT};

\node[actr,anchor=south east] (mtx) at ($(borders.south east) + (-1ex,1em)$)
  {\parbox{6em}{\center Maintenance transaction}};

\node[action,left=1.5em of mtx.left] (actmove) {};
\node[actiontext] at ($(actmove) + (0,.9ex)$) {\parbox{8em}{\center Move
committed updates}};

\node[db,above=1.5em of actmove,minimum height=3em] (vbt) {VBT};

\node[action] (actupdate) at ($(borders.east) + (0,1em)$) {};
\node[action,above=1em of actupdate] (actquery) {};
\node[actiontext] at (actupdate) {Perform updates};
\node[actiontext] at (actquery) {Issue queries};

\node[actr] (user) at ($(actupdate.east) + (3em,1em)$) {User};

\draw[userarr] (mtx.left) -- (actmove.east);
\draw[userarr] (user.left) -- (actquery.east);
\draw[userarr] (user.left) -- (actupdate.east);

\newdimen\cmimx
\newdimen\cmimy

% Query action - TMVBT
\pgfextractx{\cmimx}{\pgfpointanchor{tmvbt}{east}}
\pgfextracty{\cmimy}{\pgfpointanchor{actquery}{west}}
\draw[actarr] (actquery.west) -- (\cmimx,\cmimy);

% Update action - VBT
\pgfextractx{\cmimx}{\pgfpointanchor{vbt}{east}}
\pgfextracty{\cmimy}{\pgfpointanchor{actupdate}{west}}
\draw[actarr] (actupdate.west) -- (\cmimx,\cmimy);

\draw[actarr] (actquery.west) -- (vbt);

\draw[actarr] (vbt) -- (actmove);
\draw[actarr] (actmove) -- (tmvbt);

\end{tikzpicture}
\\
  \figcaption{The CMVBT index structure organization}
  {User transactions issue queries both to the VBT and the
  TMVBT indexes, but updates are performed only on the VBT\@.
  A system maintenance transaction is run periodically to apply the
  pending updates of committed transactions into the TMVBT index and to
  delete them from the VBT\@.}
  \label{fig:cmvbt-setup}
\end{center}
\end{figure}

\begin{figure}[htb]
\begin{center}
%  \includegraphics[width=0.4\textwidth]{images/cmvbt-example}\\
  \begin{tikzpicture}[x=1.5em,y=1.5em]

\node[dbl,minimum width=3em] (tmvbt)
  {\footnotesize\parbox{2em}{\centering $1$\\$2$\\$3$\\~}};
\node[below=0.2 of tmvbt] {\footnotesize TMVBT};

\node[dbl,minimum width=3em,right=1 of tmvbt] (vbt)
  {\footnotesize\parbox{2em}{\centering $101$\\$102$\\$103$\\$104$}};
\node[below=0.2 of vbt] (vbtlab) {\footnotesize VBT};

\node[right=1.8 of vbtlab] (ctilab) {\footnotesize CTI};
\node[rounded rectangle,draw=black!100,above=0.2 of ctilab] (cti) 
  {\footnotesize\parbox{4em}{\centering $4 \to 103$\\$5 \to 101$}};

\node[above=0.5 of cti] (variables) 
  {\footnotesize\parbox{5em}{\flushright $\stablever = 3$\\$\comver = 5$}};

\end{tikzpicture}
\\
  \figcaption{Example of the logical contents of a CMVBT index}
  {The database contains the updates of five committed versions, versions
  \range{1}{5}. 
  The TMVBT index contains all the updates of stable versions
  \range{1}{3}, and the updates of committed transient versions \range{4}{5}
  are still located in the VBT index, identified with transaction
  identifiers \num{103} and~\num{101}.
  The VBT additionally contains updates by two active updating transactions
  that have transaction identifiers \num{102} and~\num{104}.}
  \label{fig:cmvbt-example}
\end{center}
\end{figure}

\begin{figure}[htb]
\begin{center}
  \subfigure{
\begin{tikzpicture}
\node[rounded rectangle,draw=black!100] (tmvbt) {%
\contentrows{6}{6.2em}{%
$(1, [1, 2), w_1)$\\
$(2, [1,\infty), w_2)$\\
$(3, [2,3), w_3)$\\
$(3, [3,\infty), w_3')$\\
$(4, [3,\infty), w_4)$}};
\node[above=0pt of tmvbt] {\textbf{TMVBT}};

\node[rounded rectangle,draw=black!100,right=1em of tmvbt] (vbt) {%
\contentrows{6}{5em}{%
$(1, 102, w_1')$\\
$(2, 101, w_2')$\\
$(4, 103, \deletemark)$\\
$(4, 104, w_4')$\\
$(6, 101, w_6)$\\
$(7, 103, w_7)$}};
\node[above=0pt of vbt] {\textbf{VBT}};

\node[rounded rectangle,draw=black!100,right=1em of vbt] (cti) {%
\contentrows{6}{3.4em}{%
$4 \to 103$\\
$5 \to 101$}};
\node[above=0pt of cti] {\textbf{CTI}};
\end{tikzpicture}
}
\subfigure{
  \begin{tabular}{c}
  \textbf{History of transactions}\\
  \begin{tabular}{l}
  $T_1$, $\commitver{T_1} = 1$: Insert $(1, w_1)$ and $(2, w_2)$\\
  $T_2$, $\commitver{T_2} = 2$: Insert $(3, w_3)$ and delete item with
  key~\num{1}\\ 
  $T_3$, $\commitver{T_3} = 3$: Insert $(3, w_3')$ and $(4, w_4)$\\
  $T_4$, $\commitver{T_4} = 4$: Insert $(7, w_7)$ and delete item with
  key~\num{4}\\ 
  $T_5$, $\commitver{T_5} = 5$: Insert $(2, w_2')$ and $(6, w_6)$\\
  $T_6$, $\txid{T_6} = 102$: Insert $(1, w_1')$\\ 
  $T_7$, $\txid{T_7} = 104$: Insert $(4, w_4')$\\ 
  \end{tabular}\\
  \end{tabular}
}

  \figcaption{Example of data entries stored in a CMVBT index}
  {This example represents the same situation as
  \figref{fig:cmvbt-example}.
  The format of entries in the TMVBT is (key, life span, data), and the
  format of entries in the VBT is (key, transaction identifier, update).
  In addition to the committed transactions, the VBT also contains the
  updates of two active transactions with transaction identifiers \num{102}
  and \num{104}.}
  \label{fig:cmvbt-contents}
\end{center}
\end{figure}

The theory of general transaction processing in traditional 
databases is well-defined and mature, and the basic principles are
well presented in many
textbooks~\cite{bernstein:1987:cc-n-r,gray:1993:transactionprocessing,papadimitriou:1986:cc-theory}.
In this chapter, we concentrate on the theory of multiversion
transaction processing, and highlight the differences to the classical
transaction theory.
The traditional read-write model assumes that transactions are sequences
of reads and writes on data items, without distinguishing item deletions and
insertions from
updates~\cite{bernstein:1987:cc-n-r,papadimitriou:1986:cc-theory}.
The theory of transaction processing in this dissertation is based on
the recoverable transaction model presented by Sippu and
Soisalon-Soininen~\cite{sippu:2001:theory}, which in turn is based on the
model proposed by C.~Mohan~\cite{mohan:1990:aries-kvl,mohan:1992:aries-im}.
In this model, data-item insertions and deletions are made explicit,
and structure-modification operations are included in the model.
We assume the partially persistent transaction-time model, as
described in the introduction.
