%%=====================================================================
%% Multiversion Index Structures
%%=====================================================================
\chapter{Multiversion Index Structures}
\label{chapter:mv-index}

We have now discussed the theory behind temporal databases, concentrating
mostly on multiversion databases.
In this chapter, we describe some of the index structures used in
multiversion databases.
We begin the chapter by defining a few general properties of multiversion
index structures in \secref{sec:mv-index:properties}. 
In \secref{sec:mv-index:btree}, we will demonstrate that a single-version
index is not an efficient structure for indexing multiversion data. 
To be able to properly determine the efficiency of multiversion index
structures, \secref{sec:mv-index:optimality} defines what we mean by an
optimal multiversion index structure, and \secref{sec:mv-index:common} lists
common design ideas used in efficient multiversion indexes.
In \secref{sec:mv-index:early}, we describe some of the early multiversion
index structures.
For a comprehensive presentation and comparison of different multiversion
access methods, the reader is referred to Salzberg and
Tsotras~\cite{salzberg:1999:comparison}, and \sic{�zsoyo{\v g}lu} and
Snodgrass~\cite{ozsoyoglu:1995:survey}.
The rest of this chapter is dedicated to different kinds of
structures that have been used to index multiversion data or are
otherwise related: spatial indexes~(\secref{sec:mv-index:spatial}), hashing
structures~(\secref{sec:mv-index:hashing}), version-control
systems~(\secref{sec:mv-index:vcs}), and
other structures~(\secref{sec:mv-index:other}).


%% Properties of Multiversion Indexes
%%---------------------------------------------------------------------
\section{Properties of Multiversion Indexes}
\label{sec:mv-index:properties}

To begin, let us define what we mean by a multiversion index structure:

\thmskip
\begin{definition}
\label{def:mv-index}
A \emph{multiversion index structure} is a transaction-time index 
that is partially persistent and enables efficient \qtype{$x$/$-$/point}
queries on the data items, where~$x$ is either \qtype{point}, \qtype{range},
or~$*$.
The index is a collection of nodes that forms a tree or a directed acyclic
graph (DAG\phantomsection\label{def:dag})\@.
The nodes of the graph are fixed-size database pages.
The graph contains one or possibly many \emph{root pages}, which
serve as starting points for search operations.
Pages that have child pages are called \emph{index pages} or \emph{parent
pages}, and pages that do not have child pages are called \emph{leaf pages}.
Each page contains \emph{entries} that represent either data items~(called
\emph{data entries}, see \defref{def:mv-item}) or routers to child pages
(called \emph{index entries}). 
Page capacity \capacity\phantomsection\label{def:capacity} tells how
many entries fit into the page. 
The capacity is dictated by the entry format and the page size, but for the
simplicity of the theoretical discussion, we assume that the page
capacity~\capacity\ is the same for all index and leaf pages. 
The data entries stored in the index may contain either the actual data
stored with the key (the row in the relation), in the case of a \emph{primary}
or \emph{sparse index}; or a pointer to a separate storage location, in the
case of a \emph{secondary} or \emph{dense index}.  
\end{definition}
\thmskip

The multiversion index structure defines the way the data items are stored
and accessed.
Similar to the \Btree, most often the index pages of a multiversion
index contain only index entries, and the leaf pages contain only data
entries. 
Searches in a multiversion index follow the same logic as searches in a
single-version index structure: each node has a number of child nodes, and
each child page covers a more restricted area of the search space.
The search spaces of sibling pages usually do not overlap, but there are
exceptions.
In a multiversion index, the search space is the key-version space. 
Each page thus covers a region in key-version space. 
If the multiversion index contains a single root page, then  that root page
covers the entire key-version space.
A child page's search-space region overlaps with the parent page's region,
and often the child page's region is a subset of the parent page's region.
A key~$k$ that is part of version~$v$ (alive at version $v$) is located at
the leaf page whose key-version region covers the key-version
coordinate~$(k, v)$.

In multiversion index structures, the most important property to optimize is
the number of pages that an action needs to read or write to perform an
action, because \abbr{I/O} operations on disk storage are still the most
significant bottleneck in most database
applications~\cite{kanellakis:1996:indexing}. 
A good index structure requires a minimal number of page accesses for its 
actions. 
If a search operation on a database index requires access to $m$ pages to
locate key~$k$, then $m$ is normally logarithmic in the number of data
items indexed by the structure, if the index is a tree structure.
For analyzing the performance of index structures, we define the cost of an
action:

\thmskip
\begin{definition}
\label{def:action-cost}
The \emph{cost} of an action or an operation is the number of 
index-structure pages the action needs to access (read and/or write).
In the case of a sparse index, this includes all the pages the actions
need to access.
In the case of a dense index, this includes only the pages of the index
structure itself, and not the data pages that may need to be accessed
additionally for each data item.
\end{definition}
\thmskip



%% Versioned B-tree
%%---------------------------------------------------------------------
\section{Versioned \Btree}
\label{sec:mv-index:btree}

The
B-tree~\cite{bayer:1972:btree,bayer:1977:btree-concurrency,comer:1979:btree}
is a widely used search tree structure that is optimized for use as a database
index structure. 
% The difference to other search trees is that 
A single node of the B-tree is stored in a single database page.
Each database page is designed to fit in a single disk block (or a fixed
number of consecutive blocks), so that reading and writing database pages is
as efficient as possible with the underlying storage medium.
Because disk block sizes range from
\SIrange{4}{64}{\kibi\byte}~\cite{shao:2004:clotho}, the nodes of the B-tree
have a huge number of children, ranging from hundreds to even thousands.
For example, if a \SI{4}{\kibi\byte} index page contains index entries that
consist of a four-byte key separator and a four-byte child page identifier,
the page can contain in the excess of \num{500}~entries; an
\SI{8}{\kibi\byte} page can contain a thousand index entries, and so on.
The fan-out of a B-tree is therefore high, and the trees tend to be very low.
There are typically only three to five pages on a path from the root to a
leaf page in even a very large database system.
 
The most widely used variant of the B-tree is the \Btree, which stores
data entries only in leaf nodes.
The index nodes of a \Btree\ thus contain only index entries.
%, as noted in the introduction to this chapter. 
The leaf pages of a \Btree\ index are at level one, and index pages are 
at consecutively higher levels.
The height of the \Btree\ is the level of its root
page\footnote{There are overlapping definitions for tree height and
page levels; we adopt the convention used by Bayer~\cite{bayer:1972:btree}
because it seems most natural.}\phantomsection\label{def:tree-height}.
With this convention, 
an empty \Btree\ has a height of zero (no pages allocated\footnote{For
practical efficiency reasons, an empty root page might need be kept allocated
for an empty \Btree\ index.});
a single leaf-page root page forms a \Btree\ of height one; 
and a \Btree\ with $n$~levels of index pages and a single level of leaf pages
forms a \Btree\ of height~$n+1$. 
A standard \Btree\ index stores entries of the form $(k, w)$ in its leaf
pages and entries of the form $(k, p)$ in its index pages, where $k$ is the
data item key, $w$ is the data value, and $p$ is the page identifier of a
child page that resides at the next lower level.
The key~$k$ used in the index entries is also called a \emph{router} that
directs the search to the correct child page.
The data value $w$ is either the value itself, in the case of a sparse
index; or a pointer to where the data is stored (a record identifier of the
data), in the case of a dense index. 
The entries are ordered by the key~$k$.
An update on a data item in a standard \Btree\ is performed by physically
deleting the old entry from the index, and by inserting a new entry to
replace the old one. 
No version history is recorded in a \Btree.
However, with a slight modification to the entry structure we can record the
history of data-item changes in the \Btree\ index.

As discussed in \secref{sec:mv-data:mv-data}, multiversion data items
are tuples of the form $(k, \vrmark{v}, w)$, where $\vrmark{v}$ is the life
span of the data item.
Furthermore, the data items can be uniquely identified by the key-version
pair $(k, v_1)$, when $\vrmark{v} = [v_1, v_2)$.
An update to a database index is logically either a key insertion or
update (a write action), or a key deletion (a delete action); recall the
update model from \secref{sec:mv-data:updating-tx}.
The multiversion history can thus be stored in a \Btree\ index if we simply
change the data entry format to $(k, v_1, w)$ for a write action and $(k, v_2,
\deletemark)$ for an item deletion.
The version $v_1$ represents the commit-time version of the
transaction that inserted the new value $w$, and $v_2$ represents the
commit-time version of the transaction that performed the deletion.
In this convention, each deleted multiversion data item (i.e., a multiversion
data item with a life span $\vrmark{v} = [v_1,v_2) : v_2 \neq \infty$) is
represented by two entries: one marking the insertion of the multiversion
data item, and one the deletion or updation of the item.
We call this extended \Btree\ structure the \emph{versioned \Btree},
or VBT\phantomsection\label{def:vbt} for short.
The entries in a VBT are ordered first by the keys, and then by the
versions, so that $(k, v, w) < (k', v', w')$ if either $k < k'$ or
$k = k' \wedge v < v'$.
This defines a total ordering, because entries are uniquely identified by the
pair $(k, v)$, so that no two updates can have the same key and version.

Because the entries are now ordered by the key-version pairs $(k, v)$, the
index entries in index pages need to reflect this.
The index entries of a VBT are therefore of the form $(k, v, p)$, where
$p$ is the identifier of a child page.
As with a \Btree, the $(k, v)$ pairs in index entries are separator values
that are used to direct searches.
If an index page contains $n$ entries $(k_i, v_i, p_i)$, with $i \in
\{1,2, \ldots, n\}$, and $(k_j, v_j) \leq (k, v) < (k_{j+1}, v_{j+1})$, then   
the page identifier $p_j$ is the identifier of the child page whose
key-version range contains the entry $(k, v)$.
%The operator $\leq$ is defined as 
We define $(k, v) \leq (k',v')$ if either $(k, v) < (k',v')$ or $k = k'
\wedge v = v'$.
The key-version pair $(k_i, v_i)$ in an index entry thus defines a
lower limit for key-version pairs in the child page pointed to by the index
entry, and the key-version pair $(k_{i+1}, v_{i+1})$ in the next entry
defines the upper limit.
This could as well be defined the other way around, so that the key-version
separator value stored with an index entry defines the upper limit for the
page pointed to by that entry; and the separator value stored in the previous
entry defines the lower limit.

Searching for a single key~$k$ at version~$v$ in the VBT is performed by
locating the entry $(k, v', \updatemark)$ with the largest $v'$ such that $v'
\leq v$. 
If no such entry is found, or if $\updatemark = \deletemark$, then the query
should return $\nullmark$ to indicate that no entry for the given key at the
given version exists in the database.
Otherwise, the query returns the value~$\updatemark$.
Because all the versions of the data items are stored in the same index tree,
%the length of the history affects the number of pages that the search needs
%to process.
the cost of a single-key retrieval, write, or delete action is 
\OhT{\log_{\capacity} m} pages, where~$m$ is the total number of entries
stored in the VBT\@.
This holds regardless of the version that is queried.
Let us denote by $\entries{v}$\phantomsection\label{def:entries} the number
of data items that are alive at version~$v$. 
Note that $\entries{v}$ might be significantly smaller than $m$, if the
database history contains many deletions.
Early versions may also contain very few live data items.
However, because even large \Btree{}s tend to be low in height, this access
cost is usually acceptable.

The problematic operation in the VBT is the range query operation.
An efficient range query in a single-version \Btree\ index needs to process 
\OhT{\log_\capacity m + r/\capacity} \Btree\ index pages, where $m$~is the
number of entries stored in the index, and $r$~is the number of data items
in the queried range. 
This is because data item entries that are next to each other in the key
dimension are stored next to each other in the leaf pages and each leaf page
contains \OhT{\capacity} entries.
\Btree{}s can generally guarantee a minimum number of
entries per page, such as $\nicefrac{\capacity}{3}$, for example.
% Another variant of the B-tree, the \Blinktree, has links between sibling
% pages so that all consecutive leaf pages can be traversed through the links
% without backtracking through the index pages, so that only the
% \OhT{r/\capacity} leaf pages need to be accessed.
Most of the \Btree\ implementations have the leaf-page level siblings linked 
to enhance range queries, so the range can be scanned without backtracking
to index pages once the other end of the range is located.  
However, even backtracking the search through index pages to locate all the
leaf pages requires access to only \OhT{r/\capacity} pages, since the
number of leaf pages is asymptotically much higher than the number of index
pages required to index them (see proof below).

\thmskip
\begin{theorem}
\label{thm:btree-range-cost}
Locating the $r$ entries in a queried range $[k_1, k_2)$ in a \Btree\ index 
requires access to $\OhT{\log_\capacity m + r/\capacity}$ \Btree\ pages,
where $r$ is the number of entries in the queried range $[k_1, k_2)$, and $m$
is the number of entries stored in the index structure.
\end{theorem}
\begin{proof}
Because each index page in a \Btree\ has a fan-out of \OhT{\capacity}, and
all the root-to-leaf paths are of the same length, the height of the \Btree\
index is \OhT{\log_\capacity m}. 
This explains the logarithmic part $\log_\capacity m$ of the costs, as the
search tree must be traversed from the root to the correct leaf node to
locate key~$k_1$.
The $r$~entries returned by the query require \OhT{r/\capacity} pages
to store them, which gives the second part of the cost.
If the index has direct links between sibling pages (like the \Blinktree),
then the pages can be directly traversed, and the proof is complete.
If there are no sibling links, the leaf pages must be located by backtracking
through the index pages, resulting in additional page accesses.
The number of index pages required to locate the leaf pages is however
asymptotically smaller than the number of leaf pages indexed by the index
pages; with the possible exception of an extra root-to-leaf traversal of the
entire index.
This result is proven by, for example, Brown and
Tarjan~\cite{brown:1980:sorted-lists} in a more general form.
We formulate a short proof for this result here.
In each index page at level~$l+1$ (except for the leftmost
and rightmost index pages), the search will locate and use at least
\OhT{\capacity} pointers to pages at level~$l$.
Thus, to locate all $n = \OhT{r/\capacity}$ leaf pages, at most
\OhT{\log_\capacity n} pages at level two need to be processed.
Similarly, only \OhT{\log_\capacity \log_\capacity n} pages at level three
need to be processed.
In general, exponentially fewer pages are required at each higher level,
and the sum of all the required pages is bounded from the above by the
power series of \nicefrac{1}{2} times $n$, which in turn converges to
$2 n$ = \OhT{n} = \OhT{r/\capacity}.
% Therefore, for a search tree of height~$k$, the number of pages that need
% to be processed is at most:
% \begin{align*}
% n + \log_\capacity n + \log_\capacity \log_\capacity n + \cdots +
% \underbrace{\log_\capacity \cdots \log_\capacity}_{k-1} n &\leq \\
% n + \frac{n}{2} + \frac{n}{2^2} + \cdots + \frac{n}{2^{k-1}} &\leq 2n =
% \OhT{n}
% \end{align*}
% The inequality holds, because the base of the logarithm $\capacity$ is larger
% than two.
% Each term in the formula is therefore at most half of the previous term, and
% the entire formula is bounded from above by the power series of \num{1/2},
% which in turn converges to \num{2}.
% Therefore processing the index pages does not add any asymptotic overhead to
% the leaf-page traversal.
Furthermore, while the leftmost index pages at each level might contain
fewer than \OhT{\capacity} entries that are relevant to the query (and
similarly for the rightmost index pages), if we reserve two extra pages for
each level, these are accounted for. 
Because the initial traversal from the root page to the leftmost leaf page
has already added an asymptotic cost of one page for each level, these extra
pages reserved for each level do not add to the asymptotic cost of the range
query.
\end{proof}
\begin{corollary}
\label{cor:vbt-not-clustered}
In the VBT, the entries are not clustered next to each other by the key
values---the different versions of these entries are in the way.
The only guaranteed cost limit for a key-range query of the range $[k_1,
k_2)$ in a VBT index is of the form \OhT{\log_\capacity m + (n \times
\entries{k}) / \capacity}, where $m$ is the number of entries in the entire
index, $n$ is the number of versions in the database history, and
$\entries{k}$ is the maximum possible amount of discrete keys in the queried
range (for databases that store integer keys, $\entries{k} = k_2 - k_1$).
\end{corollary}
\thmskip

\corollaryref{cor:vbt-not-clustered} implies that in the worst case, there
are $n$ different versions of each data item in the range, and they must all
be scanned to find the relevant entries.
Note that this does not have anything to do with the size of the
result set of the query---none of the records in the range need to be alive
at the queried version, resulting in an empty answer set to the query.
Thus, the VBT is not sufficient for use as a multipurpose multiversion
database index structure.


%% Asymptotic Optimality
%%---------------------------------------------------------------------
\section{Asymptotic Optimality}
\label{sec:mv-index:optimality}

As we saw in the previous section, the \Btree\ index is not efficient when
used as a multiversion index structure.
To properly categorize the efficiency of index structures, let us now define
what the action costs of an optimal index structure are.
When designing a dynamic index structure where item search is based on
the comparison of key values, and the items are ordered, the minimum number
of key comparisons required for the search is \OhT{\log_2 m}, when the
structure contains $m$ entries.
This can be achieved, for example, with a binary search tree. 
Similarly, the minimum number of key comparisons for locating a range
of entries is \OhT{\log_2 m + r}, if we do not assume that the $r$ entries of
the queried range are stored in consecutive storage locations so that it would
suffice to locate only the storage locations of the range endpoints.

% \thmskip
% \begin{definition}
% \label{def:optimal-snapshot}
% A general-purpose snapshot index structure is \emph{asymptotically optimal},
% if all the action costs are logarithmically dependent in the number of
% indexed items.
% In more detail, all single-key user actions (item retrieval, item write, and
% item deletion) must have a cost of at most 
% $c_1 = \Oh{\log_{\capacity} m}$ index pages, where $m$ is the number of data
% items indexed by the index structure, and \capacity\ is the page capacity. 
% Additionally, the key-range-query action must have a cost of at most 
% $c_2 = \Oh{\log_{\capacity} m + r/\capacity}$ index pages, where $r$ is the
% number of data items returned by the range query.
% \end{definition}
% \thmskip

As we have shown in the previous section, the \Btree\ is an optimal
snapshot index structure that achieves these bounds, although in the case of
database tree structures, we are calculating the number of page accesses
instead of item value comparisons.
If we wish to consider the number of key value comparisons in a \Btree, note
that a single-key search in a \Btree\ index requires access to 
\OhT{\log_\capacity m} pages, as shown in \secref{sec:mv-index:btree}, where
\capacity\ is the page capacity and $m$ is the number of entries stored in
the index.
Because each page contains at most \capacity\ entries, at most \OhT{\log_2
\capacity} key value comparisons need to be performed for each page if the
entries of the page are ordered.
The total number of key comparisons is thus \OhT{\log_2 \capacity
\log_\capacity m}.
Once the page capacity~\capacity\ is fixed, the term $\log_2 \capacity$
becomes a constant, and can be omitted from the asymptotic analysis.

% Note that there are index structures that can be used to perform some of
% the operations more efficiently. 
%(such as hashing
% structures~\cite{kollios:2002:hashing}, for which the single-key query
% cost is only \Oh{1} pages). 
% However, in these structures some of the other actions (such as key-range
% queries) cannot be performed efficiently.
% Thus, this definition of optimality is justified, in our opinion,
% for a general-purpose database index structure.

Based on the discussion above, we formally define the requirement for
optimality in a multiversion database index structure:

\thmskip
\begin{definition}
\label{def:optimal-mv} 
A general-purpose multiversion database index structure is
\emph{optimal}, if the action costs are logarithmically
dependent in the number of data items alive at the queried version. 
More specifically, the corresponding action costs must be at most 
$c_1 = \Oh{\log_{\capacity} \entries{v}}$ pages of the index structure for
single-key actions and $c_2 = \Oh{\log_{\capacity} \entries{v} +
r/\capacity}$ pages of the index structure for the range-query action in the
worst case, where $\entries{v}$~is the number of data items that are alive at
the queried version~$v$.
\end{definition}
\thmskip

An optimal multiversion index structure must be as efficient as an optimal
single-version index structure that only indexes the data items alive at the
queried version.
The definition presented here is the same as the definition of Becker
et~al.~\cite{becker:1996:mvbt,becker:1993:optimal}, and stricter than the one
assumed by Varman and Verma~\cite{varman:1997:multiversion}, in which the
logarithms are taken from the total number of updates performed to the index
structure, instead of the total number of entries that are alive at the queried
version.
% The definitions converge when querying for the latest version, if the
% database history only contain new item insertions, because then the number of
% live items at a version~$v$ equals the number of updates performed before
% version~$v$.

In practice, the index structure must also be able to index multiple
data items that receive the same version, corresponding to multiple
data items inserted or deleted by the same transaction.
We naturally also require that the structural consistency of the index must
be preserved in the presence of multiple updating transactions.
The logical key-level consistency of the index, and of the set of data items
themselves, is assumed to be preserved by using an appropriate multiversion
concurrency-control algorithm, such as snapshot
isolation~\cite{berenson:1995:sql-critique} (see \secref{sec:mv-data:cc}).
These are practical requirements that are necessary for the index to be
useful in a modern multi-user database environment.


%% Common Multiversion Index Design
%%---------------------------------------------------------------------
\section{Common Multiversion Index Design}
\label{sec:mv-index:common}

It is apparent that a specifically tailored structure is required for
efficiently storing the history of data items.
A common design in newer multiversion index structures is that each database
page covers a region in key-version space, and that these regions at any
level of the index do not overlap.
The root page covers the entire key-version space, and each page lower
in the index structure covers a smaller region. 
These multiversion index structures are often directed acyclic
graphs, rather than trees, and child pages may have more than one parent
page.

An example of the difference between a single-version \Btree\ index and a
common multiversion index structure design is shown in
\figref{fig:sv-mv-index-comparison}.
The figure also shows the key ranges (or regions of key-version
space) covered by each page, and an example of the search tree of a
single version is shown for the multiversion index structure.
In this multiversion index structure, there is a unique \emph{search tree}
for each version of the database.
This concept is defined formally in \defref{def:search-tree} in
page~\pageref{def:search-tree}. 
The search trees of different versions of the database may share pages, as
shown by the leaf page~$p_5$ in the figure.
The page~$p_5$ is shared by all the search trees, as indicated by its life
span $[-\infty,\infty)$, and therefore it also has multiple parents so that
it can be reached from each search tree.

\begin{figure}[htb]
\begin{center}
\subfigure[A single-version \Btree\ index]{
\begin{tikzpicture}[x=4em,y=3.7em,label distance=.3em]
  \tiny

  \node[rnode,draw] (p1) at (0,2)
    {\centerrows{4.2em}{$p_1$\\$[-\infty,\infty)$}};
  \node[rnode,right=0 of p1] (p1text) 
    {\raggedrows{10em}{= Page identifier\\= Key range}};
  
  \node[rnode,draw] (p2) at (-1.5,1) 
    {\centerrows{4.2em}{$p_2$\\$[-\infty,0)$}};
  \node[rnode,draw] (p3) at (0,1) 
    {\centerrows{4.2em}{$p_3$\\$[0,100)$}};
  \node[rnode,draw] (p4) at (1.5,1) 
    {\centerrows{4.2em}{$p_4$\\$[100,\infty)$}};

  \node[rnode,draw] (p5) at (-1,0) 
    {\centerrows{4.2em}{$p_5$\\$[0,25)$}};
  \node[rnode,draw] (p6) at (0,0) 
    {\centerrows{4.2em}{$p_6$\\$[25,75)$}};
  \node[rnode,draw] (p7) at (1,0) 
    {\centerrows{4.2em}{$p_7$\\$[75,100)$}};

  % Connectors
  \foreach \x in {p2,p3,p4}
    \draw[-] (p1.south) -- (\x.north);
  \foreach \x in {p5,p6,p7}
    \draw[-] (p3.south) -- (\x.north);

  % Not shown pages
  \newdimen\btloy
  \newdimen\btlox

  \pgfextracty{\btloy}{\pgfpointanchor{p5}{north}}
  \pgfextractx{\btlox}{\pgfpointanchor{p2}{south}}
  \draw[-] (p2.south) -- ($(\btlox,\btloy) + (-.2,0)$);
  \draw[-] (p2.south) -- ($(\btlox,\btloy) + (-.5,0)$);
  \draw[-] (p2.south) -- ($(\btlox,\btloy) + (-.8,0)$);
  \node at ($(\btlox,\btloy) + (-.5,-3ex)$) {\ldots};

  \pgfextracty{\btloy}{\pgfpointanchor{p5}{north}}
  \pgfextractx{\btlox}{\pgfpointanchor{p4}{south}}
  \draw[-] (p4.south) -- ($(\btlox,\btloy) + (.2,0)$);
  \draw[-] (p4.south) -- ($(\btlox,\btloy) + (.5,0)$);
  \draw[-] (p4.south) -- ($(\btlox,\btloy) + (.8,0)$);
  \node at ($(\btlox,\btloy) + (.5,-3ex)$) {\ldots};

  % Key ranges and key-space cover
  \coordinate[right=1 of p4] (l2p1);
  \coordinate[label=below:{$0$}] (l2p2) at ($(l2p1) + (0.25,0)$);
  \coordinate[label=below:{$100$}] (l2p3) at ($(l2p2) + (1,0)$);
  \coordinate (l2p4) at ($(l2p3) + (0.25,0)$);

  \draw[-|] (l2p1) -- node[above] {$p_2$} (l2p2);
  \draw[|-|] (l2p2) -- node[above] {$p_3$} (l2p3);
  \draw[|-] (l2p3) -- node[above] {$p_4$} (l2p4);

  \coordinate[above=1 of l2p1] (l3p1);
  \coordinate (l3p2) at ($(l3p1) + (1.5,0)$);
  \draw[-] (l3p1) -- node[above] {$p_1$} node[below] {Keys} (l3p2);

  \coordinate[below=1 of l2p1] (l1p1);
  \coordinate[label=below:{$0$}] (l1p2) at ($(l1p1) + (0.25,0)$);
  \coordinate[label=below:{$25$}] (l1p3) at ($(l1p2) + (0.25,0)$);
  \coordinate[label=below:{$75$}] (l1p4) at ($(l1p3) + (0.5,0)$);
  \coordinate[label=below:{$100$}] (l1p5) at ($(l1p4) + (0.25,0)$);
  \coordinate (l1p6) at ($(l1p5) + (0.25,0)$);

  \draw[-|] (l1p1) -- (l1p2);
  \draw[|-|] (l1p2) -- node[above] {$p_5$} (l1p3);
  \draw[|-|] (l1p3) -- node[above] {$p_6$} (l1p4);
  \draw[|-|] (l1p4) -- node[above] {$p_7$} (l1p5);
  \draw[|-] (l1p5) -- (l1p6);
  
\end{tikzpicture}
}
\subfigure[A common design in multiversion indexes]{
\label{fig:sv-mv-index-comparison:mv}
\begin{tikzpicture}[x=4em,y=3.7em,label distance=0pt]
  \tiny

  \node[rnode,draw] (p1) at (0,2)
    {\centerrows{4.2em}{$p_1$\\$[-\infty,\infty)$\\$[-\infty,\infty)$}};
  \node[rnode,right=0 of p1] (p1text) 
    {\raggedrows{10em}{= Page identifier\\= Key range\\= Life span}};
  
  \node[rnode,draw] (p2) at (-1.5,1) 
    {\centerrows{4.2em}{$p_2$\\$[-\infty,\infty)$\\$[-\infty,10)$}};
  \node[rnode,draw] (p3) at (0,1) 
    {\centerrows{4.2em}{$p_3$\\$[-\infty,100)$\\$[10,\infty)$}};
  \node[rnode,draw] (p4) at (1.5,1) 
    {\centerrows{4.2em}{$p_4$\\$[100,\infty)$\\$[10,\infty)$}};

  \node[rnode,draw] (p5) at (-1,0) 
    {\centerrows{4.2em}{$p_5$\\$[-\infty,25)$\\$[-\infty,\infty)$}};
  \node[rnode,draw] (p6) at (0,0) 
    {\centerrows{4.2em}{$p_6$\\$[25,75)$\\$[10,\infty)$}};
  \node[rnode,draw] (p7) at (1,0) 
    {\centerrows{4.2em}{$p_7$\\$[75,100)$\\$[10,\infty)$}};

  % Connectors
  \foreach \x in {p2,p3,p4}
    \draw[-] (p1.south) -- (\x.north);
  \foreach \x in {p5,p6,p7}
    \draw[-] (p3.south) -- (\x.north);
  
  \draw (p2.south) -- (p5.north);

  % Not shown pages
  \newdimen\tmpx
  \newdimen\tmpy

  \pgfextracty{\tmpy}{\pgfpointanchor{p5}{north}}
  \pgfextractx{\tmpx}{\pgfpointanchor{p2}{south}}
  \draw[-] (p2.south) -- ($(\tmpx,\tmpy) + (-.2,0)$);
  \draw[-] (p2.south) -- ($(\tmpx,\tmpy) + (-.5,0)$);
  \draw[-] (p2.south) -- ($(\tmpx,\tmpy) + (-.8,0)$);
  \node at ($(\tmpx,\tmpy) + (-.5,-3ex)$) {\ldots};

  \pgfextracty{\tmpy}{\pgfpointanchor{p5}{north}}
  \pgfextractx{\tmpx}{\pgfpointanchor{p4}{south}}
  \draw[-] (p4.south) -- ($(\tmpx,\tmpy) + (.2,0)$);
  \draw[-] (p4.south) -- ($(\tmpx,\tmpy) + (.5,0)$);
  \draw[-] (p4.south) -- ($(\tmpx,\tmpy) + (.8,0)$);
  \node at ($(\tmpx,\tmpy) + (.5,-3ex)$) {\ldots};

  \pgfextracty{\tmpy}{\pgfpointanchor{p7}{south}}
  \pgfextractx{\tmpx}{\pgfpointanchor{p4}{east}}

  \path[draw,dashed,rounded corners] ($(p1.north west) + (-2ex,2ex)$) -- 
    ($(p1.north east) + (2ex,2ex)$) --
    ($(p3.north east) + (2ex,2ex)$) --
    ($(p4.north east) + (2ex,2ex)$) --
    ($(\tmpx,\tmpy) + (2ex,-2ex)$) -- 
      node[below,near start] {Search tree of version \num{10}} 
    ($(p5.south west) + (-2ex,-2ex)$) --
    ($(p5.north west) + (-2ex,2ex)$) --
    ($(p6.north west) + (-2ex,2ex)$) -- cycle;

  % Key ranges and key-space cover
  % Level 2
  \coordinate[right=1 of p4,label=left:{$10$}] (l2p1);
  \coordinate (l2p2) at ($(l2p1) + (1.25,0)$);
  \coordinate (l2p3) at ($(l2p2) + (0.25,0)$);
  \coordinate[label=above:{$100$}] (l2p2u) at ($(l2p2) + (0,.25)$);

  \draw (l2p1) rectangle node {$p_3$} 
    (l2p2u) rectangle node {$p_4$} 
    (l2p3) rectangle node {$p_2$}
    ($(l2p1) + (0,-.25)$); 

  % Level 3 (root page)
  \coordinate[above=1 of l2p1] (l3p1);
  \coordinate (l3p2) at ($(l3p1) + (1.5,0)$);
  \draw ($(l3p1) + (0,-0.3)$) rectangle
     node {$p_1$} 
     node[below=1.5em] {\tiny Keys}
     ($(l3p2) + (0,0.3)$);
  \node[rotate=90,left=0pt of l3p1,anchor=south] {\tiny Versions};

  % Level 1 (leaf pages)
  \coordinate[below=1 of l2p1,label=left:{$10$}] (l1p1);
  \coordinate (l1p2) at ($(l1p1) + (0.34,0)$);
  \coordinate[label=above:{$25$}] (l1p2u) at ($(l1p2) + (0,.25)$); 
  \coordinate (l1p3) at ($(l1p2) + (0.50,0)$);
  \coordinate[label=above:{$75$}] (l1p3u) at ($(l1p3) + (0,.25)$); 
  \coordinate (l1p4) at ($(l1p3) + (0.41,0)$);
  \coordinate[label=above:{$100$}] (l1p4u) at ($(l1p4) + (0,.25)$); 
  \coordinate (l1p5) at ($(l1p4) + (0.25,0)$);

  \draw ($(l1p1) + (0,.25)$) rectangle ($(l1p5) + (0,-.25)$);
  \draw ($(l1p1) + (0,-.25)$) rectangle node {$p_5$} 
    (l1p2u) rectangle node {$p_6$} 
    (l1p3) rectangle node {$p_7$} 
    (l1p4u);

%   \draw[-|] (l1p1) -- (l1p2);
%   \draw[|-|] (l1p2) -- node[above] {$p_5$} (l1p3);
%   \draw[|-|] (l1p3) -- node[above] {$p_6$} (l1p4);
%   \draw[|-|] (l1p4) -- node[above] {$p_7$} (l1p5);
%   \draw[|-] (l1p5) -- (l1p6);
%   
\end{tikzpicture}
}
\figcaption{Comparison of single-version and multiversion indexes}{}
\label{fig:sv-mv-index-comparison}
\end{center}
\end{figure}

Because the pages of a multiversion index structure cover regions in
key-version space instead of just key ranges, they can be split either by
keys or by versions.
These operations are generally called \emph{key split} and
\emph{version split}, respectively.
The main challenge in multiversion index structures is to design
these operations so that the data is distributed in such a way that
the operations remain efficient in the presence of varying histories
of user actions.

It is also important to merge pages to preserve data locality after key
deletions. 
It is true that entries are generally not physically removed from a
multiversion index, because the history information must be
preserved. 
However, once a page $p$ is version-split into pages $p'$ and $p''$ using a
version~$v$ as the separating version, so that the entries in $p'$
have creation times (\defref{def:lifespan}) that precede the creation times
in $p''$, the entries that have been logically deleted before~$v$ need not be
present in page~$p''$, because the items represented by those entries are not
alive at the versions following~$v$.
For an example, suppose that a page in a multiversion database holds an entry
created by a transaction with commit-time version $v_1$, and that this
entry is later deleted by a transaction with version $v_2$.
Page~$p$ is now version-split into a historical page~$p'$, and a current 
page~$p''$.\footnote{Depending on the index structure, the physical
page~$p$ will be reused as either~$p'$ or~$p''$.} 
Now, page~$p'$ must hold the information of the insertion at $v_1$ and
the deletion at $v_2$, but the new current page $p''$ does not need to
contain any traces of the key~$k$, because $k$ is not alive at any version $v
\geq v_3$ unless explicitly re-inserted into the database. 
Key deletions may thus physically remove entries from new copies of old
pages, and in this way the number of entries in the new pages may fall below
the acceptable minimum so that a page merge is required.

It may seem that the requirement for merging pages is critical only in the
asymptotic sense, and not in practice, as the pages can still be
version-split and reused for storing new data items.
The fact is, however, that unless pages are merged, the key ranges they cover
can only shrink, and never expand.
Because the search tree of the current version must cover the entire key
dimension, this means that the search tree of the current version cannot
shrink unless pages are merged.

Imagine now a warehouse inventory application for keeping track of the
goods stored in the warehouse.
The application creates a data item for each product stored in the warehouse.
Each item is assigned an increasing integer identifier and stored in a
multiversion database, indexed by the identifier. 
As the database fills, pages are key-split and thus the leaf pages cover
key-ranges that are close to the lower end of the integer key space.
When the products are taken out of the warehouse, the data items are
deleted.
New items have increasing identifiers and are thus inserted to the leaf
pages in the higher end of the key space.  
Eventually, all the pages with lower key ranges contain only deleted
entries, and all the live entries are clustered in the leaf pages with higher
key ranges.
Suppose that a reporting transaction performs a range query for the
entire key range of the database. 
The range query must scan through all the pages at the lower end of
the key range that contain only deleted entries before it reaches the pages
where the live entries are stored.
The performance of the reporting process will only get worse as the database
accumulates more historical entries, even if the database contains the
same number of live entries.
As shown here, it is important to guarantee that all pages that are part of
the search tree of a version~$v$ contain enough entries that are alive at
version~$v$ so that range queries that target version~$v$ are efficient.

Merging pages can lead to a tree height decrease in a single-version index
structure.
In multiversion structures, the height of the entire index cannot
decrease, because the heights of the version trees of historical versions
must remain as they are. 
The search trees of different versions can, however, be of varying heights.
The MVBT~\cite{becker:1996:mvbt,becker:1993:optimal}, for example, has a
separate structure called the \rootstar\ (see \defref{def:rootstar} below)
that stores the page identifiers of the root pages of different versions. 
%holds links to the root pages of different versions. 
This is illustrated in \figref{fig:mv-height-decrease}.
The \TSBtree~\cite{lomet:1989:tsb}, in contrast, has a single root
and thus all the search trees of different versions are of the same
height.

\thmskip
\begin{definition}
\label{def:rootstar}
A \rootstar\ is an auxiliary structure that can be used to efficiently
retrieve the page identifiers of the root pages of different versions. 
\end{definition}
\thmskip

\begin{figure}[htb]
\begin{center}
  \begin{tikzpicture}[x=3.5em,y=3.7em,% 
vlabel/.style={font=\scriptsize,left=0.2em,near start}]

%\fill[blue!50] (-1,-1) rectangle (5,3);

\node[page,dead] (p1) at (0,0) {};
\node[page] (p2) at (1,0) {};
\node[page,dead] (p3) at (2,0) {};
\node[page] (p4) at (3,0) {};
\node[page] (p5) at (4,0) {};
\node[page,dead] (p6) at (5,0) {};

\node[page,dead] (p7) at (0.8,1) {};
\node[page,dead] (p8) at (2.2,1) {};

\node[page] (p9) at (3.5,1) {};

\node[page,dead] (p10) at (1.5,2) {};

\node[db] (roots) at (4.5, 2.5) {\footnotesize$\rootstarp$};

\draw (p7.south) -- (p1.north);
\draw (p7.south) -- (p2.north);
\draw (p8.south) -- (p3.north);
\draw (p8.south) -- (p4.north);
\draw (p10.south) -- (p7.north);
\draw (p10.south) -- (p8.north);

\draw (p9.south) -- (p2.north);
\draw (p9.south) -- (p4.north);
\draw (p9.south) -- (p5.north);

 
\draw[->,densely dotted] (roots) to[bend left=10] (p6.north);
\draw[->] (roots) to[bend right=10] 
           node[vlabel] {\parbox{4em}{\center Selected version}}
           (p9.north); 
\draw[->,densely dotted] (roots) to[bend right=10] (p10.east);

\end{tikzpicture}

  \figcaption{Search trees of different heights}
  {Search trees of different versions in a multiversion index can
  have different heights.}
  \label{fig:mv-height-decrease}
\end{center}
\end{figure}




%% Early Multiversion Index Structures
%%---------------------------------------------------------------------
\section{Early Multiversion Index Structures}
\label{sec:mv-index:early}

The idea of storing historical versions in the database is not new.
Overmars discussed general methods for making data structures persistent 
in 1981~\cite{overmars:1981:search-past-i,overmars:1981:search-past-ii}, and
the classification of the time concepts ranges back to the 
mid-1980s~\cite{lum:1984:designing,snodgrass:1985:taxonomy,snodgrass:1986:temporal}.
The earliest specifications used the term \emph{rollback database} when
talking of multiversion databases (that is, transaction-time databases);
\emph{historical database} when discussing valid-time databases; and
\emph{temporal databases} when talking of bitemporal databases (see
\secref{sec:mv-data:time} for current definitions of the time concepts).
This section reviews some of the early index structures and the design ideas
behind them.

% Reverse chaining
One of the earliest approaches to data persistence is \emph{reverse
chaining}\phantomsection\label{def:reverse-chaining} that is used to chain
the history of the data entries together. 
Lum et~al.~\cite{lum:1984:designing} have described an index structure that
uses a current version tree to index the current version, and a historical
tree that is used to index historical versions. 
Both trees contain pointers to a reverse-chained linked list of entry
values, ordered by the update times, so that the latest version is at the
start of the list.
While both of these index structures may be efficient (optimal) B-trees, the
number of updates on a data item directly affects the length of the
version-history chain, and thus the query actions for previous versions on
this index structure can have very high costs, especially for key-range
queries.

% Path copying
Another technique used in early multiversion index structures is \emph{path
copying}\phantomsection\label{def:path-copy}~\cite{sarnak:1986:persistent,soisalon-soininen:1999:full-text}.
This method achieves persistence by creating copies of changed nodes, so that
the old nodes are left untouched. 
Because the new nodes need to be attached to parent nodes, and the
original parent nodes cannot be changed, the entire path from
the changed node up to the root node needs to be copied.
This method therefore effectively creates a new path that can be used for
current-version queries and retains the old path for historical queries. 
Sarnak and Tarjan~\cite{sarnak:1986:persistent} describe how to make
red-black trees~\cite{bayer:1972:red-black,guibas:1978:red-black} persistent
by creating a new copy of the path along which a change has occurred.
Search trees of different versions share common subtrees, and differ only
on the single copied path.
As the entire path is copied, a new root node is created for each update.
Soisalon-Soininen and Widmayer also use path copying with \abbr{AVL}-trees to
make the tree structure recoverable~\cite{soisalon-soininen:1999:full-text}.

Because there are multiple roots, all the roots of different versions must be
stored in some structure. 
Sarnak and Tarjan~\cite{sarnak:1986:persistent} use an array of page
identifiers, ordered by the creation time of the corresponding pages. 
This array is essentially a \rootstar\ structure of 
\defref{def:rootstar}.
While this method is efficient for binary search trees, with a logarithmic
cost of \OhT{\log_2 \entries{v}} pages for both updates and query operations,
where \entries{v} denotes the number of entries that are alive at
version~$v$, the method is not feasible for disk storage, because only a
single entry is stored per node and each update requires \OhT{\log_2
\entries{v}} space for the new copied path.
Sarnak and Tarjan further enhance the space consumption of their method by
allowing nodes to grow fat.
% (see \defref{def:fat-node} below).
A \emph{fat node} is a node that can contain an arbitrary number of
entries; in this case, an arbitrary number of left and right pointers
in the binary search tree, corresponding to paths in different versions. 
The fat node can be implemented, for example, by chaining together a
list of database pages.
Path copying for each update can thus be avoided.
However, the enhanced structure still stores only a single data-item entry in
each node, so the approach remains unsuitable as a disk-based access method.

% \thmskip
% \begin{definition}
% \label{def:fat-node}
% A \emph{fat node} is a node that can contain an arbitrary number of entries.
% The fat node can be implemented, for example, by chaining together a
% list of database pages.
% Traversing all the entries of a fat node implemented as a list of database
% pages costs $\OhT{n/\capacity}$ \abbr{I/O} operations, where $n$ is the
% number of entries stored in the fat node, and $\capacity$ is the page
% capacity.
% \end{definition}
% \thmskip

% Time sequences
Shoshani and Kawagoe~\cite{shoshani:1986:temporal} have presented a
more general framework for indexing data with different types of \emph{time
sequences}. 
A time sequence is the collection of changes associated with a data item.
In our multiversion data item model, a data item has a life span during which
it is alive, and any change to the data item creates a new data item (see
\secref{sec:mv-data:mv-data}).
In contrast, a time sequence records all the updates that target a single
key. 
A data item in Shoshani's and Kawagoe's framework is represented by
a tuple $(k, S)$, where $S = \{(v_i, w_i) : i \in \{0, 1, \ldots, n-1\} \}$
is the time sequence of the data item.
An individual tuple $(v_i, w_i)$ of the time sequence denotes that the
data item with the key~$k$ was assigned the value~$w_i$ at version~$v_i$. 

Shoshani's and Kawagoe's framework furthermore identifies time sequences with
different update patterns. 
Item costs in a grocery store database, for example, have continuous,
irregularly and step-wisely changing values, while the number of items sold
is a discrete value that is measured at regular intervals.
The dynamic index structures that Shoshani and Kawagoe propose assigns
cells or database pages to data keys (surrogates). 
As the pages fill up, new pages are linked to form a chain of pages.
These pages are further indexed into an ordered list of pointers so that
the entire structure is doubly indexed---first by the key attribute to locate
the secondary index and then by the version to locate the correct page.
Finally, the data page itself must be searched for the correct version.
Single-key operations in these index structures have a logarithmic cost, but
the constant overhead is high as there are multiple indexes that have to be
traversed.
Furthermore, the data items are not clustered on the key attribute, and
key-range queries are thus inefficient.

% WOBT
Easton's \emph{write-once balanced tree}
(\abbr{WOBT},~\cite{easton:1986:wobt}\phantomsection\label{def:wobt}) is
a multiversion index structure that stores multiple versions of data
items on indelible storage. 
The structure can therefore be used with
\abbr{WORM}\phantomsection\label{def:worm} (write once, read many) media.
This index structure is based on the \Btree, but rather than overwriting old
data, new versions are written next to the old ones.
When pages fill up with different versions, a new copy is created and
possibly split into two separate pages. 
The old one, however, remains as it was.
Root pages of the WOBT are forward-chained, so that the most recent root page
can be located by starting a search from the first root (first page on the
database index) when the structure is loaded from the disk during
database startup. 
The latest root, and possibly the other roots as well, are then cached in main
memory for fast access. 
The WOBT is not very space-efficient, because no data can be overwritten. 
It is, however, an important structure, because the more recent \TSBtree\
(described in \secref{sec:tsbmvbt:tsb}) is based on it, and the \TSBtree\ in
turn is used in the multiversion database engine that Microsoft is
developing on top of the SQL Server (see
\secref{sec:tsbmvbt:immortaldb}).

% Fully persistent structures
Driscoll et~al.\ have discussed a more general way of making main-memory data
structures either partially or fully
persistent~\cite{driscoll:1989:persistent}\phantomsection\label{def:driscoll-persistent}.
Their solution to partial persistence is based on the ideas presented by
Overmars~\cite{overmars:1981:search-past-i,overmars:1981:search-past-ii}.
They suggest using either fat nodes 
% (see \defref{def:fat-node}) 
or node copying to make binary search trees persistent.
In the \emph{fat node} method, the binary tree nodes can grow arbitrarily
large.
In the \emph{node copying} method, the nodes may contain a fixed number of
left and right pointers to child nodes.
The pointers have a version attached to them that is used to select the
correct pointer to traverse.
When a node fills, a new copy of it is created.
Only the most recent pointers are copied to the new node, and a pointer to the
new node is attached to the parent.
The old node is thus left in place and can be used for historical queries.
While Driscoll et~al.\ designed the algorithms for in-memory structures, the
node copying approach has been adapted for use with disk-based temporal index
structures.
The more recent index structures, such as the TBS-tree, the MVBT, and the
MVAS (see \chapref{chapter:tsbmvbt}), all employ page-copying methods such as
these to organize the database pages.
For discussion on the fully persistent index structures, refer to the
article by Driscoll et~al.~\cite{driscoll:1989:persistent}.

% Time index and time index+
The \emph{time index} is yet another early multiversion index structure
that was proposed by Elmasri et
al.~\cite{elmasri:1990:time-index,elmasri:1991:time-index}. 
The authors had noted that the other index structures chained the versions
of data items separately, and thus did not cluster the data items of a
given version next to each other, and the time index was designed to correct
this.
Elmasri et~al.\ talk about valid
time~\cite{elmasri:1990:time-index,elmasri:1991:time-index}, but they make the
assumption that changes occur in an increasing time order, and that changes to
previous times do not happen, so the time index is more properly classified as
a transaction-time index structure (recall the definition from
\secpageref{def:tx-time}). 

The time index is based on the \Btree, but it is organized by the data-item
versions (transaction-time instants), instead of data-item keys. 
Each leaf page holds a range of database versions.
The leaf-page entries are tuples of the form $(v_i, b_i)$, where $v_i$ is the
version of the entry and $b_i$ a pointer to a bucket (i.e., database page or
a collection of pages) containing information about the entries of that
version. 
To save space, only the bucket identified by $b_0$ contains a full snapshot
of entries; the rest of the buckets (identified by $b_i$, $i \neq 0$) store
updates.
Elmasri et~al.\ enhance this index by various techniques
(such as separating incremental and decremental
buckets~\cite{elmasri:1991:time-index}), and Kouramajian et~al.\
further develop the structure into the \emph{time
index$^+$}~\cite{kouramajian:1994:time-index}.
The enhanced structure uses a compression technique that shares the live
entries of two sibling leaf pages in a shared bucket.
This technique can also be used by a larger set of leaf pages, so that
entries shared by all of the leaf pages are placed in a shared bucket.
The compression technique greatly enhances the space usage, but the bucket
updating is costly, and the situation at any given version needs to be
reconstructed based on multiple sources.

% Fully persistent B-tree
Lanka and Mays have applied the persistence methods of Driscoll et~al.\
discussed above to make a fully persistent
\Btree~\cite{lanka:1991:persistent}. 
They propose various different schemes for converting a standard \Btree\ into
a persistent one; namely, the \emph{fat node} method, the \emph{fat field}
method, and the \emph{pure version} method.
All the methods are built on top of a standard, ephemeral \Btree.
A \emph{fat node} in the persistent \Btree\ of Lanka and Mays is the
extension of a \Btree\ node.
It is a logical collection of \Btree\ nodes, with an added version block that
is used to locate the \Btree\ node corresponding to a given version.
Different versions may point to the same \Btree\ node, if the contents
of the subtree rooted at that node are identical for both versions.
The version block is attached to the beginning of the fat node.
In the fat node method, the logical index thus contains alternating index
blocks and version blocks, so that a search traversing from a root page first
selects the correct pointer based on the searched key value, then selects
the correct pointer based on the version attribute, and so on.
The problem with the fat node method is space usage---each update creates
a new disk block.
In the \emph{fat field} method, some of the versioning information is attached
to the \Btree\ pages, but the version blocks are still used in some situations.
The fat field method is in fact quite close to the structure of the more
recent indexes. 
Because the tree contents and the pointers in index pages change between
versions, the root of the index also changes between different versions.
The fully persistent \Btree\ thus uses a \rootstar\ structure to store
pointers to the roots of different versions.
The \emph{pure version} method, on the other hand, is a straightforward
extension of a \Btree, where the index pages are left as they are, and the
versioning information is attached to the entries at leaf pages.
This approach is more space efficient, but it does not cluster different keys
of a single version close to each other, and range queries are therefore not
efficient.

The fully persistent \Btree\ was designed to be fully persistent,
so that new versions can be based on any previous version
(see \defref{def:fully-persistent}). 
This is achieved by maintaining an auxiliary structure that describes the
version history. 
The history is modelled as a directed acyclic graph, so that different
version branches can be merged.
In effect, the auxiliary structure is used to fetch the ancestor set~$A_v$
for each version~$v$.
For convenience, $A_v$ is defined to include the version $v$.
When a query for a version~$v$ is performed on the fully persistent \Btree,
only versions that are present in~$A_v$ are considered.
For partial persistence, we can omit the auxiliary structure and
simply consider all versions.

% Snapshot index
Tsotras and Kangelaris have discussed I/O optimality and propose an I/O
optimal (according to their definition) index structure called the
\emph{snapshot
index}~\cite{tsotras:1995:snapshot-index}\phantomsection\label{def:snapshot-index}.
The snapshot index stores new data items sequentially in a doubly-linked list
of pages. 
The underlying structure is therefore basically a log file of item
insertions.
At all times, one of the pages is an \emph{accessor} page that receives
all new records that represent item insertions or updates.
Access to current entries is made efficient by keeping only pages that are
\emph{useful} on the list.
Useful pages have at least a minimum amount of entries that are alive.
Pages that are no longer useful are moved away from the main item
list but they are kept linked to the pages in the list.
The index also maintains an auxiliary dynamic hash structure for locating the
most recent update to any given data item, hashed by the item key.
The hashing function is used to make updates efficient: any item that is to
be deleted can be located via the hashing function in expected constant
time, and any new item to be inserted can be placed at the single accessor
page in constant time.
In practice, the snapshot index has an expected constant update
cost. 
Queries, however, need to search through all the entries that are alive at
the queried version, because the entries are not ordered on the key
attribute. 
The snapshot index is therefore efficient for constructing the entire
(unordered) set of data items belonging to any given version, but not
for querying a range of keys or a single key of a given version.
More formally, the snapshot index is I/O optimal for \qtype{$*$/$-$/point}
queries, but not for \qtype{range/$-$/point} or \qtype{point/$-$/point}
queries.



%% Multidimensional Index Structures
%%---------------------------------------------------------------------
\section{Multidimensional Index Structures}
\label{sec:mv-index:spatial}

Multidimensional index structures have also been used for indexing
multiversion data.
Recall from \secref{sec:mv-data:mv-data} that multiversion data items are 
tuples of the form $(k, \vrmark{v}, w)$, where $k$ is the data key and 
\vrmark{v} is life span of the data item; that is, the range of versions
(transaction-time instants) during which the item is alive.
The key and version dimensions are orthogonal, and the data item may thus be
uniquely identified by a two-dimensional line segment in key-version
space that is parallel to the version axis (see \figref{fig:md-mv-items}).
When considered in this way, the data items can then be indexed by a
multidimensional index structure.
In this case, the pages of the multidimensional index structure cover regions
of key-version space. 
These regions are generally known as \emph{minimum bounding regions}, or
MBRs\phantomsection\label{def:mbr}.
\figref{fig:md-mv-items:mbr} shows one possible way to cover a set of
multiversion data items with suitable MBRs.

\begin{figure}
\centering
\subfigure[Data items]{\begin{tikzpicture}[x=1.8em,y=1.5em]
  \coordinate (o) at (0,0);
  \coordinate (ox) at (0.3,0.3);
  \coordinate (ok) at ($(ox) + (0,4.2)$);
  \coordinate (ov) at ($(ox) + (7,0)$);

  \coordinate (a1) at ($(o) + (1,2)$);
  \coordinate (b1) at ($(o) + (2,3)$);
  \coordinate (c1) at ($(o) + (3,1)$);
  \coordinate (d1) at ($(o) + (4,4)$);

  \coordinate (a2) at ($(a1) + (4,0)$);
  \coordinate (b2) at ($(b1) + (2,0)$);
  \coordinate (c2) at ($(c1) + (4,0)$);
  \coordinate (d2) at ($(d1) + (3,0)$);

  \coordinate (b3) at ($(b2) + (3,0)$);
  
%  \foreach \point in {a1,b1,c1,d1,b2,c2}
%    \fill[black] (\point) circle (2pt);

  \draw[->] (ox) -- node[lbl,above] {Keys} (ok);
  \draw[->] (ox) -- node[lbl,below] {Versions} (ov);

  \draw[|-|] (a1) -- node[lbl,above] {$(2,\alpha)$} (a2);
  \draw[|-|] (b1) -- node[lbl,above] {$(3,\beta)$} (b2);
  \draw[->] (b2) -- node[lbl,above] {$(3,\beta')$} (b3); 
  \draw[|->] (c1) -- node[lbl,above] {$(1,\gamma)$} (c2);
  \draw[|->] (d1) -- node[lbl,above] {$(4,\delta)$} (d2);
\end{tikzpicture}
}
\subfigure[Page MBRs]{\begin{tikzpicture}[x=1.8em,y=1.5em]
  \coordinate (o) at (0,0);
  \coordinate (ox) at (0.3,0.3);
  \coordinate (ok) at ($(ox) + (0,4.2)$);
  \coordinate (ov) at ($(ox) + (7,0)$);

  \coordinate (a1) at ($(o) + (1,2)$);
  \coordinate (b1) at ($(o) + (2,3)$);
  \coordinate (c1) at ($(o) + (3,1)$);
  \coordinate (d1) at ($(o) + (4,4)$);

  \coordinate (a2) at ($(a1) + (4,0)$);
  \coordinate (b2) at ($(b1) + (2,0)$);
  \coordinate (c2) at ($(c1) + (4,0)$);
  \coordinate (d2) at ($(d1) + (3,0)$);

  \coordinate (b3) at ($(b2) + (3,0)$);
  
%  \foreach \point in {a1,b1,c1,d1,b2,c2}
%    \fill[black] (\point) circle (2pt);

  \draw[->] (ox) -- node[lbl,above] {Keys} (ok);
  \draw[->] (ox) -- node[lbl,below] {Versions} (ov);

  \draw[|-|] (a1) -- (a2);
  \draw[|-|] (b1) -- (b2);
  \draw[->] (b2) -- (b3); 
  \draw[|->] (c1) -- (c2);
  \draw[|->] (d1) -- (d2);

  \draw[dashed] ($(a1) + (-0.2,0.3)$) rectangle ($(c2) + (0.2,-0.3)$);
  \draw[dashed] ($(b1) + (-0.2,-0.3)$) rectangle ($(d2) + (0.2,0.3)$);

\end{tikzpicture}
\label{fig:md-mv-items:mbr}}
\figcaption{Multiversion data indexed in a multidimensional index}%
{%
%The data items can be considered as two-dimensional
%line-segments in key-version space.
Reading from the left, first items with keys~$2$, $3$ and~$1$ were
consecutively inserted into the database at different versions.
After that, data item with key~$3$ was updated and $4$ was inserted at the
same version. 
Finally, the item with key~$2$ was deleted.}
\label{fig:md-mv-items}
\end{figure}

Perhaps the most widely used multidimensional index structure is the \Rtree\ 
of Guttman~\cite{guttman:1984:rtree} and its variants, such
as the \Rstartree\ of Beckmann et~al.~\cite{beckmann:1990:rtree}.
\Rtree{}s do not, in general, guarantee logarithmic access times in all
situations. 
Even an exact-match query in the standard \Rtree\ may require traversing
multiple paths, because the key-version regions of sibling pages may overlap.
However, when discussing multiversion data, it is worth noticing that
the key-version ranges of data items stored in the leaf pages cannot
overlap.
This means that the overlap in index pages may be reduced when storing
multiversion data.
On the other hand, standard multidimensional index structures are designed
to index data items with static spatial dimensions.
Multiversion data items have a life span \vrmark{v} that is initially an
infinite range $[v_1,\infty)$ starting from $v_1$, as the item is not yet
deleted.
When the multiversion data item is deleted, its life span is cropped to a
finite range $[v_1, v_2)$.
The life span is thus not static, and the data items are not perfectly
suited for multidimensional indexing.
The initially infinite ranges of the data items also cause problems for 
indexing.

Once a multiversion data item has been deleted, it can no longer be
modified, and its life span and key-version range become static.
The \abbr{PostgreSQL} database system, initially introduced in an article by
Michael Stonebraker~\cite{stonebraker:1987:postgres}, takes advantage of this
fact. 
Initial versions of the PostgreSQL database clustered the data item
entries in a standard snapshot index (i.e., a \Btree) based on the key
attribute. 
Different versions of the data item were linked in a chain of \emph{delta
records} that described the item updates.
A \emph{vacuum cleaner} process was run periodically to move the
earlier versions of data items into an \Rtree\ index.
At this point, the moved data items were static, and could be
properly indexed in an \Rtree.
The \Rtree\ was thus used as a storage for historical entries.
Kolovson and Stonebraker~\cite{kolovson:1989:historical} have also designed
different variants of this design that utilize magnetic disks or combinations
of magnetic and optical disks.
Current versions of the PostgreSQL~\cite{postgresql:2009:postgresql}
include \abbr{GiST} indexes (generalized search
trees~\cite{hellerstein:1995:gist}) that can be used to implement \Rtree{}s.

There are still problems even when storing only static multiversion data
items (i.e., multiversion data items that have already been deleted) in an
\Rtree, because the data items have highly varying lengths.
A single multiversion data item with a long life span causes the entire leaf
page where it resides in to have a wide bounding region.
A wide leaf page causes the entire path from the leaf page up to the root
page to have wide bounding regions, because the MBRs of the parent
pages must cover the MBRs of their child pages.
This in turn leads to high overlap between pages, which causes search
performance to degrade.
Kolovson and Stonebraker have designed the \emph{segment \Rtree}, or
\abbr{SR-tree} to alleviate these issues~\cite{kolovson:1991:segment-index}.
The \abbr{SR-tree} allows data items with long life spans to be stored higher
in the tree, thus reducing the overlap on lower levels of the tree.
Because the pages on the higher levels should also have space for enough
index entries (i.e., pointers to child pages), the \abbr{SR-tree} allows
pages that are higher on the tree to be larger.
The index structure can be further enhanced if some assumption of the
distribution of the data items can be made.
Kolovson and Stonebraker also introduce \emph{skeleton \abbr{SR-trees}} that
make this initial assumption and organize the structure of the tree based on
it. 
The index is dynamic, however, so it will adapt to the actual distribution of
the data items.
According to their tests, the skeleton \abbr{SR-tree} outperforms both the
\abbr{SR-tree} and the \Rtree\ when used as a multiversion index.

Regardless of the enhanced performance of the skeleton \abbr{SR-tree}, the
fact remains that \Rtree{}s and other methods based on them do not have
logarithmic cost guarantees, and cannot therefore be considered 
optimal for indexing multiversion data.  
Another problematic issue with \Rtree{}s is concurrency control and recovery.
The MBRs of the \Rtree\ pages need to be consistent with the MBRs
of the child pages, so they need to be updated whenever entries are inserted or deleted. 
In standard \Rtree{}s, insertion causes MBR enlarging on the path from root
to a leaf page.
This can normally be processed during the search for a proper page to
accommodate the new entry.
Item deletion, however, causes MBRs to shrink, and the shrinking must be done
bottom-up.
This is challenging for concurrency control, because large parts of the tree
need to be kept locked during the deletion operation to ensure correct
operation and to avoid deadlocks.


%% Hashing Structures
%%---------------------------------------------------------------------
\section{Hashing Structures}
\label{sec:mv-index:hashing}

Hashing is generally used for very efficient, constant-time membership
queries, when a suitable hashing function for the indexed values exists.
A hashing function $h : X \to \mathbb{I}$ maps all
values~$x$ in the source domain~$X$ into a finite set of integers.
The size of the target set is normally much smaller than the size of
the source set.
Some source values are therefore necessarily hashed into the same integer
value in the target set, thus causing collisions.
Each hashed value~$h(x)$ identifies a
\emph{bucket}\phantomsection\label{def:bucket} that conceptually holds all
the objects that have the same hash value.
The range of hashed integers is thus $0, 1, \ldots, n-1$, where $n$ is the
number of buckets allocated for the index.
Each bucket can be implemented, for example, as a linked list of database
pages.
% The hashed integer value $h(x)$ can then be used as an array index, or
% as page identifier in database systems.
% For the hashing method to work, all the colliding values must however be
% stored in the same location that is identified by the hash value $h(x)$.
% One approach is to chain together all the objects with the same hash value. 
% The storage location identified by the hashed value is generally called
% a \emph{bucket}, as it 

With traditional hashing methods, the hashing function is static, and the
size of the target set ($n$) is therefore predefined.
If too small a target set was initially selected, hashing becomes inefficient
because collisions occur more and more frequently, and the object chains
become longer.
In these situations, a complete reorganization (a \emph{re-hashing}) of the
hash table is required.
Litwin has presented a linear hashing scheme that dynamically changes the
hashing function so that the size of the target set can
grow~\cite{litwin:1980:linear}.
Periodically, a single bucket~$b_i$ is \emph{split} into two buckets $b_i$
and $b_j$ by changing the hashing function so that instead of mapping the
entries $x \in b_i$ into~$b_i$, some of them are mapped to~$b_i$ and some
into~$b_j$. 
This can be achieved by organizing the bucket splits correctly.
In practice, if a hashing function~$h_i$ used at round~$i$ maps objects into
a range $0, 1, \ldots, n_i-1$, then the hashing function~$h_{i+1}$ used for
the next round $i+1$ maps objects into the range $0, 1, \ldots,
2n_i-1$, thus doubling the range size $n_{i+1} = 2n_i$. 
For all the objects~$x$ either $h_{i+1}(x) = h_i(x)$ or $h_{i+1}(x) = h_i{x}
+ n$. 
This internal reorganizing happens one bucket at a time, so there is no
extensive one-time reorganization. 
At most two different hashing functions need to be used to access any object,
and so the expected performance of the hashing remains the same, even if the
set of stored objects grows.

Kollios and Tsotras have expanded the linear hashing method for hashing
multiversion data items~\cite{kollios:2002:hashing}.
In their partially persistent linear hashing method, each data bucket
stores multiversion data items similar to our definition (see
\defref{def:mv-item}). 
% Each entry records a life span during which the data item is alive. 
When a bucket split occurs, the hashing function changes and some entries
need to be moved to a new bucket.
In this situation, the entries are moved \emph{logically}; that is, 
a live multiversion item $(k, [v_1, \infty), w)$ that is moved during a split
at version~$v_2$ from bucket~$b_i$ to bucket~$b_j$ remains in~$b_i$, but the
entry representing it is changed to the historical entry $(k, [v_1,
v_2), w)$, and a new live entry $(k, [v_2, \infty), w)$ is inserted into the
new bucket~$b_2$.
The historical query operation then reduces to finding the correct hashing
function used at the historical version, and then finding the historical
entry from the correct page.
For the first part of the query, Kollios and Tsotras maintain an array that
stores information about the hashing functions used for different versions.
Each array slot represents a version during which the hashing function has
changed, and locating the correct function thus requires a binary search on
the array. 
This search adds a logarithmic cost on the historical query, which is
otherwise an expected amortized constant-time operation.
Kollios and Tsotras note, however, that the array should be small enough to
remain in main memory and thus the effect of this search on the overall query
performance should be negligible.

Hashing functions are efficient for solving the \qtype{point/$-$/point}
query, with (almost) an expected \OhT{1} constant query and update costs.
The trade-off is that other query types cannot be processed efficiently.
Key-range queries are especially inefficient, because the keys are not
clustered near each other.
In fact, finding the next key in a hashing structure requires either
traversing through the entire structure or, even worse, trying out all
possible next keys.
Hashing structures are therefore not suitable as a general multiversion index
structure.


%% Version-Control Systems
%%---------------------------------------------------------------------
\section{Version-Control Systems}
\label{sec:mv-index:vcs}

Software engineers use version-control systems (\abbr{VCS}) to share the
software code between the developers and to maintain the history of all the
previous versions of the software.
A developer \emph{checks out} the code from the central version-control
system, creates his modifications, and finally \emph{commits} them to the
\abbr{VCS}, thus creating a new version.
In this sense, version-control systems are in fact transaction-time
databases.
Most version-control systems allow software engineers to create branches on
the version history, so that different teams can work with different features
without disturbing each other.
Version-control systems are therefore fully persistent.
Examples of version-control systems are the concurrent versions system
(\abbr{CVS})~\cite{cederqvist:2008:cvs},
Subversion~\cite{collins-sussman:2008:version}, and
Git~\cite{git:2010:user-manual}.

The main difference between version-control systems and traditional databases
is in their usage. 
Multiversion database indexes are used to locate data item tuples based on a
single key or a key range, by either fixing the version to a single point or
by specifying a range of versions (recall the different query types from
\secref{sec:mv-data:queries}).
Version-control systems, on the other hand, are used for much more specific
queries, such as ``find all the differences between versions $v_1$ and
$v_2$'' (with $v_1$ and $v_2$ possibly on totally different branches), 
``find all the changes to the data item identified by $k$'', or ``find the
version during which the line $l$ was changed in the text file identified
by $k$''.
These are queries that could not be effectively answered using a traditional
database index.
% Because the version-control systems are optimized for answering these kinds of
% queries, it would be reasonable to assume that they are not optimal
% for answering the queries defined for traditional databases.
A multiversion database system could, however, conceivably use an
existing version-control system for indexing the multiversion data items, and
it is therefore worth it to examine these systems a little further to determine
whether they would be efficient for this purpose.

Fendt has evaluated the performance of some of the more recent
version-control systems in an online article at the Linux
Foundation~\cite{fendt:2009:dvcs}.
He shows that the Git version-control system is the fastest one, alongside
with another \abbr{VCS} called Mercurial. 
We will therefore concentrate on Git. 

The Git version-control system is based on indexing objects by the
\abbr{SHA-1}%
\footnote{\abbr{SHA-1}, the Secure Hash Algorithm, is a cryptographic
hash function that computes a \num{160}-bit hash value (called \emph{digest}) from
its input.} 
hash values of their contents~\cite{virtanen:2009:git,git:2010:user-manual}.
The objects are stored in the file system, named with the hexadecimal
representation of the \abbr{SHA-1} hash value.
The file system therefore works as the primary index for the stored objects.
Four types of objects can be stored in the Git database: blobs, trees,
commits, and tags.
The \emph{blob} type is used to represent any object that is stored in the Git
\abbr{VCS}, and so every file that is added to Git is stored in a blob
object. 
\emph{Tree} objects are used to maintain directory hierarchies, 
\emph{commit} tags represent commits (i.e., database versions), and
\emph{tags} are used to assign cryptographic tags for other objects (e.g.,
for verifying their contents).

Each tree object logically represents a single node and all its children in
a directed acyclic graph.
The tree object is basically a collection of pointers to objects stored under
that node.
A pointer that points to an object~$o$ is stored in a tuple of the form $(m,
t, s, n)$, where $m$ is the file system mode of the file that represents~$o$,
$t$ is the type, $s$ is the \abbr{SHA-1} digest of the object~$o$, and $n$ is
the name of the object.
Subdirectories are maintained by storing pointers to other tree nodes.
To proceed from a tree node $x$ to its child node, we find the correct child
pointer $(m, t, s, n)$, and use the \abbr{SHA-1} hash value $s$ to retrieve
the child object from the index (i.e., by loading the file named $s$ from the
file system).
Different versions are created by possibly creating new tree objects for that
version. 
Any child object (e.g, a text file) that has changed has an updated
\abbr{SHA-1} hash value, because the hash is based on the contents of the
stored objects.
This in turn changes the tree object, because the tree object contains the
\abbr{SHA-1} values of its children in the pointer records.
The changed tree object again needs to be indexed with a different
\abbr{SHA-1} value, and thus a change propagates all the way to the root of
the directory structure.
Unchanged nodes can however be efficiently reused between different versions.
The Git \abbr{VCS} therefore relies on a form of \emph{path copying} (see
\secpageref{def:path-copy}).

A straightforward way of using the Git version-control system as a
multiversion index structure would be to store each key in a single tree
object. 
A binary search could then be used to quickly locate the searched key among
the pointers stored in the tree object. 
The single tree object would then work as a dense index for the stored
objects.
While querying would be efficient, note that updating any key necessitates
duplicating the entire tree object, thus duplicating the entire indexing
portion of the multiversion database.
The update performance and the space usage of this structure would be
suboptimal.
Also note that this kind of a structure could not be used as a sparse
(primary) index structure, because all the child objects (data items stored
in the database) must be stored in separate files.
As a conclusion, the version-control systems are designed for maintaining
versions of files and file directories, and they are not really suited for
usage as general-purpose database indexes.



%% Other Structures
%%---------------------------------------------------------------------
\section{Other Structures}
\label{sec:mv-index:other}

In addition to the index structure already reviewed in this chapter, there
are still a few structures worth noticing.
For example, the search engine conglomerate Google has built a customized
database system for organizing the peta\-bytes of data their search engines
need to search through.
Their database system is called Bigtable~\cite{chang:2006:bigtable}, and it
is designed to be a highly scalable database system that can be used to index
vast amounts of data in a distributed database environment.
What is interesting from our point of view is that Bigtable also offers some
basic versioning functionality: data items are stored with versions,
and they can be used for querying.
There is a catch, however, as Bigtable was not designed to be primarily a
multiversion database index.
The data items are ordered on the item keys, and all the versions
of a data item are stored in the same page (or \emph{cell}, by the 
terminology of Bigtable).
Cells can have multiple versions which are chained together by placing the
most recent version of the cell in the front of the chain.
The Bigtable thus has problems similar to those of the versioned \Btree\
described in \secref{sec:mv-index:btree} and of indexes that use reverse
chaining (described in \secpageref{def:reverse-chaining}): early versions
cannot be directly accessed (because of the version chain), and key-range
queries are not efficient.
The problem for key-range queries of the most recent version is alleviated
somewhat, because not all of the versions of a given data item need to be
scanned.
Key-range queries of the earlier versions, however, must still traverse the
version-chains to locate the correct version.
From this we can arrive to the conclusion that the Bigtable is not 
optimal for indexing multiversion data.

Jouini and Jomier~\cite{jouini:2007:multiversion} have also recently
published an article comparing three different approaches for indexing
fully persistent transaction-time data (i.e., data with a possibly branched
evolution).
Their structures are called the \abbr{B+V-tree}, the \abbr{OB+tree}, and
the \abbr{BT-tree}.
From these, the \abbr{B+V-tree} resembles the versioned \Btree\ introduced in
\secref{sec:mv-index:btree}, but with support for branched history evolution.
Like the versioned \Btree, the \abbr{B+V-tree} is not efficient for range
queries because the entries are clustered primarily by their keys, and only
then by their versions.
The second structure, the \abbr{OB+tree}, builds multiple \Btree{}s that are
allowed to share unchanged branches. 
This approach is an example of path copying (see \secpageref{def:path-copy}),
and any update performed to a leaf page at database version~$v$ to create a new
version $v^{+}$ therefore necessitates the creation of a new root-to-leaf path,
thus requiring \Om{\log \entries{v}} space for each update, where \entries{v}
is the number of entries alive at version~$v$. 
The final index structure, the \abbr{BT-tree}, indexes entries based on both
the key and the version attributes, thus making this structure closer to the
more efficient approaches described in the next section.
As the details of the index structure are not discussed, we cannot really
determine the characteristics of this index structure.
The other two structures are suboptimal for indexing partially persistent
data because of the design choices they are based on, as explained above.



%% Summary
%%---------------------------------------------------------------------
% \section{Summary}
% \label{sec:mv-index:summary}
% 
% In this chapter, we have described several different techniques designed
% and used for indexing temporal data. 
% So far, none of the structures presented are optimal (by
% \defref{def:optimal-mv}), and their performance may thus degrade in some
% situations.
% In the next chapter, we discuss several efficient multiversion structures and
% show that an optimal index structure for partially persistent transaction-time
% indexing exists.

