%%=====================================================================
%% Concurrent Updating of the MVBT
%%=====================================================================
\chapter{Concurrent Updating of the MVBT}
\label{chapter:cmvbt}
\label{def:cmvbt}

The optimal TMVBT index that was presented in the previous
chapter can only be used by at most one updating transaction at a time, but
the transaction may operate concurrently with many read-only transactions.
We now describe the concurrent multiversion \Btree\ (CMVBT) index, which uses
a TMVBT index for storing the updates of committed transactions, and a
separate main-memory-resident versioned \Btree\ (VBT) index
%(see \secref{sec:mv-index:btree}) 
for storing the updates of active (and recently committed) transactions.
With this organization, multiple updating transactions can operate on the
structure concurrently. 
A system maintenance transaction is run periodically to apply the updates of
committed transactions into the TMVBT index, thus keeping the VBT index
small.
The discussion in this chapter is based on the design ideas presented in our
previous article~\cite{haapasalo:2009:cmvbt}.
The chapter begins with an overview of the concurrent index organization in
\secref{sec:cmvbt:organization}.
After that, \secref{sec:cmvbt:concurrency} describes the general principles
on how concurrency control and recovery are managed on the index.
\secref{sec:cmvbt:user-actions} shows how the user actions are implemented,
and \secref{sec:cmvbt:maintenance-tx} describes the maintenance transaction.
Finally, in \secref{sec:cmvbt:summary}, we present a summary of the CMVBT
structure.


%% Concurrent Index Organization
%%---------------------------------------------------------------------
\section{Concurrent Index Organization}
\label{sec:cmvbt:organization}

As discussed in the previous chapter, we believe that the TMVBT index
cannot be further extended so as to be updatable by multiple concurrent
updating transactions, without compromising the optimality of the structure
(specifically \invref{inv:tmvbt-live-count}).
The concurrent index structure we propose is therefore based on the idea of
collecting all the updates of active transactions and applying them to the
TMVBT index as a batch update, after the transactions have committed.
Batch updates have been previously used to enhance the performance of \Btree\
indexes~\cite{pollari-malmi:1996:batch-updates,pollari-malmi:2000:differential-index}
and also to create indexes on large existing data sets while simultaneously
allowing concurrent updating transactions to modify the data
set~\cite{mohan:1992:side-file}.
In both of these cases, a separate structure is used to record the updates of
active transactions. 
The updates are then applied to the main index structure later, after the
transactions have committed (for the former case) or after the index has been
created (for the latter case).

% A similar approach, called \emph{differential indexes}, is employed
% by Pollari-Malmi et~al.~\cite{pollari-malmi:2000:differential-index}
% to group updates falling into the same leaf page and to simplify the
% recovery algorithms.
% Mohan and Narang also use temporary indexes called \emph{side
% files}~\cite{mohan:1992:side-file} to allow online index construction
% while the data is being queried and updated.
% Differential indexes were first introduced by Lang
% et~al.~\cite{lang:1986:differential}.

Our \emph{concurrent multiversion \Btree}, or CMVBT, is a multiversion index
structure that is composed of two parts: a main-memory-resident versioned
\Btree\ index (VBT, as defined in \secref{sec:mv-index:btree}) that is used
as a temporary storage for the pending updates created by active
transactions, and a transactional multiversion \Btree\ index (TMVBT, as
described in the previous chapter) that is used as a final, stable storage in
which all the committed multiversion data items are eventually stored. 
We assume that the VBT index has sibling pointers at each level pointing to
the next page at the same level, like in a
\Blinktree~\cite{lehman:1981:b-link-tree}.
These pointers will be used to accelerate key-range scans in the VBT index. 
The transactions operating on the CMVBT index follow the transaction model
presented in Sections~\ref{sec:mv-data:read-only-tx} (for read-only
transactions) and~\ref{sec:mv-data:updating-tx} (for updating transactions).
For user transactions, the TMVBT index is a read-only structure. 
The TMVBT is only updated by a system \emph{maintenance transaction}, which
is run periodically to apply the updates of committed transactions from the
VBT index into the main TMVBT index, and to delete the pending updates from
the VBT\@.
This setup is illustrated in \figref{fig:cmvbt-setup}.
We assume that only a single instance of the maintenance transaction is running
at a time; for discussion on multithreading the maintenance transaction, see
the discussion at the end of \secref{sec:cmvbt:maintenance-tx}.

\begin{figure}[htb]
\begin{center}
  \input{images/cmvbt-setup}\\
  \figcaption{The CMVBT index structure organization}
  {User transactions issue queries both to the VBT and the
  TMVBT indexes, but updates are performed only on the VBT\@.
  A system maintenance transaction is run periodically to apply the
  pending updates of committed transactions into the TMVBT index and to
  delete them from the VBT\@.}
  \label{fig:cmvbt-setup}
\end{center}
\end{figure}

We expect that the VBT index will remain small under typical database
workloads, as the maintenance transaction is run periodically to apply
the updates to the TMVBT and delete them from the VBT\@.
For efficiency, we thus reserve a portion of the page buffer specifically for
the VBT index. 
This way the entire VBT index can be kept in main memory during normal
transaction processing.
In all of our tests, the VBT size was between \num{0} and \num{20} pages,
with a typical size of just a few pages (see \secref{sec:performance:general}, 
page \pageref{test:vbt-size}).

Like the TMVBT, the CMVBT maintains a variable \comver\ that records the
version of the latest committed transaction.
In the CMVBT index, the updates are performed by inserting pending updates
(see \defref{def:pending-update}) to the VBT index.
The pending updates are then later on applied to the TMVBT index and deleted
from the VBT\@.
If there is a large number of concurrent updating transactions that commit at
around the same time, the TMVBT index will ``lag behind'' until the
maintenance transaction has had time to apply the pending updates into the
TMVBT\@. 
Therefore, in the context of the CMVBT index, the \comver\ variable
does not tell the maximum committed version of the TMVBT index.
Accordingly, the CMVBT maintains a separate variable \stablever\ that tells
which commit-time versions are already reflected in the TMVBT index. 
In other words, for each version $v \leq \stablever$, all the updates
of a transaction $T$ with $\commitver{T} = v$ have been applied to
the TMVBT index.

\thmskip 
\begin{definition}
\label{def:stablever}
In the CMVBT index, the variable \comver\ tells the version of the latest
committed transaction.
A separate variable $\stablever \leq \comver$ tells the
commit-time version of the latest transaction whose updates have been applied
to the TMVBT index.
The commit-time versions $v \leq \stablever$ are called \emph{stable
versions}, and the transactions that created the versions are called
\emph{stable transactions}.
All the updates of stable transactions have been applied to the TMVBT index
by the maintenance transaction.
The commit-time versions $v > \stablever$ are called \emph{transient
versions}, and the corresponding transactions \emph{transient transactions}.
All the updates of transient transactions are located in the VBT index.
\end{definition}
\thmskip 

By this definition, an active updating transaction is always transient,
while a committed transaction can be either transient or stable.
An example of the commit-time versions and transaction identifiers used in
a CMVBT database is given in \figref{fig:cmvbt-example}\@.
The example shows a database with five committed transactions
(versions~\range{1}{5}), and two active, uncommitted transactions (with
temporary identifiers~\num{102} and~\num{104}).
Three out of the five committed transactions are stable, namely the
transactions with commit-time versions~\range{1}{3}, and 
the transaction identifiers~\num{103} and~\num{101} correspond to
transient committed versions \num{4} and \num{5}, respectively. 
The CTI table present in the figure is explained in the following
paragraphs.

\begin{figure}[htb]
\begin{center}
%  \includegraphics[width=0.4\textwidth]{images/cmvbt-example}\\
  \begin{tikzpicture}[x=1.5em,y=1.5em]

\node[dbl,minimum width=3em] (tmvbt)
  {\footnotesize\parbox{2em}{\centering $1$\\$2$\\$3$\\~}};
\node[below=0.2 of tmvbt] {\footnotesize TMVBT};

\node[dbl,minimum width=3em,right=1 of tmvbt] (vbt)
  {\footnotesize\parbox{2em}{\centering $101$\\$102$\\$103$\\$104$}};
\node[below=0.2 of vbt] (vbtlab) {\footnotesize VBT};

\node[right=1.8 of vbtlab] (ctilab) {\footnotesize CTI};
\node[rounded rectangle,draw=black!100,above=0.2 of ctilab] (cti) 
  {\footnotesize\parbox{4em}{\centering $4 \to 103$\\$5 \to 101$}};

\node[above=0.5 of cti] (variables) 
  {\footnotesize\parbox{5em}{\flushright $\stablever = 3$\\$\comver = 5$}};

\end{tikzpicture}
\\
  \figcaption{Example of the logical contents of a CMVBT index}
  {The database contains the updates of five committed versions, versions
  \range{1}{5}. 
  The TMVBT index contains all the updates of stable versions
  \range{1}{3}, and the updates of committed transient versions \range{4}{5}
  are still located in the VBT index, identified with transaction
  identifiers \num{103} and~\num{101}.
  The VBT additionally contains updates by two active updating transactions
  that have transaction identifiers \num{102} and~\num{104}.}
  \label{fig:cmvbt-example}
\end{center}
\end{figure}

As explained in the transaction model for updating transactions in
\secref{sec:mv-data:updating-tx}, we assign a transaction identifier
called \txid{T} for each new transaction~$T$\@.
The updates of the transaction~$T$ are stored into the VBT index using
transaction identifiers in the VBT entries (identifiers
\range{101}{104} in the example situation of \figref{fig:cmvbt-example}). 
The commit-time versions \commitver{T} of committed transactions define
the ordering of the transactions (versions \range{1}{5} in the example). 
When the updates of a committed transaction~$T$ are applied to the TMVBT
index, the commit-time version is known, and will be used in the TMVBT index.
The TMVBT thus stores commit-time versions exclusively, and the VBT
index stores only transaction identifiers.
The transaction identifiers are internal to the database system, and not
visible to the users.
The users only see the commit-time versions when issuing historical
queries to the database.
Both of these versions can be based either on the real time, or on
an increasing counter value, as long as they are unique and increasing. 
In the following discussion, we assume that the versions are based on
an increasing counter value.

Because the transaction identifiers and the actual, commit-time
versions of the transactions may differ, a mapping from commit-time
versions to the transaction identifiers is maintained in an in-memory
hash-table called the \emph{commit-time-to-identifier} table, or
CTI\phantomsection\label{def:cti} table. 
In the example of \figref{fig:cmvbt-example}, the updates of committed
transactions with commit-time versions \num{4} and \num{5} are still
located in the VBT index with transaction identifiers \num{103} and
\num{101}. 
A mapping $v_c \to v_i$, for a committed transaction~$T$ with
$\commitver{T} = v_c$ and $\txid{T} = v_i$, is removed from the
CTI table once the maintenance transaction has applied all the committed
updates of~$T$ into the TMVBT, deleted the pending updates from
the VBT, and committed.
The CTI table does not need to be backed onto disk. 
If the system fails, the CTI table can be reconstructed from the
log file contents during the analysis pass of the \abbr{ARIES} restart
recovery~\cite{mohan:1992:aries}.
 
\thmskip
\begin{definition}
\label{def:movever}
When the maintenance transaction is running, it applies the updates of
a single version called the \emph{move version}~\movever\ into the TMVBT\@.
The move version is always the earliest transient committed version.
\end{definition}
\thmskip

\thmskip
\begin{invariant}
In our numbering convention, $\movever = \stablever + 1$ whenever
the maintenance transaction is running.
The updates of the transaction with $\commitver{T} = \movever$ can be located
in both the VBT and the TMVBT indexes during the execution of the maintenance
transaction.
If the maintenance transaction is not running, $\movever = \stablever$.
\end{invariant}
\thmskip

When the maintenance transaction is running, the user transactions
still use the stable version variable~\stablever\ to direct the
search for the correct versions of data items.
In the example of \figref{fig:cmvbt-example}, the maintenance
transaction could be moving the updates of transaction~$T$ with
$\commitver{T} = 4$ from the VBT to the TMVBT\@. 

When multiple active transactions perform updates on the same data item, 
all the different updates are stored in the VBT, ordered by the transaction
identifiers of the transactions that created them.
We assume that the convention presented in \secref{sec:mv-index:btree} is
used; that is, the result of a write action by a transaction~$T$ with
$\txid{T} = v_i$ is represented by a tuple $(k, v_i, w)$ in the VBT index, and
the result of a delete action is represented by a tuple $(k, v_i,
\deletemark)$.
When updating transactions commit, the pending updates remain in the
VBT until the maintenance transaction deletes them after applying the updates
to the TMVBT index.

Because the ordering of the committed transactions may differ from
the ordering of the transaction identifiers of the transactions, the
entry values in the VBT might not be in the correct commit-time order.
Therefore, reading transactions that wish to read a version $v >
\stablever$ need to find the transaction identifiers of the
transactions with commit-time versions $v_c: \stablever < v_c
\leq v$, search the VBT for all these versions, and rearrange them according
to the commit-time ordering so that the most recent update is found.
In the example database of \figref{fig:cmvbt-example}, a user
querying for keys that are present in the commit-time version \num{5} must
find all keys in the VBT stored with transaction identifiers \num{103} and
\num{101} (corresponding to commit-time versions \num{4} and \num{5}). 
This is relevant for situations where the queried key has not been updated by
the latest preceding transaction (in the example, the transaction with
commit-time version~\num{5}), so that the data item that is alive at the
queried version has been created by an earlier transaction.

More formally, we define \transcom{v} to be the set of transient committed
versions that are relevant when querying for the version~$v$:
$\transcom{v} = \{ v_c : \stablever < v_c \leq v \}$. 
In the example of \figref{fig:cmvbt-example}, $\transcom{5} = \{4, 5\}$. 
Furthermore, we define a mapping~\idcommap{v} from the transaction
identifiers of the transactions that created the committed versions into the
commit-time versions:
$\idcommap{v}[\text{CTI}[v_c]] = v_c$ for all $v_c \in \transcom{v}$. 
In the example, $\idcommap{5} = \{101 \to 5, 103 \to 4\}$.
Now, when looking for the most recent update of key~$k$, the VBT must be
queried searching for the updates that have been created by a transaction
with an identifier~$v_i$ such that there is a mapping $v_i \to v_c$ in
\idcommap{v}. 
After all these updates for any given key have been found, they
must be rearranged according to the ordering of the corresponding
commit-time versions, so that updates with a transaction identifier~$v_i$ are
sorted by~$\idcommap{v}[v_i]$.
After this, the most recent update is known to be the last update in the
ordering. 
If no pending updates on key~$k$ are found in the VBT, the most recent update
is queried from the TMVBT index.

The ``wrong'' ordering of entries in the VBT is not specific to the CMVBT
structure, but is in fact present in all multiversion database systems that
allow concurrent updating transactions to commit in an order different
from their starting order.
Lomet et~al.~\cite{lomet:2006:transactiontime} have solved this
problem by using lazy timestamping, as described in \secref{sec:tsbmvbt:tsb}.
This technique also requires a lookup table (the persistent timestamp
table~\cite{lomet:2006:transactiontime,lomet:2009:improving}) for converting
transaction identifiers into commit-time versions. 
In our technique, the versions of entries are corrected when the
committed pending updates are applied to the TMVBT for permanent storage, and
stable versions can be queried without having to convert any versions or
rearrange any updates.



%% Concurrency Control and Recovery
%%---------------------------------------------------------------------
\section{Concurrency Control and Recovery}
\label{sec:cmvbt:concurrency}

The CMVBT allows us to use various approaches for concurrency control and
recovery. 
In this section, we describe the general idea of our concurrency-control and
recovery algorithms.
Our approach for database recovery follows the \abbr{ARIES}
algorithm~\cite{mohan:1992:aries,mohan:1992:aries-im,mohan:1990:aries-kvl}
with physiological logging and standard steal-and-no-force page buffering
policy.
Each structure-modification operation on both the VBT and the
TMVBT is logged using a single physiological redo-only log
record, so that interrupted tree-structure modifications are never
rolled back (undone) when a transaction aborts or system fails.
This approach has been described for \Btree{}s and \Blinktree{}s by Jaluta
et~al.~\cite{jaluta:2005:blink,jaluta:2006:page-server}.

The actions of user transactions on the VBT index are logged with
standard redo-undo log records, and the corresponding undo actions with  
redo-only log records, as described in more detail in
\secref{sec:cmvbt:user-actions}. 
These log records are required for total and partial rollbacks of active
transactions.
A total rollback for a transaction~$T$ could be performed by performing a
leaf-level scan of the VBT, and by deleting all the pending updates of the
form $(k, \txid{T}, \updatemark)$.
However, if a single transaction updates the same key multiple times,
the previous values stored with the key are overwritten, and need to
be restored when rolling back the transaction to a preset savepoint. 
Thus we write standard physiological redo-undo log records to log
forward-rolling update actions, and redo-only log records to log undo
actions, of user transactions.
The structure-modification operations are logged with redo-only log
records, so that they are never undone.
These log records are used to bring the VBT up-to-date after a single
transaction has crashed.
If the entire database system crashes, it is possible to bring the VBT
up-to-date either by using \abbr{ARIES}-based
recovery~\cite{mohan:1992:aries,mohan:1992:aries-im}, or by reconstructing
the VBT logically based on the log contents, including only the entries of
committed transactions.
All update actions on the TMVBT are performed by the maintenance transaction,
which is never aborted or undone. 
If a system crash occurs during the execution of the maintenance transaction,
the transaction is resumed after the system has recovered. 
Redo-only log records are thus sufficient for logging the actions
performed on the TMVBT index.

Concurrency control on the key level is provided by the snapshot
isolation~(\abbr{SI}) algorithms (\secref{sec:mv-data:cc}).
In snapshot isolation, each transaction reads data items that were alive when
the transaction started; that is, data items that are alive at
version~\snapver{T} (see \secref{def:snapver}).
The version that transaction~$T$ is reading is called the \emph{snapshot of
$T$}\@. 
As with the TMVBT index described in the previous chapter, read-only
transactions always read data from their own snapshot, and thus do not
require locks to protect the keys against concurrent modifications.
Updates to the database are performed by adding a new version of the
updated data item to the snapshot of the updating transaction (with updates
stored in the VBT index), allowing updating transactions to read their own
modifications directly from the snapshot.
Snapshot isolation is an obvious choice for multiversion structures,
because the entire history of the database is preserved, and thus
the snapshot state is available for each transaction.

Logical consistency for updating transactions is guaranteed by checking that
overlapping transactions do not make updates to the same data items. 
Transactions~$T$ and~$T'$ are \emph{overlapping}, if 
$\snapver{T} < \commitver{T'} < \commitver{T}$, or vice versa.
We assume the same approach for enforcing snapshot isolation as is 
used in PostgreSQL~\cite{alomari:2008:si-cost}, as described in
\secref{sec:mv-data:cc}.
Each updating transaction thus takes a commit-duration lock on the key~$k$ of
any data item it modifies.
Furthermore, an updating transaction~$T$ must check that the data item with
key~$k$ has not been updated by an overlapping committed transaction~$T'$.
We do not assume any specific method for implementing this check, but leave
the details open.
One possible method is to search the VBT and TMVBT to see if any updates have
been made by such a committed transaction.
Another possibility is for the committed transaction~$T'$ to leave persistent
``residual locks'' on the lock manager that conflict only with transactions
that have started before $T'$ committed.
% To confirm this, the VBT (and possibly the TMVBT) must be searched for updates
% by transactions with commit-time versions in the range $(\snapver{T},
% \comver]$.
If a conflicting update is found, the active transaction~$T$ must be aborted
to maintain snapshot isolation. 
% Searching for an update by an overlapping committed transaction is described
% in more detail in \secref{sec:cmvbt:user-actions}.
Note that the write locks taken by updating transactions in this approach
only block other updating transactions, because read-only transactions take
no locks. 

The global version variables~\comver, \stablever\ and~\movever\ are
maintained in the persistent database, and the reading and writing of~\comver\
and \movever\ are protected with locks, as in the TMVBT
(\secref{sec:tmvbt:actions}).
A \action{begin-read-only} action acquires a short-duration read lock
on~\comver\ for reading its value, and a \action{commit-update}
action acquires a commit-duration write lock on it for incrementing
its value. 
The maintenance transaction acquires a commit-duration write lock
on~\movever\ at the beginning, thus guaranteeing that at most
one maintenance transaction is active at a time.
The stable version variable~\stablever\ is read often, and thus the reading
and writing of it is protected by latching.
This approach is sufficient, because \stablever\ is only updated by the
maintenance transaction, which is never undone.
The updating of~\stablever\ is protected by a write latch taken on the
database page on which the variable is stored, and the reading of the
variable by query actions is protected by a read latch taken on the same
page.

Structural consistency of the VBT index is maintained by standard
page-latching operations, with latch-coupling applied to ensure child-link
consistency during tree traversals.
We define the latching order to be top-down, left-to-right for all
transactions, and we disallow upgrading read latches to write latches.
Furthermore, all page latches must be released whenever a write lock
cannot be acquired immediately, so that no page latches are held when the
transaction is waiting for a lock.
These are necessary (and sufficient) restrictions to avoid deadlocks
that involve page latching~\cite{gray:1993:transactionprocessing}.
For the TMVBT index, pages need to be latched, but latch-coupling is
not always required, as explained in \secref{sec:tmvbt:actions}. 
This is because the only transaction that is allowed to perform updates on
the TMVBT index is the system maintenance transaction, and there can be
only one such transaction running at a time.
The maintenance transaction therefore does not need to do latch-coupling.
Read-only transactions that read inactive data do not need latch-coupling
either, as noted in the previous chapter.
This means that read-only transactions that are reading stable versions
(from the TMVBT) do not need latch-coupling.
Further details of concurrency control and recovery are embedded in the
explanations of the algorithms that are described in the next sections,
alongside with more detailed explanations of the latching policy.



%% User Actions
%%---------------------------------------------------------------------
\section{User Actions}
\label{sec:cmvbt:user-actions}

We allow two kinds of user transactions to operate concurrently on
the \abbr{CMVBT}: read-only transactions and updating transactions. 
There are no restrictions on how many transactions of either type
can be running at the same time.
Read-only transactions follow the transaction model described in
\secref{sec:mv-data:read-only-tx}, and updating transactions follow the
model described in \secref{sec:mv-data:updating-tx}.
As in \secref{sec:tmvbt:multi-action-tx}, we write $T$ in the log records of
user actions to mean that the transaction identifier $\txid{T}$ is written in
the log record.

For the user actions of read-only and updating transactions that operate on
the CMVBT index, we need to define the following algorithms: 
\begin{enumerate}
\setlength{\itemsep}{0pt}
\item \alg{update-item} to perform an update action (insert or delete).
\item \alg{query-stable} to query for a single key of a stable version.
\item \alg{query-transient} to query for a single key of a transient
version.
\item \alg{next-key-stable} to query for the next key of a stable version.
\item \alg{next-key-transient} to query for the next key of a transient
version.
\end{enumerate}
% In the following discussion, we present algorithms for performing
% these actions, first in a general level, and then in more detail.
These algorithms are presented later on. 

With these algorithms, the actions of a read-only transaction are performed
as follows.
Key-level locking is not required for read-only transactions.
\begin{itemize}
  \setlength{\itemsep}{0pt}
  
  % Begin read-only
  \item \action{begin-read-only}$($version~$v)$: begins a new read-only
  transaction; this action takes a short-duration read lock on~\comver, 
  reads the value of the variable, checks that $v \leq \comver$, and records
  the value $\snapver{T} \gets v$ for the transaction.
  If the version check fails, the transaction is not allowed to begin.

  % Single-key query
  \item \action{query}$($key~$k)$:
  this action reads the value of the stable-version variable~\stablever.
  If $\snapver{T} \leq \stablever$, then \alg{query-stable} is run to find
  the correct data item from the TMVBT index; 
  otherwise the \alg{query-transient} algorithm is run to 
  either find the latest relevant pending update from the VBT index or to
  find the latest data-item entry from the TMVBT index.
  The reading of~\stablever\ is protected by read-latching the database
  page~$p$ that contains the variable.
  The value can be cached for the transaction, so that subsequent
  queries do not need to read the page~$p$. 

  \item \action{range-query}$($range~$[k_1, k_2))$:
  like the \action{query} action, this action reads the value of the
  stable version variable~\stablever.
  If $\snapver{T} \leq \stablever$, then \alg{next-key-stable} is run to find
  the set of data items from the TMVBT index; 
  otherwise \alg{next-key-transient} is run to retrieve the set of data
  items from both the VBT and the TMVBT indexes.
  The set of data items is retrieved by first finding the first entry from the
  range with the query $(k, w) \gets \proc{N}(k_1, \snapver{T})$, and
  \proc{N} is either \alg{next-key-stable} or \alg{next-key-transient}, as
  explained above.
  If $k < k_2$, the snapshot data item $(k, w)$ is added to the result
  set, and the next data item is fetched with $(k', w')
  \gets$ $\proc{N}(k, \snapver{T})$.
  This iteration is continued until a data item that is outside the
  queried range $[k_1, k_2)$ is found, or until the algorithm~\proc{N}
  returns no more results.
  To be more precise, the first entry is located with a slightly different
  query algorithm because the first entry in the TMVBT index may have the
  key~$k_1$, and thus the query actually find the first entry $(k,w)$ with $k
  \geq k_1$, instead of $k > k_1$.
  When querying for the rest of the keys, the next key must be greater than
  the previous key.
  We have omitted the description of the first-key-query algorithm because it
  is almost identical to the next-key-query algorithm. 

  \item \action{commit-read-only}: commits the transaction by removing
  it from the system. 
  No further actions are required.
\end{itemize}

For updating transactions, the actions are implemented as follows. 
Key-level locking is only required where specifically stated.
\begin{itemize}
  \setlength{\itemsep}{0pt}
 
  % Begin updating transaction
  \item \action{begin-update}: begins a new updating transaction~$T$;   
  this action creates an identifier for the transaction ($\txid{T} \gets$
  \emph{new identifier}), acquires a short-duration read lock on the
  variable~\comver, and records the snapshot version~$\snapver{T} \gets
  \comver$.
  The action finishes by writing the log record \lrb{T},
  \logact{begin}, \lre{\snapver{T}}, but the log is not forced to disk.

  % Single-key query
  \item \action{query}$($key~$k)$: this action queries the VBT and the TMVBT
  indexes to find the data item that is alive at \snapver{T}.
  Although \snapver{T} might be stable, this action must always use
  the \alg{query-transient} algorithm to query both the VBT and the TMVBT
  indexes, because the transaction might have itself updated the key~$k$, and
  pending updates are located only in the VBT index. 

  % Range-query
  \item \action{range-query}$($range~$[k_1, k_2))$: like the \action{query}
  action, this action always uses the \alg{next-key-transient} algorithm to
  query both VBT and TMVBT indexes to locate the data items that are alive at
  \snapver{T} in the range $[k_1, k_2)$.

  % Write action
  \item \action{write}$($key~$k,$ data~$w)$: this action takes a
  commit-duration write lock on the key~$k$, and invokes the
  \alg{update-item} algorithm to insert a pending update into the VBT\@. 
  This action writes a redo-undo log record
  \lrb{T}, \logact{write}, $p$, $k$, $w$, $\updatemark'$, \lre{n}, where
  (1)~$p$~is the page identifier of the VBT page on which the pending update
  was inserted; 
  (2)~$\updatemark'$~is the value of an overwritten pending update, if
  the action replaced an earlier update created by~$T$; or
  \nullmark, otherwise; and 
  (3)~$n$ is the Undo-Next-LSN of the log record; that is, the
  LSN of the last (not-yet-undone) action performed by~$T$
  before this action.

  % Delete action
  \item \action{delete}$($key~$k)$: the operation of this action is identical
  to the \action{write} action, except that in the \alg{update-item} algorithm
  a deletion is performed instead of data-item insertion, and the
  redo-undo log record that is written is \lrb{T}, \logact{delete}, $p$, $k$,
  $w'$, \lre{n}. 
  We denote the possibly replaced value of a pending update here by~$w'$
  instead of~$\updatemark'$, because in this case, the replaced value cannot
  be $\deletemark$.

  % Set savepoint
  \item \action{set-savepoint}: sets a savepoint and returns the savepoint
  identifier~$s$ to the transaction.
  The action first writes a redo-undo log record \lrb{T}, \logact{savepoint},
  \lre{n}. 
  The savepoint identifier~$s$ is the log sequence number of the savepoint
  log record.

  % Rollback to savepoint
  \item \action{rollback-to-savepoint}$($LSN~$s)$: rolls the transaction
  back to a preset savepoint identified by the log sequence number~$s$.
  This action is followed by the \action{undo-write} and
  \action{undo-delete} actions for the \action{write} and
  \action{delete} actions performed after setting savepoint~$s$, executed
  in the reverse order. 
  The \action{write} and \action{delete} actions are located with a reverse
  scan of the log records, starting from the last action performed by~$T$,
  and by locating the previous records via the Undo-Next-LSN
  pointers in the log records that point to the log record of the previous
  action, until the log record for the \action{set-savepoint} action is found
  for the savepoint~$s$.

  % Commit
  \item \action{commit-update}: commits the updating transaction; this action
  (1)~acquires a commit-duration write lock on~\comver,
  (2)~increments the variable $\comver \gets \comver + 1$,
  (3)~assigns a commit-time version to the transaction
  $\commitver{T} \gets \comver$,
  (4)~adds the mapping CTI$[\commitver{T}] \gets \txid{T}$, 
  (5)~writes a log record \lrb{T}, \logact{commit}, \lre{\comver},
  (6)~forces the log onto disk, 
  (7)~releases the lock on \comver, and
  (7)~removes the transaction~$T$ from the system.
  The release-version action is performed by the system maintenance
  transaction some time after the commit operation has finished.
  The commit-time version \commitver{T} can be queried by other
  transactions immediately after this action has completed, before the
  maintenance transaction has run.

  % Version release
  \item \action{release-version}: incorporates the updates performed by the
  earliest transient committed transaction into the TMVBT index, and removes
  the pending updates from the VBT index.
  This action is implemented by the maintenance transaction that is explained
  in more detain in \secref{sec:cmvbt:maintenance-tx}.

  % Abort 
  \item \action{abort}: labels the transaction as aborting and starts the
  backward-rolling phase.
  This action writes the log record \lrb{T}, \logact{abort},
  \lre{n}. 
  In the backward-rolling phase, all the actions of the transaction are
  undone, in reverse order, by following the previous-record pointers (i.e.,
  the Undo-Next-LSN values) in the log records. 
  This action is thus similar to the \action{rollback-to-savepoint} action,
  except that the log is rolled back until the \action{begin} log record is
  encountered, at which point the \action{finish-rollback} action is executed.
%   Another possibility for undoing all the actions of transaction~$T$ is to
%   scan through the VBT index and physically remove all entries of the
%   form $(k, \txid{T}, \updatemark)$.

  % Undo the write action
  \item \action{undo-write}$($log record~$r)$: this action undoes a write
  action by reading the log record $r =
  \lrb{T}$, \logact{write}, $p$, $k$, $w$, \lre{\updatemark',n} and by
  physically removing the entry $(k, \txid{T}, w)$ from the index. 
  If $\updatemark' \neq \nullmark$, then the write action replaced an earlier
  pending update created by transaction~$T$, and the undo action inserts
  the pending update $(k, \txid{T}, \updatemark')$ into the VBT index to
  restore the overwritten entry. 
  Note that $\updatemark'$ may also be $\deletemark$, if the transaction had
  deleted the key~$k$ earlier.
  The undo action finishes by writing the redo-only compensation log record 
  \lrb{T}, \logact{undo-write}, $p'$, $k$, $w$, $\updatemark'$, \lre{n},
  where $p'$ is the page identifier of the VBT page on which the update was
  performed. 
  If a structure-modification operation has occurred on page~$p$
  after the \action{write} actions, it is possible that $p' \neq p$.

  % Undo the delete action
  \item \action{undo-delete}$($log record~$r)$: this action undoes a delete
  action by reading the log record $r =
  \lrb{T}$, \logact{delete}, $p$, $k$, $w'$, \lre{n} and by physically
  removing the entry $(k, \txid{T}, \deletemark)$ from the VBT index. 
  If $w' \neq \nullmark$, then the delete action replaced an earlier pending
  update created by transaction~$T$, and the undo action inserts the
  pending update $(k, \txid{T}, w')$ into the VBT index to restore the 
  overwritten pending update. 
  The undo action finishes by writing the redo-only compensation log record
  \lrb{T}, \logact{undo-delete}, $p'$, $k$, $w'$, \lre{n}.

  % Finish rollback 
  \item \action{finish-rollback}: finishes the rollback of a backward-rolling
  transaction by writing a log record \lrb{T},
  \lre{\logact{finish-rollback}}, forcing the log onto disk, and
  by removing the transaction~$T$ from the system.
\end{itemize}

Let us now consider the algorithms required to implement the user actions.
The algorithm \alg{update-item} which performs an update action on the
CMVBT index is reasonably straightforward: all that needs to be done is
to record the action to the VBT, using top-down, left-to-right
latch-coupling for tree traversal.
An insertion of a data item with key~$k$ by a transaction~$T$
with $\txid{T} = v$ is recorded by adding the entry $(k, v, w)$, where
$w$ is the entry value;
and deletion of a data item with key~$k$ by transaction~$T$ is recorded
by adding the entry $(k, v, \deletemark)$. 
If the VBT index already contains an entry $(k, v, \updatemark')$ with the
same key and version, the old entry will be replaced by the new entry. 
% Otherwise, all update actions always add a new key into the VBT index.
% The algorithm finishes by writing a redo-undo log record
% \logrecord{T,\logact{write},p,k,w,w',n} for a \action{write} action
% and a redo-undo log record \logrecord{T,\logact{delete},p,k,w',n}
% for a \action{delete} action.
% Here, $p$ denotes the page where the new pending update was inserted, and 
% $w'$ is the value of a replaced data item, if the action replaced an earlier
% update create by the same transaction.
The possible structure-modification operations on the VBT needed to
accommodate the update are performed level-by-level as separate atomic
actions that are never undone, before the installation of the update,
similarly to the SMOs for the TMVBT index (see \secref{sec:tmvbt:smo}).

Querying for stable versions with the \alg{query-stable} algorithm is also
straightforward. 
When a read-only transaction queries a stable version, the transaction 
can directly query for the version from the TMVBT index, as explained in
\secref{sec:tmvbt:actions}, by using the TMVBT query action.
In this situation, the TMVBT index can be traversed without latch-coupling,
and saved paths can be used for key-range and next-key queries without any
need to check for page validity when relatching the pages on the path. 
This is possible because all pages traversed and entries read are
inactive, and thus guaranteed to remain where they are located, by
\invref{inv:tmvbt-static-inactive-entries}.
The read-only transaction can determine whether a version is stable by reading
the stable version variable~\stablever.
The reading of the variable is protected by read-latching the page on which it
is stored. 
The latch can be released immediately after the value of the variable
has been determined.
It is also possible to read the variable value once, at the beginning of the
transaction, and then cache the result for the transaction.
% This may lead to a read-only transaction trying to find pending updates from
% the VBT that have already been applied to the TMVBT and deleted from the VBT,
% but this does not cause problems, because the updates will then be found from
% the TMVBT later on.

When a read-only transaction is querying for a transient version, and when an
updating transaction is performing any key query, the \alg{query-transient}
algorithm must be used.
Both read-only and updating transactions use the same
\alg{query-transient} algorithm for querying the index structures, and we
will thus use the term \emph{reader} to refer to a read-only transaction or
an updating transaction that is performing a query action.
When the queried version is transient, the reader cannot know
beforehand which index structure contains the most recent update that
precedes the queried version.
The relevant version might be the last update in the TMVBT, or
there may be an intermediate update in the VBT\@.
For illustration of such a situation,
an example of the possible contents of a CMVBT index is given in
\figref{fig:cmvbt-contents}.
This example represents the same logical situation as the example shown in 
\figref{fig:cmvbt-example}, but now the actual entries stored in the
different indexes are shown.
Furthermore, when an updating transaction has itself performed an update
action, the pending update is only located in the VBT, and it is thus not
sufficient to restrict the search to the TMVBT, even if the snapshot
version~\snapver{T} of the updating transaction is stable.

\begin{figure}[htb]
\begin{center}
  \subfigure{
\begin{tikzpicture}
\node[rounded rectangle,draw=black!100] (tmvbt) {%
\contentrows{6}{6.2em}{%
$(1, [1, 2), w_1)$\\
$(2, [1,\infty), w_2)$\\
$(3, [2,3), w_3)$\\
$(3, [3,\infty), w_3')$\\
$(4, [3,\infty), w_4)$}};
\node[above=0pt of tmvbt] {\textbf{TMVBT}};

\node[rounded rectangle,draw=black!100,right=1em of tmvbt] (vbt) {%
\contentrows{6}{5em}{%
$(1, 102, w_1')$\\
$(2, 101, w_2')$\\
$(4, 103, \deletemark)$\\
$(4, 104, w_4')$\\
$(6, 101, w_6)$\\
$(7, 103, w_7)$}};
\node[above=0pt of vbt] {\textbf{VBT}};

\node[rounded rectangle,draw=black!100,right=1em of vbt] (cti) {%
\contentrows{6}{3.4em}{%
$4 \to 103$\\
$5 \to 101$}};
\node[above=0pt of cti] {\textbf{CTI}};
\end{tikzpicture}
}
\subfigure{
  \begin{tabular}{c}
  \textbf{History of transactions}\\
  \begin{tabular}{l}
  $T_1$, $\commitver{T_1} = 1$: Insert $(1, w_1)$ and $(2, w_2)$\\
  $T_2$, $\commitver{T_2} = 2$: Insert $(3, w_3)$ and delete item with
  key~\num{1}\\ 
  $T_3$, $\commitver{T_3} = 3$: Insert $(3, w_3')$ and $(4, w_4)$\\
  $T_4$, $\commitver{T_4} = 4$: Insert $(7, w_7)$ and delete item with
  key~\num{4}\\ 
  $T_5$, $\commitver{T_5} = 5$: Insert $(2, w_2')$ and $(6, w_6)$\\
  $T_6$, $\txid{T_6} = 102$: Insert $(1, w_1')$\\ 
  $T_7$, $\txid{T_7} = 104$: Insert $(4, w_4')$\\ 
  \end{tabular}\\
  \end{tabular}
}

  \figcaption{Example of data entries stored in a CMVBT index}
  {This example represents the same situation as
  \figref{fig:cmvbt-example}.
  The format of entries in the TMVBT is (key, life span, data), and the
  format of entries in the VBT is (key, transaction identifier, update).
  In addition to the committed transactions, the VBT also contains the
  updates of two active transactions with transaction identifiers \num{102}
  and \num{104}.}
  \label{fig:cmvbt-contents}
\end{center}
\end{figure}

In the example of \figref{fig:cmvbt-contents}, when performing queries that
target the transient commit-time version~\num{5}, the relevant version for
key~\num{3} is the last update in the TMVBT index (created by transaction
$T_3$), while the relevant version for key~\num{2} is the update in the VBT
by transaction $T_5$ with transaction identifier $101$.
Both structures may need to be checked whenever querying for a transient
version.
For single-key queries (with the key given), it is sufficient to
restrict the search to the VBT index if an update on the key is found from
there; only if no such update is found from the VBT do we need to consult the
TMVBT to find the latest update on the key.
However, when performing next-key queries, both structures always need to be
checked, because it is impossible to determine which structure contains the
nearest key otherwise.

Because the maintenance transaction may be moving the updates
of the move version~\movever\ to the TMVBT during the execution of the query
action, the reader must be prepared to miss some of the updates of
version~\movever\ in the VBT, and to possibly re-encounter some already
encountered updates in the TMVBT\@. 
The situation is amended by organizing the read actions in such a way that
the VBT is always consulted first, and the TMVBT only afterwards.
This guarantees that no update is missed.

\begin{algorithm}[htb]
$\proc{query-transient}(k, v, T)$:

\begin{algorithmic}[1]
\STATE $\updatemark \gets \proc{key-query-VBT}(k, v, T)$
\IF[If no updates found from \abbr{VBT}]{$\updatemark = \nullmark$}
  \RETURN $\proc{query-stable}(k, v, \TRUE)$
\ELSIF[If latest update in VBT is a deletion]{$\updatemark = \deletemark$}
  \RETURN \nullmark
\ELSE[Latest update in VBT is an insertion]
  \RETURN \updatemark
\ENDIF
\end{algorithmic}
\figcaption{Key-query algorithm for transient versions}%
{This algorithm is used by read-only transactions when querying for transient
versions, and by updating transactions when performing any key queries.}
\label{alg:query-transient}
\end{algorithm}

The general algorithm for \alg{query-transient} is given in
\algref{alg:query-transient}\@.
Querying for a single key from the TMVBT index with the procedure
\alg{query-stable} works exactly like the single-key query action of the
TMVBT, explained in \secref{sec:tmvbt:actions}, except that the third
parameter is used to indicate that the reader may need to read active data
and must therefore use latch-coupling when traversing the paths of the TMVBT
index. 
Note that the version~$v$ may be greater than the most recent version
of the TMVBT (i.e., \stablever\ or \movever).
The searches in the TMVBT still work, because the live pages (and entries) in
the TMVBT have a life span of the form $[v',\infty)$, which covers~$v$.

When reading the pages of the TMVBT index, the reader can determine whether a
page has been invalidated by a concurrent structure-modification operation
(for example, when relatching pages on a previously released saved path) by
internally caching the page-LSN before releasing latches and by then
comparing the cached value to the value in the page after it has been
relatched. 
If the new page-LSN is greater than the cached one, then the page has
been modified by the maintenance transaction, and it is necessary to either
re-traverse the entire path from the root of the TMVBT index or to backtrack
up the saved path until an unmodified page is found.

Querying for a single pending update from the VBT is described in
\algref{alg:key-query-vbt}\@.
The algorithm generates a reverse identifier-to-commit-time~(\abbr{ITC})
mapping for all relevant commit-time versions, mapping the
possible updates created by the transaction itself to infinity so
that they always take precedence over other updates.
A transaction identifier list~$I$ is also generated.
This list contains the transaction identifiers for which there is a mapping
in the \abbr{ITC}\@.
The \alg{query-all-VBT}$(k, I)$ operation finds all pending updates $(k, v',
\updatemark')$ from the VBT with key~$k$ and transaction identifier $v' \in
I$. 
This function is implemented using a standard VBT range-query operation with
latch-coupling applied when traversing in the index.
After finding all the possibly relevant updates, the \alg{key-query-VBT}
function orders the updates in transaction commit order, and returns the
value $\updatemark'$ of the latest update.
If there are no updates on the key~$k$ in the VBT, the function
returns the null marker~$\nullmark$. 
If the last update is a deletion, the returned value is $\updatemark' = 
\deletemark$.
These values must be separate so that the single-key-query algorithm
knows whether to continue searching from the TMVBT or not.

In the example database of \figref{fig:cmvbt-contents}, the
reverse \abbr{ITC} mapping when querying for commit-time version~\num{5}
with a read-only transaction~$T_r$ is $\{101 \to 5, 103
\to 4\}$. 
If the querying transaction is an updating transaction~$T_u$ with
transaction identifier $\txid{T_u} = 104$, the
\abbr{ITC} mapping is $\{101 \to 5$, $103 \to 4$,
$104 \to \infty\}$. 
For example, if $T_u$ queries for key~\num{4} (at transient commit-time
version~\num{5}), the \abbr{ITC} is generated as explained
above, and the transaction identifier list would be~$I = \{101, 103, 104\}$.
The \alg{query-all-VBT} operation finds the entries $(4, 103, \deletemark)$
and $(4, 104, w_4')$, which are then converted to commit-time entries
and placed in the ordered result map $K_c = \{4 \to
\deletemark, \infty \to w_4' \}$.
The query returns the value with the highest key from this map, $w_4'$, as
the result.

\begin{algorithm}[htb]
$\proc{key-query-VBT}(k, v, T)$:

\begin{algorithmic}[1] 
\STATE $\textrm{\abbr{ITC}} \gets \emptymark$ 
\STATE $I \gets \emptymark$
\FOR{each $v_c \in \{ \stablever + 1, \ldots, v \}$}
  \STATE $v_i \gets \text{CTI}[v_c]$
  \COMMENT{Find the transaction identifier}
  \STATE $\textrm{\abbr{ITC}}[v_i] \gets v_c$
  \STATE $I \gets I \cup \{ v_i \}$
\ENDFOR
\IF{$T$ is an updating transaction}
  \STATE $\textrm{\abbr{ITC}}[\txid{T}] \gets \infty$
  \COMMENT{Updates by $T$ have precedence}
  \STATE $I \gets I \cup \{ \txid{T} \}$
\ENDIF
\STATE $K_i \gets \proc{query-all-VBT}(k, I)$
\STATE $K_c \gets \emptymark$
\FOR{\textbf{each} $(k, v', \updatemark') \in K_i$}
  \STATE $v_c \gets$ \abbr{ITC}$[v']$
  \STATE $K_c[v_c] \gets \updatemark'$
\ENDFOR
\RETURN entry $K_c[v_c]$ with highest version $v_c$; or \nullmark, if $K_c$
is empty
\end{algorithmic}
\caption{Algorithm for querying for a single key from the VBT index.}
\label{alg:key-query-vbt}
\end{algorithm}

When a user transaction~$T$ is querying for a single key~$k$ of a transient
version $v > \stablever$, and the maintenance transaction~$T_m$ is ongoing,
one of the following situations may occur, regarding an update of
version~\movever\ when that update is the latest update on the key~$k$:
\begin{enumerate}
\setlength{\itemsep}{0pt}

\item $T$ finds the pending update in the VBT\@.
The update has not yet been applied to the TMVBT index.
This is the normal situation, and no special processing is required.

\item $T$ finds the pending update in the VBT\@.
The update has been applied to the TMVBT by the maintenance
transaction $T_m$. 
Because the pending update was found in the VBT, the TMVBT is not
searched, and thus this situation is similar to the first one.

\item $T$ does not find the pending update of version~\movever\ in the
VBT, because the maintenance transaction $T_m$ has already deleted it. 
Because no update was found in the VBT, the TMVBT is searched to find
the latest update.
At this point, the maintenance transaction $T_m$ has already applied the
update to the TMVBT, so the update is found in there.
\end{enumerate}

If there is a more recent update with commit-time version~$v_c$ such that 
$\movever < v_c \leq \snapver{T}$, then this update will be found
in the VBT in all of the above situations, and it will be
returned directly without consulting the TMVBT index.

For next-key queries, the general algorithm follows the same structure as the
single-key query: if a read-only transaction is querying for a stable version,
the algorithm \alg{next-key-stable} is used to query the TMVBT index directly. 
If the queried version is transient, or if an updating transaction is
performing the range-query, the algorithm~\alg{next-key-transient}
(\algref{alg:next-query}) is run.
In this case, both the VBT and the TMVBT must always be searched to locate 
the data item with the next key.

\begin{algorithm}[!htb]
$\proc{next-key-transient}(k, v, T)$:

\begin{algorithmic}[1]
\STATE $k_c \gets k$ 
\COMMENT{Initialize the current key}
\label{alg:next-query:init-keys}
\LOOP 
  \STATE $(k_1, \updatemark_1) \gets \proc{next-key-VBT}(k_c, v,
  T)$ 
  \STATE $(k_2, w_2) \gets \proc{next-key-stable}(k_c, v,
  \TRUE)$
  \IF[Data item with next key is in TMVBT]{$k_1 > k_2$}
  \label{alg:next-query:next-in-tmvbt}
    \RETURN $(k_2, w_2)$
  \ELSIF[Pending update for the next key is in VBT]{$k_1 < k_2$}
  \label{alg:next-query:next-in-vbt}
    \IF[Is $\updatemark_1$ an insertion or update]{$\updatemark_1 \neq
    \deletemark$} 
      \RETURN $(k_1, \updatemark_1)$
    \ENDIF
  \ELSE[Same key returned from both structures]
  \label{alg:next-query:next-in-both}
    \IF[No next keys in either index]{$k_1 = \nullmark$}
      \RETURN \nullmark
    \ELSIF[Is $\updatemark_1$ an insertion?]{$\updatemark_1 \neq
    \deletemark$} 
      \RETURN $(k_1, \updatemark_1)$
    \ENDIF
  \ENDIF
  \STATE $k_c \gets k_1$
  \COMMENT{Scan forward in both indexes}
\ENDLOOP
\end{algorithmic}
\caption{Algorithm for finding the next live data-item starting from a
given key.}
\label{alg:next-query}
\end{algorithm}

When searching for a transient version, the next keys from both structures
need to be retrieved alternatingly. 
In \algref{alg:next-query}, starting from 
line~\ref{alg:next-query:init-keys}, the current key is initialized to the
previously found key. 
After this, at the beginning of the infinite loop, both of the index
structures are searched to find the next key, and the algorithm checks which
key is nearest to the previously found key. 
If the next nearest key is found in the TMVBT
(line~\ref{alg:next-query:next-in-tmvbt}), we can return the key and
the corresponding value directly. 
If, on the other hand, the nearest key is in a pending update found in the
VBT (line~\ref{alg:next-query:next-in-vbt}), we must check whether the
pending update is an insertion (and not a deletion) before returning the key.
Similarly, if the keys fetched from both structures are the same
(line~\ref{alg:next-query:next-in-both}), we need to check that the
pending update on the key in the VBT is not a deletion.
If the pending update is a deletion, we need to skip this key and scan
forward to find the next key.
Whenever a pending deletion is found, and the indexes need to be scanned
forward, both of the structures need to be queried again.
This means that the tuple $(k_2, w_2)$ found from the TMVBT cannot be reused,
because an ongoing maintenance transaction might have changed the nearest key
in the TMVBT index by inserting a new entry $(k_2', w_2')$ with $k_2' < k_2$
into the TMVBT\@.

Like the \alg{key-query-VBT} function, the \alg{next-key-VBT}
function needs to find the transaction identifiers of all the commit-time
versions between \stablever\ and~$v$, find pending updates with these
transaction identifiers, and order the updates based on their corresponding
commit-time versions.
The actual implementation of the next-key query should use saved paths to
accelerate the next-key queries from the VBT and TMVBT indexes.
With saved paths, most of the consecutive next-key calls to the VBT and CMVBT
indexes will fall to the same leaf page, and they will reuse the existing
saved path without requiring any additional \abbr{I/O} operations.
To further enhance the operation, page fixes and latches in the VBT and TMVBT
index structures need not be released between the next-key-query
sub-operations in the main loop of \algref{alg:next-query}.

As an example of a next-key query, suppose a read-only transaction~$T_r$ is
querying for the key next to key~\num{3} at version~\num{4}, in the example
database of \figref{fig:cmvbt-contents}.
At the beginning of the next-key-query function, the next entries
are fetched from both structures: $(k_1, \updatemark_1) = (4, \deletemark)$
(from the VBT) and $(k_2, w_2) = (4, w_4)$ (from the TMVBT)\@.
Because the keys of both of the entries are the same, the one retrieved from
VBT takes precedence.
However, because it is a deletion marker, we need to continue the
search to find the next entries.
The algorithm thus continues by finding the entries next to key~\num{4} from
both of the structures: $(k_1, \updatemark_1) = (7, w_7)$ (from the VBT) and
$(k_2, w_2) = \nullmark$ (from the TMVBT)\@.
Because the TMVBT has no more entries, the VBT entry is the next key, and the
key-value pair $(7, w_7)$ is returned to the user.

The following theorems state the correctness and complexity of the user
actions, expressed in terms of how many pages need to be visited:

\thmskip
\begin{theorem}
\label{theorem:cmvbt:update} 
The update action correctly records a key update (key insertion or
deletion) in the VBT\@. 
A single update action has a cost of \OhT{\log_\capacity n_V} page accesses,
where $n_V$ is the number of entries in the VBT\@.
\end{theorem}
\begin{proof}
The update action must traverse the VBT once to insert the update marker, so
the complexity \OhT{\log_\capacity n_V} comes from the tree traversal.
As with standard \Btree{}s, any required structure-modification
operations do not affect the asymptotic complexity of the action.
\end{proof}
\thmskip

\thmskip
\begin{theorem}
\label{theorem:cmvbt:key-query} 
The key-query action for key~$k$ and version~$v$ correctly returns the
most recent committed version~$v'$ of the queried key, relative to
version~$v$ so that $v' \leq v$.
A single key-query action for a stable version~$v$ has a cost of
\OhT{\log_\capacity \entries{v}} page accesses, where \entries{v} is the
number of entries in the TMVBT index that are alive at version~$v$, and
\capacity\ is the page capacity.
A single key-query action for a transient version~$v$ requires at
most \OhT{\log_\capacity n_V + (n_a + n_t)/\capacity + \log_\capacity
\entries{\stablever}} page accesses, where $n_V$ is the number of entries in
the VBT, \entries{\stablever} is the number of entries in the TMVBT that are
alive at the stable version~\stablever, $n_a$ is the number of active
updating transactions, and $n_t$ is the number of transient versions.
\end{theorem}
\begin{proof}
The key-query action for a stable version performs a single-key query
action on the TMVBT, and thus has a cost of \OhT{\log_\capacity
\entries{v}} page accesses (\thmref{thm:tmvbt-query-cost}). 
For a transient version, the VBT search requires \OhT{\log_\capacity n_V}
page accesses for the initial tree traversal, and at most an
additional \OhT{(n_a + n_t)/\capacity} leaf page accesses to locate all
pending updates that might be relevant to the key query.
The latter part of the cost is derived from the fact that there can be at 
most $n_a + n_t$ pending updates that have the key~$k$ in the VBT index,
and all of these may have to be scanned.
Because the entries with the same key are clustered together, it
suffices to scan $(n_a + n_t) / \capacity$ VBT pages.
Finally, the query for a transient version may need to further query the
stable version from the TMVBT, thus adding the \OhT{\log_\capacity
\entries{\stablever}} term to the complexity of the action.
\end{proof}
\thmskip

\thmskip
\begin{theorem}
\label{theorem:cmvbt:range-query} 
A range-query action for the key range $[k_1, k_2)$ targeting a
stable version~$v$ has a cost of \OhT{\log_\capacity \entries{v} +
r/\capacity} page accesses, where \entries{v} is the number of entries in the
TMVBT index that are alive at version~$v$, $r$ is the number of entries
returned by the query, and \capacity\ is the page capacity.
A range-query action targeting a transient version~$v$ has a cost of
at most \OhT{\entries{k} \log_\capacity n_V + \log_\capacity
\entries{\stablever} + r/\capacity} pages, where \entries{k} is the maximum
number of discrete keys at the queried interval (for databases that store
integer keys, $k = k_2 - k_1$), and $n_V$ is the number of entries in the
VBT\@.
\end{theorem}
\begin{proof}
The range-query action of a stable version queries the TMVBT index directly,
and the proof is thus the same as the proofs of
Theorems~\ref{thm:tmvbt-query-cost} and~\ref{thm:mvbt-cost}.
When querying for a transient version, it is possible that there are
transient pending deletions in the VBT recorded for every possible discrete
key value in the queried range, and the algorithm must process them all.
In such a situation, the \alg{next-key-transient} algorithm needs to perform
next-key queries for each possible discrete key value in the queried range.
Each of these queries requires at most a single root-to-leaf traversal of
both the VBT and the TMVBT\@.
In the VBT, it is possible that the saved path cannot be efficiently reused
to obtain the next key directly, because there may be other pending updates
with the same key but different versions in the way.
In the TMVBT, however, the saved path will be reused and the same number of
pages need to be accessed in total as when querying for the stable version.
\end{proof}
\thmskip



%% Maintenance Transaction
%%---------------------------------------------------------------------
\section{Maintenance Transaction}
\label{sec:cmvbt:maintenance-tx}

The maintenance transaction is a system-generated transaction that is run
periodically to apply the pending updates of committed transactions, one
transaction at a time, into the TMVBT index, and to delete the pending
updates from the VBT\@.
To ascertain correct operation with concurrent user transactions,
the updates must be applied in such a way that the system transaction
does not cause user transactions to miss any of them.
This is accomplished by first applying the pending updates into the
TMVBT, then increasing the stable version variable~\stablever,
and only then removing the pending updates from the VBT index.
A consequence of this approach is that the user transactions must be
prepared to possibly encounter the same update twice when scanning
the index structures. 

The maintenance transaction performs the following steps:
\begin{enumerate}
\setlength{\itemsep}{0pt}

\item 
\label{maintenance:movever} 
Acquire a commit-duration write lock on the global variable~\movever, update
the variable $\movever \gets \stablever + 1$, and find out the
corresponding transaction identifier $v_i \gets \text{CTI}[\movever]$.
Reading of the stable version variable~\stablever\ is protected by 
the read latch taken on the database page on which the variable is 
stored.

\item
\label{maintenance:copy} 
Scan through the VBT index to find the pending updates $(k, v_i,
\updatemark)$ created by the transaction $T$ with $\txid{T} = v_i$.
Apply the updates into the TMVBT index, changing the version
from~$v_i$ to~\movever.

\item 
\label{maintenance:stablever} 
Update the stable version to $\stablever \gets \movever$.
Updating the variable is protected by taking a write latch on the database
page on which the variable is stored. 

\item 
\label{maintenance:delete} 
Scan through the VBT index a second time, and physically delete all the
entries of the form $(k, v_i, \updatemark)$.

\item 
\label{maintenance:finish} 
Remove the mapping $\movever \to v_i$ from the CTI table.

\end{enumerate}

The actions performed by the maintenance transaction are logged using
redo-only log records. 
If the system crashes during the execution of the maintenance
transaction~$T_m$, the redo pass of restart recovery will redo all
actions performed by~$T_m$ to bring the database pages into a consistent
state and restart the maintenance transaction.
All the steps of the maintenance transaction are idempotent, meaning that
performing them multiple times has the same effect as performing them once. 
That is, $f(f(x)) = f(x)$ for all actions~$f$ of the maintenance
transaction $T_m$ and for all possible states~$x$ of the combined CMVBT
(TMVBT + VBT) index structure.
When the maintenance transaction is restarted after a system crash, it will
automatically skip those actions that already have been performed.

At the beginning, in step~\ref{maintenance:movever}, the move
version~\movever\ is incremented to record that the maintenance transaction is
active.
Using the actions described for the TMVBT index in \secref{sec:tmvbt:actions},
the maintenance transaction invokes the \action{begin-update} action of the
TMVBT to update the active version variable~\actver\ of the TMVBT to
\movever. 
In practice, however, the active version variable should be the same as the
move-version variable, when TMVBT is used as a part of the CMVBT index.

The database management system must guarantee that there is only a
single maintenance transaction running at any time.
Thus, at the very beginning, the maintenance transaction takes a
commit-duration write lock on~\movever.
This action is logged by a redo-only log record
\lrb{T_m}, \logact{begin-maintenance}, \movever, \lre{v_i}, where $v_i$ is the
transaction identifier of the transaction that has the commit-time
version~\movever.

At the next step, step~\ref{maintenance:copy}, the pending updates recorded
in the VBT are applied to the TMVBT\@.
The maintenance transaction scans through both structures at the same
time, using a saved path for the TMVBT and latch-coupling for the VBT\@.
Because the system transaction is the only transaction updating the TMVBT, no
latch-coupling on the TMVBT is required, and only one page needs to be
latched at a time, except during structure-modification operations.
Because the TMVBT does not have sibling links, the saved path needs to be
backtracked to locate the next leaf page.
Pages on the saved path can be safely relatched, because there is no other
transaction that can update the TMVBT, but pages lower on in the path must
still be unlatched before latching pages on a higher level to maintain
top-down, left-to-right ordering on page latching.
In the \abbr{VBT}, the sibling links assumed in
\secref{sec:cmvbt:organization} are used to traverse through the leaf
pages efficiently.
Because $T_m$ only reads the updates from the VBT at this point,
it is sufficient to perform a leaf-level scan, and thus no saved path is
required.

A pending update $(k, v_i, \updatemark)$ is applied on the TMVBT by using the
actions defined in \secref{sec:tmvbt:actions}.
If $\updatemark \neq \deletemark$, the action \action{write}$(k, 
\updatemark)$ is performed; otherwise the action \action{delete}$(k)$ is
invoked.
Each update that is performed to the TMVBT is logged using redo-undo
log records. 
The log records defined in \secref{sec:tmvbt:actions} could also be used, but
they contain undo information which is not needed, because the actions of
the maintenance transaction are never undone.
The redo-only log record written for a write action, recorded by the pending
update $(k, v_i, w)$, $w \neq \deletemark$, is thus
\lrb{T_m}, \logact{apply-write}, $p$, $k$, \movever, \lre{w}, where $p$ is
the page identifier of the TMVBT page on which the update was performed. 
Note that the version~\movever\ is used by the TMVBT algorithms when applying
the update to the TMVBT index, because we require that $\actver = \movever$.
The redo-only log record written for a delete action, recorded by the pending
update $(k, v_i, \deletemark)$, is \lrb{T_m}, \logact{apply-delete}, $p$,
$k$, \lre{\movever}.

Step~\ref{maintenance:stablever} updates the stable version
variable~$\stablever \gets \movever$. 
The variable is protected by write-latching the page on which the
variable is stored.
At this point we know that all the updates of transaction~$T$ with
$\commitver{T} = \movever$ have been applied to the TMVBT\@. 
New read-only transactions can therefore perform queries that target
the version~\movever\ directly on the TMVBT index.
% This step therefore acquires a write latch on the page containing the
% variable, update the variable to $\stablever \gets \movever$, and release the
% latch.
To speed up recovery, this action is also logged with a single
redo-only log record \lrb{T_m}, \logact{increment-stable}, \lre{\stablever},
and the log is forced onto the disk at this point.
If this log record is found after a system crash,
steps~\ref{maintenance:movever}--\ref{maintenance:stablever}  
of the maintenance transaction are skipped entirely, and the
maintenance transaction continues at step~\ref{maintenance:delete}
to finish the transaction.

Next, at step~\ref{maintenance:delete}, the pending updates are deleted from
the VBT index.
It is safe to delete these updates, because although deleting the entries may
cause concurrent user transactions to miss an update they expected to find in
the VBT, the transactions will find the missed update later on from the
TMVBT, where it has already been applied to at this point.
To correctly maintain the structural consistency of the VBT, the tree must be
traversed from the root, using a saved path, so that structure-modification
operations can be performed if pages contain too few entries after entry
deletions.
During the search, latch-coupling will be applied, first top-down, then
left-to-right, as explained earlier. 
The delete action for a pending update $(k, v_i, \updatemark)$ is logged with
a redo-only log record \lrb{T_m}, \logact{delete-update}, $p$, $k$, \lre{v_i},
where $p$ is the page identifier of the VBT page on which the pending update
was located.

Finally, in step~\ref{maintenance:finish}, the temporary transaction
identifier mapping is removed from the CTI table.
The maintenance transaction commits by writing a redo-only log record
\lrb{T_m}, \lre{\logact{commit-maintenance}}. 
The log must be forced onto disk at this point.

The following theorem states the correctness and complexity of the
maintenance transaction:

\thmskip
\begin{theorem}
\label{theorem:cmvbt:maintenance} 
Let~$n$ denote the number of updates applied by the earliest
transient committed transaction with version~\movever, $n_V$ the
number of pending updates in the VBT and $n_T = \entries{\movever-1}$ the
number of entries in the TMVBT that are alive at version~$\movever - 1$. 
The maintenance transaction correctly transforms the transient version
\movever\ into a stable version by applying the updates of the version
\movever\ into the TMVBT, and by removing the pending updates from
the VBT\@. 
The maintenance transaction requires access to at most \Oh{n_V/\capacity +
\min \{ n \log_\capacity (n_T + n), (n_T + n)/\capacity\}} pages, where
\capacity\ is the page capacity.
\end{theorem}
\begin{proof}
The complexity of the maintenance transaction is composed of the VBT and
TMVBT index scans of steps
\ref{maintenance:copy}~and~\ref{maintenance:delete}.
The complexity of both scans of the VBT is the same, \OhT{n_V/B}, because a
full leaf-level scan of the VBT is required for both of the steps. 
Maintaining the entire saved path from root to leaf for the VBT in the
deletion phase does not add to the asymptotic complexity of the scan (see
the proof of \thmref{thm:btree-range-cost}). 
Applying the $n$ updates into the TMVBT requires \Oh{n \log_\capacity
n_T} page accesses in the worst case, because a single update operation in the
TMVBT requires \OhT{\log_\capacity (n_T + n)} page accesses (see
\secref{sec:tmvbt:actions}).
However, the leaf pages of the search tree $S_{\movever-1}$ of the latest
version of the TMVBT are accessed at most once, so the operation is also
bound by \Oh{(n_T + n)/\capacity}; that is, a full leaf-level page scan of
the latest version of the TMVBT plus the new pages created by insertions.
\end{proof}
\thmskip

Throughout this chapter, we have assumed that a single thread or process is
running a single maintenance transaction to apply the updates of a
single committed transaction from the VBT into the TMVBT at a time.
If the maintenance transaction becomes a bottleneck for the system, it should
be possible to extend the algorithms presented here so that multiple threads or
processes run multiple maintenance transactions concurrently. 
The multiple maintenance transactions still have to apply all the updates of a
single committed transaction at a time, before starting to move the updates of
another committed transaction, because only the updates of a single version can
be applied to the TMVBT at a time (see \secref{sec:tmvbt:multiupdate}).
In this approach, the maintenance transactions are set up to scan
different portions of the VBT index each, and to apply the updates to the TMVBT
concurrently.
Because there are now more then one transaction updating the
TMVBT index, all the updating transactions have to perform latch-coupling while
traversing the TMVBT index, because the active pages may be modified by a
concurrent instance of the maintenance transaction.

Another approach for multithreading the maintenance transaction is to allow
step~\ref{maintenance:delete} of the maintenance transaction for a
version~$v_1$ to run concurrently with step~\ref{maintenance:copy} for the
next version~$v_2$. 
This means that the updates of version~$v_2$ can be applied into the TMVBT at
the same time as the pending updates of version~$v_1$ are deleted from the
VBT by another thread, or group of threads.



%% Summary
%%---------------------------------------------------------------------
\section{Summary}
\label{sec:cmvbt:summary}

We have now described the concurrent MVBT (CMVBT), a general-purpose
multiversion database structure that can be used concurrently by multiple
updating transactions.
The CMVBT is optimal when querying for stable versions, provided that the
correct root page of the TMVBT index is known (see
\secref{sec:tmvbt:summary}).
When querying for transient versions, the separate VBT index must also be
searched.
However, because the VBT is expected to remain in main memory during normal
transaction processing, these queries should not require any additional
disk \abbr{I/O} operations.

The update actions of user transactions are efficient, because the pending
updates are inserted into the small VBT index.
This improves the latency of transactions, but does not increase the overall
system throughput, because the maintenance transaction still needs to apply
the pending updates into the main TMVBT index.
As we will show in the next chapter, the overall performance of the
CMVBT is on par with the \TSBtree\ of Lomet and
Salzberg~\cite{lomet:1989:tsb,lomet:1990:tsb-performance,lomet:2005:immortaldb,lomet:2006:transactiontime,lomet:2008:version-compression,lomet:2009:improving},
indicating that the main-memory-resident VBT index does not affect the
performance of CMVBT significantly.

