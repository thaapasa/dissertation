%%=====================================================================
%% Introduction
%%=====================================================================
\chapter{Introduction}
\label{chapter:introduction}

% Sections are not needed in introduction.

% Some notes:
% - 1st paragraph: why do we need versions?
% - Explain how different applications use versions
% - Optimality not too soon, some definition needed
%   - Same for non-optimality
% - Goal goes to the end of the introduction, not at the beginning


% General introduction to the subject (we need data history)
Modern society has become dependent on large software systems that
rely on databases for storing important data.
These database systems traditionally maintain only a single version of the
stored data set. 
Updates to the database erase the information content of any
previous states, and therefore these databases cannot be
used efficiently by any application that needs to query the history
of the database.
There are many applications, however, that do require access to
previous database states, such as trend-processing and inventory control
applications.
%For example, online analytical processing (\abbr{OLAP})
%applications need to query historical data to find out commercially
%relevant trends or patterns of behaviour,  
%engineering design applications need to store the different versions of a
%product's design,
%and inventory control applications need to have access to the
%previous states of the inventory.
Furthermore, any business-critical applications would benefit from the
accountability and traceability that is obtained when the past database
states are not deleted, so that it is always possible to find out
\emph{if} a change to the database state has occurred, and \emph{what} was
changed.

% More details on applications that need access to data history
In general, we can identify several major application areas
that require access to past database states. 
These are, for example, applications that deal with evolving data,
such as engineering design, land register, and scheduling applications,
inventory control systems and moving object
databases~\cite{becker:1996:mvbt,bercken:1996:multiversion,ozsoyoglu:1995:survey};
statistical applications such as decision support systems and
online analytical processing (\abbr{OLAP})
applications~\cite{bercken:1996:multiversion,jensen:2001:timestamping};
and applications that require accountability, such as medical,
legal, financial, accounting, banking, and insurance
applications~\cite{jensen:2001:timestamping,lomet:1989:tsb,ozsoyoglu:1995:survey}.
It is theoretically possible to reconstruct past database states
based on database logs and system backups, but this is time-consuming
and costly, and thus not feasible for everyday statistical queries.
The applications listed here need a database structure that is
specifically tailored for storing the evolution of the data
items alongside with the data items themselves.
These kind of databases are generally called \emph{temporal
databases}~\cite{bohlen:1998:consensus}.

% What is time in temporal databases?
When new updates are made to a temporal database, the database states are
updated to reflect the state of the world at different time instants.
In temporal database theory, there are two orthogonal
definitions for system-supported time: transaction time and valid
time~\cite{bohlen:1998:consensus}.
\emph{Valid time} is used to record the time when a modelled event happens in
the real world, and \emph{transaction time} is used to mark the time when an
event or a data item has been recorded in the database. 
Temporal databases that support both of these time dimensions are
called \emph{bitemporal databases}~\cite{bohlen:1998:consensus}.
A traditional database that does not incorporate either valid time or
transaction time is called a \emph{snapshot database}.

% Persistence
Some transaction-time databases allow data items to evolve along
different time lines, so that a new version may be based on any
previous version, not just the latest current version.
This time model is required by applications such as engineering design
software and version control systems, and the databases that
follow this model are called \emph{fully
persistent databases}~\cite{salzberg:2004:framework}. 
Most of the temporal databases, however, are designed to work with a single
linear data history, in which a new version of the data set is always created
based on the most recent database version.
These database systems are called \emph{partially persistent databases}, and
they are used by applications that record events that happen in the
real world.
Partially persistent transaction-time databases are called
\emph{multiversion databases} by many
authors~\cite{becker:1996:mvbt,bercken:1996:multiversion,jouini:2007:multiversion,lomet:1989:tsb,varman:1997:multiversion},
and the different transaction-time instants in these databases
are called \emph{versions}.
The database thus starts with an initial version, and any updates to the
database create new versions of the stored data item set.
The most recent version of the database is called the \emph{current version}
of the database.
In this dissertation, we concentrate our research on multiversion
databases.

% What kind of index structures there are for temporal databases?
Temporal databases are not a new idea.
The need for supporting the time dimension in database systems has
been long identified; for example, Bubenko has presented a conceptual
framework for incorporating the temporal dimension into database
management systems some thirty years ago~\cite{bubenko:1977:temporal}.
Consequently, there now exists a multitude of different temporal
database index structures.
Most of these are specifically tailored for
some operations, and they are either inefficient for other standard
database operations, or their performance may deteriorate with some
histories of user actions.
Because of this, these structures are not suitable for use as 
general-purpose temporal indexes.
For example, multiversion index structures that are based on hashing
techniques are very efficient for single-key
queries~\cite{kollios:2002:hashing}, but they cannot be used for
efficient key-range queries.
Similarly, tree-duplicating 
%and path-duplicating 
indexes can be queried efficiently~\cite{salzberg:1999:comparison}, but
updating them is inefficient.
To our knowledge, the most efficient multiversion 
indexes are the time-split \Btree~(\TSBtree) of Lomet and
Salzberg~\cite{lomet:1990:tsb-performance,lomet:1989:tsb}, the multiversion
\Btree~(MVBT) of Becker et~al.~\cite{becker:1993:optimal,becker:1996:mvbt},
and the multiversion access method~(MVAS) of Varman and
Verma~\cite{varman:1997:multiversion}.

% Definition of efficiency and intro to B-trees
The comparison of different multiversion index structures is
naturally dependent on how we define the efficiency of a given index
structure. 
One such ranking is based on the \emph{asymptotic optimality} of the
index operations.
That is, if we can define what the performance of an optimal
multiversion index structure is, we can compare the structures by
analyzing how much their performance differs from the optimal
performance.
In single-version (snapshot) databases, the most commonly used index
structure is the 
B-tree~\cite{bayer:1972:btree,bayer:1977:btree-concurrency,comer:1979:btree}. 
The B-tree has many variants, of which the \Btree\ is the most
often used.
The \Btree\ is regarded to be an asymptotically optimal database
index structure for snapshot databases.
All single-key operations on the \Btree\ (item query, insertion, and
deletion) require access to \OhT{\log_\capacity m} pages of the \Btree\
index~\cite{comer:1979:btree}, where $m$ is the number of data items indexed
by the \Btree\ and \capacity\ is the page capacity; and the range-query
operation requires access to \OhT{\log_\capacity m + r/\capacity} \Btree\
pages, where $r$ is the number of data items in the queried range. 
This holds for all histories of user actions and guarantees that the
performance of the \Btree\ index never deteriorates.

% Definition of MV index efficiency
When querying for data items of a given version~$v$, an ideal multiversion
index structure should work as efficiently as a single-version index
structure that only contains the items that are part of the queried
version~$v$.
Conceptually, an index structure that creates a new index tree for each
database version achieves this optimality for all query actions, assuming
that the root page of the queried version is known; but not for update actions.
The single-key operations in an asymptotically optimal multiversion index
structure may thus require access to at most \Oh{\log_\capacity
\entries{v}}~disk pages, where \entries{v} is the number of data items that
are part of the queried version~$v$; and the range-query operation may
require access to at most \Oh{\log_\capacity \entries{v} + r/\capacity}
pages. 
With this definition, only the
MVBT~\cite{becker:1993:optimal,becker:1996:mvbt} is optimal. 
The MVAS~\cite{varman:1997:multiversion} guarantees a slightly different
upper bound on the number of page accesses per operation (see
\secref{sec:tsbmvbt:mvas}), and the
\TSBtree~\cite{lomet:1989:tsb,lomet:1990:tsb-performance} has no guarantees if
the database history contains deletions.

% Key-ranges are important
%Another important query action for databases is 
%The key-range query is a crucial operation in databases.
% Efficient key-range queries are fundamental to a database index structure,
% because all index scan and index join operations are based on the key-range
% query.
% Again, with snapshot databases, a sparse \Btree\ index can be used to
% retrieve the $r$ data items that belong to a given key range by accessing at
% most \Oh{\log_\capacity m + r/\capacity} database pages, where $B$ is the
% number of entries that can fit into a single page.
% A dense \Btree\ index needs to access at most \Oh{\log_\capacity m +
% r/\capacity + r} = \Oh{\log_\capacity m + r} pages, because each of the $r$
% data items may be stored in different pages. 
% Because $r/B$ is the theoretical minimum number of pages that need to
% be accessed to retrieve $r$ data items from the pages of a sparse index that
% can contain at most $B$ entries each, and $r$ is the minimum number of pages
% that need to be accesses with a dense index if each data item is stored in a
% different page, the \Btree\ is also optimal for key-range
% queries. 
% Note that both dense and sparse indexes access only \Oh{\log_\capacity m +
% r/\capacity} pages of the index structure. 
% The \Oh{r} pages that a dense index may need to access are the data pages
% where the data items are stored, which are not part of the index structure.
% We therefore require that an optimal multiversion
% database index needs to access at most \Oh{\log \entries{v} + r/B} database
% pages for a key-range query that returns $r$~data items.
% The MVBT and the MVAS both meet this requirement. 

% More requirements for MV index
There are still other requirements for a multiversion database index
structure that is to be used in a general-purpose database management system.
Firstly, it must be possible to perform multiple updates within a
single version in the database to accommodate all the updates of a
multi-action updating transaction. 
Both the MVBT and the MVAS unfortunately require that each update creates a
new version, so that multiple updates always result in multiple versions of
the database.
Secondly, it must be possible to use concurrency-control and recovery
algorithms with the index structure so that multiple transactions
may query and update the index structure concurrently.
The MVBT and the MVAS only allow a single updating transaction to make
modifications to the structure at a time, and the transaction cannot
be rolled back.
The TBS-tree~\cite{lomet:1989:tsb,lomet:1990:tsb-performance} does
not have these restrictions, but it fails to guarantee optimality for any of
its operations if the database history contains deletions.

% Purpose of dissertation
The purpose of this dissertation is to design and implement an efficient
multiversion database index structure for indexing the evolution of the 
logical data set. 
The structure should be optimal, or as close to optimal as possible; and the
structure must be usable in a multi-user environment with multiple updating
and read-only transactions that are running concurrently.
We have selected the asymptotically optimal multiversion \Btree\ structure of
Becker et~al.~\cite{becker:1993:optimal,becker:1996:mvbt} as the basis of our
research work. 
Our index structure is based on the ideas we have presented in two
conference articles~\cite{haapasalo:2009:tmvbt,haapasalo:2009:cmvbt}
and one PhD workshop article~\cite{haapasalo:2008:concurrency}; 
I have been the primary author in all of these articles.
We have shown how to extend the MVBT structure into the \emph{transactional
multiversion \Btree} (TMVBT) so as to allow a single multi-action transaction
to operate on the structure at a time~\cite{haapasalo:2009:tmvbt}, but 
we have not been able to extend the TMVBT further so that multiple updating
transactions could operate on it concurrently, without compromising the
optimality of the index structure.
We have therefore used a separate main-memory-resident versioned \Btree\
(VBT) structure to store the updates of active transactions, so that a system
maintenance transaction can apply the updates of committed transactions to
the TMVBT successively, one transaction at a
time~\cite{haapasalo:2009:cmvbt}.
We call the combined index structure the \emph{concurrent multiversion
\Btree}, or CMVBT\@.

Logical data consistency in the CMVBT can be maintained by using any standard
multiversion concurrency-control algorithm, such as snapshot
isolation~\cite{berenson:1995:sql-critique,fekete:2005:snapshot-isolation,cahill:2008:snapshot}.
Our algorithms are based on the \abbr{ARIES} recovery
algorithm~\cite{mohan:1992:aries,mohan:1990:aries-kvl}, and we perform
structure-modification operations (SMOs) on all of the index
structures atomically, so that each SMO transforms a structurally valid and
balanced index structure into another structurally valid and balanced
index~\cite{jaluta:2003:recoverable,jaluta:2005:blink,jaluta:2006:page-server}.

We show in this dissertation that the combined CMVBT index is efficient:
the performance of the CMVBT is comparable to that of the \TSBtree\ in
general transaction processing, but the CMVBT outperforms the \TSBtree\ in
range queries when the database history contains key deletions.
The \TSBtree\ performs worse with key deletions because it does not merge
pages.
We further strengthen the test results by showing that the performance of the
combined CMVBT index organization is comparable to the performance of the
optimal TMVBT index.
In fact, with longer transactions, the CMVBT performs better, because the
main-memory-resident VBT index groups all the updates of a single transaction
together, thus enabling them to be inserted in the TMVBT as an efficient
batch update. 
Batch updating is commonly used to increase the performance of index
structures~\cite{pollari-malmi:1996:batch-updates,pollari-malmi:2000:differential-index}.

% Outline of dissertation
We begin the dissertation with a general introduction to multiversion
database theory in \chapref{chapter:mv-data}. 
After that, in \chapref{chapter:mv-index}, we describe the evolution of
multiversion index structures and present a survey of some of the existing
structures.
In \chapref{chapter:tsbmvbt}, we review three of the most efficient
multiversion structures; namely, the \TSBtree\ of Lomet and
Salzberg~\cite{lomet:1989:tsb,lomet:1990:tsb-performance}, the MVBT of
Becker et~al.~\cite{becker:1993:optimal,becker:1996:mvbt}, and the
MVAS of Varman and Verma~\cite{varman:1997:multiversion}.
\chapref{chapter:tmvbt} describes our transactional extension to
the MVBT, and shows how the TMVBT index retains the optimality of the MVBT\@.
In \chapref{chapter:cmvbt}, we present the concurrent multiversion \Btree,
which is a fully concurrent index structure that is composed of the
TMVBT index and a main-memory-resident versioned \Btree\ index 
that is used for storing the updates of active transactions. 
After that, in \chapref{chapter:performance}, we evaluate the performance of
the CMVBT index and compare it to the \TSBtree. 
We have implemented both indexes and present the results of our
test runs on them.
Finally, in \chapref{chapter:conclusions}, we give the conclusions
of the research work.
% and outline possible areas for future research.
 

