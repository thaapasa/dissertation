%%=====================================================================
%% Transactions on the MVBT
%%=====================================================================
\chapter{Transactions on the MVBT}
\label{chapter:tmvbt}
\label{def:tmvbt}

As shown in the previous chapter, there are efficient multiversion index
structures available, but there is no single structure that is both optimal
and that can be used in a concurrent transactional environment.  
We reviewed the result by Becker
et~al.~\cite{becker:1993:optimal,becker:1996:mvbt} that showed that the
multiversion \Btree~(MVBT) is an optimal multiversion index structure,
% if the root page of the queried version is known
% (\secref{sec:tsbmvbt:mvbt})
but it follows a single-update model, 
% in which a transaction can only consist of a single update that 
and the update cannot be rolled back.
In this chapter, we present our redesigned MVBT, called the
\emph{transactional multiversion \Btree} (TMVBT)\@.
The TMVBT adds transactions to the MVBT by redesigning the
structure-modification operations (SMOs) so that multiple data-item updates
can be performed within a single transaction, and the updates can be rolled
back.
% We show that the optimality of the index structure is preserved, and that
% each page still contains at least \minlive\ entries that are alive
% at version~$v$ for all versions in which the page is part of the search
% tree~$S_v$. 
% The extended structure-modification operations do not need to access any more
% pages than the MVBT structure-modification operations.
% We also show how concurrency-control and recovery algorithms can be used to
% allow a single updating transaction to operate on the index concurrently with
% multiple read-only transactions.
% The TMVBT is thus not yet an optimal general-purpose
% multiversion database index, because multiple updating transaction cannot
% operate of the index concurrently.
% Multiple updating transactions are discussed on the next chapter.
The TMVBT structure was first introduced in our previous
article~\cite{haapasalo:2009:tmvbt}.
In the discussion here, we explain the algorithms in more detail, and provide
detailed proofs for the properties of the structure.

We begin the chapter by describing the implementation of the transaction
model of Sections~\ref{sec:mv-data:read-only-tx}
and~\ref{sec:mv-data:updating-tx} for the TMVBT in
\secref{sec:tmvbt:multi-action-tx}.
After that, \secref{sec:tmvbt:active-entries} defines the concept of active
entries, which is needed for maintaining the optimality in the presence of
multi-action transactions, and
\secref{sec:tmvbt:structure} describes the structure of the TMVBT\@.
In \secref{sec:tmvbt:actions}, we show how the user actions are performed,
and in \secref{sec:tmvbt:smo}, we describe the structure-modification
operations triggered by the user actions.
Finally, in \secref{sec:tmvbt:multiupdate}, we illustrate why we cannot
allow multiple updating transactions to operate on the index concurrently,
and in \secref{sec:tmvbt:summary}, we summarize the discussion on the TMVBT
index.


%% Multi-Action Transactions
%%---------------------------------------------------------------------
\section{Multi-Action Transactions}
\label{sec:tmvbt:multi-action-tx}

We allow two kinds of transactions to operate on the TMVBT concurrently: 
any number of read-only transactions (as defined in
\secref{sec:mv-data:read-only-tx}) and at most one updating transaction (as
defined in \secref{sec:mv-data:updating-tx}) at a time. 
For reasons explained in \secref{sec:tmvbt:multiupdate}, we cannot allow more
than one updating transaction to operate on the TMVBT at a time.
Concurrent updating transactions are discussed in the next chapter.

In contrast to the MVBT (\secref{sec:tsbmvbt:mvbt}), each updating
transaction operating on the TMVBT can perform any number of updates, and the
updates all receive the same version.
Because only one updating transaction can operate on the TMVBT at a time, the
commit order of the transactions is known during the execution of the
transactions, and each data-item update can be directly performed with the
correct transaction-time version.
In the context of the TMVBT, we use the term \emph{version} to denote these
transaction-time instants. 
This does mean, however, that the version assigned to an updating
transaction cannot be based on the real time instant of the commit action,
because that is not known at the beginning of the transaction.
We thus assume that the versions used in the TMVBT are increasing
integer numbers that are assigned at the beginning of the updating
transaction.
The versions can be based on an increasing counter value or they can be based
on the real-time instant of the begin action of the transaction, as long as
they are increasing values.
For simplicity, in the discussions in this chapter, we assume that the
active version variable (defined below) is based on an integer value that is
incremented by one each time a new updating transaction begins.

Like the MVBT, the TMVBT also maintains a commit version variable~\comver\
that records the version of the latest committed transaction.
The TMVBT also maintains an active version
variable~\actver\phantomsection\label{def:actver} that holds the version
of the current updating transaction.
If there is no active updating transaction operating on the index, $\actver =
\comver$.
When an updating transaction starts, the active version variable is
incremented, $\actver \leftarrow \actver + 1$.
When the single updating transaction commits, the commit version is
incremented to match the active version variable $\comver \leftarrow
\actver$.
These version variables therefore tell whether there is an active updating
transaction running on the TMVBT index: if the active version variable is
larger than the commit version variable, then there is an updating
transaction running, and no other updating transaction can begin.
Read-only transactions can always target any version that is less than
or equal to the commit version~\comver, unless purging of old versions is
implemented (see discussion in \secref{sec:tsbmvbt:mvbt} and in the MVBT
articles~\cite{becker:1993:optimal,becker:1996:mvbt}).
In that case, the minimum version that can be accessed must also be
maintained in a separate variable.

The transaction model used for the TMVBT is the transaction model
explained in Sections~\ref{sec:mv-data:read-only-tx}
and~\ref{sec:mv-data:updating-tx}.
The log records written by a transaction~$T$ for the actions presented here
must also contain the transaction identifier \txid{T}.
Because the common form of log records
\lrb{T}, \logact{action}, \lre{\ldots} already contains the identifier~$T$,
we use $T$ to mean that the transaction identifier $\txid{T}$ is written in
the log records.
The control actions of the transaction model are implemented as shown below;
the query and update actions are discussed in \secref{sec:tmvbt:actions}.

\begin{itemize}
  \setlength{\itemsep}{0pt}

  % Begin
  \item \action{begin-read-only}$($version~$v)$: begins a new read-only
  transaction; this action takes a short-duration read lock on the~\comver\
  variable, checks if $v \leq \comver$, and records the value $\snapver{T}
  \leftarrow v$ for the transaction. 
  If $v > \comver$, the transaction is aborted.
 
  % Begin
  \item \action{begin-update}: begins a new updating transaction~$T$; this
  action takes a commit-duration write lock on the active version
  variable~\actver, increments the variable $\actver \leftarrow \actver +
  1$, and assigns it to the transaction: $\txid{T} \leftarrow \actver$.
  A redo-undo log record \lrb{T}, \logact{begin}, $v$, \lre{\actver} is
  written, with $v$ denoting the previous value of the variable~\actver,
  but the log is not forced to disk. 

  % Commit
  \item \action{commit-update}: commits the active updating transaction~$T$ 
  by  
  (1)~taking a commit-duration write lock on the committed version 
  variable~\comver, 
  (2)~updating the variable $\comver \leftarrow \actver$; 
  (3)~writing a log record \lrb{T}, \logact{commit}, \lre{\comver}; 
  (4)~forcing the log onto disk; and
  (5)~calling the release-version action.

  % Version release
  \item \action{release-version}: this action does nothing, because the
  updating transaction has already assigned correct versions to each updated
  data item entry.
  If the versions of the TMVBT index should be timestamps that are based on
  the time of the commit action (which is not known at the beginning of the
  updating transaction), then this action can perform the necessary changes
  to update the entries.

  % Abort
  \item \action{abort}: labels the updating transaction as aborted and
  starts the backward-rolling phase.
  This action writes the log record \lrb{T}, \lre{\logact{abort}}.

  % Finish-rollback
  \item \action{finish-rollback}: finishes the rollback of an aborting
  transaction by decrementing the active version variable $\actver \leftarrow
  \actver - 1$, writing a log record
  \lrb{T}, \logact{finish-rollback}, \lre{\actver}, and 
  forcing the log to disk.

\end{itemize}
 
% (1)~when an updating transaction~$T$ begins, the \actver\ variable is
% incremented and assigned to the transaction, $\txid{T} \to
% \actver$;
% (2)~when the updating transaction commits, the same version \actver\ is used
% as the commit-time version of the transaction; and
% (3)~the version release method does nothing, because all the items already
% have the correct version.

All the update actions of an updating transaction are logged using
the write-ahead logging protocol as in
\abbr{ARIES}~\cite{mohan:1992:aries}\@.
In addition to the log records described above, redo-undo log records are
written for an \action{insert} action and a \action{delete} action, 
while redo-only log records are written for an \action{undo-insert} action
and an \action{undo-delete} action.
These log records are described in \secref{sec:tmvbt:actions}. 
A read-only transaction does not create any log records; it only stores
transient control information in the active-transactions table when it
begins, and removes that information when it commits.


%% Active Entries 
%%---------------------------------------------------------------------
\section{Active Entries}
\label{sec:tmvbt:active-entries}

Recall from \figref{fig:mvbt-invalid-split} on
page~\pageref{fig:mvbt-invalid-split} the problem of inserting multiple
data-item entries into the MVBT index with the same version.
When performing a version-split operation on page~$p$, a new copy~$p'$ of the
page is created and the life span of the original page~$p$ is truncated to
the current version and the page is left as it is for use in historical
queries. 
If the page~$p$ was created by the same transaction that triggers the
version split, the life span of the page will degenerate into an empty range,
and the page will thus not be part of any search tree in the database.
In these situations, the key split can be performed directly on the
page~$p$, without applying the version-split operation first.

Let us now define the concepts of active entries and active pages to
classify the situations where a version-split operation is not required and
in fact must not be performed. 
Remember that the single updating transaction always has the version~\actver\
as its identifier and uses that version to stamp the data-item updates
and structure-modification operations.

\thmskip
\begin{definition}
\label{def:active-entries-pages}
An \emph{active entry} (or \emph{active page}, respectively) in the TMVBT
index is an entry (page) that has a life span of $[\actver, \infty)$. 
An active entry (page) has been created earlier on by the same updating
transaction.
Entries (pages) that are not active are called \emph{inactive entries}
(\emph{inactive pages}).
\end{definition}
\thmskip

As stated in the previous chapter, read-only transactions may only read
versions that have a commit-time version of at most \comver.
This leads to the following observation:

\thmskip
\begin{invariant}
\label{inv:tmvbt-read-only-inactive}
Read-only transactions in the TMVBT index only read inactive entries and
pages. 
Active entries and pages are only seen by the single active updating
transaction.
\end{invariant}
\thmskip

If the active updating transaction is deleting an active entry, the entry can
be physically removed from the index, instead of changing its life span.
This does not invalidate partial persistence, because the active entry was
created by the same transaction, and thus did not exist before the updating
transaction first inserted it.
Updates that are internal to the transaction are not visible outside
the transaction and must not consume space in the index.

\thmskip
\begin{invariant}
\label{inv:active-entries-physical-delete}
When a single updating transaction~$T$ deletes an active entry (created by
$T$), the entry is physically removed from the TMVBT index.
Similarly, if $T$ updates an active entry it physically removes the old entry
and creates a new active entry to replace the old one.
\end{invariant}
\thmskip

When performing a version-split operation on a page~$p$ at version~$v$ in the
original MVBT index, Becker et~al.\ suggested that the
entries that are left in page~$p$ may be left
unmodified~\cite{becker:1996:mvbt}, so that the life spans of the
entries~$e_i$ that were alive at version~$v$ remain unbounded on the above;
that is, of the form $[v_i, \infty)$, where $v_i < v$.
This does not affect any queries, because only historical queries targeting
versions $v' < v$ will ever end up in the historical page~$p$; thus, even if
an entry~$e_i$ is deleted by a transaction with a version $v'' > v$, the
queries targeting those newer versions will never encounter the now-outdated
entry~$e_i$ on the historical page~$p$. 
However, if a transaction~$T$ stores a previously used path as a saved path
(see p.~\pageref{def:saved-path}) and reuses the path later on, it is
possible that the pages in the saved path are no longer valid.
The transaction~$T$ cannot ascertain the validity of the pages unless the
consistency of the life spans of all the entries and pages is maintained, so 
that the deletion times of entries in historical pages are set to the
deletion time of the historical page. 
In the TMVBT index, we explicitly require that the life spans of entries that
are left on a historical page are cropped so that they end at the version
during which the page was split:

\thmskip
\begin{invariant}
\label{inv:tmvbt-live-entry-copy}
When a page~$p$ in the TMVBT index is version-split into a new page~$p'$ at
version~\actver, all live entries $(k, [v, \infty), w)$ such that $v <
\actver$ are processed as follows: 
a live copy $(k, [\actver, \infty), w)$ is created and inserted into the new
live page~$p'$, and the live entry at page~$p$ is changed to the historical
entry $(k, [v, \actver), w)$.
All active live entries of the form $(k, [\actver, \infty), w)$ are physically
moved to the new page~$p'$.
\end{invariant}
\thmskip

This invariant is required in order that the key-version regions of all
the entries of a given level of the TMVBT index do not overlap, as shown in
\figref{fig:mvbt-space-partition} on page~\pageref{fig:mvbt-space-partition}.
By adhering to these rules, we can also obtain the following lemmata: 

\thmskip
\begin{lemma}
\label{lemma:active-pages-active-entries}
Active pages in the TMVBT only contain active entries.
\end{lemma}
\begin{proof}
An active page is a page that was created by the active updating transaction.
When the transaction commits, the page immediately becomes inactive.
When an active page~$p$ was created, all the live entries that were copied to
it were changed so that their life spans start at the split boundary
(i.e., at version~\actver), thus making the copies of the entries active.
If the active entries are changed in any way by the same transaction, they
will be physically deleted or replaced by new active entries, as per
\invref{inv:active-entries-physical-delete}.
\end{proof}
\thmskip

\thmskip
\begin{lemma}
\label{lemma:active-pages-single-parent}
Active pages have at most one parent.
\end{lemma}
\begin{proof}
Multiple parents in a multiversion index are caused by multiple routers to
the same page~$p$ in the index pages above the page~$p$.
When an active page~$p$ is created, a new index entry $i_p$ is inserted to the
parent page~$p'$. 
Note that the index entry~$i_p$ is also active, and will remain active until
the current active transaction commits.
When the current transaction commits, both the entry~$i_p$ and the page~$p$
will immediately become inactive.
By \invref{inv:tmvbt-live-entry-copy}, active entries are physically moved
during a version-split operation.
If the parent page~$p'$ is version-split before the active transaction
commits, the index entry $i_p$ is physically moved to the new page,
thereby preventing the creation of new copies of~$i_p$.
Because there can be only a single index entry~$i_p$ pointing to an
active page~$p$, active pages can only have a single parent.
\end{proof}
\thmskip

All the entries of active TMVBT pages have the same life span of $[\actver,
\infty)$. 
This holds for both leaf pages and index pages, and is illustrated in
\figref{fig:tmvbt-active-entries}.
Because of this fact, we can in fact disregard the life spans of entries when
performing an operation on active pages: in effect, we can treat active pages
as if they were pages in a non-versioned \Btree\ index.
The extended TMVBT algorithms are based on this observation.
The algorithms themselves are explained in detail in \secref{sec:tmvbt:smo}.

\begin{figure}[!htb]
\begin{center}
\subfigure[Active index entries]{\begin{tikzpicture}[x=2em,y=2em,every node/.style={font=\small}]
  \small
  \coordinate (o) at (0,0);
  \coordinate (ok) at ($(o) + (0,5)$);
  \coordinate (ov) at ($(o) + (3,0)$);
  \coordinate (v0) at (o); 
  \coordinate[label=below:{\actverp}] (v1) at ($(o) + (1,0)$); 

  \draw (1,0) -- (1,4.5);
%  \draw (1,1) -- (2,1);
%  \draw (1,3) -- (2,3);

  \draw[opendash] decorate [opendeco] { (1,4.5) -- (2,4.5) };

  \draw[opendash] decorate [opendeco] { (2,0) -- (2,1) };
  \draw[opendash] decorate [opendeco] { (2,1) -- (2,3) };
  \draw[opendash] decorate [opendeco] { (2,3) -- (2,4.5) };

  \draw[dashed] { (1.1,0.1) -- (2,0.1) };
  \draw[dashed] { (1.1,0.1) -- (1.1,0.95) };
  \draw[dashed] { (1.1,0.95) -- (2,0.95) };

  \draw[dashed] { (1.1,1.05) -- (2,1.05) };
  \draw[dashed] { (1.1,1.05) -- (1.1,2.95) };
  \draw[dashed] { (1.1,2.95) -- (2,2.95) };

  \draw[dashed] { (1.1,3.05) -- (2,3.05) };
  \draw[dashed] { (1.1,3.05) -- (1.1,4.5) };

  \node at (1.55,0.5) {$\ptr p_1$};
  \node at (1.55,2) {$\ptr p_2$};
  \node at (1.55,3.75) {$\ptr p_3$};

  \draw[->] (o) -- node[below=1em,lbl] {Versions} (ov);
  \draw[->] (o) -- node[above,sloped,lbl] {Keys} (ok);
\end{tikzpicture}

\label{fig:tmvbt-active-entries:index}}
\subfigure[Active leaf entries]{\begin{tikzpicture}[x=2em,y=2em,every node/.style={font=\small}]
  \small
  \coordinate (o) at (0,0);
  \coordinate (ok) at ($(o) + (0,5)$);
  \coordinate (ov) at ($(o) + (3,0)$);
  \coordinate (v0) at (o); 
  \coordinate[label=below:{\actverp}] (v1) at ($(o) + (1,0)$); 

  \draw[|->] (1,1) -- node[lbl,above] {$e_1$} (2,1);
  \draw[|->] (1,2) -- node[lbl,above] {$e_2$} (2,2);
  \draw[|->] (1,3.5) -- node[lbl,above] {$e_3$} (2,3.5);

  \draw[dashed,->] (1,0) -- (1,4.5);

  \draw[->] (o) -- node[below=1em,lbl] {Versions} (ov);
  \draw[->] (o) -- node[above,sloped,lbl] {Keys} (ok);
\end{tikzpicture}

\label{fig:tmvbt-active-entries:leaf}}
\figcaption{Active entries in the TMVBT index}%
{The index page contains three index entries with routers to pages $p_1$,
$p_2$, and $p_3$.}
\label{fig:tmvbt-active-entries}
\end{center}
\end{figure}

For an example, let us review the problem scenario in MVBT as depicted in
\figref{fig:mvbt-invalid-split} on page~\pageref{fig:mvbt-invalid-split}.
In the TMVBT, the page~$p_1$ is active, and thus it can be key-split directly
without version-splitting it first.
The operation of the same transaction, executing on the TMVBT index, is shown
in \figref{fig:mvbt-invalid-split-solved}.

\begin{figure}[htb]
\begin{center}
  \input{images/mvbt-problem-solved}
  \figcaption{Key split without version split in the TMVBT}%
  {A key-split is triggered by the insertion of key~\num{4}.
  The leaf page contains three entries, namely $e_1$, $e_2$, and $e_3$.
  The format of the page header is key range, life span; 
  and the format of the entries is (key, life span, data).}
  \label{fig:mvbt-invalid-split-solved}
\end{center}
\end{figure}


%% Transactional Multiversion B-tree
%%---------------------------------------------------------------------
\section{Transactional Multiversion \Btree}
\label{sec:tmvbt:structure}

As we explained in the previous chapter, only the MVBT
index~\cite{becker:1993:optimal,becker:1996:mvbt} can be
considered optimal when updating transactions follow a single-update model,
although the MVAS of Varman and Verma has access cost guarantees that are close
to optimal ($m$-optimal, see \secref{sec:tsbmvbt:mvas}).
We have chosen the MVBT as the basis of our work, instead of the MVAS,
because
(1)~the page reusing rules of the MVAS make the structure of the pages more
complicated without improving the space complexity bounds, 
(2)~the lack of a separate \rootstar\ structure makes history queries less
efficient, and
(3)~the access list incurs a high maintenance cost.
Nevertheless, the improvements presented in this chapter could also be
implemented on the MVAS index structure.

The \emph{transactional multiversion \Btree} (TMVBT) index, which was first
introduced in our previous article~\cite{haapasalo:2009:tmvbt}, is a directed
acyclic graph with multiple root pages that is based on the multiversion
\Btree\ of Becker et~al.~\cite{becker:1993:optimal,becker:1996:mvbt}.
The original MVBT structure was reviewed in \secref{sec:tsbmvbt:mvbt}.
The different roots of the TMVBT index are stored in a \rootstar\ structure,
exactly as in the MVBT index.
The page format in the \abbr{TMVBT} is identical to that of the MVBT, with
the addition of recovery information required for our \abbr{ARIES}-based
recovery algorithm, such as a Page-LSN field that stores the log
sequence number (LSN) of the log record
of the latest update on the page. 
We assume that each page~$p$ explicitly stores the life span $\vr{p}$,
the key range $\kr{p}$, and also the height of the page.
%, denoted $\height{p}$. 
The height of a page is one for all leaf pages, and greater 
% $\height{p} > 1$ 
for index pages.

As discussed in \secref{sec:tsbmvbt:mvbt}, the MVBT has three variables that
determine how many live entries there are in each page and how often pages
are split or merged.
These variables are used in the TMVBT in the same meaning.
The variable \minlive\ determines the minimum number of live entries that
must be present in each live page (see \invref{inv:mvbt-live-count} on
page~\pageref{inv:mvbt-live-count}), and variables \minsplit\ and \maxsplit\
%, $\minsplit < \maxsplit$, 
control how many live entries must be present in each live page created by a
structure-modification operation. 
Becker et~al.\ use the term \emph{weak version condition} to refer to the
first requirement, and the term \emph{strong version
condition}\phantomsection\label{def:strong-version} to refer to the
second~\cite{becker:1993:optimal,becker:1996:mvbt}.

The variables \minsplit\ and \maxsplit\ are defined as 
$\minsplit = \minlive + s$, and $\maxsplit = \capacity - s$, where $s$~is
a \emph{split tolerance variable} that determines how many actions must at
least be performed on the page before a new structure-modification operation
is required.
If the strong version condition holds, then at least $s$~entries can be
deleted from the page before the number of live entries falls below \minlive,
and similarly at least~$s$ entries can be inserted to the page before the page
becomes full.
In effect, $s$~is used to prevent thrashing.
When a page has more than \maxsplit\ entries immediately after a
version-split, it will be key-split into two pages. 
We thus require that $\maxsplit \geq 2 \times \minsplit$ so that the two new
pages will have at least \minsplit\ entries each.

\thmskip
\begin{invariant}
\label{inv:tmvbt-minsplit-maxsplit}
All the live pages at level~$l$ that are involved in a structure-modification
operation that targets a page~$p$ at level~$l$ must contain
from \minsplit\ to \maxsplit\ live entries immediately after the \abbr{SMO}\@.
\end{invariant}
\thmskip

Note that these requirements do not need to hold for the parent page~$q$ at
level $l+1$, because only a small constant number of updates is applied to it
during any SMO\@.
The router entries in a parent page are only updated by the SMOs at a lower
level, so the updates performed on the parent page~$q$ correspond to inserting
or deleting entries from a leaf page.

The values chosen to the variables affect the size of the index structure and
the frequency of structure-modification operations. 
It is theoretically possible to set \minlive\ as high as $\capacity/2$, if $s
= 0$, but this means that thrashing is not prevented.
The upper limit of the value of $s$ is $\capacity/3$, but with this setting
$\minlive = 0$ and thus the optimality constraints are lost.
For the discussion in this dissertation, we assume that the following values
are used: $\minlive = \nicefrac{1}{5}\, \capacity$, $s = \nicefrac{1}{5}\,
\capacity$, $\minsplit = \minlive + s = \nicefrac{2}{5}\, \capacity$, and
$\maxsplit = \capacity - s = \nicefrac{4}{5}\, \capacity$.

Although the representations of the variables differ from the
definition used by Becker et~al., we can show that the variables are the
same as in the \abbr{MVBT} article~\cite{becker:1996:mvbt}. 
Becker et~al.\ require that $\minlive = d = \capacity/k$, $\minsplit =
(1 + \epsilon) \times d$ and $\maxsplit = (k - \epsilon) \times d$, where
$k$ and $\epsilon$ are variables that can be selected.
If we assign $s = \epsilon d$, we obtain $\minsplit = d + \epsilon d =
\minlive + \epsilon d = \minlive + s$, and $\maxsplit = k d - \epsilon
d = \capacity - s$, which are the definitions used here.

For optimality of the index structure, we wish to keep the structure of
the TMVBT index as close to the MVBT index as possible.
Most importantly, we wish to maintain \invref{inv:mvbt-live-count}, so that
all pages of each search tree~$S_v$ contain at least \minlive\ entries that
are alive at version~$v$, for all versions~$v$.
Let us first restate \invref{inv:mvbt-static-entries} for the TMVBT:

\thmskip
\begin{invariant}
\label{inv:tmvbt-static-inactive-entries}
All inactive entries in the TMVBT pages remain in place. 
They are never moved to another page.
Only the deletion time of an inactive entry may be changed, always to the
current active version~\actver.
Active entries in the TMVBT pages may be physically deleted,
updated, or moved to another page (see
Invariants~\ref{inv:active-entries-physical-delete}
and~\ref{inv:tmvbt-live-entry-copy}).
\end{invariant}
\thmskip

This means, in practice, that the structure of the search tree~$S_v$ of a
version~$v$ can only change when $v = \actver$.
After $T$ commits, version~$v$ becomes inactive, and the structure of
the search tree~$S_v$ becomes static.
By this we mean that the set of pages that forms the search tree~$S_v$ can
no longer change, and the entries that are alive at version~$v$ are 
never physically deleted or moved to another page. 
If we design the algorithms in such a way that the search tree of the active
version is balanced (\defref{def:consistent-balanced}) in all situations,
this implies that search trees of all versions are balanced, and thus
optimal.
This follows from the fact that the search tree~$S_v$ of the active
version is balanced immediately before the active transaction commits and thus
also at the moment version~$v$ becomes inactive. 
Furthermore, because inactive search trees are static, $S_v$ will always remain
balanced. 
We will show in the next sections that the TMVBT algorithms maintain 
\invref{inv:mvbt-live-count} for the TMVBT\@.
This, together with the observation that all the root-to-leaf paths in the
search tree~$S_v$ are always of the same length, implies that the balance
conditions of the active-version search tree are also maintained.

% TMVBT live count
\thmskip
\begin{invariant}
\label{inv:tmvbt-live-count}
\invref{inv:mvbt-live-count} holds for the TMVBT index.
That is, for all versions~$v$ and all pages~$p$, page~$p$ contains at least
\minlive\ entries that are alive at version~$v$; or $p$ is a root page of
$S_v$, in which case it contains at least \num{2}~entries that are alive at
version~$v$; or $p$ is the only page of $S_v$ and contains at least one entry
that is alive at version~$v$; or $p$ is not part of the search tree~$S_v$ and
therefore contains no entries that are alive at version~$v$.
\end{invariant}
\thmskip

Figures~\ref{fig:example-oper-1}--\ref{fig:example-oper-3} show
an example of the TMVBT page operations. 
In this illustrative example, the index is structurally consistent and
balanced, with suboptimal settings of $\minlive = 1$ and $s = 1$ for a page
capacity of $\capacity = 5$. 
All the following examples have been generated by our visualization
software \TreeLib\ (see \chapref{chapter:performance}). 
Pages $p_1$, $p_2$, and $p_4$ are not shown in the figures, because  
$p_1$ and $p_2$ are used as database information pages, and page
$p_4$ is the root page of the \rootstar\ index. 

\begin{figure}[!hbt]
\begin{center}
  \begin{tikzpicture}[x=0.135em,y=-0.058em]
\node[treepage] (p6) at (29,5) {%
\treeindexpagecontents{6.5}{$[\!-\infty,\!\infty\!),[1,\!\infty\!)$}%
  {%%
   $([\!-\infty,\!4),[1,\!\infty\!),p_{3})$\\
   $([4,\!\infty\!),[1,\!\infty\!),p_{5})$}};
\treepageheader{p6}{$p_{6}$};
\node[treepage] (p3) at (5,146) {%
\treepagecontents{6.5}{$[\!-\infty,\!4),[1,\!\infty\!)$}%
  {%%
   $(1,[1,\!\infty\!))$\\
   $(2,[1,\!\infty\!))$\\
   $(3,[1,\!\infty\!))$}};
\treepageheader{p3}{$p_{3}$};
\node[treepage] (p5) at (53,146) {%
\treepagecontents{6.5}{$[4,\!\infty\!),[1,\!\infty\!)$}%
  {%%
   $(4,[1,\!\infty\!))$\\
   $(5,[1,\!\infty\!))$\\
   $(6,[1,\!\infty\!))$\\
   $(7,[2,\!\infty\!))$\\
   $(8,[2,\!\infty\!))$}};
\treepageheader{p5}{$p_{5}$};
\draw[parentlink] (p6.south) -- ($(p3header.north) + (0,-2)$);
\draw[parentlink] (p6.south) -- ($(p5header.north) + (0,-2)$);

\end{tikzpicture}

  \figcaption{Example of a TMVBT index after insertions}
  {
  The page header shows the page identifier followed by the key range
  and version of the page; 
  the format of index-page entries is (key range, life span, page
  identifier); and 
  the format of leaf-page entries is (key, life span, data), but
  the associated data has been left out for clarity. 
  This TMVBT has been created by transaction~$T_1$
  inserting keys~\range{1}{6} and transaction~$T_2$ inserting
  keys~\num{7} and~\num{8}.
  }
  \label{fig:example-oper-1}
\end{center}
\end{figure}

In \figref{fig:example-oper-1}, the index contains six inactive live
entries inserted by transaction~$T_1$ (entries with keys \range{1}{6}), and
two active entries inserted by transaction~$T_2$ (entries with keys \num{7}
and \num{8}). 
During the execution of~$T_1$, the leaf page~$p_3$ was key-split into
pages~$p_3$ and~$p_5$, and a new root page~$p_6$ was created, thus
incrementing the height of the search tree~$S_1$ by one.

\begin{figure}[!htb]
\begin{center}
  \input{images/tmvbt-oper-2}
  \figcaption{\abbr{TMVBT} after inserting a data item with key~\num{9}}
  {
  The format of the figure is the same as in \figref{fig:example-oper-1}.
  White rectangles denote live pages, and gray rectangles denote dead pages.
  Transaction~$T_2$ has caused a version-split on~$p_5$ by inserting
  key~\num{9}.}
  \label{fig:example-oper-2}
\end{center}
\end{figure}

\figref{fig:example-oper-2} shows the result of a version split
after transaction~$T_2$ tried to insert key~\num{9} to the full page~$p_5$. 
The page~$p_5$ was version-split into pages $p_7$~and~$p_8$.  
The historical entries are left stored in the dead page $p_5$, and active
copies of the entries have been created into pages~$p_7$ and~$p_8$.
Note that all the active entries have been physically moved away from
page~$p_5$.

\begin{figure}[!htb]
\begin{center}
  \begin{tikzpicture}[x=0.13em,y=-0.058em]
\node[treepage,dead] (p6) at (29,5) {%
\treeindexpagecontents{6.5}{$[\!-\infty,\!\infty\!),[1,\!2)$}%
  {%%
   $([\!-\infty,\!4),[1,\!2),p_{3})$\\
   $([4,\!\infty\!),[1,\!2),p_{5})$}};
\treepageheader{p6}{$p_{6}$};
\node[treepage,dead] (p3) at (5,146) {%
\treepagecontents{6.5}{$[\!-\infty,\!4),[1,\!2)$}%
  {%%
   $(1,[1,\!2))$\\
   $(2,[1,\!2))$\\
   $(3,[1,\!2))$}};
\treepageheader{p3}{$p_{3}$};
\node[treepage,dead] (p5) at (53,146) {%
\treepagecontents{6.5}{$[4,\!\infty\!),[1,\!2)$}%
  {%%
   $(4,[1,\!2))$\\
   $(5,[1,\!2))$\\
   $(6,[1,\!2))$}};
\treepageheader{p5}{$p_{5}$};
\node[treepage] (p8) at (101,146) {%
\treepagecontents{6.5}{$[\!-\infty,\!\infty\!),[2,\!\infty\!)$}%
  {%%
   $(1,[2,\!\infty\!))$\\
   $(2,[2,\!\infty\!))$\\
   $(3,[2,\!\infty\!))$}};
\treepageheader{p8}{$p_{8}$};
\draw[parentlink] (p6.south) -- ($(p3header.north) + (0,-2)$);
\draw[parentlink] (p6.south) -- ($(p5header.north) + (0,-2)$);

\end{tikzpicture}

  \figcaption{\abbr{TMVBT} after deleting most of the entries}
  {Transaction~$T_2$ deleted keys~\range{4}{9}, thus shrinking
  the current-version search tree to a single page.}
  \label{fig:example-oper-3}
\end{center}
\end{figure}

\figref{fig:example-oper-3} shows the status of the database after
transaction~$T_2$ has deleted entries \range{4}{9}. 
Deleting the active entries has caused the number of live entries in
pages~$p_7$ and~$p_8$ to fall below \minlive, so the pages have been
consolidated by merging them.
In more detail, first~$p_7$ was merged with~$p_8$ by moving the active
entries of~$p_7$ to~$p_8$, which caused~$p_7$ to be deallocated. 
Note that also the active router to~$p_7$ was deleted from the parent
page~$p_6$. 
When the rest of the entries in~$p_8$ were deleted, page~$p_8$ was further
merged with $p_3$ by killing the page~$p_3$ and by creating active copies of
the live entries in~$p_3$ into~$p_8$. 
As we will show in~\secref{sec:tmvbt:smo}, the algorithms actually created
a new live copy of~$p_3$ when killing it (call it $p_9$), and the active live
copy~$p_9$ was then merged with~$p_8$, causing $p_9$ to be deallocated.
At this point~$p_8$ was the only live page at level~\num{1}, so the height of
the current-version search tree~$S_2$ was decremented by making
$p_8$ the root page of version~\num{2}. 
The auxiliary structure \rootstar\ now contains page identifiers of root
pages $p_6$ (for version~\num{1}) and $p_8$ (for version~\num{2}).

A more diverse example of a TMVBT index is shown in
Figures~\ref{fig:tmvbt-example:1}--\ref{fig:tmvbt-example:3}, with the same
settings used as in the previous examples. 
This example has been generated by our visualization software with
the action sequence given below:

\begin{itemize}
  \setlength{\itemsep}{0pt}
  \item Transaction~$T_1$: insert data items with keys~\range{1}{9}
  (\figref{fig:tmvbt-example:1}).
  \item Transaction~$T_2$: delete data items with keys~\range{7}{9}
  (\figref{fig:tmvbt-example:2}); insert data items with keys~\range{10}{15}
  (\figref{fig:tmvbt-example:3}).
\end{itemize}

\begin{figure*}[htb]
\begin{center}
  \begin{tikzpicture}[x=0.13em,y=-0.058em]
\node[treepage] (p6) at (53,5) {%
\treeindexpagecontents{6.5}{$[\!-\infty,\!\infty\!),[1,\!\infty\!)$}%
  {%%
   $([\!-\infty,\!4),[1,\!\infty\!),p_{3})$\\
   $([4,\!7),[1,\!\infty\!),p_{5})$\\
   $([7,\!\infty\!),[1,\!\infty\!),p_{7})$}};
\treepageheader{p6}{$p_{6}$};
\node[treepage] (p3) at (5,146) {%
\treepagecontents{6.5}{$[\!-\infty,\!4),[1,\!\infty\!)$}%
  {%%
   $(1,[1,\!\infty\!))$\\
   $(2,[1,\!\infty\!))$\\
   $(3,[1,\!\infty\!))$}};
\treepageheader{p3}{$p_{3}$};
\node[treepage] (p5) at (53,146) {%
\treepagecontents{6.5}{$[4,\!7),[1,\!\infty\!)$}%
  {%%
   $(4,[1,\!\infty\!))$\\
   $(5,[1,\!\infty\!))$\\
   $(6,[1,\!\infty\!))$}};
\treepageheader{p5}{$p_{5}$};
\node[treepage] (p7) at (101,146) {%
\treepagecontents{6.5}{$[7,\!\infty\!),[1,\!\infty\!)$}%
  {%%
   $(7,[1,\!\infty\!))$\\
   $(8,[1,\!\infty\!))$\\
   $(9,[1,\!\infty\!))$}};
\treepageheader{p7}{$p_{7}$};
\draw[parentlink] (p6.south) -- ($(p3header.north) + (0,-2)$);
\draw[parentlink] (p6.south) -- ($(p5header.north) + (0,-2)$);
\draw[parentlink] (p6.south) -- ($(p7header.north) + (0,-2)$);

\end{tikzpicture}

  \figcaption{Example of a TMVBT index after insertions}{In this
  figure, transaction $T_1$ has inserted keys \range{1}{9}.}
  \label{fig:tmvbt-example:1}
\end{center}
\end{figure*}

\begin{figure*}[htb]
\begin{center}
  \begin{tikzpicture}[x=0.125em,y=-0.058em]
\node[treepage] (p6) at (77,5) {%
\treeindexpagecontents{6.5}{$[\!-\infty,\!\infty\!),[1,\!\infty\!)$}%
  {%%
   $([\!-\infty,\!4),[1,\!\infty\!),p_{3})$\\
   $([4,\!7),[1,\!2),p_{5})$\\
   $([4,\!\infty\!),[2,\!\infty\!),p_{8})$\\
   $([7,\!\infty\!),[1,\!2),p_{7})$}};
\treepageheader{p6}{$p_{6}$};
\node[treepage] (p3) at (5,146) {%
\treepagecontents{6.5}{$[\!-\infty,\!4),[1,\!\infty\!)$}%
  {%%
   $(1,[1,\!\infty\!))$\\
   $(2,[1,\!\infty\!))$\\
   $(3,[1,\!\infty\!))$}};
\treepageheader{p3}{$p_{3}$};
\node[treepage,dead] (p5) at (53,146) {%
\treepagecontents{6.5}{$[4,\!7),[1,\!2)$}%
  {%%
   $(4,[1,\!2))$\\
   $(5,[1,\!2))$\\
   $(6,[1,\!2))$}};
\treepageheader{p5}{$p_{5}$};
\node[treepage] (p8) at (101,146) {%
\treepagecontents{6.5}{$[4,\!\infty\!),[2,\!\infty\!)$}%
  {%%
   $(4,[2,\!\infty\!))$\\
   $(5,[2,\!\infty\!))$\\
   $(6,[2,\!\infty\!))$}};
\treepageheader{p8}{$p_{8}$};
\node[treepage,dead] (p7) at (149,146) {%
\treepagecontents{6.5}{$[7,\!\infty\!),[1,\!2)$}%
  {%%
   $(7,[1,\!2))$\\
   $(8,[1,\!2))$\\
   $(9,[1,\!2))$}};
\treepageheader{p7}{$p_{7}$};
\draw[parentlink] (p6.south) -- ($(p3header.north) + (0,-2)$);
\draw[parentlink] (p6.south) -- ($(p5header.north) + (0,-2)$);
\draw[parentlink] (p6.south) -- ($(p8header.north) + (0,-2)$);
\draw[parentlink] (p6.south) -- ($(p7header.north) + (0,-2)$);

\end{tikzpicture}

  \figcaption{Example of a TMVBT index after deletions}{In this
  figure, transaction $T_2$ has deleted keys \range{7}{9}.}
  \label{fig:tmvbt-example:2}
\end{center}
\end{figure*}

\begin{figure*}[htb]
\begin{center}
  \begin{tikzpicture}[x=0.108em,y=-0.058em]
\node[treepage,dead] (p6) at (53,5) {%
\treeindexpagecontents{6.5}{$[\!-\infty,\!\infty\!),[1,\!2)$}%
  {%%
   $([\!-\infty,\!4),[1,\!2),p_{3})$\\
   $([4,\!7),[1,\!2),p_{5})$\\
   $([7,\!\infty\!),[1,\!2),p_{7})$}};
\treepageheader{p6}{$p_{6}$};
\node[treepage] (p3) at (5,146) {%
\treepagecontents{6.5}{$[\!-\infty,\!4),[1,\!\infty\!)$}%
  {%%
   $(1,[1,\!\infty\!))$\\
   $(2,[1,\!\infty\!))$\\
   $(3,[1,\!\infty\!))$}};
\treepageheader{p3}{$p_{3}$};
\node[treepage,dead] (p5) at (53,146) {%
\treepagecontents{6.5}{$[4,\!7),[1,\!2)$}%
  {%%
   $(4,[1,\!2))$\\
   $(5,[1,\!2))$\\
   $(6,[1,\!2))$}};
\treepageheader{p5}{$p_{5}$};
\node[treepage,dead] (p7) at (101,146) {%
\treepagecontents{6.5}{$[7,\!\infty\!),[1,\!2)$}%
  {%%
   $(7,[1,\!2))$\\
   $(8,[1,\!2))$\\
   $(9,[1,\!2))$}};
\treepageheader{p7}{$p_{7}$};
\node[treepage] (p10) at (125,5) {%
\treeindexpagecontents{6.5}{$[\!-\infty,\!\infty\!),[2,\!\infty\!)$}%
  {%%
   $([\!-\infty,\!4),[2,\!\infty\!),p_{3})$\\
   $([4,\!10),[2,\!\infty\!),p_{8})$\\
   $([10,\!13),[2,\!\infty\!),p_{9})$\\
   $([13,\!\infty\!),[2,\!\infty\!),p_{11})$}};
\treepageheader{p10}{$p_{10}$};
\node[treepage] (p8) at (149,146) {%
\treepagecontents{6.5}{$[4,\!10),[2,\!\infty\!)$}%
  {%%
   $(4,[2,\!\infty\!))$\\
   $(5,[2,\!\infty\!))$\\
   $(6,[2,\!\infty\!))$}};
\treepageheader{p8}{$p_{8}$};
\node[treepage] (p9) at (197,146) {%
\treepagecontents{6.5}{$[10,\!13),[2,\!\infty\!)$}%
  {%%
   $(10,[2,\!\infty\!))$\\
   $(11,[2,\!\infty\!))$\\
   $(12,[2,\!\infty\!))$}};
\treepageheader{p9}{$p_{9}$};
\node[treepage] (p11) at (245,146) {%
\treepagecontents{6.5}{$[13,\!\infty\!),[2,\!\infty\!)$}%
  {%%
   $(13,[2,\!\infty\!))$\\
   $(14,[2,\!\infty\!))$\\
   $(15,[2,\!\infty\!))$}};
\treepageheader{p11}{$p_{11}$};
\draw[parentlink] (p6.south) -- ($(p3header.north) + (0,-2)$);
\draw[parentlink] (p6.south) -- ($(p5header.north) + (0,-2)$);
\draw[parentlink] (p6.south) -- ($(p7header.north) + (0,-2)$);
\draw[parentlink] (p10.south) -- ($(p3header.north) + (0,-2)$);
\draw[parentlink] (p10.south) -- ($(p8header.north) + (0,-2)$);
\draw[parentlink] (p10.south) -- ($(p9header.north) + (0,-2)$);
\draw[parentlink] (p10.south) -- ($(p11header.north) + (0,-2)$);

\end{tikzpicture}

  \figcaption{Example of a TMVBT index after more insertions}{In this
  figure, transaction $T_2$ has inserted keys \range{10}{15}.}
  \label{fig:tmvbt-example:3}
\end{center}
\end{figure*}

The transactions on this TMVBT index have induced the following 
structure-modification operations:

\begin{itemize}
  \setlength{\itemsep}{0pt}
  \item The first six insertions by transaction~$T_1$ have triggered a
  key-split, splitting page~$p_3$ to~$p_3$ and~$p_5$. 
  At this point, the root page~$p_6$ was created to hold the routers to these
  pages, and \rootstar\ was updated by replacing the page
  identifier stored for version~\num{1} from $p_3$ to $p_6$.
  \item The further three insertions by~$T_1$ have triggered another
  key-split on~$p_5$, creating the new leaf page~$p_7$. 
  The situation after these SMOs is depicted in \figref{fig:tmvbt-example:1}.
  \item After~$T_2$ has deleted the entries with entries \range{7}{9}, a 
  page-merge operation was triggered on~$p_7$ to merge the page with~$p_5$. 
  Because both of these pages were inactive, they were first killed, creating
  two new active pages. 
  These were then merged into the active leaf page~$p_8$.
  The situation after this SMO is shown in \figref{fig:tmvbt-example:2}.
  \item The insertions by~$T_2$ further induced two page splits; first on the
  active page~$p_8$, creating the active page~$p_9$; and then on~$p_9$,
  thus creating page~$p_{11}$.
  \item Insertion of the router to~$p_{11}$ to the parent page~$p_6$ caused a
  split operation on the parent page~$p_6$. 
  Because~$p_6$ was inactive, it was first version-split into~$p_{10}$.
  At this point~$p_{10}$ had enough space to hold the router to $p_{11}$,
  so~$p_{10}$ was not further key-split into two pages.
  The page identifier $p_{10}$ was inserted to the \rootstar\ to mark that the
  root page of version~\num{2} differs from the root page of version~\num{1}.
  The situation after these SMOs is shown in \figref{fig:tmvbt-example:3}.
\end{itemize}

\figref{fig:tmvbt-example:3} shows that the page~$p_3$ containing entries
with keys \range{1}{3} is shared by both roots of the TMVBT index. 
Note that page~$p_3$ is alive but not active, because $\actver = 2$ 
(assuming that transaction $T_2$ has not yet committed), and $p_3$ has a
life span other than $[2, \infty)$.
It is thus possible for this page to have more than one parent. 
The pages $p_8$ to $p_{11}$ are active and only contain entries of
the most recent version. 
Also note that the index page~$p_{10}$ is active even though it contains a
router to the inactive page~$p_3$, because the router itself is active.


% LinkRef cannot be used with TMVBT
In the previous chapter, we briefly discussed efficient version-range queries
(i.e., \qtype{$x$/$-$/range} queries) on the MVBT index structure.
These were introduced by van~den~Bercken and
Seeger~\cite{bercken:1996:multiversion}.
Even though the TMVBT is based on the MVBT index, the most efficient
\LinkRef\ technique cannot be used with the TMVBT index.
This is because the technique relies on storing links to historical pages
that temporally precede a page~$p$.
In the MVBT, each page can have at most two temporal predecessors (see the
discussion in the end of \secref{sec:tsbmvbt:mvbt}), and the links to those
pages can therefore be tracked.
In the TMVBT, pages can have an unlimited number of temporal predecessors,
because merging active pages combines the temporal predecessors of the
merged pages.


%% User Actions
%%---------------------------------------------------------------------
\section{User Actions}
\label{sec:tmvbt:actions}

Having defined the transactions and the structure of the TMVBT index, we
will now describe the implementation of the user actions in this section.
As a general rule, we assume that the physical consistency of the database
during normal processing is maintained by short-duration
latching~\cite{mohan:1992:aries} of pages, so that the server process or
thread that executes a transaction keeps a page~$p$ read-latched for the time
a read action is performed on~$p$, and write-latched for the time an update
action is performed.
We also assume that the buffer manager applies the standard
steal-and-no-force buffering policy~\cite{gray:1993:transactionprocessing}.
These assumptions are in accordance with the \abbr{ARIES} recovery
algorithm~\cite{mohan:1992:aries}.
No logical key-level locking is required for the TMVBT, because 
(1)~for read-only transactions, the historical versions that the read-only
transactions read are never deleted from the index; and 
(2)~for updating transactions, there can be only one updating transaction
operating on the index at a time. 

% We also assume that only the single active updating transaction can perform
% any updates on the index structure.
% The undo actions performed during restart recovery are thus performed by an
% updating system transaction which prevents new updating transactions from
% beginning before restart recovery is finished.
% Furthermore, this also means that historical version
% purging~\cite{becker:1996:mvbt} cannot be performed concurrently with the
% updating transactions. 
% If historical version purging is desired as a background process that can be
% run concurrently with the updating transaction, then the latching policy
% described in this section is not sufficient.
% Rather, the updating transaction must also read-latch pages to
% prevent collisions with the concurrent version purging process.

The global version variables~\comver\ and~\actver\ are maintained
in the permanent database and their reading and writing is protected
by locking. 
A \action{begin-read-only} action acquires a short-duration read lock
on~\comver\ for reading its value, and a \action{commit-update}
action acquires a commit-duration write lock on it for incrementing
its value. 
A \action{begin-update} action acquires a commit-duration write lock
on~\actver, thus guaranteeing that at most one updating
transaction is active at a time.
The decrement of \actver\ in a \action{finish-rollback} action 
is performed under the protection of that lock.
The \action{begin-read-only} and \action{commit-read-only} actions do not
write any log records, because read-only transactions do not involve any
logging.

In a fully dynamic index structure in which any inserted data can be
physically deleted at any time,
\emph{latch-coupling} (called \emph{crabbing} by Gray and
Reuter~\cite{gray:1993:transactionprocessing}) is the standard way to
guarantee the validity of traversed search paths in all circumstances.
In a general situation, the validity of the traversed path can be
ascertained by releasing the latch on the parent page only after
a latch on a child page has been acquired.
Latch-coupling is deadlock-free if the latches are acquired in a
predefined order, such as first top-down, then left-to-right.
However, in the case of the TMVBT index the fact that
inactive data always remains in place
(\invref{inv:tmvbt-static-inactive-entries}), together with our assumption
that a read-only transaction only reads inactive data
(\invref{inv:tmvbt-read-only-inactive}), implies that the \action{query} and
\action{range-query} actions of read-only transactions do not need to perform
latch-coupling, and a parent page may be unlatched during tree traversal
before acquiring a latch on the child page.

Accordingly, an action \action{query}$(k)$ in a read-only transaction
that is reading the version~\snapver{T} can be implemented as follows.
First, the root page for version~\snapver{T} is located from \rootstar\ and
read-latched.
Then the TMVBT is traversed using read latches without latch-coupling until
the leaf page~$p$ is found that covers key~$k$ and version~\snapver{T}; that
is, $k \in \kr{p}$ and $\snapver{T} \in \vr{p}$. 
At each index page~$p'$ on the traversed path, the next page on the
path is the child page~$p''$ of $p'$ with $k \in \kr{p''}$ and $\snapver{T} \in
\vr{p''}$. 
Once the identifier of the child page~$p''$ has been determined, the read latch
on the parent page~$p'$ is released and the child page~$p''$ is read-latched.
When the correct leaf page~$p$ has been found, the proper entry $(k, [v_1,v_2),
w)$ with $v_1 \leq \snapver{T} < v_2$ is located, and page~$p$ is unlatched.

An action \action{range-query}$([k_1,k_2))$ is implemented similarly,
except that for each index page~$p'$ in the search path we need to
traverse all subtrees rooted at each child page~$p''$ such that
$[k_1,k_2) \cap \kr{p''} \neq \emptymark$ and $\snapver{T} \in \vr{p''}$.
If there are more than one such child page~$p''$, then the page identifiers
of all but the first child page are pushed into a stack, and the traversal
proceeds to the subtree rooted at the first child.
When a subtree has been searched, a page identifier (if any) is popped from
the stack, the corresponding page is read-latched, and the search is
continued at the subtree rooted at that page.
Because the inactive entries and pages are static
(\invref{inv:tmvbt-static-inactive-entries}), the pages do not need to be
latched while the page identifiers are queued in the stack.
Latching is used only to prevent inconsistent reads if the updating
transaction needs to modify a page at the same time the read-only transaction
is reading it.

The following theorem follows directly from the definitions of the query
actions of the read-only transactions and from the fact that only one
updating transaction can be active at a time:

\thmskip
\begin{theorem}
\label{theorem:tmvbt:serializable} 
The TMVBT algorithms produce a
snapshot-isolated schedule~\cite{berenson:1995:sql-critique} for the
transactions.
\end{theorem}
\begin{proof}
Firstly, because there can only be a single active updating transaction that
operates on the TMVBT at a time, the updating transactions are processed in a
fully serialized manner, thus fulfilling the requirements for
snapshot-isolated transactions. 
Secondly, read-only transaction only read committed data that is never deleted,
so they also form snapshot-isolated schedules.
\end{proof}
\thmskip

An updating transaction begins with the \action{begin-update} action and ends
with the \action{commit-update} action, as described in
\secref{sec:tmvbt:multi-action-tx}, unless the transaction is aborted and
rolled back.
% The action takes a commit-duration write lock on~\actver, increments
% it, and assigns it to the transaction as described in
% \secref{sec:tmvbt:multi-action-tx}.
% A transaction~$T$ executing this action writes a redo-undo log record
% \lrb{T}, \logact{begin}, $v$, \lre{\actver}, where $v$ is the previous value
% of the active version variable, and \actver\ is the incremented value.
% The \action{commit-update} action, in addition to taking the lock and updating
% the commit version variable~\comver, as explained in
% \secref{sec:tmvbt:multi-action-tx}, writes a redo-only log record
% \lrb{T}, \logact{commit}, \lre{\comver}.
% Finally, this action forces the database log to disk.
The \action{query} and \action{range-query} actions are the same as for
read-only transactions, except that they now target the version~\actver, and
the actions may read active entries and pages.
As with read-only transactions, these actions in an updating transaction do
not write log records, because they do not create changes to the database
that would have to be redone or undone during restart recovery.

For efficiency, we assume that the TMVBT index records the page identifier of
the root page of version~\actver\ separately so that the queries in updating
transactions do not need to use the \rootstar\ structure to find it.
Similarly, because read-only transactions reading the most recent committed
version always target the version~\comver, the page identifier of the root
page of that version is also maintained separately.

\thmskip
\begin{theorem}
\label{thm:tmvbt-query-cost}
When the root of the search tree of version~$v$ is known, the cost of a
single-key query action in the TMVBT targeting version~$v$ is
\OhT{\log_\capacity \entries{v}} pages, and the cost of the key-range query
action for version~$v$ is \OhT{\log_\capacity \entries{v} + r/\capacity}
pages of the TMVBT structure, where $\entries{v}$ denotes the number of data
items that are alive at version~$v$, $r$ is the number of entries returned by
the range query and \capacity\ is the page capacity.
\end{theorem} 
\begin{proof}
Assuming that \invref{inv:tmvbt-live-count} holds, each page of the TMVBT
that is part of the search tree~$S_v$ has at least \minlive\ entries that are
alive at version~$v$. 
The proof is therefore the same as the proof of \thmref{thm:mvbt-cost}.
We will show later on in
Lemmas~\ref{lemma:kill-page-split-counts},
\ref{lemma:split-page-split-counts}, and~\ref{lemma:merge-page-split-counts} 
that all the SMOs maintain \invref{inv:tmvbt-live-count}, thereby confirming
this result. 
\end{proof}
\thmskip

We assume that all TMVBT page traversals maintain a \emph{saved
path}\phantomsection\label{def:saved-path}~\cite{lomet:1992:conc-rec,lomet:1997:concurrency};
that is, an array \emph{path} local to the server process or thread in
question and indexed by the height of pages.
An entry \emph{path}$[i]$ holds the page identifier, key range, life span,
and Page-LSN of the page that was located at level~$i$ when traversing the
root-to-leaf path.
The saved-path concept can be used to accelerate the user actions by starting
the traversal at the lowest-level page in the saved path that, according to
the saved information, covers the queried search space.
This page is known to be the correct page to start the tree
traversal, because 
(1)~for read-only transactions, the inactive data is never moved away
from the pages; and 
(2)~for updating transactions, there can be no other updating
transaction that would invalidate the data in the saved path of the
current updating transaction.
This holds regardless of whether a concurrent purging process is allowed,
because the purging process only deletes pages that are part of
historical versions that are no longer queried.

% Write and delete action general idea
For the \action{write}$(k,w)$ and \action{delete}$(k)$ actions of the
updating transaction, the TMVBT is traversed using read latches without
latch-coupling as for the \action{query}$(k)$ action of the updating
transaction, except that the target leaf page~$p$ is write-latched.
% As in \action{query}$(k)$ and \action{range-query}$([k_1, k_2))$, no
% latch-coupling is needed.
If the target leaf page~$p$ can accommodate the update, then the update is
applied on page~$p$ directly; otherwise a structure-modification operation
is performed before the action can proceed.
After the update has been applied, a redo-undo log record for the action
is generated, its LSN is stamped in the Page-LSN field of~$p$, and the
write latch on~$p$ is released.

% Write action specifics
In the \action{write}$(k,w)$ action, if the index contains a live entry
of the form $(k,[v,\infty),w')$, then that entry is logically deleted by
either replacing it with a new entry $(k,[v,\actver),w')$, if $v \neq
\actver$; or by physically removing the old entry, if $v = \actver$.
After the existing entry has been deleted, a new entry
$(k,[\actver,\infty),w)$ is inserted into the page~$p$. 
The page~$p$ can accommodate this update action, if the operations explained
above can be carried out without the page overflowing.
The redo-undo log entry written for this action contains the version and data
of the replaced entry, in addition to the version and data of the inserted
entry.
The log entry written by an updating transaction~$T$ is thus
\lrb{T}, \logact{write}, $p$, $k$, \actver, $w$, $v$, $w'$, \lre{n}, where
$n$ is the log sequence number of the previous not-yet-undone action
of~$T$, and $v$ and $w'$ are null if the index contained no live entry with
the key~$k$.
 
% Delete action specifics
In the case of the \action{delete}$(k)$ action, page~$p$ can
accommodate the update if replacing the entry $(k,[v,\infty),w)$ by
$(k,[v,\actver),w)$ (in the case $v \neq \actver$), or physically removing
the entry $(k,[\actver,\infty),w)$ (otherwise) does not decrease the number
of live entries in the page below the required minimum number of live entries,
$\minlive$.
% The redo-undo log record written for this action contains the version and
% data of the deleted entry.
An updating transaction~$T$ writes a redo-undo log record
\lrb{T}, \logact{delete}, $p$, $k$, \actver, $v$, $w$, \lre{n} for this
action.

When the target leaf page~$p$ cannot accommodate the update,
structure modifications are needed. 
These operations are explained in \secref{sec:tmvbt:smo}. 
For writes, the operation \alg{split-page} is called before
the write action can proceed. 
For deletes, the page~$p$ needs to be consolidated by the operation
\alg{merge-page} before the entry can be deleted from the page.
When the structure-modification operations are initiated, the page~$p$,
recorded on the saved path, is left write-latched. 
After the operations, the saved path contains the correct write-latched leaf
page~$p'$ whose key range covers the key~$k$. 
As with the earlier situation, the update is now performed on page~$p'$, a
redo-undo log record is generated, the LSN is stamped on~$p'$ and page~$p'$
is unlatched.

As will be explained in the next section, the structure modifications
(page splits or merges) are applied in a top-down, level-by-level
manner, logging the structure modification done at each level using
a single redo-only log record.
Each of these structure modifications involves a maximum of five pages
at two adjacent levels.
The sequence of structure modifications results in a target leaf page that
can accommodate the \action{insert} or \action{delete} action in question.

An undo action, \action{undo-write}$(r)$ or \action{undo-delete}$(r)$,
is performed as a physical undo if possible and as a logical undo
otherwise~\cite{mohan:1992:aries}.
For a physical undo, the page mentioned in the log record~$r$ is
write-latched and the Page-LSN field is examined.
If the Page-LSN field still contains the LSN of~$r$, or if
the page contents show that the page is the correct target for the
undo action and the page can accommodate the undo action, then the undo
action is performed on the page, a redo-only log record is generated, its
LSN is stamped in the Page-LSN field of the page, and the page is
unlatched.
If the page mentioned in the log record~$r$ cannot be seen
to be the correct target for the undo action or if the page cannot
accommodate the undo action, a \emph{logical undo}
is performed, starting with a search for the key mentioned in~$r$
and performing any structure modifications that may be necessary to
make the target page accommodate the undo action.
The undo actions write redo-only compensation log records, as dictated
by the ARIES algorithm.
If the page cannot accommodate the undo action, the page is first split
or merged after write-latching it, as with the forward-rolling write
and delete actions.

The \action{undo-write} action reads the log record
\lrb{T}, \logact{write}, $p$, $k$, \actver, $w$,
$v$, $w'$, \lre{n} created by the \action{write} action that is to be undone. 
Let page~$p'$ be the current correct leaf page that covers key~$k$ at
version~\actver.
As discussed above, $p'$~is either~$p$ or it has been located with a
root-to-leaf traversal of the TMVBT index structure.
This action locates the active entry $(k,[\actver,\infty),w)$ from page~$p'$
and physically removes it.
If $v$ and $w'$ are not null, then this action furthermore restores the
previous entry.
If $v < \actver$, the currently dead inactive entry $(k,[v,\actver),w')$ is
located (if it is still present on~$p$) and restored to life by replacing it
with $(k,[v,\infty),w')$. 
If $v = \actver$, or if $p'$ has been version-split so that the historical
entry $(k,[v,\actver),w')$ is not present on~$p'$, then the active entry
$(k,[\actver,\infty),w')$ is inserted to page~$p'$ to undo the write
action.
The leaf page~$p'$ can accommodate the undo action if the operations
described can be carried out without the page overflowing, and without the
number of live entries on the page decreasing below \minlive.
The action finishes by writing a redo-only compensation log record
\lrb{T}, \logact{undo-write}, $p'$, $k$, \actver, $v$, $w'$, \lre{n}.

The \action{undo-delete} action examines the log record \lrb{T},
\logact{delete}, $p$, $k$, \actver, $v$, $w$, \lre{n} created by the delete
action, and locates the correct leaf page~$p'$.
If $v < \actver$, then this action locates the logically deleted entry 
$(k,[v,\actver),w)$, and replaces it with $(k,[v,\infty),w)$.
If $v = \actver$, or if the logically deleted entry $(k,[v,\actver),w)$ is no
longer present on $p'$, then this action inserts a new active entry
$(k,[\actver,\infty),w)$ into page~$p'$. 
The leaf page~$p'$ can accommodate the undo action if the operation
described above can be performed on $p'$ without it overflowing. 
The action finishes by writing a redo-only compensation log
record \lrb{T}, \logact{undo-delete}, $p'$, $k$, \actver, $v$, $w$, \lre{n}.


%% Structure-Modification Operations
%%---------------------------------------------------------------------
\section{Structure-Modification Operations}
\label{sec:tmvbt:smo}

In this section, we will describe the structure-modification operations
(SMOs) used with the TMVBT index. 
The two main operations described here, \alg{split-page} and
\alg{merge-page}, are triggered by the user actions described in the previous
section. 
For convenience, we will use the same notation for entries in both index
and leaf pages.
The format of an index-page entry is $(\kr{p}, \vr{p}, p)$; that is
$([k_1, k_2), [v_1, v_2), p)$, where $\kr{p} = [k_1, k_2)$ is the key range
and $\vr{p} = [v_1, v_2)$ is the life span of the page~$p$. 
A leaf-page entry $(k,\vrmark{v},w)$, for key~$k$ and data-item
value~$w$, that is alive at versions in the range $\vrmark{v} = [v_1,
v_2)$, is represented by the tuple $([k, k^{+}), [v_1, v_2), w)$,
where $k^{+}$ is the key immediately following key~$k$ in the key space (for
databases that store integer keys, $k^{+} = k + 1$).
In this way we can use the same algorithms for manipulating both the leaf
pages and the index pages.

The convention in the TMVBT structure-modification operations is
that each SMO transforms a balanced (\defref{def:consistent-balanced}) TMVBT
index into another balanced TMVBT\@.
Each operation is logged with a single redo-only log record, so that
structure modifications are never undone when a transaction aborts or
system fails~\cite{jaluta:2005:blink,jaluta:2006:page-server}. 
The structure-modification operations are performed top-down,
starting from the highest page on the search path that requires
splitting or merging.

The actual implementation of the operations traverses the search path
bottom-up in order to determine which kind of a structure modification
is needed at each level, yet without performing any modification. 
When a parent page which does not need any modification is encountered, the
search path is traversed top-down, and the structure modifications are
performed level-by-level, logging each operation with a single redo-only log
record.
The search path state is guaranteed to remain valid throughout the
operations because only one updating transaction can be active at a
time.
The information about the traversed search path stored in the
saved path can thus be trusted.
For clarity, the algorithms presented in this section only describe
the structure-modification operations applied at a single level.

% Operation: Page killing
Both of the SMOs \alg{split-page} and \alg{merge-page} are based on the
page-killing operation \alg{kill-page}.
Page killing is not a separate SMO, but it is used in both of the actual
SMOs.
This operation creates a live copy of a page~$p$, and marks the original
page~$p$ as killed.
%This operation assumes a write-latched inactive live
%page~$p$ and its write-latched parent page~$q$, marks $p$ as killed by
%replacing its life span $[v, \infty)$ with $[v, \actver)$, and creates a
%new active live page~$p'$ that holds the live contents of~$p$. 
%As per \invref{inv:tmvbt-live-entry-copy}, the active entries of~$p$ are
%physically moved to page~$p'$, and new active copies of the inactive live
%entries of page~$p$ are created into page~$p'$.
The operation is described in \algref{alg:killpage}.
The page~$p$ and its parent~$q$ (located from the saved path) are both
write-latched before the operation is performed.
As discussed above, we assume that all required \abbr{SMOs} have been
applied to the parent page~$q$, so that $q$ can accommodate the entry
inserted by the algorithm.

\begin{algorithm}
$\proc{kill-page}(p, q)$:

\begin{algorithmic}[1]
\STATE $p' \leftarrow$ allocate, write-latch, and format a new page
% format $p'$ as a TMVBT page, and set the height of $p'$ to be the height
% of $p$
\STATE move all active live entries of $p$ to $p'$
\STATE create active copies of all inactive live entries of $p$ to $p'$
\STATE kill all live entries of $p$ 
% by setting the end point of their life span to \actver
\STATE $r \leftarrow$ find the router $([k_1, k_2), [v,\infty), p)$ to $p$
from $q$ 
\STATE replace the retrieved router $r$ in $q$ with $([k_1,k_2),
[v,\actver), p)$
\STATE insert a new router $([k_1, k_2), [\actver,\infty), p')$ to
$q$
\RETURN $p'$
\end{algorithmic}
\figcaption{Page-killing algorithm}%
{The algorithm assumes an inactive page~$p$ and its parent page~$q$, kills
the page~$p$, creates an active live copy~$p'$ of~$p$ and updates the
routers in the parent page~$q$.}
\label{alg:killpage}
\end{algorithm}

The page-killing operation begins by allocating a new page~$p'$,
write-latching it and formatting it as a TMVBT page. 
All the live entries of page~$p$ are now either copied or moved to page~$p'$
in such a way that all active entries are physically moved to $p'$, and 
all inactive live entries are copied to~$p'$. 
The life spans of the copied entries are split at version
\actver, so that a life span $[v, \infty)$ is changed to $[v,
\actver)$ in the entry that remains in page~$p$ and to $[\actver, \infty)$
in the new entry created to page~$p'$. 
The logical state of the database at all versions prior to \actver\ is thus
maintained in page~$p$, but the page~$p$ is no longer part of the
search trees of versions $v \geq \actver$. 
The router to page~$p$ in the parent page~$q$ must be updated by setting its
end version to \actver, and the router to the new page must also be inserted
into the parent. 
After this, the old page~$p$ is replaced with the active page~$p'$ in
the saved path. 
All pages modified by this operation are kept write-latched, because the
acquired page latches can be released only after the associated log record
has been written. 

\thmskip
\begin{lemma}
\label{lemma:kill-page-split-counts}
The \alg{kill-page} operation maintains \invref{inv:tmvbt-live-count}.
\end{lemma}
\begin{proof}
Because the TMVBT index is balanced at the point when the SMO that uses the
page-killing operation is triggered, page~$p$ must contain at least \minlive\
entries that are alive at version~$v'$, for each version $v' \in [v,
\actver]$, where $v$ is the starting point of the life span of $p$. 
There are thus at least \minlive\ entries in page~$p$ that are alive at
version \actver. 
These entries are either copied or moved to the new page~$p'$, which thus has
at least \minlive\ live entries after the page-killing operation.
Because the life spans of the copied entries were cropped to start from
version \actver, page~$p'$ contains no entries that are alive at any
earlier version $v' < \actver$.
All of the entries that are alive at version~\actver\ in page~$p$ are 
deleted from~$p$---either physically, if they were active entries; or
logically, if they were inactive entries.
After the page-killing operation, page~$p$ thus does not have any entries that
are alive at version~\actver.
The number of entries in page~$p$ that were alive at an earlier version $v' <
\actver$ remains unchanged. 
Clearly, the invariant holds for pages~$p$ and~$p'$.
Furthermore, a single router that is alive at version \actver\ is logically
deleted from the parent page~$q$, and a new active router is inserted to the
parent to replace it. 
The number of live entries in the parent page is therefore not affected for
any version.
Because all the pages retain a valid number of live entries for all
versions, we conclude that the page-killing operation maintains
\invref{inv:tmvbt-live-count}.
It is clear that \invref{inv:tmvbt-static-inactive-entries} is also
maintained.
\end{proof}
\thmskip

\begin{algorithm}[htb]
$\proc{split-page}(p, q)$:

\begin{algorithmic}[1]
\IF[\figref{fig:split-active}, key split]{$p$ is active}
  \STATE $p' \gets$ allocate, write-latch, and format a new
  page 
  \STATE distribute entries of $p$ between $p$ and $p'$
%  \STATE insert router to $p'$ in $q$
%  \STATE adjust router to $p$ in $q$
\ELSE[Figures~\ref{fig:split:is1}--\ref{fig:split:ima2}, version
split] 
  \STATE $\liveentries{p} \gets$ count the number of live entries in $p$
  \STATE $p' \gets$ \proc{kill-page}($p$)
  %\STATE insert router to $p'$ in $q$
  \IF[\figref{fig:split:is2}]{$\liveentries{p} >
  \maxsplit$} 
    \STATE $p'' \gets$ allocate, write-latch, and format a new page 
    \STATE distribute entries of $p'$ between $p'$ and $p''$
%    \STATE insert router to $p''$ in $q$
%    \STATE adjust router to $p'$ in $q$
  \ELSIF[Figures~\ref{fig:split:imi1}--\ref{fig:split:ima2}]{$\liveentries{p} <
  \minsplit$} 
    \STATE $s \gets$ find a live sibling page of $p'$ from $q$ and
    write-latch it
    \STATE $\liveentries{s} \gets$ count the number of live entries in $s$
    \STATE $p'' \gets s$ if $s$ is active, \proc{kill-page}($s$)
    otherwise
    %\IF[Figures~\ref{fig:split:ima1},\ref{fig:split:ima2}]{$s$ is active} 
    %  \STATE $p'' \leftarrow s$
    %\ELSE[Figures~\ref{fig:split:imi1},\ref{fig:split:imi2}]
    %  \STATE $p'' \leftarrow$ \proc{kill-page}($s$)
    %\ENDIF
    \IF[Figures~\ref{fig:split:imi2},\ref{fig:split:ima2}]{$\liveentries{p} + \liveentries{s} > \maxsplit$}
      \STATE redistribute entries of $p'$ and $p''$ 
%      \STATE adjust router to $p'$ in $q$ 
%      \STATE adjust router to $p''$ in $q$
    \ELSE[Figures~\ref{fig:split:imi1},\ref{fig:split:ima1}]
      \STATE move all entries of $p''$ to $p'$
%      \STATE adjust router to $p'$ in $q$
%      \STATE remove router to $p''$ from $q$
      \STATE deallocate $p''$
    \ENDIF
  \ELSE[\figref{fig:split:is1}]
    \STATE \COMMENT{No further action required}
  \ENDIF
\ENDIF
%\STATE log the operation using a redo-only log record
%\STATE release page latches
\end{algorithmic}
\figcaption{The page-splitting algorithm}%
{Splits a page~$p$, possibly redistributing the live entries with a sibling
page~$s$. 
Updates the routers in the parent page~$q$.}
\label{alg:split}
\end{algorithm}

% SMO: Split
The \alg{split-page} operation is a structure-modification operation
that splits a page that has become full. 
The operation is similar to the MVBT version-split operation, with the
exception that active pages are directly key-split without first killing the
page. 
This operation is triggered by a \action{write} action when a data
page has become full, by both \action{undo-write} and \action{undo-delete}
actions if the leaf page cannot accommodate the undo action, and the
operation is also used to split index pages along the search path. 
At the beginning, the page~$p$ to be split is retrieved from the
saved path along with its parent page~$q$, and both pages are
write-latched for modification; unless the operation was triggered by a user
action, in which case page~$p$ is already write-latched.
As explained in the beginning of this section, we expect that parent
pages in the saved path have already been split so that the parent
page~$q$ can accommodate the routers to the new pages created by
the split operation. 

An overview of the actual \alg{split-page} operation is simple: if page~$p$
is active, it will be key-split, and if it is inactive, it will be
version-split. 
Key-splitting and version-splitting will be defined in more detail below. 
An overview of the page-split operation is shown in \algref{alg:split}. 
In the algorithms presented in this section, we denote by \liveentries{p} the
number of live entries in page~$p$. 
Note that the bottom-up checking phase needs to do the same checks
that are described in \algref{alg:split} to determine which
kind of a split needs to be done. 
When performing the actual structure modification, the same checks
must be performed again (as described here), or results saved
during the checking phase can be used.

% Split-page: key-split
The first part of the split algorithm, the \alg{key-split} operation, is
similar to the page-split operation in a standard \Btree. 
This operation begins by allocating, write-latching, and formatting a new
page~$p'$.
After that, the entries of the old page~$p$ are distributed between~$p$
and~$p'$ by moving half of the entries from~$p$ to~$p'$.
Note that all the entries of page~$p$ are alive and active at the
beginning of the operation because~$p$ is active, and thus the entries
may be physically distributed between the pages.
The operation finishes by adjusting the the router to page~$p$ in the parent
page~$q$, and by inserting a router to the new page~$p'$ into~$q$. 
The key-split operation is illustrated in
\figref{fig:split-active}.

\begin{figure}[!htb]
\begin{center}
  \input{images/tmvbt-split-a.tex}
  \figcaption{Key-splitting an active page~$p$}%
  {The horizontal axis represents life spans, and the vertical axis key
  ranges.}
 % [$p$ active, $\liveentries{p} = \capacity$]
  \label{fig:split-active} 
\end{center}
\end{figure}

% Create new root
An exception to the key-split algorithm is the situation when page~$p$
has no parent page, and the saved path only contains the page~$p$. 
This happens when page~$p$ is a root page.
In this situation a new parent page~$r$ is allocated, write-latched and
formatted; and routers to pages~$p$ and~$p'$ are inserted to it. 
The new root is attached to the TMVBT by inserting the tuple $(\actver, r)$
into \rootstar, and by updating the cached root page identifier for
version~\actver.
This will replace the existing tuple $(\actver, p)$ in the \rootstar,
which pointed to the previous root page~$p$.
The rest of the key-split operation is otherwise similar to the normal
situation.
Note that this operation increases the height of the search tree
$S_{\actver}$ by one.

% Split-page: version-split
The \alg{version-split} operation begins by killing the inactive page~$p$
with the page-killing operation defined earlier. 
The new active copy of page~$p$ is denoted by~$p'$.
At this point, page~$p'$ may contain too few or too many entries to satisfy
\invref{inv:tmvbt-minsplit-maxsplit}.
%the strong version condition (see p.~\pageref{def:strong-version}).
If the number of entries in~$p'$ (i.e., \liveentries{p}, the number of live
entries in~$p$ before the page-killing operation) is less than $\minsplit$,
the page will be merged with a sibling page by consolidating it in the same
way as pages are merged in the \alg{merge-page} operation that will be
shortly described. 
If the number of entries in~$p'$ is more than $\maxsplit$, page~$p'$ will
be key-split in the same way as active pages are split. 

% Update root pointer for version-split
As with the key-split algorithm, it is possible that the inactive page~$p$
has no parent, if it is the root page of the current-version search tree.
In this case, the page-killing operation cannot update the parent page, but
instead a new tuple $(\actver, p')$ is inserted to \rootstar\ to indicate
that the new active page~$p'$ is the current root page.
If page $p'$ contains more than \maxsplit\ entries after the page-killing
operation, it is further key-split, a new parent page~$r$ is created, 
and the tuple $(\actver, p')$ in \rootstar\ is replaced by $(\actver, r)$.
The cached root page identifier must also be updated.

% Log record
The entire split operation, consisting of either a key split or a version
split (possibly followed by a key split or consolidation), and including the
possible tree-height increase or root-page update, is logged with a single
redo-only log record containing the page identifiers of all the pages
involved in the SMO; that is, a subset of the pages $p$, $p'$, $p''$, $s$,
$q$, and $r$.
The redo-only log record must also contain information of all the
entries moved or copied between pages, as explained in
\secref{sec:mv-data:smo}. 
For example, the log record of a version split followed by a key split (shown
in \figref{fig:split:is2}) is
\lrb{T}, \logact{version-split-key-split}, $p$, $p'$, $p''$, $q$,
$E_{p'}$, \lre{E_{p''}}, where $E_x$ denotes the set of entries that were
written on page~$x$. 
This operation can be redone on $p$ by deleting all the live
entries from $p$; and on pages $p'$ and $p''$ by clearing the page and
inserting the entries from the corresponding set $E_{p'}$ and $E_{p''}$. 
As usual, all the pages are kept latched until the log record has been
generated and its LSN stamped in the Page-LSN fields of the
pages. 
The split operation finishes by replacing $p$ in the saved
path with the active page whose key range covers~$k$ (either~$p'$ or~$p''$),
and by unlatching all the other pages involved in the operation.

\begin{figure}[!htb]
\begin{center}
  \subfigure[$\minsplit \leq \liveentries{p} \leq
  \maxsplit$]{\label{fig:split:is1}
    \begin{tikzpicture}[x=\tmvbtsmox,y=\tmvbtsmoy]
  \small
\coordinate (lo) at (0,0);
\coordinate (hi) at (0,3);
\coordinate (loact) at (2,0);
\coordinate (hiact) at (2,3);
\coordinate (loinf) at (3,0);
\coordinate (hiinf) at (3,3);


  \node[anchor=base] at ($(lo) + (1.5, 1.3)$) {$p$};

  \draw (lo) -- (hi);
  \draw (lo) -- (loinf);
  \draw (hi) -- (hiinf);
  \draw[opendash] decorate [opendeco] { (loinf) -- (hiinf) };

\end{tikzpicture}
\raisebox{1.25\tmvbtsmoy}{$\rightarrow$}
\begin{tikzpicture}[x=\tmvbtsmox,y=\tmvbtsmoy]
  \small
\coordinate (lo) at (0,0);
\coordinate (hi) at (0,3);
\coordinate (loact) at (2,0);
\coordinate (hiact) at (2,3);
\coordinate (loinf) at (3,0);
\coordinate (hiinf) at (3,3);


  \node[anchor=base] at ($(lo) + (1, 1.3)$) {$p$};
  \node[anchor=base] at ($(loact) + (0.5, 1.3)$) {$p'$};

  \draw (lo) -- (hi);
  \draw (lo) -- (loinf);
  \draw (hi) -- (hiinf);
  \draw (hiact) -- (loact);
  \draw[opendash] decorate [opendeco] { (loinf) -- (hiinf) };

\end{tikzpicture}
}
  \subfigure[$\liveentries{p} > \maxsplit$]{\label{fig:split:is2} 
    \begin{tikzpicture}[x=\tmvbtsmox,y=\tmvbtsmoy]
  \small
\coordinate (lo) at (0,0);
\coordinate (hi) at (0,3);
\coordinate (loact) at (2,0);
\coordinate (hiact) at (2,3);
\coordinate (loinf) at (3,0);
\coordinate (hiinf) at (3,3);


  \node[anchor=base] at ($(lo) + (1.5, 1.3)$) {$p$};

  \draw (lo) -- (hi);
  \draw (lo) -- (loinf);
  \draw (hi) -- (hiinf);
  \draw[opendash] decorate [opendeco] { (loinf) -- (hiinf) };

\end{tikzpicture}
\raisebox{1.25\tmvbtsmoy}{$\rightarrow$}
\begin{tikzpicture}[x=\tmvbtsmox,y=\tmvbtsmoy]
  \small
\coordinate (lo) at (0,0);
\coordinate (hi) at (0,3);
\coordinate (loact) at (2,0);
\coordinate (hiact) at (2,3);
\coordinate (loinf) at (3,0);
\coordinate (hiinf) at (3,3);


  \node[anchor=base] at ($(lo) + (1, 1.3)$) {$p$};
  \node[anchor=base] at ($(loact) + (0.5, 2.3)$) {$p'$};
  \node[anchor=base] at ($(loact) + (0.5, 0.8)$) {$p''$};

  \draw (lo) -- (hi);
  \draw (lo) -- (loinf);
  \draw (hi) -- (hiinf);
  \draw (hiact) -- (loact);
  \draw ($(loact)+(0,2)$) -- ($(loinf)+(0,2)$);
  \draw[opendash] decorate [opendeco] { (loinf) -- ($(loinf)+(0,2)$) };
  \draw[opendash] decorate [opendeco] { ($(loinf)+(0,2)$) -- (hiinf) };

\end{tikzpicture}
}
  \\
  \subfigure[$s$ inactive, $\liveentries{p} < \minsplit$ and 
    $\liveentries{p} + \liveentries{s} \leq
  \maxsplit$]{\label{fig:split:imi1}
    \begin{tikzpicture}[x=\tmvbtsmox,y=\tmvbtsmoy]
  \small
\coordinate (lo) at (0,0);
\coordinate (hi) at (0,3);
\coordinate (loact) at (2,0);
\coordinate (hiact) at (2,3);
\coordinate (loinf) at (3,0);
\coordinate (hiinf) at (3,3);


  \node[anchor=base] at ($(lo) + (1.5, 1.8)$) {$p$};
  \node[anchor=base] at ($(lo) + (1.5, 0.3)$) {$s$};

  \draw (lo) -- (hi);
  \draw (lo) -- (loinf);
  \draw (hi) -- (hiinf);
  \draw ($(lo) + (0,1)$) -- ($(loinf) + (0,1)$);
  \draw[opendash] decorate [opendeco] { (loinf) -- ($(loinf) + (0,1)$) };
  \draw[opendash] decorate [opendeco] { ($(loinf) + (0,1)$) -- (hiinf) };

\end{tikzpicture}
\raisebox{1.25\tmvbtsmoy}{$\rightarrow$}
\begin{tikzpicture}[x=\tmvbtsmox,y=\tmvbtsmoy]
  \small
\coordinate (lo) at (0,0);
\coordinate (hi) at (0,3);
\coordinate (loact) at (2,0);
\coordinate (hiact) at (2,3);
\coordinate (loinf) at (3,0);
\coordinate (hiinf) at (3,3);


  \node[anchor=base] at ($(lo) + (1, 1.8)$) {$p$};
  \node[anchor=base] at ($(lo) + (1, 0.3)$) {$s$};

  \node[anchor=base] at ($(loact) + (0.5, 1.3)$) {$p'$};

  \draw (lo) -- (hi);
  \draw (lo) -- (loinf);
  \draw (hi) -- (hiinf);
  \draw ($(o) + (0,1)$) -- ($(loact) + (0,1)$);
  \draw (hiact) -- (loact);
  \draw[opendash] decorate [opendeco] { (loinf) -- (hiinf) };

\end{tikzpicture}
}
  \subfigure[$s$ inactive, $\liveentries{p} < \minsplit$ and  
    $\liveentries{p} + \liveentries{s} >
  \maxsplit$]{\label{fig:split:imi2}
    \input{images/tmvbt-split-imi2.tex}}
  \\
  \subfigure[$s$ active, $\liveentries{p} < \minsplit$ and  
    $\liveentries{p} + \liveentries{s} \leq
  \maxsplit$]{\label{fig:split:ima1}
    \begin{tikzpicture}[x=\tmvbtsmox,y=\tmvbtsmoy]
  \small
\coordinate (lo) at (0,0);
\coordinate (hi) at (0,3);
\coordinate (loact) at (2,0);
\coordinate (hiact) at (2,3);
\coordinate (loinf) at (3,0);
\coordinate (hiinf) at (3,3);


  \node[anchor=base] at ($(lo) + (1.5, 1.8)$) {$p$};
  \node[anchor=base] at ($(loact) + (0.5, 0.3)$) {$s$};

  \pattern[bgpattern] (lo) rectangle ($(loact) + (0,1)$);

  \draw ($(lo) + (0,1)$) -- (hi);
  \draw ($(lo) + (0,1)$) -- ($(loinf) + (0,1)$);
  \draw (hi) -- (hiinf);
  \draw[opendash] decorate [opendeco] { (loinf) -- ($(loinf) + (0,1)$) };
  \draw[opendash] decorate [opendeco] { ($(loinf) + (0,1)$) -- (hiinf) };
  \draw (loact) -- (loinf);
  \draw (loact) -- ($(loact) + (0,1)$);

\end{tikzpicture}
\raisebox{1.25\tmvbtsmoy}{$\rightarrow$}
\begin{tikzpicture}[x=\tmvbtsmox,y=\tmvbtsmoy]
  \small
\coordinate (lo) at (0,0);
\coordinate (hi) at (0,3);
\coordinate (loact) at (2,0);
\coordinate (hiact) at (2,3);
\coordinate (loinf) at (3,0);
\coordinate (hiinf) at (3,3);


  \node[anchor=base] at ($(lo) + (1, 1.8)$) {$p$};
  \node[anchor=base] at ($(loact) + (0.5, 1.3)$) {$p'$};

  \pattern[bgpattern] (lo) rectangle ($(loact) + (0,1)$);

  \draw ($(lo) + (0,1)$) -- (hi);
  \draw ($(lo) + (0,1)$) -- ($(loact) + (0,1)$);
  \draw (hi) -- (hiinf);
  \draw[opendash] decorate [opendeco] { (loinf) -- (hiinf) };
  \draw (loact) -- (loinf);
  \draw (loact) -- (hiact);

\end{tikzpicture}
}
  \subfigure[$s$ active, $\liveentries{p} < \minsplit$ and  
    $\liveentries{p} + \liveentries{s} >
  \maxsplit$]{\label{fig:split:ima2}
    \begin{tikzpicture}[x=\tmvbtsmox,y=\tmvbtsmoy]
  \small
\coordinate (lo) at (0,0);
\coordinate (hi) at (0,3);
\coordinate (loact) at (2,0);
\coordinate (hiact) at (2,3);
\coordinate (loinf) at (3,0);
\coordinate (hiinf) at (3,3);


  \node[anchor=base] at ($(lo) + (1.5, 2.3)$) {$p$};
  \node[anchor=base] at ($(loact) + (0.5, 0.8)$) {$s$};

  \pattern[bgpattern] (lo) rectangle ($(loact) + (0,2)$);

  \draw ($(lo) + (0,2)$) -- (hi);
  \draw ($(lo) + (0,2)$) -- ($(loinf) + (0,2)$);
  \draw (hi) -- (hiinf);
  \draw[opendash] decorate [opendeco] { (loinf) -- ($(loinf) + (0,2)$) };
  \draw[opendash] decorate [opendeco] { ($(loinf) + (0,2)$) -- (hiinf) };
  \draw (loact) -- (loinf);
  \draw (loact) -- ($(loact) + (0,2)$);

\end{tikzpicture}
\raisebox{1.25\tmvbtsmoy}{$\rightarrow$}
\begin{tikzpicture}[x=\tmvbtsmox,y=\tmvbtsmoy]
  \small
\coordinate (lo) at (0,0);
\coordinate (hi) at (0,3);
\coordinate (loact) at (2,0);
\coordinate (hiact) at (2,3);
\coordinate (loinf) at (3,0);
\coordinate (hiinf) at (3,3);


  \node[anchor=base] at ($(lo) + (1, 2.3)$) {$p$};

  \node[anchor=base] at ($(loact) + (0.5, 1.8)$) {$p'$};
  \node[anchor=base] at ($(loact) + (0.5, 0.3)$) {$p''$};

  \pattern[bgpattern] (lo) rectangle ($(loact) + (0,2)$);

  \draw ($(lo) + (0,2)$) -- (hi);
  \draw ($(lo) + (0,2)$) -- ($(loact) + (0,2)$);
  \draw (hi) -- (hiinf);
  \draw (loact) -- (loinf);
  \draw (loact) -- (hiact);

  \draw ($(loact) + (0,1)$) -- ($(loinf) + (0,1)$);
  \draw[opendash] decorate [opendeco] { (loinf) -- ($(loinf) + (0,2)$) };
  \draw[opendash] decorate [opendeco] { ($(loinf) + (0,2)$) -- (hiinf) };

\end{tikzpicture}
}
  \figcaption{Version-splitting an inactive page~$p$}
  {The horizontal axis represents life spans, and the vertical
  axis key ranges. 
  Case 
  (a)~represents a version split, 
  (b)~a version split followed by a key split,
  (c)~a version split followed by a merge with an inactive sibling, 
  (d)~a version split followed by a redistribution of live entries
  with an inactive sibling, 
  (e)~a version split followed by a merge with an active
  sibling, and 
  (f)~a version split followed by a redistribution of live entries
  with an active sibling.}
  \label{fig:split}
\end{center}
\end{figure}

All the possible different page-split scenarios for inactive pages are shown
in \figref{fig:split}. 
In the figure, the horizontal axis represents life spans, and the
vertical axis represents key ranges. 
In the presented scenarios, page~$p$ is split. 
Page~$s$ is the sibling page that is located from the parent of~$p$
found in the saved path.
Pages~$p'$ and~$p''$ are new pages allocated by the operation. 
As can be seen from the figures, all the scenarios preserve the
initial key-version extents of~$p$ and~$s$.
That is, the new pages cover exactly the same region in key-version
space as the old pages did. 
The version-split operation therefore preserves the combined spatial extents
of the pages that are involved in the split operation, and thus can neither
cause pages to overlap nor create gaps in the key-version space. 

A more detailed example of a version split followed by a merge with an inactive
sibling (as depicted in \figref{fig:split:imi1}) is shown in
\figref{fig:split-imi1-example}.
This example shows how the live entries of a leaf page~$p$ are merged with the
live entries of a sibling page~$s$ into a newly allocated page~$p'$.
As seen from the figure, the entries labeled~\num{1} and~\num{2} are killed,
and active copies of them are created into the new page~$p'$.
The third entry, entry~\num{3}, is already active, and it is therefore
physically moved from page~$s$ into the new page~$p'$, thus reducing the number
of entries stored in~$s$ by one.
After the split operation, pages~$p$ and~$s$ are dead and only contain dead
entries.

\begin{figure}[!htb]
\begin{center}
  \input{images/tmvbt-split-imi1-example.tex}
  \figcaption{Example of a page split}
  {This figure shows a version split followed by a merge with an inactive
  sibling. 
  Entries labeled~{1} and~{2} are killed and new copies of them are
  created in page~$p'$, while the entry~\num{3} is physically moved
  from page~$s$ into page~$p'$.}
  \label{fig:split-imi1-example}
\end{center}
\end{figure}

\thmskip
\begin{lemma}
\label{lemma:split-page-split-counts}
The \alg{split-page} operation maintains
\invref{inv:tmvbt-live-count}.
\end{lemma}
\begin{proof}
First of all, if the page~$p$ that is to be split is an active page, then 
the split operation is triggered because the page became full and thus
contains $\liveentries{p} = \capacity$ live entries.
In this case, the entries will be split between the page~$p$ and a newly
allocated page~$p'$, resulting in $\nicefrac{\capacity}{2}$ entries in each
page, satisfying $\minsplit \leq \nicefrac{\capacity}{2} \leq \maxsplit$. 
Secondly, if the page~$p$ is inactive, it is first killed, and the resulting
page~$p'$ contains $\liveentries{p}$ entries, where $\minlive \leq
\liveentries{p} \leq \capacity$. 
This is because the \alg{split-page} operation is only invoked when the
page~$p$ has become full, and by \invref{inv:tmvbt-live-count} it must
contain at least $\minlive$ live entries.
If $\liveentries{p} < \minsplit$, the page will be merged with another page
to avoid thrashing. 
Merging pages maintains the invariant, which is proven in the proof of
\lemmaref{lemma:merge-page-split-counts} for the \alg{merge-page} operation.
If $\liveentries{p} > \maxsplit$, the page will be key-split into two pages,
both of which will contain more than \minsplit\ entries. 
\end{proof}
\thmskip


\begin{algorithm}[htb]
$\proc{merge-page}(p, q)$:

\begin{algorithmic}[1]
\STATE $\liveentries{p} \gets$ count the number of live entries in $p$
\STATE $p' \leftarrow p$ if $p$ is active, \proc{kill-page}($p$) otherwise
%\IF[Figures~\ref{fig:merge:ama1}--\ref{fig:merge:ami2}]{$p$ is active}
%  \STATE $p' \leftarrow p$
%\ELSE[Figures~\ref{fig:merge:ima1}--\ref{fig:merge:imi2}]
%  \STATE $p' \leftarrow$ \proc{kill-page}($p$)
%\ENDIF
\STATE $s \leftarrow$ find a live sibling page of $p$ from $q$ and
write-latch it
\STATE $\liveentries{s} \gets$ count the number of live entries in $s$
\STATE $p'' \leftarrow s$ if $s$ is active, \proc{kill-page}($s$) otherwise 
%\IF[Figures~\ref{fig:merge:ama1},\ref{fig:merge:ama2},\ref{fig:merge:ima1},\ref{fig:merge:ima2}]{$s$
%is active} 
%  \STATE $p'' \leftarrow s$
%\ELSE[Figures~\ref{fig:merge:ami1},\ref{fig:merge:ami2},\ref{fig:merge:imi1},\ref{fig:merge:imi2}]
%  \STATE $p'' \leftarrow$ 
%\ENDIF
\IF[Figures~\ref{fig:merge:ama1},\ref{fig:merge:ami1},\ref{fig:merge:ima1},\ref{fig:merge:imi1}]{$\liveentries{p}
+ \liveentries{s} \leq \maxsplit$} 
  \STATE move all entries of $p''$ to $p'$ 
  %\STATE remove router to $p''$ from $q$ 
  %\STATE adjust router to $p'$ in $q$ 
  \STATE deallocate $p''$
\ELSE[Figures~\ref{fig:merge:ama2},\ref{fig:merge:ami2},\ref{fig:merge:ima2},\ref{fig:merge:imi2}]
  \STATE redistribute entries between $p'$ and $p''$
  %\STATE adjust router to $p'$ in $q$
  %\STATE adjust router to $p''$ in $q$
\ENDIF
%\STATE log the operation using a redo-only log record
%\STATE release page latches
\end{algorithmic}
\figcaption{The page-merging algorithm}%
{Merges a page~$p$ with a sibling page~$s$ located from the parent page~$q$,
killing the pages if they are inactive. 
Updates the routers in the parent page~$q$ accordingly.}
\label{alg:merge}
\end{algorithm}

% SMO: merge
The \alg{merge-page} operation is a structure-modification operation that
merges a page with a sibling page when the number of live entries in the
page is about to decrease below \minlive. 
This operation is triggered by the \action{delete} or \action{undo-write}
action, if an entry deletion is about to decrease the number of live entries
below \minlive; and by SMOs at a lower level.
% This operation is also used to merge index pages in the search path. 
An overview of the operation is shown in \algref{alg:merge}.
The operation begins by retrieving the page~$p$ to be merged from the saved
path. 
The page~$p$ is assumed to be write-latched at this point.
The parent page~$q$ also needs to be modified, so it is retrieved from
the path, and write-latched for modification. 
As with the \alg{split-page} operation, it is assumed that the parent
page~$q$ can accommodate the insertions or deletions possibly triggered by
this operation (insertion of up to two new routers; or deletion of a
single router).
%Again, the bottom-up checking phase needs to do the same checks that
%are shown in \algref{alg:merge}, and these results can be saved.

The merge operation continues by finding a live adjacent sibling page~$s$
from the parent page~$q$, and by write-latching it. 
Such a page is guaranteed to be found, because by
\invref{inv:tmvbt-live-count}, the parent page must contain at least two live
entries, and live entries are adjacent to each other.
If either page~$p$ or the sibling page~$s$ is inactive, it
is killed with the page-killing operation defined earlier; otherwise the
merge operation will be performed on it directly. 
We denote the active pages by~$p'$ (from page~$p$) and~$p''$ (from
page~$s$).
The operation now merges the active pages~$p'$ and~$p''$.

The merging is similar to the standard \Btree\ merge operation. 
If the number of combined entries of pages~$p'$ and~$p''$ (equivalently,
number of live entries in pages~$p$ and~$s$ before the page-killing
operations) is larger than \maxsplit, the entries will be redistributed
between the two pages. 
If the number of entries is less than or equal to \maxsplit, the entries
will be moved to page~$p'$, and page~$p''$ will be deallocated by
removing the router to it from the parent and by deallocating the page
from the corresponding space-map page.
The router to page~$p'$ (and to page~$p''$ in the former case) in the
parent~$q$ must be updated to match the new key ranges. 
Updating the routers in the parent page~$q$ is sufficient, because the pages
are active, and therefore have only one parent by
\lemmaref{lemma:active-pages-single-parent}.

An exception to the normal operation of \alg{merge-page} is when
the page~$p$ is a root page.
First of all, by \invref{inv:tmvbt-live-count}, a root page of version~$v$
must contain at least two entries that are alive at version~$v$, unless the
root page is the only page in the search tree~$S_v$. 
If $p$ is the only page in~$S_v$, then~$p$ is left as it is until a delete
operation deletes the last live entry from~$p$.  
At this point, it is possible to insert a null marker $(\actver,\nullmark)$
to \rootstar\ to signify that the current version search tree is empty.
However, for optimizing the space usage of the index, it is also possible to
allow the searches to locate the root page~$p$. 
Because $p$ contains no live entries, the queries will immediately notice
this and stop traversing the search tree.
If the null marker is inserted to \rootstar\ and $p$ is active, then the
insertion of $(\actver, \nullmark)$ replaces an existing tuple $(\actver,
p)$, and $p$ can be deallocated. 

\begin{figure}[tb]
\begin{center}
  \subfigure[$p$ active, $s$ active, $\liveentries{p} + \liveentries{s} \leq
  \maxsplit$]{\label{fig:merge:ama1}
    \begin{tikzpicture}[x=\tmvbtsmox,y=\tmvbtsmoy]
  \small
\coordinate (lo) at (0,0);
\coordinate (hi) at (0,3);
\coordinate (loact) at (2,0);
\coordinate (hiact) at (2,3);
\coordinate (loinf) at (3,0);
\coordinate (hiinf) at (3,3);

  \pattern[bgpattern] (lo) rectangle (hiact);

  \node[anchor=base] at ($(loact) + (0.5, 2.3)$) {$p$};
  \node[anchor=base] at ($(loact) + (0.5, 0.8)$) {$s$};

  \draw (loact) -- (hiact);
  \draw (loact) -- (loinf);
  \draw (hiact) -- (hiinf);
  \draw ($(loact)+(0,2)$) -- ($(loinf)+(0,2)$);
  \draw[opendash] decorate [opendeco] { (loinf) -- ($(loinf)+(0,2)$) };
  \draw[opendash] decorate [opendeco] { ($(loinf)+(0,2)$) -- (hiinf) };
\end{tikzpicture}
\raisebox{1.25\tmvbtsmoy}{$\rightarrow$}
\begin{tikzpicture}[x=\tmvbtsmox,y=\tmvbtsmoy]
\small
\coordinate (lo) at (0,0);
\coordinate (hi) at (0,3);
\coordinate (loact) at (2,0);
\coordinate (hiact) at (2,3);
\coordinate (loinf) at (3,0);
\coordinate (hiinf) at (3,3);

  \pattern[bgpattern] (lo) rectangle (hiact);

  \node[anchor=base] at ($(loact) + (0.5, 1.3)$) {$p'$};

  \draw (loact) -- (hiact);
  \draw (loact) -- (loinf);
  \draw (hiact) -- (hiinf);
  \draw[opendash] decorate [opendeco] { (loinf) -- (hiinf) };
\end{tikzpicture}
}
  \subfigure[$p$ active, $s$ active, $\liveentries{p} + \liveentries{s} >
  \maxsplit$]{\label{fig:merge:ama2}
    \input{images/tmvbt-merge-ama2.tex}}
  \subfigure[$p$ active, $s$ inactive, $\liveentries{p} + \liveentries{s}
  \leq \maxsplit$]{\label{fig:merge:ami1}
    \begin{tikzpicture}[x=\tmvbtsmox,y=\tmvbtsmoy]
  \small
\coordinate (lo) at (0,0);
\coordinate (hi) at (0,3);
\coordinate (loact) at (2,0);
\coordinate (hiact) at (2,3);
\coordinate (loinf) at (3,0);
\coordinate (hiinf) at (3,3);

  \pattern[bgpattern] ($(lo) + (0,2)$) rectangle (hiact);

  \node[anchor=base] at ($(loact) + (0.5, 2.3)$) {$p$};
  \node[anchor=base] at ($(lo) + (1.5, 0.8)$) {$s$};

  \draw (lo) -- ($(lo) + (0,2)$);
  \draw (lo) -- (loinf);
  \draw (hiact) -- (hiinf);
  \draw ($(lo)+(0,2)$) -- ($(loinf)+(0,2)$);
  \draw ($(loact)+(0,2)$) -- (hiact);
  \draw[opendash] decorate [opendeco] { (loinf) -- ($(loinf)+(0,2)$) };
  \draw[opendash] decorate [opendeco] { ($(loinf)+(0,2)$) -- (hiinf) };
\end{tikzpicture}
\raisebox{1.25\tmvbtsmoy}{$\rightarrow$}
\begin{tikzpicture}[x=\tmvbtsmox,y=\tmvbtsmoy]
\small
\coordinate (lo) at (0,0);
\coordinate (hi) at (0,3);
\coordinate (loact) at (2,0);
\coordinate (hiact) at (2,3);
\coordinate (loinf) at (3,0);
\coordinate (hiinf) at (3,3);

  \pattern[bgpattern] ($(lo) + (0,2)$) rectangle (hiact);

  \node[anchor=base] at ($(loact) + (0.5, 1.3)$) {$p'$};
  \node[anchor=base] at ($(lo) + (1, 0.8)$) {$s$};

  \draw ($(loact) + (0,2)$) -- (hiact);
  \draw (loact) -- (loinf);
  \draw (hiact) -- (hiinf);
  \draw (lo) rectangle ($(loact)+(0,2)$);
  \draw[opendash] decorate [opendeco] { (loinf) -- (hiinf) };
\end{tikzpicture}
}
  \subfigure[$p$ active, $s$ inactive, $\liveentries{p} + \liveentries{s} >
  \maxsplit$]{\label{fig:merge:ami2}
    \input{images/tmvbt-merge-ami2.tex}}
  \figcaption{Merging an active page~$p$}%
  {The horizontal axis represents life spans, and the vertical
  axis key ranges.
  Case 
  (a)~represents a merge, 
  (b)~a redistribution of live entries, 
  (c)~a merge with an inactive sibling,
  (d)~a redistribution of live entries with an inactive sibling.}
  \label{fig:merge-active}
\end{center}
\end{figure}

If $p$ is a root page and the height of~$p$ is greater than one, then the
page-merge operation is triggered by an SMO at a lower level, and we must
check whether the operation would cause the number of live entries in~$p$ to
decrease to one.
If this is the case, then the height of the search tree~$S_v$ will be
decremented by making the only remaining live child page of~$p$ the
new root of~$S_v$.
This cannot be done before the SMO at the lower lever, however, so the
tree-height-decrease operation must be performed as a part of the SMO at the
lower level.
Therefore, after the SMO at the lower level, if the number of entries in the
root page~$p$ has decreased to one, the only remaining child page~$r$ of $p$
is located from $p$, and a tuple $(\actver, r)$ is inserted to \rootstar.
If $p$ is active, this will replace the existing tuple $(\actver, p)$, and
page~$p$ can be deallocated.
The cached root page identifier of $S_{\actver}$ must also be updated.

The entire \alg{merge-page} operation, including the possible
root-page-update operation, is logged with a single redo-only log record
containing the page identifiers of all related pages---this means pages~$p$,
$s$, $p'$, $p''$, $q$, and~$r$.
The log record must contain sufficient information of all the moved
entries, like the log record of the split-page operation. 
For example, the log record for a \alg{merge-page} operation that merges an
active page with an inactive sibling, resulting in a single active page
(\figref{fig:merge:ami1}), is
\lrb{T}, \logact{merge-active-inactive}, $p$, $s$, $q$, $E_p$, \lre{E_s},
where $E_x$ denotes the set of entries present in page~$x$ after the
operation. 
Note that in this situation, the active page~$p$ is reused as~$p'$, and thus
$p'$~is not present in the log record.
After the log record has been written, the saved path is returned to a proper
state, so that the active page whose key range covers~$k$ (either~$p'$
or~$p''$) is placed in the saved path to replace the merged page.
The operation finishes by releasing the write latches on the pages.

\begin{figure}[!htb]
\begin{center}
  \subfigure[$p$ inactive, $s$ active, $\liveentries{p} + \liveentries{s}
  \leq \maxsplit$]{\label{fig:merge:ima1}
    \input{images/tmvbt-merge-ima1.tex}}
  \subfigure[$p$ inactive, $s$ active, $\liveentries{p} + \liveentries{s} >
  \maxsplit$]{\label{fig:merge:ima2}
    \input{images/tmvbt-merge-ima2.tex}}
  \subfigure[$p$ inactive, $s$ inactive, $\liveentries{p} + \liveentries{s} \leq
  \maxsplit$]{\label{fig:merge:imi1} 
    \input{images/tmvbt-merge-imi1.tex}}
  \subfigure[$p$ inactive, $s$ inactive, $\liveentries{p} + \liveentries{s} >
  \maxsplit$]{\label{fig:merge:imi2}
    \begin{tikzpicture}[x=\tmvbtsmox,y=\tmvbtsmoy]
  \small
\coordinate (lo) at (0,0);
\coordinate (hi) at (0,3);
\coordinate (loact) at (2,0);
\coordinate (hiact) at (2,3);
\coordinate (loinf) at (3,0);
\coordinate (hiinf) at (3,3);


  \node[anchor=base] at ($(lo) + (1.5, 2.3)$) {$p$};
  \node[anchor=base] at ($(lo) + (1.5, 0.8)$) {$s$};

  \draw (lo) -- (loinf);
  \draw (lo) -- (hi);
  \draw (hi) -- (hiinf);
  \draw ($(lo)+(0,2)$) -- ($(loinf)+(0,2)$);
  \draw[opendash] decorate [opendeco] { (loinf) -- ($(loinf)+(0,2)$) };
  \draw[opendash] decorate [opendeco] { ($(loinf)+(0,2)$) -- (hiinf) };
\end{tikzpicture}
\raisebox{1.25\tmvbtsmoy}{$\rightarrow$}
\begin{tikzpicture}[x=\tmvbtsmox,y=\tmvbtsmoy]
\small
\coordinate (lo) at (0,0);
\coordinate (hi) at (0,3);
\coordinate (loact) at (2,0);
\coordinate (hiact) at (2,3);
\coordinate (loinf) at (3,0);
\coordinate (hiinf) at (3,3);

  \node[anchor=base] at ($(lo) + (1, 2.3)$) {$p$};
  \node[anchor=base] at ($(lo) + (1, 0.8)$) {$s$};
  \node[anchor=base] at ($(loact) + (0.5, 1.8)$) {$p'$};
  \node[anchor=base] at ($(loact) + (0.5, 0.3)$) {$p''$};

  \draw (lo) -- (loinf);
  \draw (lo) -- (hi);
  \draw (hi) -- (hiinf);

  \draw ($(lo) + (0,2)$) -- ($(loact) + (0,2)$);

  \draw (loact) -- (hiact);
  \draw ($(loact) + (0,1)$) -- ($(loinf) + (0,1)$);
  \draw[opendash] decorate [opendeco] { (loinf) -- ($(loinf)+(0,1)$) };
  \draw[opendash] decorate [opendeco] { ($(loinf)+(0,1)$) -- (hiinf) };
\end{tikzpicture}
}
  \figcaption{Merging an inactive page~$p$}
  {The horizontal axis represents life spans, and the vertical
  axis key ranges.
  Case
  (a)~a represents a version split followed by a merge,
  (b)~a version split followed by a redistribution of live entries,
  (c)~a version split followed by a merge with an inactive
  sibling, and
  (d)~a version split followed by a redistribution of live entries
  with an inactive sibling.}
  \label{fig:merge-inactive}
\end{center}
\end{figure}

The possible page-merge scenarios for page~$p$ are shown
in Figures~\ref{fig:merge-active} and \ref{fig:merge-inactive}. 
In the figures, page~$p$ is merged with a live sibling page~$s$. 
For consistency with Figures~\ref{fig:split-active} and \ref{fig:split}, the
resulting active pages are denoted~$p'$ and~$p''$, even though $p' = p$ when
the page~$p$ is active, and $p'' = s$ when $s$ is active.
For clarity of presentation, the algorithms now create a new page~$p'$ in the
situation depicted in \figref{fig:merge:ima1}, and deallocate the existing
active sibling page~$s$, instead of reusing~$s$ as the resulting merged page. 
An actual implementation may of course correct this and reuse page~$s$
instead of creating~$p'$.

\thmskip
\begin{lemma}
\label{lemma:merge-page-split-counts}
The \alg{merge-page} operation maintains
\invref{inv:tmvbt-live-count}.
\end{lemma}
\begin{proof}
The page~$p$ that is to be merged must have exactly \minlive\ live entries,
while the sibling page~$s$ can have between \minlive\ and \capacity\ live
entries. 
The combined number of live entries $e: 2 \times \minlive \leq e \leq
\minlive + \capacity$ can always be distributed among the two pages so that
the number of entries in both pages is between \minsplit\ and \maxsplit. 
In the minimum case, $2 \times \minlive \geq \minsplit$ entries are
moved to page $p'$, and page $p''$ is deleted.
Note also that $2 \times \minlive < \maxsplit$, and
\invref{inv:tmvbt-minsplit-maxsplit} thus holds for page~$p'$.
In the maximum case, $\minlive + \capacity$ entries are distributed
between $p'$ and $p''$, resulting in more than $\minsplit$ and less
than $\maxsplit$ entries per page. 
The former holds trivially (as we are distributing more than
$\maxsplit \geq 2 \times \minsplit$ entries between two pages), and the
latter holds because $\minlive + \capacity = (\minsplit - s) + (\maxsplit +
s) = \minsplit + \maxsplit > 2 \times \minsplit$.
\end{proof}
\thmskip

\thmskip
\begin{lemma}
\label{lemma:tmvbt-latchcount}
Any structure-modification operation needed in the implementation of
the \action{insert}, \action{delete}, \action{undo-insert} and
\action{undo-delete} actions keeps at most five TMVBT pages latched
simultaneously and transforms a balanced (\defref{def:consistent-balanced})
TMVBT index into another balanced one. 
For any of the actions, at most $h+1$ structure modifications are needed,
where $h$ is the height of the search tree $S_{\actver}$ of the active
database version \actver.
\end{lemma}
\begin{proof}
As seen from the Figures~\ref{fig:split-active}--\ref{fig:merge-inactive},
when an SMO is triggered on page~$p$ at level~$l$, at most four pages at
level~$l$ are involved in the operation.
In addition to these four, the parent page~$q$ is also latched during the
SMO\@.
If a root-page update occurs (either a tree-height-increase or a
tree-height-decrease), then the root page identifier is also present in the
log records.
However, the root page does not increment the number of pages latched at a
time, because
(1)~during a tree-height-increase, there is no parent page~$q$, so the maximum
number of latches required is still five; and
(2)~during a tree-height-decrease, the new root~$r$ is the page~$p'$, because
just before the root-page update, $p'$ is the only remaining child page
of~$q$.

When considering the balance condition,
Lemmas~\ref{lemma:kill-page-split-counts},
\ref{lemma:split-page-split-counts} and~\ref{lemma:merge-page-split-counts}
show that each SMO maintains \invref{inv:tmvbt-live-count}, which proves that
if the TMVBT is balanced before the SMO, then it is also balanced  
immediately after the SMO\@.
\end{proof}
\thmskip

The following theorem states that the update actions also maintain
the asymptotic bounds of the MVBT\@:

\thmskip
\begin{theorem}
\label{thm:tmvbt-update-cost}
Assuming that the root page of the database version~$v$ is known, 
each of the actions \action{insert}, \action{delete}, \action{undo-insert},
and \action{undo-delete} is performed in time \OhT{\log_{\capacity}
\entries{\actver}}, where $\entries{\actver}$ is the number of data items in
the active database version \actver.
\end{theorem}
\begin{proof}
Searching for the correct leaf page to perform the update has a cost of
\OhT{\log_{\capacity} \entries{\actver}} pages, because the search
tree~$S_v$ is balanced.
By \lemmaref{lemma:tmvbt-latchcount}, at most $h+1$ structure-modification
operations are needed to perform the update, where $h = \OhT{\log_\capacity
\entries{\actver}}$.
At each level, at most five pages need to be accessed to perform the
structure-modification operations.
The structure-modification operations thus have a maximum cost of
\OhT{\log_\capacity \entries{\actver}}, which is the same as the initial tree
traversal.
\end{proof}
\thmskip

% Space complexity
As discussed in \secref{sec:tsbmvbt:mvbt}, the MVBT index has a space
complexity of \Oh{n/\capacity} database pages, where $n$ is the number of
updates performed during the history of the database, and \capacity\ is the
page capacity. 
This has been proven by Becker
et~al.~\cite{becker:1993:optimal,becker:1996:mvbt}.
Their proof relies on the fact that, for leaf pages,
(1)~each structure-modification operation that targets a page~$p$ creates at
most a constant number of new pages (at most two); and 
(2)~at least a constant minimum number of update operations ($\epsilon d$ in
the discussion of Becker et~al.~\cite{becker:1996:mvbt}, $s$ for the
TMVBT) must have been performed on a page~$p$ for it to require a
structure-modification operation.
Therefore, the amortized cost of the update actions is at most $\Oh{2
\capacity/s} = \Oh{1}$ new entries, if $s$ is linearly dependant on the
page capacity~\capacity.
This directly translates to the reported \Oh{n/\capacity} leaf pages for a
history of $n$ update actions, because each leaf page holds at least $\minlive
= \Oh{\capacity}$ entries.
Similarly, for the index pages of the MVBT, each SMO requires space for at most
two new index entries at the first index page level (at height two, directly
above the leaf pages). 
Becker et~al.~\cite{becker:1996:mvbt} show that these insertions at the
index level do not increase the amortized space complexity of the update
actions, if $d \geq \nicefrac{2}{\epsilon}$.
This translates to $s \geq 2$, which is a reasonable requirement for the
split tolerance variable in any case.
 
When considering the space complexity of the TMVBT, we note that the premises
of the MVBT space complexity proof remain true: 
each SMO creates at most a constant number of new pages (at most two), 
at most a constant number of new entries (again, two) are inserted into the
parent page, and 
at least a minimum number of update actions must target page~$p$ before any
SMO is required on it ($s$ actions).
We can thus conclude that the space complexity of the TMVBT is 
\Oh{n/\capacity} pages, if $s$ is linearly dependent on the page
capacity~\capacity, and $s \geq 2$.

In the discussion of space complexity for any of the efficient
multiversion index structures, the complexity has been given in terms of the
total number of update actions performed on the index structure. 
However, a transaction may insert many entries which it then later on deletes;
the index structure should not consume space for the actions that are undone.
For the TMVBT we can state the space complexity in terms of the number of
entries that are alive after transactions have committed; in these terms, the
space complexity is \Oh{(\sum_{v \in V} m_v) / \capacity} database pages,
where $V$ is the set of all versions, and $m_v$ is the number of entries that are
alive at version~$v$.

\thmskip
\begin{theorem}
The space complexity of the TMVBT index is \[\Oh{\min \{ n, \sum_{v
\in V} m_v \} / \capacity}.\]
\end{theorem}
\begin{proof}
The proof for the first part of the result, \Oh{n/\capacity}, is the same as the
proof for MVBT~\cite{becker:1993:optimal,becker:1996:mvbt} (see the discussion
above). 
For the next part, consider any fixed version~$v$.
\invref{inv:tmvbt-live-count} states that each page of the search
tree~$S_v$ contains at least \minlive\ entries that are alive at version~$v$. 
Consider now the leaf pages of the search tree~$S_v$. 
We can see from the SMOs that entries are duplicated only when creating live
copies of them; that is, duplicates differ in their life spans, not in their
keys or key ranges.
This means that within a search tree of any version~$v$, there are no
duplicates of any of the entries that are alive at version~$v$, and thus there
are at most $\Oh{m_v/\minlive} = \Oh{m_v/\capacity}$ leaf pages in~$S_v$,
because we require that \minlive\ is linearly dependent on the page
capacity~\capacity. 
Furthermore, by a proof that is similar to the proof
of~\thmref{thm:btree-range-cost}, there are at most \Oh{m_v/\capacity}
index pages in the search tree~$S_v$. 
From this we arrive to our result by summation over the set of all
versions~$V$.
\end{proof}
\thmskip

% Recovery
In restart recovery from a system crash, an
\abbr{ARIES}-based~\cite{mohan:1992:aries,mohan:1990:aries-kvl} recovery
algorithm is used.

\thmskip
\begin{theorem}
\label{thm:tmvbt-recovery}
In the event of a system crash, the redo pass of restart recovery produces a
balanced TMVBT on which the undo actions by a backward-rolling updating
transaction (if any) can be performed logically if a physical undo is
impossible.
\end{theorem} 
\begin{proof}
Each structure-modification operation is performed as a single atomic
operation that transforms a balanced TMVBT index into another balanced TMVBT
index. 
After the redo pass has finished, the TMVBT index is therefore balanced.
Thus, each of the undo actions is performed logically, if a physical undo is
not possible.
\end{proof}
\thmskip



%% Challenges With Concurrent Updates
%%---------------------------------------------------------------------
\section{Challenges With Concurrent Updates}
\label{sec:tmvbt:multiupdate}

The structure-modification operations presented in the previous section
are based on the fact that there can be only a single updating transaction
operating on the index structure at a time.
We believe that it is not directly possible to further generalize the index
structure so that multiple updating transactions could apply their updates
directly on the leaf pages of the TMVBT index.
We now present an example of the problematic situations that can arise should
multiple updating transactions be allowed to operate on the index
concurrently.

Assume that a transaction~$T_1$ has created a leaf page~$p$ that contains two
entries inserted by the transaction. 
In this illustrative example, we have set $\minlive = 2$, so that
\invref{inv:tmvbt-live-count} holds for the index structure. 
The situation after~$T_1$ has committed with a commit-time version
$\commitver{T_1} = 1$ is depicted in \figref{fig:tmvbt:multi-update:initial}.

Assume now that two active updating transactions~$T_2$ and~$T_3$ perform
updates so that (1)~$T_2$~inserts entries with keys preceding and succeeding
the keys of the entries inserted by~$T_1$; and 
(2)~$T_3$~deletes the entries inserted by~$T_1$.
If we wish to be able to store the updates of multiple active updating
transactions in the leaf pages of the TMVBT index, we must use entries such as
the pending updates defined in \chapref{chapter:mv-data} (see
\defref{def:pending-update}). 
One possible way of modeling this situation is shown in
\figref{fig:tmvbt:multi-update:active}.

\begin{figure}[!htb]
\begin{center}
\subfigure[$T_1$ has committed]{\label{fig:tmvbt:multi-update:initial}
\begin{tikzpicture}
\node[treepage] (p) {%
\treepagecontents{10.5}{$[0,10),[1,\!\infty\!)$}%
  {%%
   $(3,[1,\!\infty\!),w_3)$\\
   $(4,[1,\!\infty\!),w_4)$}};
\treepageheader{p}{$p$};
\end{tikzpicture}
\quad}
\subfigure[Updates by $T_2$ and~$T_3$]{\label{fig:tmvbt:multi-update:active}
\begin{tikzpicture}
\node[treepage] (p) {%
\treepagecontents{10.5}{$[0,10),[1,\!\infty\!)$}%
  {%%
   $(1,T_2,w_1)$\\
   $(2,T_2,w_2)$\\
   $(3,[1,\!\infty\!),w_3)$\\
   $(3,T_3,\deletemark)$\\
   $(4,[1,\!\infty\!),w_4)$\\
   $(4,T_3,\deletemark)$\\
   $(5,T_2,w_5)$\\
   $(6,T_2,w_6)$}};
\treepageheader{p}{$p$};
\end{tikzpicture}
\quad}
\subfigure[$T_2$ commits first]{\label{fig:tmvbt:multi-update:t2first}
\begin{tikzpicture}
\node[treepage] (p) {%
\treepagecontents{10.5}{$[0,10),[1,\!\infty\!)$}%
  {%%
   $(1,[2,\!\infty\!),w_1)$\\
   $(2,[2,\!\infty\!),w_2)$\\
   $(3,[1,3),w_3)$\\
   $(4,[1,3),w_4)$\\
   $(5,[2,\!\infty\!),w_5)$\\
   $(6,[2,\!\infty\!),w_6)$}};
\treepageheader{p}{$p$};
\end{tikzpicture}
\quad}
\subfigure[$T_3$ commits first]{\label{fig:tmvbt:multi-update:t3first}
\begin{tikzpicture}
\node[treepage] (p) {%
\treepagecontents{10.5}{$[0,10),[1,\!\infty\!)$}%
  {%%
   $(1,[3,\!\infty\!),w_1)$\\
   $(2,[3,\!\infty\!),w_2)$\\
   $(3,[1,2),w_3)$\\
   $(4,[1,2),w_4)$\\
   $(5,[3,\!\infty\!),w_5)$\\
   $(6,[3,\!\infty\!),w_6)$}};
\treepageheader{p}{$p$};
\end{tikzpicture}
\quad}
\figcaption{TMVBT invariant fails with concurrent updates}%
{If transaction~$T_3$ commits before~$T_2$, the number entries that are alive
at version~\num{2} is zero.}
\label{fig:tmvbt:multi-update}
\end{center}
\end{figure}

If~$T_2$ commits first with $\commitver{T_2} = 2$, and
$T_3$ afterwards with $\commitver{T_3} = 3$, the logical
state of the database after both commit operations is shown in
\figref{fig:tmvbt:multi-update:t2first}. 
We assume here that a lazy timestamping scheme is employed so that the
actual entries stored in the database may be the same entries as shown in
\figref{fig:tmvbt:multi-update:active}, but when the page is
later accessed the pending updates on the page will be converted into the
entries shown in \figref{fig:tmvbt:multi-update:t2first}.
In this situation, \invref{inv:tmvbt-live-count} holds for the page~$p$,
because $\entries{1} = 2$, $\entries{2} = 6$, and $\entries{3} = 4$; where
\entries{v} denotes the number of entries that are alive at version~$v$.

Suppose now that transactions~$T_2$ and~$T_3$ commit in reverse order, so
that $T_3$ commits first with $\commitver{T_3} = 2$, and $T_2$ commits later
with $\commitver{T_2} = 3$. 
The situation after these commits is shown in
\figref{fig:tmvbt:multi-update:t3first}.
Now, $\entries{2} = 0$, and \invref{inv:tmvbt-live-count} does not hold.
Clearly, situations such as these should be prevented if the optimality of the
TMVBT index is to be preserved.

To prevent such situations we would have to be able to (1)~identify such
situations when the updating transactions are active, and (2)~remedy the
situation by applying proper structure-modification operations.
The situations must be identified while the transactions are active, because
we cannot assume that the invariant could be checked and enforced at
transaction commit, because this would require that each page the transaction
has modified be checked during transaction commit, which is too costly to be
practical.
Secondly, even if we assume that such situations could be identified
when the updating transactions are active, it may be impossible to remedy the
situation.
To show this, consider the example situation shown in
\figref{fig:tmvbt:multi-update:active}.
As \figref{fig:tmvbt:multi-update:t3first} shows, this situation may lead to
the invalidation of \invref{inv:tmvbt-live-count}.
Normally, when the number of live entries in a page falls below \minlive, it
is possible to merge the page with a sibling page.
In this situation, however, there is no more room on the page for it to be
merged with any sibling page, and the page cannot be version-split because
the ordering of the entries created by the active transactions is not known.
We conclude that we cannot allow multiple updating transactions to apply
updates on the leaf pages of the TMVBT index directly, if the optimality of
the index is to be maintained.



%% Summary
%%---------------------------------------------------------------------
\section{Summary}
\label{sec:tmvbt:summary}

The TMVBT index is an optimal multiversion index for key-range queries (i.e.,
\qtype{$x$/$-$/point} queries, see \secref{sec:mv-data:queries}), provided
that the root of the queried version is known. 
However, only a single updating transaction can operate on it at a time.
The algorithms presented in this chapter are efficient, because no
key-level locking is required, and no latch-coupling is needed. 
As such, the TMVBT index is an optimal multiversion index structure for any
application where there is only a single source of updates, and many
clients performing queries.
Our goal is, however, to design an efficient general-purpose multiversion
index structure.
The next chapter reviews the concurrent multiversion \Btree\ (CMVBT), which
is a general-purpose multiversion index that is based on the optimal TMVBT
index.
Multiple updating transaction can operate on the CMVBT index concurrently.

