%%=====================================================================
%% Conclusions
%%=====================================================================
\chapter{Conclusions}
\label{chapter:conclusions}

A current trend in database systems is that the history of the database
contents must be accessible, in addition to the current state of the data
set.
The traditional single-version \Btree\ index can be straightforwardly
extended so that multiversion data may be indexed by it, but the versioned
extension is not efficient, as was shown in \chapref{chapter:mv-index}.
The problem with this approach is the range-query action, which is
inefficient because the data items with the same key but with different
versions are clustered close to each other, while the data items with the
same version but with different keys are not.
% the single-key actions (key query, key insertion, and key deletion) on
% a set of multiversion data can be implemented efficiently with a
% single-version \Btree\ index
Efficient indexing of the data set evolution therefore requires a
multiversion index structure.
 
When querying for any fixed version~$v$, an optimal multiversion index
structure should be as efficient as a single-version index structure that
only indexes the data items that are alive at version~$v$. 
We have defined this as the requirement for optimality
of multiversion indexes (\defref{def:optimal-mv} in
\secref{sec:mv-index:optimality}).
This guarantees that range queries remain efficient even if the database
accumulates a long history of updates.
When data items are logically deleted, the range queries that target the
latest committed version should become more efficient as fewer data items are
alive.
Range queries are an important operation in a general-purpose database
system because index scans and joins are based on them.

We have reviewed three of the most efficient multiversion index structures in
\chapref{chapter:tsbmvbt}.
These are the \TSBtree\ of Lomet and
Salzberg~\cite{lomet:1989:tsb,lomet:1990:tsb-performance}, the multiversion
\Btree\ (MVBT) of Becker et~al.~\cite{becker:1993:optimal,becker:1996:mvbt},
and the multiversion access structure (MVAS)
of Varman and Verma~\cite{varman:1997:multiversion}.
From these structures, only the MVBT is considered optimal by our definition
of optimality.
The problem with MVBT and MVAS is that they follow a single-update model, in
which the update cannot be rolled back, and therefore these indexes cannot
be used as general database indexes in a transactional multi-user
environment.
On the other hand, the \TSBtree\ does not have this restriction, but it does
not guarantee any optimal bounds for the range-query performance, either.
In particular, in the presence of logical deletions, the performance of the
\TSBtree\ degrades because the leaf pages of the index structure are not
merged.

As an initial step, we have introduced transactions to the MVBT by
redesigning it, as described in \chapref{chapter:tmvbt}.
The redesigned transactional MVBT (TMVBT) index retains the optimal bounds of
the MVBT and allows one updating multi-action transaction to operate
concurrently with multiple read-only transactions.
The TMVBT index is an efficient index structure that is usable on its own in
situations where there is only a single source of updates, such as in data
stream management systems.

In \chapref{chapter:cmvbt} we presented the design of our concurrent
multiversion \Btree\ (CMVBT) index which uses a separate main-memory-resident 
versioned \Btree~(VBT) index to store the pending updates created by active
transactions, and a TMVBT index for indexing the data items inserted by
committed transactions.
Once an active transaction~$T$ has committed, a system maintenance
transaction is run to apply the updates of~$T$ from the VBT index
into the main TMVBT index.
We say that a version~$v$ is stable, if all the updates of the
transaction~$T$ that created the version~$v$ have been applied to the TMVBT
index.
The CMVBT index is thus optimal when querying for the data items of 
stable versions, and guarantees that the performance of the queries never
degrades, even in the presence of deletions.
The separate VBT index is kept small by constantly moving the
updates of committed transactions into the TMVBT index.
The VBT can be kept entirely in main memory during general transaction
processing, and it does not incur any additional \abbr{I/O} operations.

Our CMVBT algorithms are designed to work in a multi-user environment with
multiple concurrent updating transactions. 
We allow transactions to roll back; either entirely, or up to a preset 
savepoint.
Standard concurrency-control algorithms can be used to maintain logical
data consistency.
The snapshot isolation
algorithms~\cite{berenson:1995:sql-critique}
are especially well suited for use with our multiversion index
structure, and they guarantee snapshot isolation for all transactions. 
Our algorithms are made recoverable by the ARIES recovery
algorithm~\cite{mohan:1992:aries,mohan:1992:aries-im}, and we apply
structure-modification operations on a level-by-level basis, performing each
SMO as an atomic action that transforms a balanced index into another
balanced
index~\cite{jaluta:2003:recoverable,jaluta:2005:blink,jaluta:2006:page-server}.

Because the commit-time version of a data item is not known when the
transaction that created it is active, the data-item updates must
initially be tagged with transaction identifiers, which are later 
changed to commit-time versions.
The CMVBT index organization allows the data-item versions to be efficiently
changed from transaction identifiers into commit-time versions when the 
maintenance transaction moves the updates from the VBT into the TMVBT index.
This is a non-trivial issue that often requires special book-keeping
arrangements in other multiversion index structures that support
commit-time versioning.

We have experimentally analyzed the performance of the CMVBT index
in \chapref{chapter:performance} and compared it to the performance of the
\TSBtree.
The results we obtained from our experiments agree with what we expected
from our analytical results.
The CMVBT index structure performs on par with the \TSBtree\ index in
standard transaction processing, but is more efficient for key-range
queries.
The efficiency of the key-range queries is especially apparent if the
history of the database contains deletions.
We have furthermore compared the combined CMVBT index to the TMVBT, and
conclude that the separate VBT index does not affect the overall performance
significantly.
For completeness, we have also demonstrated that the performance of 
range queries degrades rapidly if the multiversion data items are indexed
by a single-version \Btree\ index.

There is a downside to the optimal performance of the TMVBT index;
namely, that the index structure requires more space than the \TSBtree.
While the asymptotic space complexity of all the compared index structures
is the same, the size of the TMVBT (in itself, and as part of the CMVBT
index) was \SIrange{10}{60}{\percent} greater than the size of the \TSBtree\
in our tests, depending on the \TSBtree\ splitting policy and on the number
of deletions in the database history. 
Our conclusion is thus that the CMVBT structure is a good choice for a
general-purpose multiversion index when performance is more important
than storage space, especially so when it is expected that the history
will also contain key deletions.
The \TSBtree\ is a better choice when storage space is limited, particularly
if historical data is to be stored on a tertiary storage, because the
\TSBtree\ allows historical pages to be moved during time-splits. 

