%%=====================================================================
%% Transactions on Multiversion Data
%%=====================================================================
\chapter{Transactions on Multiversion Data}
\label{chapter:dummy-layout-text}

% Explanation of the details of transactions on multiversion data.
% The transaction model.
% No implementation (index structure) details.

The theory of general transaction processing in traditional 
databases is well-defined and mature, and the basic principles are
well presented in many
textbooks~\cite{bernstein:1987:cc-n-r,gray:1993:transactionprocessing,papadimitriou:1986:cc-theory}.
In this chapter, we concentrate on the theory of multiversion
transaction processing, and highlight the differences to the classical
transaction theory.
The traditional read-write model assumes that transactions are sequences
of reads and writes on data items, without distinguishing item deletions and
insertions from
updates~\cite{bernstein:1987:cc-n-r,papadimitriou:1986:cc-theory}.
The theory of transaction processing in this dissertation is based on
the recoverable transaction model presented by Sippu and
Soisalon-Soininen~\cite{sippu:2001:theory}, which in turn is based on the
model proposed by C.~Mohan~\cite{mohan:1990:aries-kvl,mohan:1992:aries-im}.
In this model, data-item insertions and deletions are made explicit,
and structure-modification operations are included in the model.
We assume the partially persistent transaction-time model, as
described in the introduction.



%% Challenges With Concurrent Updates
%%---------------------------------------------------------------------
\section{Challenges With Concurrent Updates}
\label{sec:tmvbt:multiupdate}

The structure-modification operations presented in the previous section
are based on the fact that there can only be a single updating transaction
operating on the index structure at a time.
We believe that it is not directly possible to further generalize the index
structure so that multiple updating transactions could apply their updates
directly on the leaf pages of the TMVBT index.
We now present an example of the problematic situations that can arise should
multiple updating transactions be allowed to operate on the index
concurrently.

Assume that a transaction~$T_1$ has created a leaf page~$p$ that contains two
entries inserted by said transaction.
In this illustrative example, we have $\minlive = 2$, so that
\invref{inv:tmvbt-live-count} holds for the index structure. 
The situation after~$T_1$ has committed is depicted in
\figref{fig:tmvbt:multi-update:initial}.

Assume now that two new transactions~$T_2$ and~$T_3$ perform updates so that
(1)~$T_2$~inserts entries with keys preceding and succeeding the keys of the
entries inserted by~$T_1$; and 
(2)~$T_3$~deletes the entries inserted by~$T_1$.
If we wish to be able to store the updates of multiple active updating
transactions in the leaf pages of the TMVBT index, we must use entries such as
the pending updates defined in \chapref{chapter:mv-data} (see
\defref{def:pending-update}). 
One possible way of modeling this situation is shown in
\figref{fig:tmvbt:multi-update:active}.

\begin{figure}[!htb]
\begin{center}
\subfigure[$T_1$ has committed]{\label{fig:tmvbt:multi-update:initial}
\begin{tikzpicture}
\node[treepage] (p) {%
\treepagecontents{10.5}{$[0,10),[1,\!\infty\!)$}%
  {%%
   $(3,[1,\!\infty\!),w_3)$\\
   $(4,[1,\!\infty\!),w_4)$}};
\treepageheader{p}{$p$};
\end{tikzpicture}
\quad}
\subfigure[Updates by $T_2$ and~$T_3$]{\label{fig:tmvbt:multi-update:active}
\begin{tikzpicture}
\node[treepage] (p) {%
\treepagecontents{10.5}{$[0,10),[1,\!\infty\!)$}%
  {%%
   $(1,T_2,w_1)$\\
   $(2,T_2,w_2)$\\
   $(3,[1,\!\infty\!),w_3)$\\
   $(3,T_3,\deletemark)$\\
   $(4,[1,\!\infty\!),w_4)$\\
   $(4,T_3,\deletemark)$\\
   $(5,T_2,w_5)$\\
   $(6,T_2,w_6)$}};
\treepageheader{p}{$p$};
\end{tikzpicture}
\quad}
\subfigure[$T_2$ commits first]{\label{fig:tmvbt:multi-update:t2first}
\begin{tikzpicture}
\node[treepage] (p) {%
\treepagecontents{10.5}{$[0,10),[1,\!\infty\!)$}%
  {%%
   $(1,[2,\!\infty\!),w_1)$\\
   $(2,[2,\!\infty\!),w_2)$\\
   $(3,[1,3),w_3)$\\
   $(4,[1,3),w_4)$\\
   $(5,[2,\!\infty\!),w_5)$\\
   $(6,[2,\!\infty\!),w_6)$}};
\treepageheader{p}{$p$};
\end{tikzpicture}
\quad}
\subfigure[$T_3$ commits first]{\label{fig:tmvbt:multi-update:t3first}
\begin{tikzpicture}
\node[treepage] (p) {%
\treepagecontents{10.5}{$[0,10),[1,\!\infty\!)$}%
  {%%
   $(1,[3,\!\infty\!),w_1)$\\
   $(2,[3,\!\infty\!),w_2)$\\
   $(3,[1,2),w_3)$\\
   $(4,[1,2),w_4)$\\
   $(5,[3,\!\infty\!),w_5)$\\
   $(6,[3,\!\infty\!),w_6)$}};
\treepageheader{p}{$p$};
\end{tikzpicture}
\quad}
\figcaption{TMVBT invariant fails with concurrent updates}%
{If transaction~$T_3$ commits before~$T_2$, the number entries that are alive
at version~\num{2} is zero.}
\label{fig:tmvbt:multi-update}
\end{center}
\end{figure}

If~$T_2$ commits first with a commit-time version~\num{2}, and $T_3$
afterwards with commit-time version~\num{3}, the logical state of the database
after both commit operations is shown in
\figref{fig:tmvbt:multi-update:t2first}. 
We assume here that a lazy timestamping scheme is employed so that the
actual entries stored in the database may be the same entries as shown in
\figref{fig:tmvbt:multi-update:active}, but that when the page is
later accessed the pending updates on the page will be converted into the
entries shown in \figref{fig:tmvbt:multi-update:t2first}.
In this situation, \invref{inv:tmvbt-live-count} holds for the page~$p$,
because $\entries{1} = 2$, $\entries{2} = 6$, and $\entries{3} = 4$; where
\entries{v} denotes the number of entries that are alive at version~$v$.


%% Fundamentals of Snapshot Database Theory
%%---------------------------------------------------------------------
\section{Fundamentals of Snapshot Database Theory}
\label{sec:mv-data:snapshot}

%In traditional snapshot-database transaction
%theory~\cite{bernstein:1987:cc-n-r,papadimitriou:1986:cc-theory},
%database indexes are collections 
We assume our logical database consists of data items of the form $(k,
w)$, where $k$~is the key and $w$~is the value of
the data item~\cite{sippu:2001:theory}.
%The value~$w$ can contain the actual data stored with the key (the
%row in the relation), in the case of a \emph{sparse} or
%\emph{primary index}; or a pointer to some storage location, in the
%case of a \emph{dense} or \emph{secondary index}. 
The logical model thus has no knowledge of any time dimension.
Databases that follow this data model are called
\emph{snapshot databases}\phantomsection\label{def:snapshot-database}.
As \figref{fig:tx-snapshot} shows, any change to a snapshot database
overwrites the information about previous states.

% TMVBT live count
\thmskip
\begin{invariant}
\label{inv:tmvbt-live-count}
That is, for all versions~$v$ and all pages~$p$, page~$p$ contains at least
\minlive\ entries that are alive at version~$v$; or $p$ is a root page of
$S_v$, in which case it contains at least \num{2}~entries that are alive at
version~$v$; or $p$ is the only page of $S_v$ and contains at least one entry
that is alive at version~$v$; or $p$ is not part of the search tree~$S_v$ and
therefore contains no entries that are alive at version~$v$.
\end{invariant}
\thmskip

\begin{figure}[htb]
\centering
\subfigure[Before]{%
\begin{tikzpicture}[x=\txfigx,y=\txfigy,every node/.style={font=\footnotesize}]
  \coordinate (o) at (0,0);
  \coordinate (oo) at (0,0.2);
  \coordinate (ok) at ($(oo) + (0,5.6)$);

  \coordinate[label=right:{$(1,\alpha)$}] (a) at ($(o) + (0,1)$);
  \coordinate[label=right:{$(2,\beta)$}] (b) at ($(a) + (0,1)$);
  \coordinate[label=right:{$(3,\gamma)$}] (c) at ($(b) + (0,1)$);
  \coordinate[label=right:{$(4,\delta)$}] (d) at ($(c) + (0,1)$);
  
  \foreach \point in {a,b,c,d}
    \fill[black] (\point) circle (2pt);

  \draw[->] (oo) -- node[lbl,above] {Keys} (ok);
\end{tikzpicture}
%\:
%\raisebox{2.5\txfigy}{$\Rightarrow$}
%\:
}
\subfigure[After]{%
\begin{tikzpicture}[x=\txfigx,y=\txfigy,every node/.style={font=\footnotesize}]
  \coordinate (o) at (0,0);
  \coordinate (oo) at (0,0.2);
  \coordinate (ok) at ($(oo) + (0,5.6)$);

  \coordinate[label=right:{$(1,\alpha')$}] (a) at ($(o) + (0,1)$);
  \coordinate[label=right:{$(3,\gamma)$}] (c) at ($(a) + (0,2)$);
  \coordinate[label=right:{$(4,\delta)$}] (d) at ($(c) + (0,1)$);
  \coordinate[label=right:{$(5,\epsilon)$}] (e) at ($(d) + (0,1)$);

  \foreach \point in {a,c,d,e}
    \fill[black] (\point) circle (2pt);

  \draw[->] (oo) -- node[lbl,above] {Keys} (ok);
\end{tikzpicture}%
}

\figcaption{Transaction in a snapshot database}{The transaction has
executed the schedule $b w[1,\alpha,\alpha'] d[2] w[5,\epsilon] c$.}
\label{fig:tx-snapshot}
\end{figure}

\begin{figure}[htb]
\centering
\subfigure[Before]{%
\begin{tikzpicture}[x=\txfigx,y=\txfigy,every node/.style={font=\footnotesize}]
  \coordinate (o) at (0,0);
  \coordinate (oo) at (0,0.2);
  \coordinate (ox) at ($(oo) + (0,5.6)$);
  \coordinate (ot) at ($(oo) + (5,0)$);

  \coordinate (a) at ($(o) + (0,1)$);
  \coordinate (b) at ($(o) + (1,2)$);
  \coordinate (c) at ($(b) + (-1,1)$);
  \coordinate (d) at ($(c) + (2,1)$);
  
  \draw[|-|] (a) -- node[above] {$(1,\alpha)$} ($(a) + (2,0)$);
  \draw[|-|] (b) -- node[above] {$(2,\beta)$} ($(b) + (3,0)$);
  \draw[|-|] (c) -- node[above] {$(3,\gamma)$} ($(c) + (1,0)$);
  \draw[|-|] (d) -- node[above] {$(4,\delta)$} ($(d) + (2,0)$);

  \draw[->] (oo) -- node[lbl,above] {Keys} (ox);
  \draw[->] (oo) -- node[lbl,below] {Time} (ot);
\end{tikzpicture}
%\:
%\raisebox{\txfigy}{$\Rightarrow$}
%\:
}
\subfigure[After]{%
\begin{tikzpicture}[x=\txfigx,y=\txfigy,every node/.style={font=\footnotesize}]
  \coordinate (o) at (0,0);
  \coordinate (oo) at (0,0.2);
  \coordinate (ox) at ($(oo) + (0,5.6)$);
  \coordinate (ot) at ($(oo) + (5,0)$);

  \coordinate (a) at ($(o) + (1,1)$);
  \coordinate (c) at ($(o) + (0,3)$);
  \coordinate (d) at ($(c) + (2,1)$);
  \coordinate (e) at ($(d) + (-1,1)$);

  \draw[|-|] (a) -- node[above] {$(1,\alpha')$} ($(a) + (3,0)$);
  \draw[|-|] (c) -- node[above] {$(3,\gamma)$} ($(c) + (1,0)$);
  \draw[|-|] (d) -- node[above] {$(4,\delta)$} ($(d) + (2,0)$);
  \draw[|-|] (e) -- node[above] {$(5,\epsilon)$} ($(e) + (2,0)$);

  \draw[->] (oo) -- node[lbl,above] {Keys} (ox);
  \draw[->] (oo) -- node[lbl,below] {Time} (ot);

\end{tikzpicture}%
}
\figcaption{Transaction in a valid-time database}{The transaction has
executed the schedule $b w[1,\alpha,\alpha'] d[2] w[5,\epsilon] c$.}
\label{fig:tx-valid}
\end{figure}

\begin{figure}[htb]
\centering
\subfigure[Before]{%
\begin{tikzpicture}[x=\txfigx,y=\txfigy,every node/.style={font=\footnotesize}]
  \coordinate (o) at (0,0);
  \coordinate (oo) at (0,0.2);
  \coordinate (ox) at ($(oo) + (0,6)$);
  \coordinate (ov) at ($(oo) + (4,0)$);

  \coordinate (cur) at ($(o) + (1,6.8)$);

  \coordinate (oa) at ($(o) + (0,1)$);
  \coordinate (ob) at ($(o) + (0,2)$);
  \coordinate (oc) at ($(o) + (0,3)$);
  \coordinate (od) at ($(o) + (0,4)$);

  \coordinate[label=right:{$(1,\alpha)$}] (a) at ($(oa) + (1,0)$);
  \coordinate[label=right:{$(2,\beta)$}] (b) at ($(ob) + (1,0)$);
  \coordinate[label=right:{$(3,\gamma)$}] (c) at ($(oc) + (1,0)$);
  \coordinate[label=right:{$(4,\delta)$}] (d) at ($(od) + (1,0)$);
  
  \foreach \point in {a,b,c,d}
    \fill[black] (\point) circle (2pt);

  \draw[-] (oa) -- (a);
  \draw[-] (ob) -- (b);
  \draw[-] (oc) -- (c);
  \draw[-] (od) -- (d);

  \draw[->] (oo) -- node[lbl,above] {Keys} (ox);
  \draw[->] (oo) -- node[lbl,below] {Versions} (ov);
  \draw[->] (cur) -- ($(cur) + (0,-0.5)$);
\end{tikzpicture}
%\:
%\raisebox{\txfigy}{$\Rightarrow$}
%\:
}
\subfigure[After]{%

\begin{tikzpicture}[x=\txfigx,y=\txfigy,every node/.style={font=\footnotesize}]
  \coordinate (o) at (0,0);
  \coordinate (oo) at (0,0.2);
  \coordinate (ox) at ($(oo) + (0,6)$);
  \coordinate (ov) at ($(oo) + (4,0)$);

  \coordinate (cur) at ($(o) + (2.5,6.8)$);

  \coordinate (oa) at ($(o) + (0,1)$);
  \coordinate (ob) at ($(o) + (0,2)$);
  \coordinate (oc) at ($(o) + (0,3)$);
  \coordinate (od) at ($(o) + (0,4)$);
  \coordinate (oe) at ($(o) + (0,5)$);

  \coordinate[label=right:{$(1,\alpha)$}] (a1) at ($(oa) + (1,0)$);
  \coordinate[label=right:{$(2,\beta)$}] (b1) at ($(ob) + (1,0)$);
  \coordinate (c1) at ($(oc) + (1,0)$);
  \coordinate (d1) at ($(od) + (1,0)$);

  \coordinate[label=right:{$(1,\alpha')$}] (a2) at ($(oa) + (2.5,0)$);
  \coordinate[label=right:{$(3,\gamma)$}] (c2) at ($(oc) + (2.5,0)$);
  \coordinate[label=right:{$(4,\delta)$}] (d2) at ($(od) + (2.5,0)$);
  \coordinate[label=right:{$(5,\epsilon)$}] (e2) at ($(oe) + (2.5,0)$);

  \draw[-] (oa) -- (a1);
  \draw[-] (ob) -- (b1);
  \draw[-] (oc) -- (c2);
  \draw[-] (od) -- (d2);

  \foreach \point in {a1,a2,b1,c1,c2,d1,d2,e2}
    \fill[black] (\point) circle (2pt);

  \draw[->] (oo) -- node[lbl,above] {Keys} (ox);
  \draw[->] (oo) -- node[lbl,below] {Versions} (ov);
  \draw[->] (cur) -- ($(cur) + (0,-0.5)$);
\end{tikzpicture}%
}
\figcaption{Transaction in a transaction-time database}{The
transaction has executed the schedule $b w[1,\alpha,\alpha'] d[2] 
w[5,\epsilon] c$.
The arrow on the top shows the latest committed version~\comver.}
\label{fig:tx-transaction}
\end{figure}

In the traditional transaction model, transactions in snapshot
databases consist of user actions that either \emph{write} or
\emph{read} keys to/from the database.
In addition, the transaction must issue either a \emph{commit} or an
\emph{abort action} to indicate that the transaction is finished.
After an abort action, the read and write actions are undone and the
transaction commits.  
A read action of a key $k$ for a transaction $T_i$ is denoted by 
$r_i[k]$, and a write action by $w_i[k]$.
Commit and abort are denoted by $c_i$ and $a_i$, respectively.
This classical transaction model is inadequate for
modelling key-range queries, next-key queries, and it does not
differentiate between key insertions, updates, and deletes.
Mohan has proposed a more general transaction model alongside his 
ARIES recovery algorithm~\cite{mohan:1990:aries-kvl,mohan:1992:aries-im},
which Sippu and Soisalon-Soininen have
further refined~\cite{sippu:2001:theory}. 

\begin{figure}
\begin{center}
\begin{enumerate}
\setlength{\itemsep}{0pt}
  \item $T_1$ inserts entries $(1, w_1)$, $(2, w_2)$, $(3,
  w_3)$; commits with version $v_1$.
  \item $T_2$ inserts entries $(3, w_3')$, $(4, w_4)$; commits
  with version $v_2$.
  \item $T_3$ inserts entries $(1, w_1')$, $(5, w_5)$; commits
  with version $v_3$.
\end{enumerate}
\caption{An example history of a multiversion database}
\label{fig:mv-tx-workload-old}
\end{center}
\end{figure}

\begin{figure}
\begin{center}
\begin{tikzpicture}[x=\txfigx,y=\txfigy,every node/.style={font=\footnotesize}]
  \coordinate (o) at (0,0);
  \coordinate (oo) at (0,0.2);
  \coordinate (ox) at ($(oo) + (0,6)$);
  \coordinate (ov) at ($(oo) + (6,0)$);

  \coordinate (cur) at ($(o) + (4,6.8)$);

  \coordinate[label=left:{$1$}] (oa) at ($(o) + (0,1)$);
  \coordinate[label=left:{$2$}] (ob) at ($(o) + (0,2)$);
  \coordinate[label=left:{$3$}] (oc) at ($(o) + (0,3)$);
  \coordinate[label=left:{$4$}] (od) at ($(o) + (0,4)$);
  \coordinate[label=left:{$5$}] (oe) at ($(o) + (0,5)$);

  \coordinate[label=below:{$v_1$}] (ov1) at ($(o) + (1,0)$);
  \coordinate[label=below:{$v_2$}] (ov2) at ($(o) + (2.5,0)$);
  \coordinate[label=below:{$v_3$}] (ov3) at ($(o) + (4,0)$);

  \coordinate (a1) at ($(oa) + (1,0)$);
  \coordinate[label=right:{$(1,w_1)$}] (a2) at ($(a1) + (1.5,0)$);
  \coordinate[label=right:{$(1,w_1')$}] (a3) at ($(a2) + (1.5,0)$);

  \coordinate (b1) at ($(ob) + (1,0)$);
  \coordinate (b2) at ($(b1) + (1.5,0)$);
  \coordinate[label=right:{$(2,w_2)$}] (b3) at ($(b2) + (1.5,0)$);

  \coordinate[label=right:{$(3,w_3)$}] (c1) at ($(oc) + (1,0)$);
  \coordinate (c2) at ($(c1) + (1.5,0)$);
  \coordinate[label=right:{$(3,w_3')$}] (c3) at ($(c2) + (1.5,0)$);

  \coordinate (d2) at ($(od) + (2.5,0)$);
  \coordinate[label=right:{$(4,w_4)$}] (d3) at ($(d2) + (1.5,0)$);

  \coordinate[label=right:{$(5,w_5)$}] (e3) at ($(oe) + (4,0)$);

  \draw[-] (oa) -- (a2);
  \draw[-] (ob) -- (b3);
  \draw[-] (oc) -- (c1);
  \draw[-] (c2) -- (c3);
  \draw[-] (d2) -- (d3);

  \foreach \point in {a1,a2,a3,b1,b2,b3,c1,c2,c3,d2,d3,e3}
    \fill[black] (\point) circle (2pt);

  \draw[->] (oo) -- node[lbl,above=1em] {Keys} (ox);
  \draw[->] (oo) -- node[lbl,below=1.3em] {Versions} (ov);
  \draw[->] (cur) -- ($(cur) + (0,-0.5)$);
\end{tikzpicture}

\caption{An example history of a multiversion database}
\label{fig:mv-tx-workload}
\end{center}
\end{figure}

Another approach for avoiding phantoms is to use predicate locking,
but this is often unpractical as predicate satisfiability in general
is a known \NP-complete problem and thus the approach can be highly
inefficient~\cite{gray:1993:transactionprocessing}.
A drawback of the serializable concurrency-control algorithms is that
they hinder the overall performance of the database system by forcing
other transactions to wait for access to locked keys.
Database management systems thus allow the user to set some
transactions to run on a lower isolation level to enhance the
performance of the system.

\begin{figure}[!htb]
\begin{center}
  \tikzstyle{actverline}=[dashed]
\tikzstyle{entryline}=[densely dotted]
\tikzstyle{area}=[ellipse,draw,dashed]
\tikzstyle{liveentry}=[entryline,postaction={%
  decorate,%
  decoration={%
    markings,%
    mark=at position .5pt with{\arrowreversed[line width=.35pt]{|}};,
    mark=at position 1 with{%
      \arrow[line width=.8pt]{stealth}};%
  }%
}]
\tikzstyle{deadentry}=[entryline,postaction={%
  decorate,%
  decoration={%
    markings,%
    mark=at position .5pt with{\arrowreversed[line width=.35pt]{|}};,
    mark=at position 1 with{%
      \arrow[line width=.8pt]{stealth}};%
%      \arrow[line width=.35pt,black,fill=white]{*}};% 
  }%
}]
\tikzstyle{pageborder}=[thick]
\tikzstyle{pagename}=[anchor=center,text=black!50,font=\huge]
\newlength{\spexx}
\setlength{\spexx}{1.3\tmvbtsmox}
\newlength{\spexy}
\setlength{\spexy}{.7\tmvbtsmoy}

\subfigure[Before]{
\begin{tikzpicture}[x=\spexx,y=\spexy,%
  every pin edge/.style={draw,dotted},%
  pin distance=.5\spexx]
  \small
  \coordinate (lo) at (0,-5);
  \coordinate (o) at (0,0);
  \coordinate (hi) at (0,7);
  \coordinate (loact) at (3,-5);
  \coordinate (oact) at (3,0);
  \coordinate (hiact) at (3,7);
  \coordinate (loinf) at (4,-5);
  \coordinate (oinf) at (4,0);
  \coordinate (hiinf) at (4,7);
  
  \draw[->] (lo) -- node[below,lbl] {Versions} ($(loinf) + (1em,0)$);
  \draw[->] (lo) -- node[above,lbl] {Keys} ($(hi) + (0,1em)$);
  
  \draw[pageborder] (lo) -- (o) -- (hi);
  \draw[pageborder] (lo) -- (loinf);
  \draw[pageborder] (hi) -- (hiinf);
  \draw[pageborder] (o) -- (oinf);

  \draw[opendash] decorate [opendeco] { (loinf) -- (oinf) };
  \draw[opendash] decorate [opendeco] { (oinf) -- (hiinf) };

  \node[pagename] at (2,3.5) {$p$};
  \node[pagename] at (2,-2.5) {$s$};
  
  \draw[actverline] ($(loact) + (0,-1em)$) node[below=.4em] {\actver} --  
    ($(hiact) + (0,1em)$);

  % Page p contents
  \draw[deadentry] (0,6) -- (1,6);
  \draw[deadentry] (1,6) -- (2,6);
  \draw[deadentry] (2,6) -- (3,6);
  \draw[liveentry] (0,5) -- (4,5);
  \draw[deadentry] (0,4) -- (2,4);
  \draw[deadentry] (1,3) -- (3,3);
  \draw[deadentry] (0,2) -- (1,2);
  \draw[deadentry] (2,2) -- (3,2);
  \draw[deadentry] (0,1) -- (2,1);
  \draw[deadentry] (2,1) -- (3,1);

  % Page s contents
  \draw[liveentry] (0,-1) -- (4,-1);
  \draw[deadentry] (0,-2) -- (2,-2);
  \draw[deadentry] (0,-3) -- (3,-3);
  \draw[liveentry] (3,-4) -- (4,-4);
  
  \node[area,minimum width=4.5\spexx,minimum height=.6\spexy,pin=178:{$1$}] 
    at (2,5) {};
  \node[area,minimum width=4.5\spexx,minimum height=.6\spexy,pin=182:{$2$}] 
    at (2,-1) {};
  \node[area,minimum width=1.5\spexx,minimum height=.6\spexy,pin=330:{$3$}] 
    at (3.5,-4) {};
  
\end{tikzpicture}}
\subfigure[After]{
\begin{tikzpicture}[x=\spexx,y=\spexy,%
  every pin edge/.style={draw,dotted},%
  pin distance=.5\spexx]
  \small
  \coordinate (lo) at (0,-5);
  \coordinate (o) at (0,0);
  \coordinate (hi) at (0,7);
  \coordinate (loact) at (3,-5);
  \coordinate (oact) at (3,0);
  \coordinate (hiact) at (3,7);
  \coordinate (loinf) at (4,-5);
  \coordinate (oinf) at (4,0);
  \coordinate (hiinf) at (4,7);

  \draw[->] (lo) -- node[lbl,below] {Versions} ($(loinf) + (1em,0)$);
  \draw[->] (lo) -- node[lbl,above] {Keys} ($(hi) + (0,1em)$);
  
  \draw[pageborder] (lo) -- (o) -- (hi);
  \draw[pageborder] (lo) -- (loinf);
  \draw[pageborder] (hi) -- (hiinf);
  \draw[pageborder] (o) -- (oact);

  \draw[opendash] decorate [opendeco] { (loinf) -- (oinf) };
  \draw[opendash] decorate [opendeco] { (oinf) -- (hiinf) };
  
  \node[pagename] at (1.5,3.5) {$p$};
  \node[pagename] at (1.5,-2.5) {$s$};
  \node[pagename] at (3.5,1) {$p'$};
  
  \draw[actverline] ($(loact) + (0,-1em)$) node[below=.4em] {\actver} -- (loact); 
  \draw[pageborder] (loact) -- (hiact);
  \draw[actverline] (hiact) -- ($(hiact) + (0,1em)$);

  % Page p contents
  \draw[deadentry] (0,6) -- (1,6);
  \draw[deadentry] (1,6) -- (2,6);
  \draw[deadentry] (2,6) -- (3,6);
  \draw[deadentry] (0,5) -- (3,5);
  \draw[deadentry] (0,4) -- (2,4);
  \draw[deadentry] (1,3) -- (3,3);
  \draw[deadentry] (0,2) -- (1,2);
  \draw[deadentry] (2,2) -- (3,2);
  \draw[deadentry] (0,1) -- (2,1);
  \draw[deadentry] (2,1) -- (3,1);

  % Page s contents
  \draw[deadentry] (0,-1) -- (3,-1);
  \draw[deadentry] (0,-2) -- (2,-2);
  \draw[deadentry] (0,-3) -- (3,-3);

  % Page p' contents
  \draw[liveentry] (3,5) -- (4,5);
  \draw[liveentry] (3,-1) -- (4,-1);
  \draw[liveentry] (3,-4) -- (4,-4);

  \node[area,minimum width=1.5\spexx,minimum height=.6\spexy,pin=10:{$1$}] 
    at (3.5,5) {};
  \node[area,minimum width=1.5\spexx,minimum height=.6\spexy,pin=3:{$2$}] 
    at (3.5,-1) {};
  \node[area,minimum width=1.5\spexx,minimum height=.6\spexy,pin=330:{$3$}] 
    at (3.5,-4) {};
\end{tikzpicture}
}
  \figcaption{Example of a page split}
  {This figure shows a version split followed by a merge with an inactive
  sibling. 
  Entries with keys~\num{1} and~{2} are killed and new copies of them are
  created in page~$p'$, while the entry with key~\num{3} is physically moved
  from page~$s$ into page~$p'$.}
  \label{fig:split-imi1-example}
\end{center}
\end{figure}

The form of transactions in multiversion databases is a generalization of 
the recoverable snapshot database theory presented by Sippu and
Soisalon-Soininen~\cite{sippu:2001:theory}, with the
addition of setting and restoring savepoints as presented for B-trees in
page-server database systems by Jaluta et~al.~\cite{jaluta:2006:page-server}.
\phantomsection\label{def:cmvbt}\label{def:immortaldb}
\label{def:mvas}
\label{def:mvbt}
\label{def:si}
\label{def:deletemark}
\label{def:smo}
\label{def:tmvbt}
\label{def:vtt}
\label{def:vbt}
\label{def:wal}
\label{def:lru}
\label{def:cti}
\label{def:lsn}
\label{def:dag}
\label{def:ptt}
\label{def:tsb}
\label{def:worm}
\label{def:wmrm}
\label{def:mbr}

%% General Transaction Processing
%%---------------------------------------------------------------------
\section{General Transaction Processing}
\label{sec:performance:general}
Each of the figures shows two graphs, one run with the short
workloads and another run with the long-transaction workloads.
The graphs show buffer fixes, page reads and the real time required for each
test. 
The \mbox{$x$-axis} in the graphs represents the percentage of updating
transactions in the workload. 
The update probabilities were set to \SI{0}{\percent}, \SI{10}{\percent},
\SI{20}{\percent}, \ldots, \SI{100}{\percent}. 
As explained in \secref{sec:performance:implementation}, we believe that
the interval of the maintenance transaction has a direct effect to the
performance of the CMVBT structure.
We have therefore run the tests with varying settings for the
maintenance interval. 
The interval is denoted~$m$ in the figures, and indicates that we allow $m$
transactions to commit before running the maintenance transaction.
When the maintenance transaction is run, it is run $m$ times to apply
the updates of all the committed transactions into the TMVBT index.


\phantomsection
\label{def:actver}
\label{def:capacity}
\label{def:commitver}
\label{def:entries}
\label{def:kr}
\label{def:vr}
\label{def:maxcomver}
\label{def:maxsplit}
\label{def:minlive}
\label{def:minsplit}
\label{def:movever}
\label{def:rootstar}
\label{def:snapver}
\label{def:stablever}
\label{def:txid}

Updating transactions modify the same data set, in the order the
update actions are issued.
Concurrency-control and recovery algorithms are used to maintain the
\abbr{ACID}\phantomsection\label{def:acid} properties of transactions: 
atomicity, consistency, isolation, and
durability~\cite{gray:1993:transactionprocessing}. 
